\subsection{Quantifier elimination in algebraically closed fields
and Nullstellensatz}

\begin{lem}[Terms in the language of rings are polynomials]
    \link{terms_in_RNG_are_polynomials}
    Let $X$ be a subset of a ring $A$.
    Then $\Si_\RNG(X)$-terms are interpreted as 
    polynomials with coefficients from $\<X\>$,
    the smallest subring of $A$ generated by $X$.
\end{lem}
\begin{proof}
    Let $t$ be a $\Si_\RNG(X)$-term.
    \begin{itemize}
        \item If $t$ is a constant $c$ then it's interpretation is
            $\modintp{\MM}{c}$, a constant polynomial. 
            Since $c$ is $0$, $1$ or something in $X$,
            this is in any polynomial ring over $\<X\>$.
        \item If $t$ is a variable $v_i$ then it is interpreted as
            the single variable polynomial $v_s \in \<X\>[v_s]$.
        \item If $t$ is $f(s_1,\dots,s_n)$ and each $s_i$ 
            is interpreted as a polynomial $q_i$,
            take the polynomial ring $\<X\>[v_1,\dots,v_k]$
            containing all the $q_i$ (everything is finite).
            Then $f(q_1,\dots,q_n)$ 
            is still a polynomial in $\<X\>[v_1,\dots,v_k]$
            as $f$ is either $+,-$ or $\cdot$.
    \end{itemize} 
\end{proof}

\begin{lem}[Disjunctive normal form for rings]
    \link{disjunctive_normal_form}
    Let $A$ be a ring or the empty set.
    Let $\phi$ be a quantifier free $\Si_\RNG(A)$-formula with variables indexed
    by $S$.
    Then there exist $\Si_\RNG(A)$-terms 
    (i.e. \linkto{terms_in_RNG_are_polynomials}{polynomials})
    $p_{ij}, q_{ij}$ such that
    for any $\Si_\RNG(A)$-structure $\MM$
    \[ 
        \MM \model{\Si} \bigforall{s \in S}{v_s}, \sqbrkt{\phi(v) 
        \IFF \bigor{i \in I}{} 
        \brkt{\bigand{j \in J_{i0}}{} p_{ij}(v) = 0 \AND 
        \bigand{j \in J_{i1}}{} q_{ij}(v) \ne 0}}
    \]
\end{lem}
\begin{proof}
    Applying the general 
    \linkto{disjunctive_normal_form_0}{disjunctive normal form} we have that
    any formula of the form $s = t$ or $r(t)$ is just a polynomial equation, 
    since there are no relation symbols.
    Moving everything to one side we have that they are of the form 
    $p_{ij}(v) = 0$ and $q_{ij}(v) \ne 0$.
\end{proof}

\begin{rmk}[Substructures of algebraically closed fields]
    Since $\GRP, \RNG, \ID$ are all universal axomatizations of themselves,
    by \linkto{universal_axiomatizations_make_subs_mods}{'universal 
    axiomatizations make substructures models'} we have that
    substructures of groups are groups, 
    substructures of rings are rings, 
    and substructures of integral domains are integral domains.
    Furthermore, 
    substructures of algebraically closed fields are 
    substructures of integral domains hence are integral domains.
    This becomes relevant when proving the equivalent condition on quantifier
    elimination.
\end{rmk}

\begin{nttn}
    If $A$ is a ring and 
    $I \subs A[x_1, \dots, x_n]$ is a set of polynomials then
    the vanishing of $I$ over $A$ is
    \[\V_{A}(I) := \set{c \in A^n \st \forall f \in I, f(c) = 0}\]
\end{nttn}

\begin{prop}
    \link{ACF_has_quantifier_elimination}
    $\ACF$ has quantifier elimination.
\end{prop}
\begin{proof}
    Let $\phi$ be a quantifier free $\Si_\RNG$-formula with a free variable $w$
    and index the rest by $S$.
    Let $K, L$ be algebraically closed field with $A$ embedding into each via
    $\io_K, \io_L$.
    Let $a \in A^S$.
    \linkto{improved_condition_for_q_e}{It suffices to show that}
    \[  
        K \model{\Si_\RNG} \exists w, \phi(\io_K(a),w) \implies
        L \model{\Si_\RNG} \exists w, \phi(\io_L(a),w)
    \]
    Crucially by the remark before this proposition $A$ 
    is an integral domain thus 
    we can consider $\io : A \to \Om$ the embedding into
    an algebraic closure of the field of fractions of $A$.
    As $K$ and $L$ are fields extending the fraction field of $A$,
    by the properties of field of fractions and algebraic closures
    there are field extensions $\ka : \Om \to K$ and $\la : \Om \to L$ 
    such that $\io_K = \ka \circ \io$ and $\io_L = \la \circ \io$.
    Let we write $a' := \io(a)$, 
    and have $\io_K(a) = \ka(a')$
    and $\io_L(a) = \la(a')$.
    \begin{cd}
        & & K\\
        A \ar[urr] \ar[drr] \ar[r] &F \ar[ur] \ar[dr] \ar[r] 
        & \Om \ar[u] \ar[d]\\
        & & L
    \end{cd}

    We can find the 
    \linkto{disjunctive_normal_form}{`disjunctive normal form' of} 
    $\phi$, i.e. 
    $\Si_\RNG$-terms $p_{ij}, q_{ij}$ such that any ring $\MM$
    satisfies 
    \[ 
        \MM \model{\Si} \bigforall{s \in S}{v_s},\forall w, \sqbrkt{\phi(v,w)
        \IFF \bigor{i \in I}{} 
        \brkt{\bigand{j \in J_{i0}}{} p_{ij}(v,w) = 0 \AND 
        \bigand{j \in J_{i1}}{} q_{ij}(v,w) \ne 0}}
    \]
    Hence assuming for some $b \in K$ that 
    \[
        K \model{\Si_\RNG} \bigor{i \in I}{} 
        \brkt{\bigand{j \in J_{i0}}{} p_{ij}(\ka(a'),b) = 0 \AND 
        \bigand{j \in J_{i1}}{} q_{ij}(\ka(a'),b) \ne 0}
    \] 
    it suffices to show that there exists a 
    $c \in L$ such that
    \[
        L \model{\Si_\RNG} \bigor{i \in I}{} 
        \brkt{\bigand{j \in J_{i0}}{} p_{ij}(\la(a'),c) = 0 \AND 
        \bigand{j \in J_{i1}}{} q_{ij}(\la(a'),c) \ne 0}
    \]
    Assume for a contradiction that $I$ is empty 
    then by convension we have the disjunctive normal form
    is $\bot$ and so $K \model{\Si_\RNG} \bot$ which implies $K$ is empty,
    which is a contradiction as the constant symbol $0$ 
    has an interpretation in $K$.

    Thus there exists $i \in I$ (we keep this $i$ for the rest of the proof) 
    such that 
    \[
        K \model{\Si_\RNG} \bigand{j \in J_{i0}}{} p_{ij}(\ka(a'),b) = 0 \AND 
        \bigand{j \in J_{i1}}{} q_{ij}(\ka(a'),b) \ne 0
    \]
    Now case on whether or not there exists a $j \in J_{i0}$ such that 
    $\modintp{K}{p_{ij}}$ is not the zero polynomial.
    If it exists then $\modintp{K}{p_{ij}}(\ka(a'),b) = 0$.
    As $p_{ij}$ is a $\Si_\RNG$-term and so 
    $\ka \modintp{\Om}{p_{ij}} = \modintp{K}{p_{ij}}$.
    (Intuitively it is a polynomial over $\<0,1\>$ as shown in
    \link{disjunctive_normal_form}{`disjunctive normal form'}.)
    Thus $b$ is algebraic over $\Om$, which is algebraically closed.
    Hence there is a $c \in \Om$ such that $\ka (c) = b$:
\begin{align*}
    &\quad 
        \bigand{j \in J_{i0}}{} \modintp{K}{p_{ij}}(\ka(a'),\ka(c)) = 0 \AND 
        \bigand{j \in J_{i1}}{} \modintp{K}{q_{ij}}(\ka(a'),\ka(c)) \ne 0 \\
    &\implies 
        \bigand{j \in J_{i0}}{} \ka \brkt{\modintp{\Om}{p_{ij}}(a',c)} = 0 \AND 
        \bigand{j \in J_{i1}}{} \ka \brkt{\modintp{\Om}{q_{ij}}(a',c)} \ne 0 \\
    &\implies
        \bigand{j \in J_{i0}}{} \brkt{\modintp{\Om}{p_{ij}}(a',c)} = 0 \AND 
        \bigand{j \in J_{i1}}{} \brkt{\modintp{\Om}{q_{ij}}(a',c)} \ne 0 
    \quad \quad \ka \text{ is injective} \\
    &\implies 
        \bigand{j \in J_{i0}}{} \la \brkt{\modintp{\Om}{p_{ij}}(a',c)} = 0 \AND 
        \bigand{j \in J_{i1}}{} \la \brkt{\modintp{\Om}{q_{ij}}(a',c)} \ne 0 \\
    &\implies
       \bigand{j \in J_{i0}}{} \brkt{\modintp{L}{p_{ij}}(\la(a'),\la(c))} = 0\AND 
       \bigand{j \in J_{i1}}{} \brkt{\modintp{L}{q_{ij}}(\la(a'),\la(c))} \ne 0\\
    &\implies
    L \model{\Si_\RNG} \bigor{i \in I}{} 
        \brkt{\bigand{j \in J_{i0}}{} p_{ij}(\la(a'),\la(c))) = 0 \AND 
        \bigand{j \in J_{i1}}{} q_{ij}(\la(a'),\la(c))) \ne 0}
\end{align*}
    Thus we are done with this case.
    
    In the other case we turn our attention to $J_{i1}$.
    If $j \in J_{i1}$, 
    we see $\modintp{\Om}{q_{ij}}(a',w)$ as a polynomial
    in $\Om[w]$.
    $\modintp{K}{q_{ij}}(\ka(a'),b) \ne 0$ and so 
    $\modintp{\Om}{q_{ij}}(a',w)$ is not the zero polynomial
    hence $\modintp{L}{q_{ij}}(\la(a'),w)$
    has finitely many zeros in $L$ by the division algorithm in 
    $\Om[w]$.
    Let 
    \[\V_L(q_{ij})_{j} = \set{c \in L \st \exists j \in J_{i1} 
    \st \modintp{L}{q_{ij}}(\la(a'),c) = 0}\]
    be the vanishing of the $q_{ij}$ for every $j \in J_{j1}$.
    Then $\V$ is finite and $L$ is 
    \linkto{card_of_alg_closed_fields}{infinite as it is algebraically closed}
    hence there exists a $c \in L$ 
    $L \model{\Si_\RNG} \bigand{j \in J_{i1}}{} q_{ij}(\la(a'),c) \ne 0$.
    Since each $f_{ij}$ are the zero polynomial we also have
    $L \model{\Si_\RNG} \bigand{j \in J_{i0}}{} p_{ij}(\la(a'),\la(c))) = 0$
    Hence we are done.
\end{proof}

\begin{prop}[Weak Nullstellensatz]
    If $K$ is an algebraically closed field and $\f{p}$ is a prime ideal of 
    $K[x_1, \dots, x_n]$, 
    then $\V_{K}(\f{p}) \ne \nothing$.
\end{prop}
\begin{proof}
    By \linkto{hilbert_basis}{Hilbert's basis theorem}
    $K[x_1, \dots, x_n]$ is Noetherian
    so we can find a finite set $\set{f_1, \dots, f_m}$ generating $\f{p}$.
    It suffices to find a common zero of $\set{f_1, \dots, f_m}$.
    We can write out each $f_i$ and use $a$ to represent 
    all the coefficients of all the $f_i$ as a tuple.
    We can construct some $\Si$-formula $\phi(v,x_1,\dots,x_n)$ 
    (a polynomial in variables $x_i$ with coefficients $v$
    corresponding to the tuple $a$)
    such that 
    for any $b \in K^n$,
    \[K \model{\Si_\RNG} \phi(a,b) \iff \bigand{j = 1}{m} f_i(b) = 0\]
    and for any extension $\io : K \to L$ and $b \in L^n$,
    \[  
        L \model{\Si_\RNG} \phi(\io(a),b) 
        \iff \bigand{j = 1}{m} \io(f_i)(b) = 0
    \]

    We can quotient $K[x_1, \dots, x_n] / \f{p}$ and take an algebraic closure
    $L$ (quotienting by prime ideals gives an integral domain).
    \begin{cd}
        K[x_1,\dots,x_n]  \ar[r] 
        & K[x_1,\dots,x_n] / \f{p} \ar[d, "\al"]\\
        K \ar[u] \ar[ur, "\ka"]\ar[ur, "\io"] & L
    \end{cd}
    We can then take $b := (x_1, \dots, x_n) \in (K[x_1, \dots, x_n])^n$
    and send it through to $\al(b) \in L^n$.
    \begin{align*}
        &\bigand{j = 1}{m} \ka(f_i)(b) = 0\\
        &\implies \bigand{j = 1}{m} \ka \circ \al (f_i)(\al(b)) = 0\\
        &\implies \bigand{j = 1}{m} \io(f_i)(\al(b)) = 0\\
        &\implies L \model{\Si_\RNG} \phi(\io(a),\al(b))\\
        &\implies L \model{\Si_\RNG} 
        \bigexists{i = 1}{n} x_i, \phi(\io(a),x)
    \end{align*}
    since $K$ and $L$ are both algebraically closed and 
    \linkto{ACF_has_quantifier_elimination}{$\ACF$ has quantifier elimination} 
    so it is \linkto{
        quantifier_elimination_implies_model_completeness}{model complete},
    which implies that the embedding is elementary.
    Hence we have
    \begin{align*}
        &\implies K \model{\Si_\RNG} 
        \bigexists{i = 1}{n} x_i, \phi(a,x)\\
        &\implies \exists c \in K^n, K \model{\Si_\RNG} \phi(a,c)\\
        &\implies \exists c \in K^n, \bigand{j = 1}{m} f_i(b) = 0
    \end{align*}
\end{proof}

\subsection{The Classical Zariski Topology, Chevalley, Vanishings}%

\begin{dfn}[Classical Zariski Topology]
    Let $K$ be an algebraically closed field and let 
    \[\set{\V_K(E) \st E \subs K[x_1, \dots, x_n]}\]
    be a closed basis for a topology on $K^n$.
    We call these Zariski closed sets.
\end{dfn}
The closed sets in this setting \linkto{zariski_correspondence}{correspond to }
closed sets in \linkto{prime_spec_zariski_top}{$\spec(K[x_1,\dots,x_n])$},
though the spaces are not homeomorphic.

\begin{prop}[Closed sets are finitely generated vanishings]
    \link{zariski_closed_sets_are_fin_gen}
    If $K$ is an algebraically closed field and 
    $V$ is a closed set of $K^n$ under the
    Zariski topology then $V = \V_K(S)$ for some finite subset 
    of the polynomial ring $S \subs K[x_1,\dots,x_n]$.
\end{prop}
\begin{proof}
    By definition of Zariski closed sets any element of the closed 
    basis is $\V_K(E)$ for some 
    $E \subs K[x_1,\dots,x_n]$
    We consider the ideal generated by $E$.
    Important point: This is finitely generated by the 
    \linkto{hilbert_basis}{Hilbert basis theorem}, 
    and so we can just require $E$ 
    to be finite without loss of generality.

    Moreover, 
    arbitrary intersection of such sets is also finitely generated:
    Let $E \in I$ be a collection of subsets of $K[x_1, \dots, x_n]$.
    Then $a \in \bigcap_{E \in I} \V_K(E)$ if and only if 
    for every $E \in I$ and every $f \in E, f(a) = 0$.
    This is true if and only if $a \in \V_K(\bigcup_{E \in I} E)$.
    Thus $\bigcap_{E \in I} \V_K(E) = \V_K(\bigcup_{E \in I} E)$
    which is finitely generated by the first point.

    Furthermore, any finite union of such sets is also finitely generated.
    We prove this by induction.
    For the empty case:
    \[\bigcup_{E \in \nothing} \V_K(E) = \nothing = 
    \V_K (K[x_1,\dots,x_n])\]
    which is finitely generated by the first point.
    It suffices to show that the union of two such sets is also finitely
    generated.
    \begin{align*}
        &a \in \V(F) \cup \V(G) 
        \iff \brkt{\bigand{f \in F}{} f(a) = 0} \OR 
        \brkt{\bigand{g \in G}{} g(a) = 0} \\
        & \iff \bigand{f \in F}{} \bigand{g \in G}{} 
        f(a) = 0 \OR g(a) = 0
        \iff \bigand{f \in F}{} \bigand{g \in G}{} 
        (fg)(a) = 0
    \end{align*}
    The last step is due to $K[x_1,\dots,x_n]$ being an integral domain.
    Hence we have a finite intersection
    $\V(F) \cup \V(G) = \V(\bigcap (fg)(a) = 0)$
    which is finitely generated by the first point.
\end{proof}

\begin{prop}[Constructable]
    \link{definable_is_constructable}
    Let $K$ be an algebraically closed field with the Zariski topology on $K^n$.
    Define the set $C$ inductively:
    \begin{itemize}
        \item[$\vert$] If $X \subs K^n$ is closed then it is in $C$.
        \item[$\vert$] If $X \subs K^n$ is in $C$ then $K^n \setminus X$
        is in $C$. 
        \item[$\vert$] If $X,Y \subs K^n$ are in $C$ then $X \cup Y$
        is in $C$. 
    \end{itemize}
    Then $C$ is the set of \linkto{constructable_dfn}{constructable} 
    and \linkto{definable_is_constructable_01}{equivalently} 
    the set of of definable sets in $K^n$
    (by \linkto{ACF_has_quantifier_elimination}{quantifier elimination}).

    $C$ consists of `finite boolean combinations' of closed sets, 
    and corresponds to the original definition `constructable'.
\end{prop}
\begin{proof}
    \begin{forward}
        First we show by induction on the set $C$ 
        that if $X$ is in $C$ then $X$ is constructble.
        If $X$ is closed then 
        \linkto{zariski_closed_sets_are_fin_gen}{$X = \V(S)$} for some 
        finite $S$. 
        Therefore $X$ is defined by 
        $\phi(v,b) := \bigand{f \in S}{} f(a) = 0$,
        where each $f$ is some $\Si_\RNG$ formula evaluated at $b \in K^m$.
        This is a finite and which is the negation of a finite or which is 
        (by induction) constructable.
        The rest of the induction follows immediately.
    \end{forward}
    
    \begin{backward}
        If $X$ is constructable then we show by induction that it is in $C$.
        If $X$ is defined by an atomic formula then it is either $\top$ 
        or $t = s$. If $\phi$ is $\top$ then $X = K^n$ which is closed
        hence in $C$.
        If $\phi$ is $t = s$ then 
        $\modintp{K}{t}(b)$ and $\modintp{K}{s}(b)$ are polynomials
        in $K[x_1,\dots,x_n]$.
        Writing $f = \modintp{K}{t}(b) - \modintp{K}{s}(b)$
        we have $X = \set{a \in K^n \st f(a) = 0}$,
        which is closed hence in $C$.
        The rest of the induction follows immediately.
    \end{backward}
\end{proof}

\begin{prop}[Chevalley]
    Over an algebraically closed field, 
    the image of a constructable set under a polynomial map is constructable.
\end{prop}
\begin{proof}
    Let $\rho : K^n \to K^m$ be a polynomial map defined by $(f_i)_{i=1}^m$. 
    Suppose $X \subs K^n$ is constructable.
    Then \linkto{definable_is_constructable}{
        as constructable is equivalent to definable over $K$} 
    there exists $\Si$-formula $\phi$ and $b \in K^l$ such that 
    \[X = \set{a \in K^n \st K \model{\Si_\RNG} \phi(a,b)}\]
    Then 
    \begin{align*}
        \rho(X) &= \set{c \in K^m \st \exists a \in K^m, K \model{\Si_\RNG}
        \AND \rho(a) = c}\\
        &= \set{c \in K^m \st \exists a \in K^m, K \model{\Si_\RNG}
            \AND \bigand{i = 1}{m} f_i(a) = c_i}\\
        &= \set{c \in K^m \st K \model{\Si_\RNG} \bigexists{j = 1}{n} x_j,
            \phi(x,b) \AND \bigand{i=1}{m} \phi_i(x,d) = c_i}
    \end{align*}
    The $d$ appearing in $\phi_i(x,d)$ is due to the fact that the polynomials 
    $f_i$ may have coefficients not from the language.
    Thus the image is constructable.
\end{proof}

\begin{nttn}[Radical]
    We write $r(\f{a})$ to be the radical of $\f{a}$.
\end{nttn}

\begin{dfn}[Ideal generated by subsets of $K^n$]
    If $K$ is a field, for a subset $X \subs K^n$, 
    we write $I(X)$ to mean the ideal of $X$ in $K[x_1,\dots,x_n]$ to mean
    \[\set{f \in K[x_1,\dots, x_n] \st \forall a \in X, f(a) = 0}\]
\end{dfn}

\begin{ex}[Taking ideals and vanishings are order reversing]
    \link{taking_ideals_order_reversing}
    Show that if $X \subs Y \subs K^n$ then $I(Y) \subs I(X)$.
    Show that if $E \subs F \subs K[x_1,\dots,x_n]$ then 
    $\V_K(F) \subs \V_K(E)$.
\end{ex}

\begin{prop}[Strong Nullstellensatz]
    \link{strong_nullstellensatz}
    Let $K$ be an algebraically closed field and suppose 
    $\f{a}$ is an ideal of $K[x_1,\dots,x_n]$.
    Then $r(\f{a}) = I(\V(\f{a}))$.
\end{prop}
\begin{proof}
    \linkto{strong_nullstellensatz_appendix}{See appendix.}
\end{proof}

\begin{prop}[The Zariski closed sets are Artinian]
    \link{zariski_closed_sets_artinian}
    Any descending chain of Zariski closed sets in $K^n$ for $K$ a field 
    stabilises.
\end{prop}
\begin{proof}
    Let $\dots \subs \V(\f{a}_1) \subs \V(\f{a}_0)$ 
    be a chain of Zariski closed sets.
    Then taking the ideals generated each of them we have an 
    \linkto{taking_ideals_order_reversing}{ascending chain}
    \[I(\V(\f{a}_0)) \subs I(\V(\f{a}_1)) \subs \dots\]
    This stabilises because 
    \linkto{hilbert_basis}{$K[x_1,\dots,x_n]$ is Noetherian}.
    Hence by \linkto{strong_nullstellensatz}{strong Nullstellensatz}
    we have that 
    \[r(\f{a}_0) \subs r(\f{a}_1) \subs \dots\]
    stabilises and
    taking the vanishing gives back
    the \linkto{taking_ideals_order_reversing}{descending chain} 
    $\dots \subs \V(\f{a}_1) \subs \V(\f{a}_0)$ which stabilises.
\end{proof}

\begin{dfn}[Irreducible, Variety]
    If $X$ is a topological space
    then the following are \linkto{irreducible_equiv_defs}{equivalent}:
    \begin{enumerate}
        \item Any non-empty open set is dense in $X$.
        \item Any pair of non-empty open subsets intersect non-trivially.
        \item Any two closed proper subsets do not form a cover of $X$.
    \end{enumerate}
    If any of the above hold then $X$ is said to be irreducible,
    and a subset of $X$ is irreducible if it is irreducible under the 
    subspace topology.

    A variety is a Zariski closed set that is irreducible.
\end{dfn}

\begin{cor}
    Zariski closed sets are finite unions of varieties.
\end{cor}
\begin{proof}
    For a contradiction suppose $\V_0$ 
    is a Zariski closed set that is not a finite union of varieties.
    Then $\V_0$ is not irreducible and so 
    there exists two closed proper subsets $\V_1, \V_1'$ that cover $\V_0$.
    If both $\V_1$ and $\V_1'$ are finite unions of varieties then we have a
    contradiction, 
    hence without loss of generality $\V_1$ is not a finite union of varieties.
    By induction we obtain 
    \[\dots \subset \V_1 \subset \V_0\]
    which \linkto{zariski_closed_sets_artinian}{stabilises}, a contradiction.
\end{proof}

\subsection{The Stone space and Spec}
\begin{prop}[Any polynomial is a formula]
    Let $K$ be an algebraically closed field
    and $v = (v_1,\dots,v_n)$ be variables.
    Then there exists a map 
    \[\eqzero : K[x_1,\dots,x_n] \to F(\Si_\RNG(K),v)\]
    such that for any $a \in K^n$ and any $f \in K[x_1,\dots,x_n]$,
    \[K \model{\Si_\RNG(K)} \eqzero_f(a) \iff f(a) = 0\]
    where $\eqzero_f$ is the image of an $f$.
\end{prop}
\begin{proof}
    This would be some nasty induction.
    I guess show that monomials can be made into formulas 
    and then sums of monomials can be made into formulas.
    Everything is finite so it should be okay.
\end{proof}

\begin{prop}[Bijection between the Stone space and spec]
    Let $K$ be an algebraically closed field.
    Then define the map
    \begin{align*}
        I : S_n(\eldiag{\Si}{K}) &\to \spec(K[x_1,\dots,x_n])\\
        p &\mapsto \set{f \in K[x_1,\dots,x_n] \st \eqzero_f \in p}
    \end{align*}
    We show that this is well-defined, continuous and a bijection.
    Hence $\spec(K[x_1,\dots,x_n])$ is compact.
    However, the spaces are \emph{not} homeomorphic.
\end{prop}
\begin{proof}
    Let $p \in S_n(\eldiag{\Si}{K})$.
    First we check that 
    $I_p := \set{f \in K[x_1,\dots,x_n] \st \eqzero_f \in p}$
    is a prime ideal.
    We will repeatedly use the following fact:
    since $p$ is consistent with $\eldiag{\Si}{K}$ we have that it is
    \linkto{finite_realisation_and_embeddings}{finitely realised in $K$.}

    Let $f,g \in I_p$, then $\eqzero_f,\eqzero_g \in p$.
    Suppose for a contradiction $f+g \notin I_p$, then by maximality of $p$,
    $\NOT \eqzero_{f+g} \in p$.
    Taking the finite subset of $p$ to be 
    $\set{\eqzero_f,\phi_g,\NOT \eqzero_{f+g}}$,
    by the above fact we obtain $a \in K^n$ such that 
    \[K \model{\Si_\RNG(K)} 
    \set{\eqzero_f,\eqzero_g,\NOT \eqzero_{f+g}}(a)\]
    By definition of $\eqzero$ this implies
    \[f(a) = 0,g(a) = 0,(f+g)(a) \ne 0\]
    which is a contradiction.
    Similarly we can let $f \in I_p$, $\la \in K$ 
    and suppose $\la f \notin I_p$ and so $\NOT \eqzero_{\la f} \in p$.
    We take the finite subset $\set{f, \la f} \subs p$ and get a contradiction.

    Let the product of two polynomials $fg$ be in $I_p$. 
    Suppose for a contradiction $f,g \notin I_p$.
    Then take the finite subset 
    $\set{\NOT \eqzero_f, \NOT \eqzero_g, \eqzero_{f+g}} \subs p$ and obtain
    an $a \in K^n$ such that $f(a) \ne 0, g(a) \ne 0$ but $f(a)g(a) = 0$,
    a contradiction.

    To show that the map is continuous let 
    $V(E) \subs \spec(K[x_1,\dots,x_n])$ be closed,
    where $E$ is a subset of $K[x_1,\dots,x_n]$ 
    ($V$ denotes the spec vanishing - 
    \linkto{prime_spec_zariski_top}{see appendix}).
    Then $p \in I^{-1}(U)$ if and only if $I_p \in U$
    if and only if $E \subs I_p$ if and only if 
    \[\forall f \in E, \eqzero_f \in p\]
    Hence the preimage is closed:
    \[  
        I^{-1}(U) = \set{p \in S_n(T) \st \forall f \in E, \eqzero_f \in p}
        = \bigcap_{f \in E} [\eqzero_f]
    \]
    as the basis elements $[\phi]$ are 
    \linkto{properties_of_stone_space}{clopen}.
    Thus $I$ is continuous.

    To show that it is injective suppose $I_p = I_q$ and let $\phi \in p$.
    By \linkto{ACF_has_quantifier_elimination}{quantifier-elimination 
        in $\ACF$} 
    we have $\phi$ is equivalent to quantifier free $\psi$ modulo $T$.
    Then $\psi \in p$ as $p$ is consistent with $T$
    and $\phi \in q$ if and only if $\psi \in q$ as $q$ is consistent with $T$.
    Take the \linkto{disjunctive_normal_form}{disjunctive normal form} of $\psi$
    \[ 
        \bigor{i \in I}{} 
        \brkt{\bigand{j \in J_{i0}}{} f_{ij}(v) = 0 \AND 
        \bigand{j \in J_{i1}}{} g_{ij}(v) \ne 0}
    \]
    Then (by maximality of $p$) 
    there exists some $i$ such that for all $j$,
    $f_{ij} = 0$ and $g_{ij} \ne 0$
    are in $p$.
    Thus $f_{ij} = 0$ and $g_{ij} \ne 0$ are in $q$ as they can be made into 
    polynomials in $I_p = I_q$.
    Hence by maximality of $q$ we have $\psi \in q$ and so $\phi \in q$.
    By symmetry $p = q$.

    For surjectivity, 
    let $\f{p}$ be a prime ideal of the polynomial ring.
    Consider the set
    \[  
        \eqzero(\f{p}) \cup 
        \set{\NOT \phi \st \phi \in \eqzero(K[x_1,\dots,x_n] \setminus \f{p})}
    \]
    Assuming for now that this set is 
    consistent with the elementary diagram, 
    \linkto{extend_to_maximal_type_zorn}{it can be extended 
    to $p$ a maximal $n$-type} of elementary diagram.
    Then 
    \begin{align*}
        &f \in \f{p} \implies \eqzero_f \in p \implies f \in I_p\\
        &f \notin \f{p} \implies (\NOT \eqzero_f) \in p \implies f \notin I_p
    \end{align*} 
    and so $I_p = \f{p}$ and we have found a preimage $p$.

    To show that the set is consistent 
    it suffices to show that it is 
    \linkto{finite_realisation_and_embeddings}{finitely realised in $K$}.
    Let $\De \subs \f{p}$ and $\Ga \subs K[x_1,\dots,x_n] \setminus \f{p}$
    be finite subsets.
    It suffices to show that 
    \[\exists a \in K^n, \forall f \in \De, \forall g \in \Ga, 
    f(a) = 0 \AND g(a) \ne 0\]
    which is equivalent to 
    \[\V_K(\De) \cap K^n \setminus \V_k(\Ga) \ne \nothing\]
    Suppose for a contradiction it is empty then $\V_k(\De) \subs \V_k(\Ga)$
    and taking ideals gives us 
    \[\Ga \subs r(\Ga) = I(\V_K(\Ga)) \subs I(\V_K(\De)) = r(\De) \subs \f{p}\]
    Here we used that 
    \linkto{taking_ideals_order_reversing}{taking ideals is order reversing},
    \linkto{strong_nullstellensatz}{strong Nullstellensatz}, 
    and the fact that the radical
    is a subset of any prime ideal containing its generators.
    Thus $\Ga \subs \f{p}$ which is a contradiction.
    Hence we have surjectivity.

    Suppose for a contradiction that the two spaces were homeomorphic.
    Then as the Stone space is Hausdorff, 
    $\spec(K[x_1,\dots,x_n])$ would also be Hausdorff,
    which is a \linkto{spec_not_hausdorff}{contradiction}.
\end{proof}