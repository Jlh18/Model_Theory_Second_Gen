\section{Morley Rank in Algebraically Closed Fields and Dimension}

We work towards the result that Krull dimension for algebraically closed 
fields is the same thing as Morley rank for varieties. 
The idea is that Morley rank of a variety corresponds to the Morley rank of 
types from elements of the variety, corresponds to model theoretic dimension,
corresponds to transcendence degree, 
which is the same thing as Krull dimension.
Strong minimality will play an important role in defining 
model theoretic algebraic closure,
which is then used to define dimension.

\subsection{Dimension}
In this section we will set up the important definitions needed to 
talk about dimension. 
A prerequisite for this section is knowledge of the basic results for 
\linkto{pregeometry_dfn}{pregeometries}, covered in the appendix.
\begin{dfn}[Algebraic, algebraic closure]
    Let $\MM$ be a $\Si$-structure and let $D$ be a subset of $\MM$.
    Let $A$ be a subset of $D$, 
    $a \in \MM$ is algebraic over $A$ if $a$ belongs to a finite 
    $\Si(A)$-definable set .
    Define the algebraic closure of $A$ over $D$ to be
    \[\acl_{\Si,D}(A) := \set{a \in D \st a \text{ is algebraic over } A}\]
    We drop the subscripts $\Si$ and $D$ when it is sufficiently obvious.
\end{dfn}

\begin{dfn}[Minimal, strongly minimal \cite{marker}]
    \link{strongly_minimal}
    Let $\MM$ be a $\Si$-structure.
    Let $D$ be an infinite $\Si(\MM)$-definable subset of $\MM^n$.
    $D$ is minimal in $\MM$ if any $\Si(\MM)$-definable subset of $D$
    if finite or cofinite.
    $D$ is strongly minimal if it is minimal in 
    $\NN$ for any elementary extension $\NN$ of $\MM$.
    A $\Si$-theory $T$ is strongly minimal if any $\Si$-model of $T$
    is strongly minimal 
    (note that any $\Si$-structure is definable by the formula $v = v$).
\end{dfn}

We have an equivalent definition of strong minimality, 
which if you like you can skip.
\begin{prop}[Strong minimality in terms of Morley rank and degree]
    Let $X$ be a definable subset of $\MM$, a $\Si$-structure.
    Then $X$ is strongly minimal if and only if $\MR{}{X} = \MD{X} = 1$.
\end{prop}
\begin{proof}
    \begin{forward}
        Suppose $X$ is strongly minimal.
        Then $X$ is infinite 
        \linkto{basic_facts_morley_rank_of_dfnbl_set}{hence} $1 \leq \MR{}{X}$.
        Let $\M$ be an 
        \linkto{om_sat_elem_ext_of_models}{$\om$-saturated extension of $\MM$}.
        If $2 \leq \MR{}{X}$ then there would be infinitely many 
        disjoint $\Si(\M)$ definable subsets of $X$ of Morley rank $1$
        (hence they are infinite).
        By strong minimality the only 
        $\Si(\M)$-definable subsets of $X$ are finite or cofinite.
        Thus these subsets must all be cofinite
        and so any two will intersect, a contradiction.
        Thus $1 = \MR{}{X}$.

        Since $\MR{}{X} = 1 \in \ord$ we have that $\MD{X} \in \N_{>0}$.
        Again if we have two disjoint 
        $\Si(\M)$-definable subsets of $X$ of Morley rank $1$ 
        we have a contradiction, hence $\MD{X} \leq 1$ and so $\MD{X} = 1$.
    \end{forward}

    \begin{backward}
        Suppose $X$ has Morley rank and degree $1$.
        Let $\NN$ be an elementary extension of $\MM$. 
        Let $A \subs X$ be a $\Si(\NN)$-definable subset of $X$;
        its complement is also $\Si(\NN)$-definable.
        Suppose both are infinite then they both have Morley rank greater 
        than or equal to $1$ and are disjoint,
        thus $2 \leq \MD{}{X} = 1$, which is a contradiction.
        Hence one is finite and the other cofinite.
    \end{backward}
\end{proof}

\begin{lem}[Some definable sets]
    \link{some_definable_sets}
    Let $\MM$ be a $\Si$-structure and let $B,C \subs \MM$ be $\Si$-definable
    set (i.e. $\Si(\nothing)$-definable).
    Let $\phi(x)$ be a $\Si$-formula with $n$ free variables.
    For $b \in B^n$, let $\psi(x,b)$ be a $\Si(B)$-formula 
    with $m$ free variables ($\psi(x,y)$
    is a $\Si$-formula with $n+m$ free variables).

    Then the following sets are definable by a $\Si$-formula:
    \begin{itemize}
        \item The intersection of $B$ and $C$, the union of $B$ and $C$ and
            the complement of $B$.
        \item The set of $b \in B^n$ that satisfy $\phi(x)$:
            \[\set{b \in B^n \st \MM \model{\Si} \phi(b)}\]
        \item The elements $b \in \MM^n$ such that $\psi(x,b)$ defines a set of 
            at most $k$ elements.
        \item The elements $b \in \MM^n$ such that $\psi(x,b)$ defines a set of 
            at least $k$ elements.
        \item The elements $b \in \MM^n$ such that $\psi(x,b)$ defines a set of 
            cardinality $k$.
            \[\set{b \in \MM^n \st \abs{\psi(\MM,b)} = k}\]
            and even $\set{b \in B^n \st \abs{\psi(\MM,b)} = k}$ by 
            taking the intersection of two definable sets.
    \end{itemize}
    We will become lazier when dealing with definable sets as we gain an idea 
    of what should and should not be definable.
\end{lem}
\begin{proof}
    \begin{itemize}
        \item This is clear.
        \item 
        Since $B$ is $\Si$-definable we can take $\chi(x)$ as the 
        $\Si$-formula defining $B$
        and consider the $\Si$-formula 
        \[\phi(x_1,\dots,x_n) \AND \bigand{i = 1}{n} \chi(x_i)\]
        Clearly this defines $\set{b \in B^n \st \MM \model{\Si} \phi(b)}$.
        \item 
        To make $\set{b \in \MM^n \st \abs{\psi(\MM,b)} \leq k}$
        we take the $\Si$-formula $\chi(x)$:
        \[
            \chi(x) = \bigforall{i = 1}{k + 1} x_i,
            \bigand{i = 1}{k + 1} \psi(x_i,y) \to \bigor{i \ne j}{x_i = x_j}
        \]
        where potentially $x_i$ represents $m$ variables, which we can 
        quantify over as it is finite.
        \item 
        To make $\set{b \in \MM^n \st k \leq \abs{\psi(\MM,b)}}$
        we take the $\Si$-formula $\chi(x)$:
        \[
            \chi(x) = \bigexists{i = 1}{k} x_i, {x_i \ne x_j}
        \]
    \end{itemize}
\end{proof}

\begin{prop}[Algebraic closure is a \linkto{pregeometry_dfn}{pregeometry}]
    \link{acl_d_is_pregeometry}
    Let $\MM$ be a $\Si$-structure. 
    Let $D$ be a minimal subset of $\MM$.
    Then $(D,\acl_{\Si,D})$ is a pregeometry.
\end{prop}
\begin{proof}\cite{fandom0}
    The signature we work in will always be $\Si$ and the 
    strongly minimal subset will always be $D$ so we drop the subscript here.
    Preserves order: 
    \emph{if $A \subs B \subs D$ then $\acl(A) \subs \acl(B)$.}
    Let $a \in \acl(A)$. 
    Then there exists a finite $\Si(A)$-definable set containing $a$.
    Any $\Si(A)$-formula is naturally a $\Si(B)$-formula thus $a \in \acl(B)$.

    Idempotence: \emph{for any $A \subs D$, $\acl(A) = \acl(\acl(A))$.}
    \begin{forward}
        We first show that for any subset $A \subs D$, $A \subs \acl(A)$.
        Let $a \in A$ then $a = x$ is a $\Si(A)$-formula that is defines a 
        finite set. 
        Thus $a \in \acl(A)$.
        Directly we have the corollary $\acl(A) \subs \acl(\acl(A))$.
    \end{forward}

    \begin{backward}
        We show that $\acl(\acl(A)) \subs \acl(A)$.
        Let $a \in \acl(\acl(A))$.
        Then there exists $\phi(x,v) = \phi(x,v_0,\dots,v_n)$ a $\Si$-formula 
        and $b_0,\dots, b_n \in \acl(A)$ such that $\phi(x,b)$
        defines a finite subset of $\MM$ containing $a$.
        Let $k$ be the finite cardinality of $\phi(\MM,b)$.
        \linkto{some_definable_sets}{There exists a $\Si$-formula $\psi(v)$} 
        that defines the set 
        $\set{b \in B \st \abs{\phi(\MM,b)} \leq n}$
        \[
            \phi'(x,v) := \phi(x,v) \AND \psi(v)
        \]
        We have that $a \in \phi(\MM,b) = \phi'(\MM,b)$
        and for any $c \in \MM^n$, $\phi'(\MM,c)$ is finite.
    
        For each $b_i$ appearing in $b$
        there exists a $\Si(A)$-formula $\psi_i(v_i)$ such that 
        $b_i \in \psi_i(\MM)$ and this definable set is finite.
        Define the $\Si(A)$-formula
        \[
            \phi''(x) := \bigexists{i = 1}{n} v_i,
            \phi'(x,v_0,\dots,v_n) \AND \bigand{i = 1}{n} \psi_i(v_i)
        \]
        Then $a \in \phi''(\MM)$ by taking the $v_i$ to be $b_i$ and 
        \begin{align*}
            d \in \phi''(\MM) 
            &\implies \exists c \in \MM^n, \MM \model{\Si}\phi'(d,c) 
            \text{ and for each $i$,} \MM \model{\Si}\psi_i(c_i)\\
            &\implies \text{there exist for each $i$ } c_i \in \psi_i(\MM),
            \MM \model{\Si}\phi'(d,c) \\
            &\implies d \in 
            \bigcup_{i = 0}^n \bigcup_{c_i \in \psi_i(\MM)} \phi'(\MM,c)
        \end{align*}
        The last expression is a finite union of finite sets which is finite.
        Hence $\phi''(\MM)$ is finite and $a \in \acl(A)$
    \end{backward}
    
    Finite character: \emph{if $A \subs D$ and $a \in \acl(A)$ then 
    there exists a finite subset $F \subs A$ such that $a \in \acl(F)$.}
    Take the $\Si(A)$-formula defining the finite set containing $a$.
    Pick out the (finitely many) constant symbols from $A$, 
    forming a finite subset $F \subs A$.
    Then $a \in \acl(F)$.

    Exchange: \emph{if $A \subs D$ and $a,b \in D$ such that 
    $a \in \acl(A,b)$ (shorthand for $A,\set{a}$)
    then $a \in \acl(A)$ or $b \in \acl(A,a)$.}
    Since $a \in \acl(A,b)$ there exists a $\Si(A)$-formula $\phi(v,w)$ such 
    that $a \in \phi(\MM,b)$ and $\phi(\MM,b)$ is finite - 
    say it has cardinality $n$ 
    (if $b$ does not appear in the formula then we immediately have 
    $a \in \acl(A)$).
    \linkto{some_definable_sets}{There exists a $\Si(A)$-formula} $\psi(w)$
    defining the set
    \[\psi(\MM) = \set{b' \in D \st n = \abs{\phi(\MM,b')}}\]
    As $\psi(\MM) \subs D$ and $D$ is minimal, 
    $\psi(\MM)$ is finite or cofinite.
    If it is finite then $b \in \psi(\MM)$ and so 
    $b \in \acl(A) \linkto{acl_d_is_pregeometry}{\subs} \acl(A,a)$.

    If it is $\psi(\MM)$ then consider the $\Si(A)$-formula 
    $\phi(v,w) \AND \psi(w)$.
    For each $a' \in D$ let $X(a')$ be the subset of $D$ defined by 
    $\phi(a',w) \AND \psi(w)$.
    Consider $b \in X(a)$, and case on whether it is finite or cofinite.
    If it is finite then $b \in \acl(A,a)$ as $\phi(a,w) \AND \psi(w)$
    is a $\Si(A)$-formula defining a finite set.

    If $X(a)$ is cofinite then let $m = \abs{D \setminus X(a)} \in \N$.
    \linkto{some_definable_sets}{There exists a $\Si(A)$-formula} $\chi(v)$
    defining the set
    \[\chi(\MM) = \set{a' \in D \st m = \abs{D \setminus X(a')}}\]
    If $\chi(\MM)$ is finite then $a \in \chi(\MM)$ and so $a \in \acl(A)$.
    If $\chi(\MM)$ is confinite then there exist $n + 1$ distinct elements 
    $a_i \in \chi(\MM)$ since $D$ is infinite by definition.
    Take the (finite) intersection of the cofinite $X(a_i)$,
    producing a non-empty (infinite) set.
    Take 
    \[b' \in \bigcap_{i = 1}^{n+1} X(a_i) = 
    \bigcap_{i = 1}^{n+1} \phi(a_i,\MM) \cap \psi(\MM)\] 
    Then for each $i$, $\MM \model{\Si} \phi(a_i,b')$, 
    hence $n + 1 \leq \abs{\phi(\MM,b')}$.
    However $\MM \model{\Si} \psi(b')$ implies $n = \abs{\phi(\MM,b')}$,
    a contradiction.
\end{proof}

The definition of \linkto{dimension_dfn}{dimension} 
    for pregeometries thus carries through for 
    subsets of $D$.
\begin{dfn}
    Let $\MM$ be a $\Si$-structure and let $X \subs D \subs \M$,
    where $D$ is minimal.
    We write $\dim_{\Si,D}(X)$ to mean the 
    \linkto{dimension_dfn}{dimension} 
    of $X$ in the pregeometry $(D,\acl_{\Si,D})$.
    We call this the $\Si$-dimension of $X$ in $D$.
\end{dfn}

\begin{lem}[$\acl$ preserves dimension]
    \link{acl_preserves_dimension}
    Let $\MM$ be a $\Si$-structure and let $X \subs D \subs \M$,
    where $D$ is minimal. 
    Then $\dim_{\Si,D}(X) = \dim_{\Si,D}(\acl_{\Si,D}(X))$.
\end{lem}
\begin{proof}
    Let $S \subs X$ be a basis of $X$. 
    Then it is an independent subset of $X \subs \acl(X)$ such that 
    \[\acl(X) \subs \acl(S) 
    \quad \text{ and by \linkto{acl_d_is_pregeometry}{idempotence} } \quad
    \acl(\acl(X)) \subs \acl(X) \subs \acl(S)\]
    Hence $S$ is a basis for $\acl(X)$.
\end{proof}

\subsection{The theory of algebrically closed fields is strongly minimal}
\begin{dfn}[Strongly minimal theory]
    A $\Si$-theory $T$ is (strongly) minimal if any $\Si$-model 
    of $T$ is (strongly) minimal.
\end{dfn}

\begin{lem}[Disjunctive normal form of definable sets]
    \link{dnf_for_definable_sets}
    Let $K$ be an algebraically closed field.
    Any definable set in $K^n$ can be written in the form 
    \[\bigcup_{i \in S}\brkt{V_i \cap U_i}\]
    where $V_i$ is a variety and $U_i$ is the complement of a variety in $K^n$.
\end{lem}
\begin{proof}
    Let $X$ be a definable set:
    \[X = \set{a \in K^m \st K \model{\Si_\RNG} \phi(a,b)}\]
    where $b \in K^n$ and 
    $\phi$ is some $\Si_\RNG$-formula with $n+m$ free variables.
    Then by \linkto{ACF_has_quantifier_elimination}{quantifier elimination in 
    $\ACF$}
    we have a quantifier free $\Si_\RNG$-formula $\psi$ such that 
    \[X = \set{a \in K^m \st K \model{\Si_\RNG} \psi(a,b)}\]
    We can find the `disjunctive normal form' of $\psi$ as 
    \linkto{disjunctive_normal_form}{it is quantifier free}.
    Hence for $a \in K^m$
    \begin{align*}
        &a \in X\\
        &\iff \bigor{i \in I}{} 
        \brkt{\bigand{j \in J_{i0}}{} p_{ij}(a,b) = 0 \AND 
        \bigand{j \in J_{i1}}{} q_{ij}(a,b) \ne 0}\\
        &\iff a \in \bigcup_{i \in I}
        \brkt{\bigcap_{j \in J_{i0}} \V_K(p_{ij}(x,b)) \cap 
        \bigcap_{j \in J_{i1}} K^m \setminus \V_K(q_{ij}(x,b))}\\
    \end{align*}
\end{proof}

\begin{lem}[Non-trivial vanishings are finite]
    \link{vanishing_finite_or_cofinite}
    If $K$ is a field and $S \subs K[x]$
    then $\V_K(S)$ is finite or $S = \set{0}$.
    In particular $\V_K(S)$ is either finite or cofinite in $K$.
\end{lem}
\begin{proof}
    If $\V(f)$ is finite we are done.
    If $\V(f)$ is infinite
    then each $f \in S$ has infinitely many distinct roots
    so $f = 0$ by the division algorithm.
    
    In particular, if $S = \set{0}$ then $\V(S)$ is $K$ and it is cofinite.
\end{proof}

\begin{prop}
    \link{ACF_strong_min}
    $\ACF$ is strongly minimal.
\end{prop}
\begin{proof}
    Let $K$ be an algebraically closed field.
    
    Let $D \subs K$ be definable.
    Any elementary extension of $K$ is also algebraically closed so 
    without loss of generality we only need to show minimality rather than 
    strong minimality.
    Then \linkto{dnf_for_definable_sets}{there exist 
    $p_{ij}(x,b),q_{ij}(x,b) \in K[x]$} 
    (note that the polynomials are in only one variable) such that 
        \[D = \bigcup_{i \in I}
        \brkt{\bigcap_{j \in J_{i0}} \V_K(p_{ij}(x,b)) \cap 
        \bigcap_{j \in J_{i1}} K \setminus \V_K(q_{ij}(x,b))}\]
    Which is a finite union and intersection of 
    \linkto{vanishing_finite_or_cofinite}{finite and cofinite sets},
    which is finite or cofinite.
    Hence $D$ is finite or cofinite and $\ACF$ is strongly minimal.
\end{proof}

\subsection{Dimension and transcendence degree}
We show that dimension and transcendence degree are the same thing.

\begin{lem}[Formulas defining finite sets and polynomials]
    \link{formulas_defining_finite_sets_give_poly}
    Suppose $K$ is an algebraically closed field, 
    $S$ is a subset of an extension field.
    $\phi$ is a $\Si_\RNG(K,S)$-formula with exactly $1$ free variable.
    If $\phi$ defines a finite set containing $a \in K(S)$ 
    then there exists a non-zero polynomial $p \in K(S)[x]$ such that 
    $p(a) = 0$.
\end{lem}
\begin{proof}
    First take the \linkto{disjunctive_normal_form}{
        disjunctive normal form of $\phi$}
    \[K(S) \model{\Si(K,S)} \forall v, \phi \IFF \bigor{i \in I}{} 
        \brkt{\bigand{j \in J_{i0}}{} p_{ij}(v) = 0 \AND 
        \bigand{j \in J_{i1}}{} q_{ij}(v) \ne 0}\]
    where $\Si(K,S)$-terms $p_{ij},q_{ij}$ are naturally 
    \linkto{terms_in_RNG_are_polynomials}{polynomials} in $K(S)[x]$.
    We see that there is some $i \in I$ such that 
    \[K(S) \model{\Si_\RNG(K,S)} 
        \bigand{j \in J_{i0}}{} p_{ij}(a) = 0 \AND 
        \bigand{j \in J_{i1}}{} q_{ij}(a) \ne 0
    \]
    The set defined by $\bigand{j \in J_{i0}}{} p_{ij}(v) = 0$ in $K(S)$
    is $\V_{K(S)}(\set{p_{ij} \st j \in J_{i0}})$,
    hence is \linkto{vanishing_finite_or_cofinite}{either finite
    (there exists a non-zero polynomial) 
    or all of $K(S)$ (all polynomials are zero)}.
    In the first case we obtain a non-zero polynomial $p \in K(S)[x]$ 
    such that $p(a) = 0$ and we are done.

    Assume for a contradiction the second case holds.
    For each $j \in J_{i1}$ 
    consider the set defined by $q_{ij}(v) \ne 0$ in $K(S)$,
    which is \linkto{vanishing_finite_or_cofinite}{cofinite or empty}.
    If it is empty then it is not satisfied by $a$ which is a contradiction.
    Hence 
    \[
        \bigand{j \in J_{i0}}{} p_{ij}(v) = 0 \AND 
        \bigand{j \in J_{i1}}{} q_{ij}(v) \ne 0
    \]
    defines a cofinite subset of $\phi(K(S))$, and so $\phi(K(S))$ is 
    \linkto{card_of_alg_closed_fields}{infinite}, a contradiction.
\end{proof}

A result that uses the previous lemma is that the usual algebraic closure
for fields is indeed a special case of our definition of algebraic closure.
This is included only because it is noteworthy and can be skipped.
\begin{prop}[Algebraic closure in $\ACF$ is an field theoretic 
    algebraic closure]
    Let $K \to M$ be a field extension with $M$ algebraically closed, 
    then as \linkto{ACF_strong_min}{$\ACF$ is strongly minimal} 
    we have that $M$ is strongly minimal.
    Consider a subset $A \subs M$.
    Then $\acl_{\Si(K),M}(A)$ is an algebraic closure of $K(A)$.
\end{prop}
\begin{proof}
    We write $\acl$ instead of $\acl_{\Si(K),M}$.
    We must show that $\acl(A)$ is a field that contains $K(A)$, 
    is algebraically closed and is an algebraic extension of 
    $K(A)$.

    To show that it is a field we only show closure for addition and leave
    the rest as an exercise. 
    Let $a, b \in \acl(A)$.
    Then there exist $\Si(K,A)$-formulas $\phi_a,\phi_b$ that define 
    finite subsets of $M$ containing $a$ and $b$ respectively.
    Then their sum is in the finite set defined by the formula with one free
    variable $z$:
    \[
        \exists x, \exists y, \phi_a(x) \AND \phi_b(y) \AND z = x + y
    \]

    It contains $K(A)$ if and only if it contains $A$ and the image of $K$.
    It contians $K$ since for any element $k \in K$ we can take the 
    $\Si(K,A)$-formula $x = k$. 
    Similarly for $A$.

    To show that it is algebraically closed let 
    $p \in \acl(A)[x]$ be a polynomial.
    Then we write $p$ out in terms of its coefficients $a_i \in \acl(A)$
    \[p = \sum_{i = 1}^m a_i x^i\]
    Each coefficient is in a finite subset of $M$
    defined by a $\Si(K,A)$-formula $\phi_i$.
    The formula with free variable $x$
    \[
        \bigexists{i = 1}{m} v_i, \bigand{i = 1}{m} \phi_i(v_i) \AND 
        \sum_{i = 1}^m v_i x^i
    \]
    defines the set of roots of $p$ in $M$.
    Since $M$ is algebraically closed this is all the roots and since
    $p$ only has finitely many roots it is finite.
    Hence all the roots of $p$ are in $\acl(A)$.

    To show that it is an algebraic extension of $K(A)$ we take $a \in \acl(A)$
    and obtain a formula defining a finite subset of $M$ containing $a$.
    Thus 
    \linkto{formulas_defining_finite_sets_give_poly}{there exists a polynomial}
    $p \in K(A)[x]$ with $a$ as a root, and so $a$ is algebraic over $K(A)$.
\end{proof}

\begin{prop}[Transcendence degree is dimension]
    \link{trans_deg_is_dim}
    Let $K \to L$ be a field extension.
    Suppose $S \subs L$ and 
    $M$ is an algebraically closed field extension of $L$.
    We consider the pregeometry $(M,\acl_{\Si_\RNG(K),M})$.
    \begin{enumerate}
        \item $S$ is algebraically independent over $K$ if and only if 
            $S$ is independent in the pregeometry.
        \item $K(S) \to L$ is an algebraic extension if and only if $S$ 
            spans $L$ in the pregeometry.
        \item $S$ is a transcendence basis for the extension $K \to L$ 
            if and only if 
            $S$ is a basis for $L$ in the pregeometry.
        \item Transcendence degree is the same thing as dimension:
            \[\tdeg(K \to K(S)) = \dim_{\Si(K),M}(L)\]
    \end{enumerate}
\end{prop}
\begin{proof}~
    We will write $\acl$ to mean $\acl_{\Si_\RNG(K),M}$.
    \begin{enumerate}
        \item \begin{forward}
            Let $a \in S$ and suppose for a contradiction that 
            $a \in \acl(S \setminus \set{a})$.
            Then there exists a $\Si_\RNG(K,S \setminus \set{a})$-formula $\phi$
            defining a finite subset of $\M$ containing $a$.
            Then \linkto{formulas_defining_finite_sets_give_poly}{
                there exists a non-zero polynomial} 
            $p \in K(S \setminus \set{a})[x]$ such that 
            $p(a) = 0$.
            This contradicts algebraic independence of $S$.
        \end{forward}

        \begin{backward}
            Suppose $p \in K[x_0,\dots,x_n]$ with distinct $s_0,\dots,s_n \in S$
            such that $p(s_0,\dots,s_n) = 0$.
            Then we have a polynomial in one variable
            \[p(x_0,s_1,\dots,s_n) \in K(S \setminus \set{s_0})[x_0]\]
            with $s_0$ as a root.
            Elements of $K(S \setminus \set{s_0})$ can be written as 
            polynomials over $K$ in variables from $S \setminus \set{s_0}$,
            which are $\Si_\RNG(K,S \setminus \set{s_0})$-terms;
            hence $p(x_0,s_1,\dots,s_n)$ is naturally a 
            $\Si_\RNG(K,S \setminus \set{s_0})$-term
            as well.
            Call $\phi(x_0)$ the $\Si_\RNG(K,S \setminus \set{s_0})$-formula 
            `$p(x_0,s_1,\dots,s_n) = 0$'.
            Since 
            \linkto{vanishing_finite_or_cofinite}{
                non-zero polynomials have finitely many roots} $\phi(M)$
            is finite or $p = 0$; since $s_0$ is a root, 
            \[M \model{\Si_\RNG (K,S \setminus \set{s_0})} \phi(s_0)\]
            and so if it is finite then $s_0 \in \acl(S \setminus \set{s_0})$,
            contradicting pregeometrical independence.
            Hence $p = 0$ and so $S$ is algebraically independent.
        \end{backward}
        \item \begin{forward}
            By \linkto{acl_d_is_pregeometry}{idempotence} of $\acl$, 
            for $S$ to be spanning in the pregeometry 
            it suffices to show that $L \subs \acl(S)$ 
            (take $\acl$ of both sides).
            Let $a \in L$.
            Since $K(S) \to L$ is an algebraic extension 
            there exists a non-zero polynomial 
            $p \in K(S)[x]$ such that $p(a) = 0$.
            We write $p$ as a polynomial over $K$ in variables from $S$,
            which is a $\Si_\RNG(K,S)$-term, 
            and so the $\Si_\RNG(K,S)$-formula `$p = 0$'
            defines a finite set containing $a$.
        \end{forward}
        
        \begin{backward}
            Let $a \in L$.
            We want to show that $a$ is algebraic over $K(S)$
            Since $S$ spans $L$
            \[a \in L \subs \acl(L) \subs \acl(S)\]
            and so there exists a $\Si(K,S)$-formula defining a finite set 
            containing $a$.
            \linkto{formulas_defining_finite_sets_give_poly}{Hence there exists
            a non-zero polynomial} $p \in K(S)[x]$ with $a$ as a root.
            Hence the extension is algebraic.
        \end{backward}

        \item
            $S$ is a transcendence basis of the extension $K \to L$ 
            \linkto{transcendence_bases_algebraic_extensions}{if and only if}
            $K(S) \to L$ is algebraic and $S$ is 
            algebraically independent over $K$.
            By the last two parts this is if and only if $S$ is independent 
            and spanning $L$ in the pregeometry, 
            which is if and only if $S$ is pregeometrical basis for $L$.
        
        \item This is clear.
    \end{enumerate}
\end{proof}

\subsection{Morley rank of types and dimension}
This section is purely model theoretic.
It allows us to connect our definition dimension together with Morley rank 
for types of tuples.

\begin{prop}[Projection and Morley rank]
    \link{proj_and_morley_rank}
    Suppose $\M$ is an $\om$-saturated $\Si$-structure.
    Let $\phi$ be a $\Si(\M)$-formula and let $v$ be a variable symbol.
    Then 
    \[\MR{}{\exists v, \phi} \leq \MR{}{\phi}\]
\end{prop}
\begin{proof}
    First we remove the case where $v$ is not a free variable of $\phi$ 
    by noting that if this is the case then $\exists v, \phi$ and $\phi$ 
    have the same number of free variables and so $a \in (\exists v, \phi)(\M)$
    if and only if $\M \modelsi \exists v, \phi(a)$ if and only if 
    $\M \modelsi \phi(a)$ if and only if $a \in \phi(\M)$.
    \linkto{implication_subset_inequality}{Hence} 
    they have the same Morley rank.

    Then we induct on $\al \in \ord$ to show that 
    $\al \leq \MR{}{\exists v, \phi}$ implies $\al \leq \MR{}{\phi}$.
    We will not bother with the cases $- \infty$ and $\infty$ and the former
    is trivial and the latter follows from the induction result.
    If $0 \leq \MR{}{\exists v, \phi}$ then there exists $a \in \M^n$
    such that $\M \modelsi \exists v, \phi(a,v)$. 
    Hence there is $b \in \M$ such that 
    $\M \modelsi \phi(a,b)$ and so $0 \leq \MR{}{\phi}$.

    The non-zero limit ordinal case is trivial. 
    Suppose $\al + 1 \leq \MR{}{\exists v, \phi}$.
    Then there exist $\Si(\M)$-formulas $\psi_i$ defining disjoint subsets of 
    $(\exists v, \phi) (\M)$ such that $\al \leq \MR{}{\psi_i}$.
    Then take $\Si(\M)$-formulas $\chi_i$ to be $\psi_i \AND \phi$.\footnote{
        It is important here that we notice $\psi_i \AND \phi$ and 
        $\phi$ have the same number of free variables, and so we can say that 
        $(\psi_i \AND \phi)(\M) \subs \phi(\M)$, which is not the case for 
        if we just took $\psi_i$, since it is one free variable short.
        (In general $\M^n \cap \M^{n+1} = \nothing$!)
        }
    If $a \in (\exists v, \phi)(\M)$ then 
    $a \in (\exists v, \psi \AND \phi)(\M)$ and so 
    \[
        \psi_i(\M) \subs (\exists v, \phi)(\M) 
        \subs (\exists v, \psi \AND \phi)(\M) \quad \implies \quad
        \al \leq \MR{}{\psi_i} \linkto{implication_subset_inequality}{\leq} 
        \MR{}{\exists v, \chi_i}
    \]
    by the induction hypothesis we have $\al \leq \chi_i$ and the
    $\chi_i$ define disjoint subsets of $\phi(\M)$, 
    hence $\al+1 \leq \MR{}{\phi}$.
\end{proof}

\begin{lem}[Morley rank of extended types \cite{marker}]
    \link{morley_rank_of_extended_types}
    Suppose $\M$ is a $\ka$-saturated, strongly minimal $\Si$-structure.
    Let $A \subs \M$ be such that $\abs{A} < \ka$.
    Let $a \in \M^n$ and $b \in \M$.
    Then 
    \[\MR{}{\subintp{A}{\M}{\tp}(a)} \leq \MR{}{\subintp{A}{\M}{\tp}(a,b)}\]
    Furthermore `if $b$ is dependent then removing it preserves Morley rank'
    \[b \in \acl_{\Si(A),\M}(\set{a_1,\dots,a_n}) \implies 
    \MR{}{\subintp{A}{\M}{\tp}(a)} = \MR{}{\subintp{A}{\M}{\tp}(a,b)}\]
    This says we can reduce finite sets to subsets that are independent 
    in the pregeometry whilst preserving Morley rank of the type.
\end{lem}
\begin{proof}
    We write $\MR{}{\star}$ to mean $\MR{}{\subintp{A}{\M}{\tp}(\star)}$.
    We use $x$ or $x_1,\dots,x_n$ to denote the variables corresponding to $a$
    and use $v$ to denote the variable corresponding to $b$.

    Let $\phi$ be a \linkto{morley_rank_for_types_dfn}{rank representative}
    for $\tp(a,b)$.
    Then $a \in (\exists v, \phi)(\M)$ and so $(\exists v, \phi) \in \tp(a)$.
    Hence
    \[
        \MR{}{a} \leq \MR{}{\exists v, \phi} \linkto{proj_and_morley_rank}{\leq} 
        \MR{}{\phi} \leq \MR{}{a,b}
    \]

    For the other inequality
    it suffices to show by induction on $\al \in \ord$ that 
    If $a \in \M^n$ and $b \in \M$.
    Then `if $b$ is dependent' then 
    \[\al \leq \MR{}{a,b} \implies \al \leq \MR{}{a}\]
    For the base case we note that $a \in \phi(\M)$ for any rank representative 
    $\phi$ of $\tp(a)$ and so $\phi(\M)$ is non-empty and 
    \[0 \leq \MR{}{\phi} = \MR{}{a}\]
    The non-zero limit ordinal case is trivial.

    Suppose for the successor case $\al + 1 \leq \MR{}{a,b}$. 
    Then $\al \leq \MR{}{a,b}$ and so by the induction 
    hypothesis $\al \leq \MR{}{a}$.
    \linkto{smallest_rank_rep}{There exists a `smallest' rank representative} 
    $\phi(x)$ of $\tp(a)$ such that 
    there is no $\Si(A)$-formula $\psi$ such that 
    \[\MR{}{\phi \AND \psi} = \MR{}{\phi \AND \NOT \psi} = \al\]
    Since $b \in \acl(a)$ we have a $\Si(A)$-formula $\psi(x,v)$ such that 
    $b \in \psi(a,\M)$ and $\psi(a,\M)$ is finite with cardinality $n$.
    Let $\abs{\psi(x,\M)} = n$ denote 
    \linkto{some_definable_sets}{the formula} defining the set of $a' \in \M^n$
    such that $\abs{\psi(a',\M)} = n$.
    Then define the $\Si(A)$-formula
    \[\Phi(x,v) := \phi(x) \AND \psi(x,v) \AND \abs{\psi(x,\M)} = n\]

    Note that $\Phi \in \tp(a,b)$ and so 
    $\al + 1 \leq \MR{}{a,b} \leq \MR{}{\Phi}$.
    Thus for $i \in \N$ there exist formulas $\theta_i$ with rank at least
    $\al$ defining disjoint subsets of $\Phi(\M)$.

    We show that for each $m \in \N_{>0}$ 
    \[\al \leq \MR{}{\bigand{i = 1}{m} \exists y, \theta_i}\]
    We first show for each $i$ that $\al \leq \MR{}{\exists v, \theta_i}$,
    covering the base case $m = 1$.
    \linkto{formulas_rep_by_types}{There exist representative types},
    element $c \in \M^n$ and $d \in \M$: $\theta_i \in \tp(c,d)$
    and $\MR{}{\theta_i} = \MR{}{c,d}$.
    If $d \in \acl(c)$ then by induction we have 
    \[\al \leq \MR{}{c,d} 
    \implies \al \leq \MR{}{c} \leq \MR{}{\exists v, \theta_i}\]
    Indeed $d \in \acl(c)$ since 
    $\theta_i(\M) \subs \Phi(\M)$ which says 
    \[d \in \psi(c, \M) \text{ and } \abs{\psi(c,\M)} = n\]

    Suppose it is true for $m$.
    Write $\chi(x) := \bigand{i = 1}{m} \exists v, \theta_i$.
    Then for each $i$, $(\exists v, \theta_i)(\M) \subs \phi(\M)$ implies 
    $\chi(\M) \subs \phi(\M)$ and 
    \[
        \al \leq \MR{}{\chi} \leq \MR{}{\phi} = \al 
        \implies \al = \MR{}{\chi} = \MR{}{\phi \AND \chi}
    \]
    We showed above that $\al \leq \exists y, \theta_m$.
    Partition $(\exists y, \theta_m)(\M)$ into its intersection with 
    $\chi(\M)$ and $\NOT \chi(\M)$.
    Supposing for a contradiction 
    \[\al \nleq \MR{}{\chi \AND \exists v, \theta_m}\]
    we see that $\al \leq \MR{}{\exists y, \theta_m}$ is the 
    \linkto{basic_facts_morley_rank_of_dfnbl_set}{maximum} of the two parts
    which must be $\MR{}{\NOT \chi \AND \exists v, \theta_m}$.
    Then 
    \[
        \al \leq \MR{}{\exists v, \theta_m \AND \NOT \chi} 
        \leq \MR{}{\phi \AND \NOT \chi} \leq \MR{}{\phi} = \al 
        \implies \al = \MR{}{\phi \AND \NOT \chi}
    \]
    This contradicts the property of $\phi$ being the 
    `smallest' rank representative.

    In particular we have shown, applying 
    \linkto{compactness_for_types}{compactness},
    that $\set{\exists v, \theta_i}$ is consistent and thus can be 
    \linkto{extend_to_maximal_type_zorn}{extended} to 
    an element of $S_n(\Theory_\M(A))$, which is realised by some $c \in \MM^n$ 
    as it is $\ka$-saturated.
    Thus we have for each $i$ that $\M \model{\Si(A)} \exists v, \theta_i(c,v)$,
    giving us $d_i \in \M$ such that $\M \model{\Si(A)} \theta_i(c,d_i)$.
    However the $\theta_i(c,\M)$ are disjoint subsets of 
    $\Phi(c,\M) \subs \psi(c,\M)$ and so the $d_i$ must be
    distinct elements of $\psi(c,\M)$ and so $\psi(c,\M)$ is infinite.
    However, $(c,d_1) \in \Phi(\M)$ so $\abs{\psi(c,\M)} = n$, a contradiction.
\end{proof}

\begin{lem}[Independent tuples have the same type \cite{marker}]
    \link{independent_tuples_same_type}
    Let $\MM$ be a $\Si$-structure with $A \subs D \subs \MM$,
    where $D$ is minimal and $\Si(A)$-definable.
    Let $a \in D^k$ and $b \in D^k$. 
    If $\set{a_1,\dots,a_k}$ and $\set{b_1,\dots,b_k}$ are both 
    independent in the pregeometry $(D,\acl_{\Si(A),D})$
    then 
    \[\subintp{A}{\MM}{\tp}(a) = \subintp{A}{\MM}{\tp}(b)\]
\end{lem}
\begin{proof}
    We induct on $k \in \N$. 
    We write $\tp$ for $\subintp{A}{\MM}{\tp}$.

    For the base case $k = 0$ we note that empty tuples define the same type
    \[{\tp}(a) = {\tp}(\nothing)= {\tp}(b)\]
    More concretely this is equal to $\Theory_{\MM}(A)$.

    Suppose it is true for $k$.
    Let $\set{a_1,\dots,a_{k+1}}$ and $\set{b_1,\dots,b_{k+1}}$ be independent
    in the pregeometry.
    Then by induction $\tp(a_1,\dots,a_k) = \tp(b_1,\dots,b_k)$.
    Let $\psi \in \tp(a_1,\dots,a_{k+1})$.
    It suffices to show that $\psi \in \tp(b_1,\dots,b_{k+1})$.
    Note that $a_k \in \psi(a_1,\dots,a_k, \MM) \cap D$, 
    which is a \linkto{some_definable_sets}{
        $\Si(A,a_1,\dots,a_k)$-definable subset} of $D$.
    Since by assumption $\set{a_1,\dots,a_{k+1}}$ is independent 
    this definable set must be infinite, 
    and by minimality of $D$ its complement 
    $\NOT \psi(a_1,\dots,a_k, \MM) \cap D$ is finite
    with cardinality $n$, say.
    
    \linkto{some_definable_sets}{There exists a $\Si(A)$-formula} $\chi$
    in free-variables $x_1,\dots,x_k$ defining the set of $c \in D^k$ such that
    \[\abs{\NOT \psi(c,\MM) \cap D} = n\]
    Since $\chi \in \tp(a_1,\dots,a_k) = \tp(b_1,\dots,b_k)$
    implies $\abs{\NOT \psi(b_1,\dots,b_k,\MM) \cap D} = n$,
    \[
        b_{k+1} \notin \psi(b_1,\dots,b_k,\MM) \implies 
        b_{k+1} \in \NOT \psi(b_1,\dots,b_k,\MM) \text{ finite} \implies 
        b_{k+1} \text{ is dependent, contradiction}
    \]
    We have $b_{k+1} \in \psi(b_1,\dots,b_k,\MM)$ and so 
    $\psi \in \tp(b_1,\dots,b_k)$.
\end{proof}

\begin{lem}[Monsterous structures have infinite dimension]
    \link{dim_of_monster_is_infinite}
    Let $\ka$ be a cardinal.
    Let $\M$ be a strongly minimal and $\ka$-saturated $\Si$-structure
    such that $\ka \leq \M$.
    Let $A \subs \M$ such that $\abs{A} < \ka$.
    Then $\dim_{\Si(A),\M}(\M)$ is infinite.
\end{lem}
\begin{proof}
    We work in the pregeometry $(\M,\acl_{\Si(A),\M})$ and write 
    $\dim$ for $\dim_{\Si(A),\M}$ and $\acl$ for $\acl_{\Si(A),\M}$.
    
    Suppose for a contradiction $B$ is a finite basis for $\M$ 
    in the pregeometry.
    We construct an injection to show that 
    \[\abs{\acl(B)} < \ka\]
    which is a contradiction as $\acl(B) = \M$.

    Let $b \in \acl(B)$. 
    Then there exists a $\Si(A,B)$ formula $\phi$ defining a finite subset of 
    $\M$ containing $b$.
    Then there exists $n \in \N$ and a bijection $\io_\phi : \phi(\M) \to n$.
    We define using choice
    \[
        f : \acl(B) \to \form{\Si(A,B)} \times \N, 
        \quad b \mapsto (\phi,\io_\phi(b))
    \]
    Then $f$ is injective and so 
    \[
        \abs{\acl(B)} \leq 
        \abs{\form{\Si(A,B)}} + \aleph_0 \leq 
        \aleph_0 + \abs{A} + \abs{B} = \aleph_0 + \abs{A}
        < \aleph_0 + \ka = \ka
    \]
    Hence we have our contradiction.
\end{proof}

\begin{prop}[Morley rank of types and dimension]
    \link{morley_rank_of_types_is_dim}
    Let $\ka$ be a cardinal.
    Let $\M$ be a strongly minimal and $\ka$-saturated $\Si$-structure
    such that $\ka \leq \M$.
    Let $A \subs \M$ such that $\abs{A} < \ka$.
    Let $a \in \M^k$. Then 
    \[\MR{}{\subintp{A}{\M}{\tp}(a)} = \dim_{\Si(A),\M}(\{a_1,\dots,a_k\})\]
\end{prop}
\begin{proof}
    We work in the pregeometry $(\M,\acl_{\Si(A),\M})$ and write 
    $\dim$ for $\dim_{\Si(A),\M}$ and $\acl$ for $\acl_{\Si(A),\M}$.
    We also write $\MR{}{a}$ for $\MR{}{\subintp{A}{\M}{\tp}(a)}$ and
    $\tp$ for $\subintp{A}{\M}{\tp}$.

    Let us first show that without loss of generality 
    $\{a_1,\dots,a_k\}$ is independent in the pregeometry.
    Concretely, we prove by induction on $k$ that there exists an independent
    subset $s \subs \set{a_1,\dots,a_k}$ such that 
    \[\MR{}{s} = \MR{}{a} \quad \text{ and } \quad \dim(s) = \dim(a)\]
    If $k = 0$ then it is okay as the empty set is trivially independent.
    If $0 < k$ then we case on if $\set{a_1,\dots,a_k}$ is independent
    or not.
    If it is then we are done. 
    Otherwise remove a dependent element $a_i$.
    By the theorem on \linkto{morley_rank_of_extended_types}{Morley 
        rank of extended types} we have that 
    \[\MR{}{a \setminus \set{a_i}} = \MR{}{a}\]
    Since \linkto{acl_preserves_dimension}{$\acl$ preserves dimension}
    and $a_i$ is dependent
    \[
        \dim(a \setminus \set{a_i}) = \dim(\acl(a \setminus \set{a_i}))
        = \dim(\acl(a)) = \dim(a)
    \]
    Hence by induction there is a subset $s \subs a \setminus \set{a_i} \subs a$
    such that 
    \[\MR{}{s} = \MR{}{a \setminus \set{a_i}} = \MR{}{a}
    \text{ and } \dim(s) = \dim(a \setminus \set{a_i}) = \dim(a)\]
    Hence $a$ can be replaced by an independent subset.

    Now we show that for independent $a_1,\dots,a_k$, $\MR{}{a} = k$.
    This will complete the proof since independent sets 
    are bases for themselves, which implies $\dim(a) = k = \MR{}{a}$.
    We prove this by induction on $k$.
    If $k = 0$ then $\subintp{A}{\M}{\tp}(a)$ 
    consists of $\Si(A)$-sentences satisfied by $\M$, 
    which \linkto{definable_set}{by convension} each define the set 
    $\set{\nothing}$. 
    Hence everything in $\tp(a)$ has 
    \linkto{basic_facts_morley_rank_of_dfnbl_set}{Morley rank $0$} 
    and so $\MR{}{a} = 0$.

    For the induction step (formally we should take out the $k=1$ case %? Lazy
    too but this is also straight forward) suppose 
    $a_1,\dots,a_{k+1}$ form an independent set in the 
    pregeometry.
    We first show that $k + 1 \leq \MR{}{a}$.
    Let $\phi(x_1,\dots,x_{k+1})$ be a rank representative of $a$.
    Since the \linkto{dim_of_monster_is_infinite}{dimension of $\M$ is infinite}
    we can take a countably infinite 
    subset of a basis $\set{b_i}_{i \in \N}$. 
    Let the formulas 
    \[
        \psi_i := \phi \AND (x_1 = b_i)
    \]
    define disjoint subsets of $\phi(\M)$.
    We need to show that these formulas have Morley rank at least $k$.
    Since \linkto{independent_tuples_same_type}{independent sets of the same 
        length have the same type}, for each $i \in \N$ 
    \[\tp(a_1,\dots,a_{k+1}) = \tp(b_i,\dots,b_{i+k})\]
    In particular $\phi \in \tp(b_i,\dots,b_{i+k})$ and so 
    $\psi_i \in \tp(b_i,\dots,b_{i+k})$.
    Hence, by induction and using the 
    \linkto{morley_rank_of_extended_types}{lemma on extended types}
    \[
        k = \MR{}{b_i,\dots,b_{i+k-1}} 
        {\leq} \MR{}{b_i,\dots,b_{i+k}}
        \leq \MR{}{\psi_i}
    \]
    Hence we have $k + 1 \leq \MR{}{\phi} = \MR{}{a}$.

    Suppose for a contradiction that $k + 2 \leq \MR{}{\phi}$.
    Then we have for each $i \in \N$, $\Si(\M)$-formulas $\chi_i$
    of Morley rank $k + 1$ defining disjoint subset of $\phi(\M)$.
    As they are disjoint there exists an $i$ such that $a \notin \chi_i(\M)$.
    We take $\tp(c) \in S_n(\Th_\M(A))$, 
    a \linkto{formulas_rep_by_types}{type representing this $\chi_i$}
    realised by some $c \in \M^{k+1}$ by $\ka$-saturation:
    \[\MR{}{c} = \MR{}{\chi_i} = k + 1 \text{ and } \chi_i \in \tp(c)\]
    Suppose for a contradiction that $c_1,\dots,c_{k+1}$ are independent in the 
    pregeometry.
    Then as \linkto{independent_tuples_same_type}{independent sets of the same 
        length have the same type} we have $\tp(a) = \tp(c)$,
    which implies $\chi_i \in \tp(a)$ so $a \in \chi_i(\M)$, a contradiction.
    There is a \linkto{independence_and_span_basic}{strictly smaller subset}
    $B \subs \set{c_1,\dots,c{k+1}}$ 
    that is a pregeometrical basis for $\set{c_1,\dots,c{k+1}}$.
    This will satisfy $\MR{}{B} = \MR{}{c}$ (by induction using the lemma on 
    \linkto{morley_rank_of_extended_types}{extended types}).
    By the induction hypothesis, noting that $B$ is independent, we have
    \[\MR{}{\chi_i} = \MR{}{c} = \MR{}{B} = \abs{B} < k + 1\]
    which contradicts $\MR{}{\chi_i} = k + 1$.
    Hence $k + 1 = \MR{}{\phi} = \MR{}{a}$.
\end{proof}

\subsection{Morley rank is Krull dimension}
We are now ready to show that Morley rank is the same thing as Krull dimension.
A prerequisite is the \linkto{transendence_deg_is_krull}{result} 
in algebraic geometry that transcendence degree 
is the same thing as Krull dimension.

\begin{rmk}[Defining formula for Zariski closed sets]
    \link{defining_formula_for_zariski_closed}
    Let $K$ be an alebraically closed field and $V \subs K^n$ 
    a Zariski closed set. 
    Then it is \linkto{zariski_closed_sets_are_fin_gen}{a 
        finitely generated vanishing} 
    so there exists a finite $S_V \subs K[x_1,\dots,x_n]$ such that 
    $V$ is defined by the $\Si(K)$-formula 
    \[\bigand{p \in S_V}{} p = 0\]
    where each polynomial is 
    \linkto{terms_in_RNG_are_polynomials}{naturally a $\Si(K)$-term}.
\end{rmk}

\begin{prop}[Morley rank is Krull dimension for algebraically closed fields
    \cite{marker}]
    Let $K$ be an algebraically closed field and $V \subs K^n$ a variety.
    Then the Morley rank of $V$ is equal to its 
    \linkto{dfn_krull_dimension}{Krull dimension}.
\end{prop}
\begin{proof}
    We write $\tp$ for $\subintp{K}{\M}{\tp}$.
    We show by induction that for each $n \in \N$
    if $\kdim(V) = n$
    then $\MR{}{V} = n$.
    Since \linkto{dfn_krull_dimension}{
        Krull dimension is always finite} this is sufficient.
    If $n = 0$ then \linkto{krull_dim_0}{$V$ is a singleton} 
    hence \linkto{basic_facts_morley_rank_of_dfnbl_set}{$\MR{}{V} = 0$}.

    Suppose $n > 0$ and $\kdim(V) = n$.
    Let $\ka$ be a cardinal such that $\abs{K} < \ka$.
    Then \linkto{upwards_lowenheim_skolem}{
        there exists an elementary extension of $K$
        with cardinality $\ka$};
    and then take $\M$ a 
    \linkto{existence_of_monster}{$\ka$-saturated elementary extension} of 
    that.
    We replace $K$ and $V$ with their images in $\M$ to make things simpler.
    The variety $V$ is Zariski closed by definition and so it is 
    \linkto{defining_formula_for_zariski_closed}{defined by a $\Si(K)$-formula}
    \[\phi := \bigand{p \in S_V}{} p = 0\]
    for some finite $S_V \subs K[x_1,\dots,x_n]$.
    
    Note that since 
    \linkto{formulas_rep_by_types}{formulas are represented by types} 
    and $\ka$-saturation we have
    \[
        \MR{}{V} = \MR{}{\phi} = 
        \max \set{\MR{}{q} \st \phi \in q \in S_n(\Theory_\M(K))}
        = \max \set{\MR{}{\tp(a)} \st a \in \phi(\M)}
    \]
    Thus there exists $a \in \phi(\M)$ such that $\MR{}{\tp(a)} = \MR{}{V}$.
    It suffices to show that $\MR{}{\tp(a)} = n$.

    Let $I(a)$ denote the ideal in $K[x_1,\dots,x_n]$ 
    of polyomials vanishing at $a$ 
    (extending our previous definition of $I$ beyond $K^n$).
    Then $V_a := \V_K(I(a))$ is Zariski closed and so it is 
    \linkto{defining_formula_for_zariski_closed}{defined by a $\Si(K)$-formula}
    \[\psi := \bigand{p \in S_{V_a}}{} p = 0\]
    for some finite $S_{V_a} \subs K[x_1,\dots,x_n]$.
    We first show that $V_a = V$.
    To show that $V_a \subs V$, we take $b \in V_a$ and note that 
    for each $p \in S$, $p(b) = 0$ by definition and so  
    $b \in \phi(K) = V$.
    Suppose for a contradiction $V_a \subset V$.
    Then the $\kdim(V_a) < \kdim(V) = n$.
    Hence by the induction hypothesis $\MR{}{V_a} < n$ 
    and since $a \in \psi(\M)$
    \begin{align*}
        \MR{}{V} 
        &= \MR{}{\tp(a)} \\
        &\leq \max\set{\MR{}{\tp(b) \st b \in \psi(\M)}}\\
        &= \max \set{\MR{}{q} \st \psi \in q \in S_n(\Theory_\M(K))}\\
        &\linkto{formulas_rep_by_types}{=} \MR{}{\psi} \\
        &= \MR{}{V_a} < n
    \end{align*}
    Hence by the induction hypothesis $V$ 
    has Krull dimension strictly less than $n$, a contradiction.
    Thus $V_a = V$.

    Now we have that $K(a)$ is isomorphic to the function field $K(V)$,
    and \linkto{iso_field_ext_same_trans_deg}{
        hence have the same transcendence degree}:
    \[
        K(a_1,\dots,a_n) \iso 
        K[x_1,\dots,x_n] / I(V_a) = 
        K[x_1,\dots,x_n] / I(V) = K(V)
    \]
    As \linkto{transendence_deg_is_krull}{Krull dimension is 
        transcendence degree, } 
    $n = \kdim(V) = \tdeg(K \to K(V)) = \tdeg(K \to K(a))$. 
    This in turn is the same as 
    $\dim_{\Si_\RNG(K),\M}(\set{a_1,\dots,a_n})$
    since \linkto{trans_deg_is_dim}{transcendence degree is dimension}.
    \linkto{ACF_strong_min}{$\ACF$ is strongly minimal}
    and $\ka$-saturated satisfying $\ka \leq \M$,
    thus \linkto{morley_rank_of_types_is_dim}{dimension 
        corresponds to Morley rank of the type}
    \[
        k = \dim_{\Si_\RNG(K),\M}(\set{a_1,\dots,a_n}) 
        = \MR{}{\tp(a)} = \MR{}{V}
    \]
\end{proof}