\section{Boolean Algebras, Ultrafilters and the Stone Space}
\subsection{Boolean Algebras}
There is a very detailed wikipedia page \cite{wiki0} on Boolean algebras.

\begin{dfn}[Partially ordered set]
    \link{partial_ordering}
    The signature of partially ordred sets $\Si_\PO$ consists of 
    $(\nothing,\nothing ,n_f,\set{\leq},m_f)$,
    where $n_\leq = 2$.
    The theory of partially ordered sets $\PO$ consists of 
    \begin{itemize}
        \item[$\vert$] Reflexivity: 
            $\forall x, 
            x \leq x$ (this is just notation for $\leq(x,x)$)
        \item[$\vert$] Antisymmetry: 
            $\forall x \forall y, 
            (x \leq y \AND y \leq x) \to x = y$
        \item[$\vert$] Transitivity:
            $\forall x \forall y \forall z, 
            (x \leq y \AND y \leq z) \to x \leq z$
    \end{itemize}
\end{dfn}

\begin{dfn}[Boolean algebra]
    \link{dfn_boolean_algebra}
    The signature of Boolean algebras $\Si_\BLN$ consists of 
    $(\set{1,0},\set{\leq,\cap,\cup,\NEG},n_f,\nothing,m_f)$,
    where $n_\leq = 2$, $n_\cap = n_\cup = 2$ and $n_\NEG = 1$.
    The theory of Boolean algebras $\BLN$ consists of the theory of 
    partially ordered sets\footnote{Often 
        the ordering is defined afterwards as 
        $a \leq b$ if and only if $a = a \cap b$.} 
    $\PO$ together with the formulas
    \begin{itemize}
        \item[$\vert$] Assosiativity of adjunction: 
            $\forall x \forall y \forall z, 
            (x \cap y) \cap z = x \cap (y \cap z)$
        \item[$\vert$] Assosiativity of disjunction: 
            $\forall x \forall y \forall z, 
            (x \cup y) \cup z = x \cup (y \cup z)$
        \item[$\vert$] Identity for adjunction:
            $\forall x, x \cap 1 = x$ 
        \item[$\vert$] Identity for disjunction:
            $\forall x, x \cup 0 = x$ 
        \item[$\vert$] Commutativity of adjunction: 
            $\forall x \forall y, x \cap y = y \cap x$
        \item[$\vert$] Commutativity of disjunction: 
            $\forall x \forall y, x \cup y = y \cup x$
        \item[$\vert$] Distributivity of adjunction:
            $\forall x \forall y \forall z, 
            x \cap (y \cup z) = (x \cap y) \cup (x \cap z)$
        \item[$\vert$] Distributivity of disjunction:
            $\forall x \forall y \forall z, 
            x \cup (y \cap z) = (x \cup y) \cap x (\cup z)$
        \item[$\vert$] Negation on adjunction: 
            $\forall x, x \cap \NEG x = 0$ 
        \item[$\vert$] Negation on disjunction: 
            $\forall x, x \cup \NEG x = 1$ 
        \item[$\vert$] Order on adjunction:
            $\forall x \forall y, (x \cap y) \leq x$
        \item[$\vert$] Maximal property of adjunction:
            $\forall x \forall y \forall z, 
            (x \leq y) \AND (x \leq z) \to (x \leq y \cap z)$ 
        \item[$\vert$] Order on disjunction:
            $\forall x \forall y, x \leq (x \cup y)$ 
        \item[$\vert$] Minimal property of disjunction:
            $\forall x \forall y \forall z, 
            (x \leq z) \AND (y \leq z) \to (x \cup y \leq z)$ 
        \end{itemize}
    Often `absorption' is also included, 
    but it can be deduced from the other axioms.
    I have not used the usual logical symbols due to obvious clashes with 
    our notation, 
    and we will be using this in the context of sets anyway.
    If $B$ is a $\Si_\BLN$-model of $\BLN$ we call it a Boolean algebra.
\end{dfn}

\begin{lem}[Facts about Boolean algebras]
    \link{bare_boolean_facts}
    Let $B$ be a Boolean algebra, let $\FF$ be an ultrafilter
    on $B$, let $a,b \in B$ and let $f : B \to C$ be a morphism.
    \begin{itemize}
        \item $a \cap a = a$ and $a \cup a = a$.
        \item $a \cup 1 = 1$ and $a \cap 0 = 0$.
        \item If $a \cap b = 0$ and $a \cup b = 1$ then $a = \NEG b$
            (negations are unique.)
        \item $\NEG (a \cup b) = (\NEG a) \cap (\NEG b)$ and its dual. 
            (De Morgan)
        \item ($a \in \FF$ or $b \in \FF$) if and only if $a \cup b \in \FF$.
        \item Morphisms are order preserving.
        \item Morphisms commute with negation.
        \item $a \cap b = 1$ if and only if $a = 1$ and $b = 1$.
            Similarly for $\cup$ with $0$.
        \item If $a \cap b = 0$ then $b \leq \NEG a$.
    \end{itemize}
\end{lem}
\begin{proof}~
    \begin{itemize}
        \item We prove only $a \cap a = a$ as the other has the same proof.
            \[a \cap a = (a \cap a) \cup 0 = (a \cap a) \cup (a \cap \NEG a)
            = a \cap (a \cup \NEG a) = a \cap 1 = a\]
        \item Again, we prove only $a \cup 1 = 1$. Using the previous part,
            \[a \cup 1 = a \cup (a \cup \NEG a) = (a \cup a) \cup \NEG a
            = a \cup \NEG a = 1\]
        \item \begin{align*}
            a &= a \cup 0 = a \cup (b \cap \NEG b) = 
            (a \cup b) \cap (a \cup \NEG b) = 1 \cap (a \cup \NEG b)\\
            &= 0 \cup (a \cup \NEG b) = (b \cap \NEG b) \cup (a \cup \NEG b)
            = (b \cup a) \cup \NEG b = 0 \cup \NEG b \\
            &= \NEG b
        \end{align*} 
        \item By the previous part it suffices to show that 
            \[\brkt{(\NEG a) \cap (\NEG b)} \cap (a \cup b) = 0 \text{ and }
            \brkt{(\NEG a) \cup (\NEG b) } \cup (a \cup b) = 1\]
            These are clear.
        \item \begin{forward}
            In the case that $a \in \FF$ we have $a \leq a \cup b \in \FF$ 
            as $\FF$ is under superset. 
            The other case is the same.
        \end{forward}
        \begin{backward}
            Suppose $a \notin \FF$ and $b \notin \FF$ then
            \linkto{negation_is_in_ultrafilter}{$\NEG a \in \FF$ and 
                $\NEG b \in \FF$} hence $\NEG a \cap \NEG b \in \FF$
                by closure under intersection. 
                By the previous part this is equal to $\NEG (a \cup b)$,
                which implies $(a \cup b) \notin \FF$ as $\FF$ 
                is an \linkto{negation_is_in_ultrafilter}{ultrafilter}.
        \end{backward}
        \item Suppose $b \leq a$. 
            We show that $f(a) \leq f(b)$.
            By the minimal property of disjunction,
            \[b \leq a \AND a \leq a \implies b \cup a \leq a\]
            Clearly $a \leq b \cup a$ and so $a = b \cup a$
            Hence $f(b) \leq f(b) \cup f(a) = f(b \cup a) = f(a)$.
        \item Suppose $f(a) = b$.
            Then $f(a) \AND f(\NEG a) = f(a \AND \NEG a) = f(0) = 0$
            and $f(a) \OR f(\NEG a) = f(a \OR \NEG a) = f(1) = 1$.
            As negations are unique (shown above) 
            this gives us that $f(\NEG a) = \NEG f(a)$.
        \item Suppose $a \cap b = 1$ then $1 \leq a \cap b \leq a$ hence $1 = a$
            and similarly $1 = b$.
        \item\begin{align*}
                &a \cup \NEG a = 1\\
                \implies & b \cap (a \cup \NEG a) = b \cap 1 = b\\
                \implies & (b \cap a) \cup (b \cap \NEG a) = b\\
                \implies & 0 \cup (b \cap \NEG a) = b\\
                \implies & b = b \cap \NEG a \leq \NEG a
            \end{align*}
    \end{itemize}
\end{proof}


\begin{dfn}[Filters and ultrafilters on Boolean algebras]
    \link{dfn_filter_on_bool_alg}
    Let $B$ be a Boolean algebra.
    A subset $\FF$ of $B$ is an filter on $B$ if
    \begin{itemize}
        \item $1 \in \FF$.
        \item For any two members $a,b \in \FF$ the adjunction
            $a \cap b$ is in $\FF$. (Closure under finite intersection.)
        \item If $a \in \FF$ then any $b \in B$ 
            such that $a \leq b$ is also a member of $\FF$. 
            (Closure under superset.)
    \end{itemize}
    We say a filter on $B$ is proper if it does not contain $0$.
    A proper filter $\FF$ on $B$ is an ultrafilter (maximal filter) when
    for any filter $\GG$ on $B$, if $\FF \subs \GG$ then $\GG = \FF$ 
    or $\GG = B$.
\end{dfn}

\begin{prop}[Equivalent definition of ultrafilter]
    \link{negation_is_in_ultrafilter}
    Let $B$ be a boolean algebra. 
    Let $\FF$ be a proper filter on $B$.
    The following are equivalent:
    \begin{enumerate}
        \item $\FF$ is an ultrafilter over $B$.
        \item If $a \cup b \in \FF$ then $a \in \FF$ or $b \in \FF$.
            `$\FF$ is prime'.
        \item For any $a$ of $B$, 
            $a \in \FF$ or $(\NEG a) \in \FF$.
    \end{enumerate}
    For $2.$ the or is in fact `exclusive or' since if both $a$ 
    and its negation were in $\FF$ then $\FF$ would not be proper.
\end{prop}
\begin{proof}
    $(1. \implies 2.)$
        Suppose $a_0 \cup a_1 \in \FF$.
        \[\GG_{a_0} = \set{b \in B \st \exists c \in \FF, c \cap a_0 \leq b}\]
        Then clearly $\GG_{a_0}$ is a filter containing $\FF$ and $a_0$.
        Similarly have $\GG_{a_1}$.
        Since $\FF$ is an ultrafilter, 
        we can case on whether $\GG_{a_i}$ is $\FF$ or $B$.
        If either is $\FF$ then $a_i \in \FF$ and we are done.
        If $\GG_{a_0} = \GG_{a_1} = B$ then both contain $0$ and so 
        there exist $c_i \in \FF$ such that 
        $c_i \cap a_i = 0$.
        \linkto{bare_boolean_facts}{Hence $c_i \leq \NEG a_i$,}
        and so $(\NEG a_i) \in \FF$.
        Thus $\NEG (a_0 \cup a_1) = \NEG a_0 \cap \NEG a_1 \in \FF$,
        and so $0 \in \FF$,
        a contradiction.

    $(2. \implies 3.)$ 
        Let $a \in \FF$.
        Then we have that $a \cup \NEG a = 1$, 
        which is by definition a member of $\FF$.
        By assumption this implies that $a \in \FF$ or $\NEG a \in \FF$.
    
    $(3. \implies 1.)$
        Let $\GG$ be a proper filter such that $\FF \subs \GG$.
        It suffices to show that $\GG = \FF$.
        Then let $a \in \GG$.
        Suppose $a \notin \FF$.
        Then $\NEG a \in \FF$ and so $\NEG a \in \GG$.
        Thus $0 = (a \cap \NEG a) \in \GG$ 
        thus $\GG$ is not proper, a contradiction.
        Hence $\GG = \FF$.
\end{proof}

\begin{prop}[Extending filters to ultrafilters]
    \link{zorn_ultrafilter}
\end{prop}


\begin{dfn}
    The category of Boolean algebras consists of Boolean algebras as the objects
    and for any two Boolean algebras $B, C$ morphisms $f:B \to C$ such that
    for any $a,b \in B$
    \[f(0) = 0, f(1) = 1, f(a \cup b) = f(a) \cup f(b), 
    f(a \cap b) = f(a) \cap f(b)\] 
\end{dfn}

\begin{dfn}
    The category of Stone spaces has
    $0$-dimensional, 
    compact and Hausdorff topological spaces as objects and 
    continuous maps as morphisms.
\end{dfn}

\begin{prop}[Contravariant functor from Boolean algebras to Stone spaces
        \cite{tent}]
    \link{bool_alg_to_stone_functor}
    The map
    $S(\star)$ that sends a Boolean algebra $B$ to the Stone space
    \[S(B) := \set{\FF \subs B \st \FF \text{ is an ultrafilter}}\]
    and a Boolean algebra morphism
    $f: A \to B$ to a continuous map of Stone spaces:
    \[S(f) := f^-1(\star) : S(B) \to S(A)\]
    is a contravariant functor from Boolean algebras to the category of Stone
    spaces.
\end{prop}
\begin{proof}
    We give the topology on $S(B)$ the topology generated by the clopen sets:
    Let $b \in B$, 
    then an element of the clopen basis is given by
    \[[b] := \set{\FF \subs B \st b \in \FF}\]
    Thus $S(B)$ is $0$-dimensional by definition.
    It is Hausdorff because if $\FF, \GG \in S(B)$ are not equal then 
    without loss of generality there exists $b \in \FF$ such that 
    $b \notin \GG$. 
    As ultrafilters are proper, $[b] \cap [\NEG b] = \nothing$ and so
    we have obtained disjoint open sets such that 
    $\FF \in [b]$ and $\GG \in [\NEG b]$.
    
    Next we show that it is compact. 
    To do this, 
    we will look at $S(B)$ as a subspace of the power set of $B$
    which is isomorphic to
    $2^B$, 
    the set of functions from $B$ to $2$.
    We endow $2 = \set{0,1}$ with the discrete topology and note that it is 
    compact.
    Then $2^B = \prod_{a \in B} 2$ has an induced product topology which is 
    compact by Tychonoff's Theorem.
    The isomorphism from $2^B$ to the power set of $B$ is given by 
    \[2^B \to \PP(B) := f \mapsto f^{-1}(1)\]
    We take the topology on $\PP(B)$ induced by this isomorphism.
    We must show that 
    \begin{itemize}
        \item $S(B)$ is the image of 
            $\mor{B}{2}{} := 
            \set{f \in 2^B \st f \text{ is a Boolean algebra morphism}}$
            under this isomorphism.
        \item $\mor{B}{2}{}$ is a closed subset of $2^B$ and therefore
            compact.
        \item The induced topology on $S(B)$ 
            is the same as its original topology.
    \end{itemize}
    With these facts we see that $S(B)$ is also compact.

    Let $f : B \to 2$ be a Boolean algebra morphism. 
    We must show that $\FF := f^{-1}(1)$ is an ultrafilter.
    Since $f(1) = 1$, $1 \in \FF$. 
    If $a,b \in \FF$ then $f(a) = f(b) = 1$ so 
    $f(a \cap b) = f(a) \cap f(b) = 1$ thus $\FF$ is closed under intersection.
    If $a \subs b$ and $a \in \FF$ then as $f$ is 
    \linkto{bare_boolean_facts}{order preserving} $f(a) \subs f(b)$.
    It is a proper filter as $f(0) = 0 \ne 1$.
    To show it is an ultrafilter we use the
    \linkto{negation_is_in_ultrafilter}{equivalent definition}:
    let $a \in \PP(B)$. 
    If $f(a) = 1$ then we are done,
    otherwise
    \[f(\neg a) = \neg f(a) = \neg 0 = 1 \implies \neg a \in \FF\]
    and we are done.
    To show that this is a surjection we use the inverse:
    \[\FF \mapsto \brkt{a \to \begin{cases}
        1, a \in \FF\\
        0, a \notin \FF
    \end{cases}}\] 
    To show that this is a morphism we note that $\FF$ 
    is proper and contains $1$ thus $f(0) = 0$ and $f(1) = 1$. 
    Also 
    \begin{align*}
        & f(a \cap b) = 1 \iff a \cap b \in \FF\\ 
        &\iff a \in \FF \text{ and } b \in \FF 
        \quad \text{ by closure under finite intersection and superset}\\
        &\iff f(a) = 1 \text{ and } f(b) = 1 \\
        &\iff f(a) \cap f(b) = 1 
        \quad \text{ \linkto{bare_boolean_facts}{ as proven before}}
    \end{align*}
    Hence $f(a \cap b) = f(a) \cap f(b)$.
    Lastly since 
    \[\NEG f(a) = 1 \iff f(a) = 0 \iff a \notin \FF \iff \NEG a \in \FF 
    \iff f(\NEG a) = 1\]
    thus by \linkto{bare_boolean_facts}{De Morgan} and 
    \linkto{bare_boolean_facts}{uniqueness of negations} we have
    \[f(a \cup b) = f(\NEG(\NEG a \cap \NEG b)) = 
    \NEG \sqbrkt{\NEG f(a) \cap \NEG f(b)} = 
    \brkt{\NEG \NEG f(a)} \cup \NEG \NEG f(b) = f(a) \cup f(b)\]
    Thus the inverse map gives back a Boolean algebra morphism and 
    $\mor{B}{2}{} \iso S(B)$ under the isomorphism.

    To show that $\mor{B}{2}{}$ is a closed subset we write it as 
    \[\mor{B}{2}{} = \set{f \st f(0) = 0} \cap \set{f \st f(1) = 1}
        \cap \set{f \st f \text{ commutes with $\cap$}} \cap 
        \set{f \st f \text{ commutes with $\cup$}}\]
    and show that all of these four sets are closed.
    Call them $\min,\max,C_{\cap},C_\cup$ respectively and for each $a \in B$
    call the projection map $\pi_a : 2^B \to 2$ (these send $f \mapsto f(a)$
    such that $\pi_a(f) = f(a)$)
    and note that by definition of the product topology each 
    $\pi_a$ is continuous; 
    the closed sets of the product are generated by preimages of closed sets.
    \[f(0) = 0 \iff \pi_0(f) = 0 \iff f \in \pi_0^{-1}(0)\]
    Hence $\min = \pi_0^{-1}(0)$ is closed as $\set{0}$ 
    is closed in the discrete topology on $2$.
    Similarly $\max = \pi_1^{-1}(1)$ is closed.
    \[C_\cap = \bigcap_{a,b \in B} \set{f \st f(a \cap b) = f(a) \cap f(b)}
    = \bigcap_{a,b \in B} \set{f \st \pi_{a \cap b}^{-1}(f(a) \cap f(b))}\]
    Thus $C_\cap$ is an arbitrary intersection of preimages of closed sets 
    since each ${f(a) \cap f(b)}$ is closed in the discrete toipology on $2$,
    hence $C_\cap$ is closed. 
    Similarly 
    \[C_\cup = \bigcap_{a,b \in B} 
    \set{f \st \pi_{a \cup b}^{-1}(f(a) \cup f(b))}\]
    is closed and so $\mor{B}{2}{}$ is closed.

    With regards to compactness it remains to show that the topologies 
    on $S(B)$ are the same.
    It suffices to show that any (closed) basis element of each can be written 
    as a closed set in the other.
    Let $[b]$ be an element of the basis for $S(B)$ under the Stone topology.
    Then this is the image of the closed subset 
    $\pi_b^{-1}(1) \subs \mor{B}{2}{}$ under the isomorphism:
    \[\mathrm{iso}(\pi_b^{-1}(1)) = \set{f^{-1}(1) \st f(b) = 1}
    = \set{\FF \text{ ultrafilter} \st b \in \FF}\]
    Conversely, 
    any element of the closed basis for $\mor{B}{2}{}$ is of the form 
    $\pi_b^{-1}(X)$ where $b \in B$ and $X \subs 2$.
    Hence 
    \[\mathrm{iso}(\pi_b^{-1}(X)) = \set{f^{-1}(1) \st f(b) \in X}\]
    We can case on if $X = \nothing, \set{0}, \set{1}, 2$ and deduce that 
    respectively $\mathrm{iso}(\pi_b^{-1}(X))$ 
    becomes $\nothing, [\NEG b], [b], S(B)$, 
    all of which are closed in the Stone topology.
    Thus the topologies are the same under this isomorphism and hence 
    $S(B)$ is compact.

    To show that $S(\star)$ 
    is a contravariant functor we need to check that the morphism map
    \[S(f) := f^-1(\star) : S(B) \to S(A)\]
    is a well-defined, respects the identity and composition. 
    We show that $S(f)$ is continuous: 
    it suffices that preimages of clopen elements are clopen.
    Let $[b] \subs S(A)$ be clopen. 
    \begin{align*}
        & S(f)^{-1}[b]\\
        =& \set{\FF \in S(B) \st f^{-1}(\FF) \in [b]}\\
        =& \set{\FF \in S(B) \st f(b) \in \FF}\\
        =& [f(b)]
    \end{align*}
    which is clopen.
\end{proof}

\begin{prop}[Stone Duality]%?Incomplete
    There is an equivalence between the category of Stone
    spaces and the category of Boolean algebras.
    Given by the functor $\BB_\star$ 
    sending any topological space $X$ to the set of 
    its clopen subsets (this is a basis of $X$ as it is $0$ dimensional):
    \[\BB_X := \set{a \subs X \st a \text{ is clopen}}\]
    and its inverse \linkto{bool_alg_to_stone_functor}{$S(\star)$} .
\end{prop}
\begin{proof}
    Let $X$ be a $0$-dimensional compact Hausdorff topological space.
    There is an obvious Boolean algebra to take on 
    \[\BB_X := \set{a \subs X \st a \text{ is clopen}}\]
    which is interpreting $0$ to be $\nothing$, $1$ to be $X$, 
    $\leq$ as $\subs$,
    adjunction as intersection, disjunction as union and negation to be 
    taking the complement in $X$.
    One can check that this is a Boolean algebra.

    We make this a contravariant functor by taking any continuous map 
    $f : X \to Y$
    to an induced map $f^{\diamond}:= f^{-1}(\star):\BB_Y \to \BB_X$.
    One can check that this is a well-defined functor.

    Lastly we prove the equivalence of categories by giving 
    natural transformations $S(\BB_{\star}) \to \id{\star}$ in
    the category of topological spaces
    and $\id{\star} \to \BB_{S(\star)}$ in the category of Stone spaces.
    %?incomplete
\end{proof}

\subsection{Isolated points of the Stone space}
\begin{dfn}[Isolated point]
    \link{dfn_isolated_point}
    Let $X$ be a topological space and $x \in X$. 
    We say $x$ is isolated if $\set{x}$ is open.
\end{dfn}

\begin{dfn}[Derived set]
    \link{dfn_derived_set}
    Let $X$ be a topological space. 
    The derived set of $X$ is defined as
    \[X' : = X \setminus \set{x \in X \st x \text{ isolated}}\]
\end{dfn}

\begin{ex}[Equivalent definition of derived set]
    Let $U$ be a subspace of $X$ a topological space.
    $x \in U$ is not isolated (in the subspace topology) if and only if 
    for any open set $O_x$ containing $x$, 
    $O_x \cap U \setminus \set{x}$ is non-empty. 
    (it is a limit point of $U$.)
\end{ex}

\begin{dfn}[Atom]
    Let $B$ be a Boolean algebra. 
    We say $a \in B$ is an atom if it is non-zero and for any $b \in B$
    if $b \leq a$ then $b = a$ or $b = 0$.
\end{dfn}

\begin{dfn}[Principle filter]
    Let $B$ be a Boolean algebra and $a \in B$.
    Suppose $a$ is non-zero.
    Then principle filter of $a$ is defined as 
    \[\upa{a} := \set{b \in B \st a \leq b}\]
    One should check that this is a proper filter.
\end{dfn}

\begin{prop}
    Let $a \in B$ a Boolean algebra.
    Then $\upa{a}$ is an ultrafilter if and only if $a$ is atomic. 
\end{prop}
\begin{proof}
    \begin{forward}
        Suppose $b \leq a$.
        As $\upa{a}$ is \linkto{negation_is_in_ultrafilter}{ultrafilter} 
        either $b \in \upa{a}$ or $\NEG b \in \upa{a}$.
        In the first case $a \leq b$ hence $a = b$.
        If $\NEG b \in \upa{a}$ then $b \leq a \leq \NEG b$ and so 
        \[1 = b \cup \NEG b \leq \NEG b \cup \NEG b = \NEG b\]
        and so $1 = \NEG b$ and $b = 0$.
    \end{forward}

    \begin{backward}
        Suppose $a$ is an atom. 
        We show that for any $b \in B$, 
        it is in $\upa{a}$ or its negation is in $\upa{a}$
        Since $1 \in \upa{a}$, 
        $b \cup \NEG b \in \upa{a}$ and so $a \leq b \cup \NEG b$.
        Thus
        \[a = a \cap (b \cup \NEG b) = (a \cap b) \cup (a \cap \NEG b)\]
        Since $a$ is non-zero, either $a \cap b$ or $a \cap \NEG b$ is non-zero.
        If $a \cap b \ne 0$ then together with the fact that $a \cap b \leq a$
        we conclude that $a \cap b = a$ as $a$ is atomic.
        Hence $a \leq b$ and $b \in \upa{a}$.
        Similarly the other case results in $\NEG b \in \upa{a}$.
    \end{backward}
\end{proof}

\begin{prop}[Correspondence between atoms and isolated points]
    Let $a \in B$ a Boolean algebra.
    If $a$ is an atom if and only if $[a]$ is a singleton in $S(B)$.
    Hence $\upa{a}$ is an isolated point in $S(B)$.
\end{prop}
\begin{proof}
    \begin{forward}
        If $a$ is an atom then $[a] = \set{\upa{a}}$ is the only ultrafilter 
        containing $a$ is the principle filter of $a$:
        Let $\FF$ be an ultrafilter containing $a$.
        Then for any $b \in \FF$, 
        $a \cap b \in \FF$ and non-zero.
        Therefore $a \cap b = a$ as $a$ is an atom and $a \leq b$.
        Hence $b \in \upa{a}$.
        By maximality $\FF = \upa{a}$.
        Hence $[a]$ is a singleton.
    \end{forward}

    \begin{backward}
        Suppose $a$ is not atomic. 
        Then there exists $b \leq a$ such that $b \ne 0$ and $b \ne a$.
        \begin{align*}
            & b \cup (a \cap \NEG b)
            = (a \cap b) \cup (a \cap \NEG b)\\
            = & a \cup (b \cap \NEG b) = a \cup 0 = a
        \end{align*}
        There exist \linkto{zorn_ultrafilter}{ultrafilters} 
        extending the principle filters of $b$ and 
        $a \cap \NEG b$.
        These are not equal since $b$ and $\NEG b$ cannot be in the same 
        proper filter.
        Both filters contain $a$. 
        Hence $[a]$ is not a singleton.
    \end{backward}
\end{proof}