\subsection{Pregeometries}
This subsection walks through the basics of pregeometries,
leading to the theorem on bases having the same cardinality, 
allowing us to define dimension.
It follows the lecture notes by Prof. C. Ward Henson \cite{henson}.

\begin{dfn}[Pregeometry]
    \link{pregeometry_dfn}
    Let $X$ be a set and $\cl$ be a map from the power set of $X$ to itself,
    called the `closure map'.
    $(X,\cl)$ is a pregeometry the following hold
    \begin{enumerate}
        \item $\cl$ is a morphism of ordered sets: 
            for any $U \subs V \subs X$, $\cl(U) \subs \cl(V)$.
        \item Idempotence: for any $U \subs X$, $\cl(\cl(U)) = \cl(U)$.
        \item Finite character: if $U \subs X$ and $a \in \cl(U)$ then 
            there exists a finite subset $F \subs U$ such that $a \in \cl(F)$.
        \item Exchange: if $a \in \cl(U \cup \set{b})$
            then $a \in \cl(U)$ or $b \in \cl(U \cup \set{a})$.
    \end{enumerate}
\end{dfn}

\begin{dfn}[Span, independence, basis]
    Let $(X,\cl)$ be a pregeometry.
    Let $U,V \subs X$.
    \begin{itemize}
        \item $U$ spans $V$ if $U \subs V$ and $\cl(U) = \cl(V)$.
        \item $U$ is independent if for any $a \in U$, 
            $a \notin \cl(U \setminus \set{a})$.
        \item $U$ is a basis of $V$ if $U$ spans $V$ and is independent.
    \end{itemize}
\end{dfn}

\begin{prop}[Independence and span]
    \link{independence_and_span_basic}
    Let $(X,\cl)$ be a pregeometry, $a \in X$ and $U,V \subs X$.
    \begin{enumerate}
        \item If $U$ is independent then any subset of $U$ is independent.
        \item If $U$ is independent and $a \notin \cl(U)$ then then 
            $U \cup \set{a}$ is independent.
        \item $U$ is independent if and only if every finite subset of $U$ is 
            independent.
        \item $U$ spans $V$ if and only if there exists a subset of 
            $U$ that is a basis of $V$. 
        \item Any independent subset $U$ is contained in a basis.
        \item $U$ is a basis of $V$ if and only if $U$ 
            is a maximally independent subset of $V$
            if and only if $U$ is a minimally spanning subset of $V$.
            (i.e. for any independent $W \subs V$, $U \not\subset W$;
            for any spanning $W \subs V$, $W \not\subset U$)
    \end{enumerate}
\end{prop}
\begin{proof}~
    \begin{enumerate}
        \item Let $a \in A \subs U$. 
        Suppose for a contradiction 
        $a \in \cl(A \setminus \set{a})$ then as $\cl$ is a morphism of 
        ordered sets $a \in \cl(U \setminus \set{a})$, 
        which is false as $U$ is independent.
        Hence $A$ is independent.
        \item The forward direction follows from the part $1$. 
        
        \begin{backward}
            Let $a \in U$ and suppose for a contradiction 
            $a \in \cl(U \setminus \set{a})$. 
            As $\cl$ has `finite character' there exists a finite subset 
            $F \subs U \setminus \set{a}$ such that $a \in \cl(F)$.
            This is a contradiction as every subset of $U$ is independent
            so $F$ is independent.
        \end{backward}

        \item Let $x \in U \cup \set{a}$.
        Suppose for a contradiction that 
        $x \in \cl(U \cup \set{a} \setminus \set{x})$.
        We will show that $a \in \cl(U)$, which is false.
        If $x = a$ then $a \in \cl(U \cup \set{a} \setminus \set{a}) = \cl(U)$
        and we are done.
        Otherwise $x \in U$ then by the exchange property
        we have $x \in \cl(U \setminus \set{x})$ or $a \in \cl(U)$.
        The first case is false since $U$ is independent,
        hence we are done.

        \item The backward direction follows from the fact that 
        $\cl$ is a morphism of ordered sets.

        We prove the forward direction and part $4$ at once with a lemma:
        Suppose $U_i \subs U$ and $U_i$ is independent whilst 
        $U$ spans $V$. Consider 
        \[
            \set{W \subs U \st U_i \subs W \text{ and } 
            W \text{ is independent}}
        \]
        This set is non-empty since it contains $U_i$.
        Let $C$ be a chain in the set. 
        Then $\bigcup C$ is a subset of $U$.
        It is independent because every finite subset of it is independent
        by part $3$.

        Noting that $\nothing \subs U$ is independent we have $3$.

        \item Apply the above lemma with $V$ which is spanning.

        \item We show that $U$ is basis of $V$ if and only if it is a 
            maximally independent subset of $V$.

            \begin{forward}
                Suppose for a contradiction that
                $U \subset W$ and $W$ is independent.
                Let $a \in W \setminus U$.
                Since $U$ is a basis 
                $a \in \cl(U) \subs \cl(W \setminus \set{a})$, 
                contradicting independence of $W$.
            \end{forward}

            \begin{backward}
                Suppose $U$ is maximally independent.
                Then to show $U$ is also spanning we show that $V \subs \cl(U)$.
                Let $a \in V$. 
                If $a \in U$ then we are done.
                If $a \notin U$ then $U \cup \set{a}$ is dependent by maximality
                and so there exists $b \in U \cup \set{a}$ such that 
                $b \in \cl(U \cup \set{a} \setminus \set{b})$.
                In the case that $b \in U$ by independence of $U$ and the 
                exchange property
                \[b \in \cl(U \setminus {b} \cup \set{a}) \AND 
                b \notin \cl(U \setminus \set{b}) \implies 
                a \in \cl(U)\]
                In the case that $b = a$ then 
                \[a \in \cl(U \cup \set{a} \setminus \set{a}) = \cl(U)\]
            \end{backward}

            Now we show that $U$ is a basis of $V$ if and only if it is a 
            minimally spanning subset of $V$.

            \begin{forward}
                Suppose for a contradiction we have $W \subset U$ such that 
                $W$ spans $V$.
                Then let $a \in U \setminus W$.
                Since $W$ is spanning we have 
                \[a \in \cl(W) \subs \cl(U \setminus \set{a})\]
                contradicting independence of $U$.
            \end{forward}

            \begin{backward}
                Let $U$ be a minimally spanning subset.
                To show it is independent, 
                let $a \in U$ and suppose for a contradiction 
                $a \in \cl(U \setminus \set{a})$.
                Then $U \subs \cl(U \setminus \set{a})$ and so 
                \[V \subs \cl(U) \subs \cl(U \setminus \set{a})\]
                and $U \setminus \set{a}$ spans $V$
                which contradicts the minimality of $U$.
            \end{backward}
    \end{enumerate}
\end{proof}

\begin{lem}[Support of an element]
    \link{support_pregeometry}
    Let $(X,\cl)$ be a pregeometry and let $U$ be a basis of $V \subs X$.
    Then for any $a \in \cl(V)$ there exists a unique finite subset $F \subs U$
    - the `support' of $a$ in basis $U$ - such that 
    \begin{itemize}
        \item It spans $a$: $a \in \cl(F)$
        \item It is minimal: 
        if $a \in \cl(G)$ for some $G \subs V$ then $F \subs G$
    \end{itemize}
\end{lem}
\begin{proof}
    Since $\cl$ has finite character we have that there exists a finite subset 
    of $U$ containing $a$ in its closure, 
    and by well ordering of the naturals (and choice) we can take one such 
    subset $F$ with minimal cardinality.

    For minimality (which gives uniqueness),
    let $G$ be another subset containing $a$ in its closure.
    Suppose for a contradiction there exists $b \in F \setminus G$.
    Then $a \in \cl(G) \subs \cl(U \setminus \set{b})$.
    Hence 
    \[
        U \setminus \set{b} \cup \set{a} \subs \cl(U \setminus \set{b})
        \implies 
        \cl(U \setminus \set{b} \cup \set{a}) = \cl(U \setminus \set{b})
    \]
    We show that this contradicts the independence of $U$.
    Since $b \in F$ and $F$ is minimal,
    $a \notin \cl(F \setminus \set{b})$ and by exchange 
    \[
        b \in \cl(F \setminus \set{b} \cup \set{a}) 
        \subs \cl(U \setminus \set{b} \cup \set{a}) 
        = \cl(U \setminus \set{b})
    \]
    contradicting the independence of $U$.
\end{proof}

\begin{prop}[Existence of dimension]
    \link{existence_of_dim}
    Let $(X,\cl)$ be a pregeometry.
    Then for any $U$ there exists a basis of $U$.
\end{prop}
\begin{proof}
    We apply Zorn's lemma.
    The set $\set{W \subs U \st W \text{ is independent}}$
    contains the empty set so is non-empty.
    Let $C$ be a chain in there.
    Then $\bigcup C$ is a subset of $U$.
    $\bigcup C$ is independent 
    \linkto{independence_and_span_basic}{if and only if} 
    every finite subset of it is independent, 
    which is true since any part of the chain is independent.
    Hence there exists a maximally independent subset of $U$,
    \linkto{independence_and_span_basic}{i.e. a basis of $U$}.
\end{proof}

\begin{prop}[Uniqueness of dimension]
    \link{uniqueness_of_dim}
    Let $(X,\cl)$ be a pregeometry.
    Then if $U$ and $V$ are bases of $Y \subs X$ then $\abs{U} = \abs{V}$.
\end{prop}
\begin{proof}
    Without loss of generality $\abs{U} \leq \abs{V}$.
    It suffices to show that $\abs{V} \leq \abs{U}$.
    We case on whether $V$ is finite or not.

    If $V$ is infinite then we reconstruct $V$ in the following way:
    for each $a \in U$ there exists a finite subset $F_a$ of $V$ such that 
    $a \in \cl(F_a)$.
    By choice we can construct $(F_a)_{a \in U}$ and take the union 
    $\bigcup_{a \in U} F_a$.
    For any $a \in U$, the union satisfies 
    $a \in \cl(F_a) \subs \cl(\bigcup F_a)$
    thus $U \subs \cl(\bigcup F_a)$ and
    \[Y \subs \cl(U) \subs \cl(\bigcup F_a)\]
    Since $\bigcup F_a$ is a subset of $V$ that spans $Y$, 
    and bases are \linkto{independence_and_span_basic}{minimally spanning sets}, 
    we have $V = \bigcup F_a$.

    Suppose for a contradiction $U$ is finite. 
    Then $V = \bigcup_{a \in U} F_a$ would be a finite union of finite sets 
    which is finite,
    contradicting our assumption about $V$.
    Hence 
    \[
        \abs{V}  = \abs{\bigcup_{a \in U} F_a} 
        \leq \abs{U}
    \]
    Thus we are done.

    If $V$ is finite then we use induction on $n \in \N$ to show that 
    if $V_0$ is a independent set of cardinality $n$ then there exists 
    $W \subs U$ such that $V_0$ and $W$ are disjoint, 
    $V_0 \cup W$ is a basis of $Y$ 
    and $\abs{V_0 \cup W} = \abs{U}$.
    This will imply the existance of such a $W$ for $V$ such that 
    \[\abs{V} \leq \abs{V \cup W} = \abs{U},\]
    completing the proof.

    For the base case we take $W = U$.
    For the induction step we suppose $V_0$ is non-empty (containing $a$) and 
    obtain via the induction hypothesis 
    $W \subs U$ such that $V_0 \setminus \set{a}$ and $W$ are disjoint, 
    $V_0 \setminus \set{a} \cup W$ is a basis of $Y$ 
    and $\abs{V_0 \setminus \set{a} \cup W} = \abs{U}$.
    If $a \in W$ then we can take our new $W$ to be $W \setminus \set{a}$
    and see that $V_0$ and $W \setminus \set{a}$ are disjoint, 
    $V_0 \cup W \setminus \set{a}$ is a basis of $Y$ 
    and $\abs{V_0 \cup W \setminus \set{a}} = \abs{U}$.
    
    Otherwise, take $F$ the 
    \linkto{support_pregeometry}{support of $a$ in the basis 
        $V_0 \setminus \set{a} \cup W$}.
    Supposing for a contradiction that $F \cap W$ were empty,
    we get $F \subs V_0 \setminus \set{a}$, which implies 
    $a \in \cl(F) \subs \cl(V_0 \setminus \set{a})$, 
    contradicting independence of $V_0$.
    Hence there exists $b \in F \cap W$.

    We claim $W \setminus \set{b}$ is our new $W$.
    They are disjoint: since $V_0 \setminus \set{a} \cap W$ is empty 
    and we are in the case where $a \notin W$, we have $V_0 \cap W$ is empty
    and so $V_0 \cap W \setminus \set{b}$ is empty.

    To show that their union is independent:
    we first note that as subsets of independent sets are independent
    \[(V_0 \setminus \set{a}) \cup (W \setminus \set{b})
    = (V_0 \cup W \setminus \set{b}) \setminus \set{a}\]
    is independent.
    \linkto{independence_and_span_basic}{It suffices to show that}
    \[a \notin \cl((V_0 \setminus \set{a}) \cup (W \setminus \set{b}))\]
    Suppose for a contradiction 
    $a \in \cl((V_0 \setminus \set{a}) \cup (W \setminus \set{b}))$.
    Then by minimality of $F$ we have 
    \[
        b \in F \subs (V_0 \setminus \set{a}) \cup (W \setminus \set{b})
        \implies 
        b \in V_0 \setminus \set{a}
        \implies
        V_0 \setminus \set{a} \cap W \ne \nothing
    \]
    The last of which is a contradiction.

    To show that their union is spanning it suffices to show that 
    $b \in \cl(V_0 \cup W \setminus \set{b})$ since this implies 
    \[V_0 \setminus \set{a} \cup W \subs V_0 \cup W 
    \subs \cl(V_0 \cup W \setminus \set{b})\]
    and so $Y \subs \cl(V_0 \setminus \set{a} \cup W)
    \subs \cl(V_0 \cup W \setminus \set{b})$.
    Suppose $b \in V_0$, 
    then $b \in \cl(V_0 \cup W \setminus \set{b})$ and we are done.
    Otherwise 
    \[
        (V_0 \cup W  \setminus \set{b}) \setminus \set{a} =
        (V_0 \setminus \set{a}) \cup (W \setminus \set{b}) = 
        (V_0 \setminus \set{a} \cup W) \setminus \set{b}
    \]
    Since $V_0 \cup W \setminus \set{b}$ is independent
    $a \notin \cl(V_0 \setminus \set{a} \cup W \setminus \set{b})$. 
    Thus by the exchange principle
    \[a \in \cl((V_0 \setminus \set{a}) \cup (W \setminus \set{b}) \cup \set{b}) 
    \implies 
    b \in \cl(V_0 \cup W \setminus \set{b})\]
    and so the union is spanning.

    Finally we note that 
    \[\abs{V_0 \cup W \setminus \set{b}} = \abs{V_0} + \abs{W} - 1 = 
    \abs{V_0 \setminus \set{a} \cup W} = \abs{U}\]
    and the proof is complete.
\end{proof}

\begin{dfn}[Dimension]
    \link{dimension_dfn}
    Let $U$ be a subset of $(X,\cl)$ a pregeometry.
    Then the dimension of $U$ is defined to be the cardinality of any basis of 
    $U$.
    The \linkto{existence_of_dim}{existence} and 
    \linkto{uniqueness_of_dim}{uniqueness} 
    of dimension are given by the previous two theorems.
\end{dfn}