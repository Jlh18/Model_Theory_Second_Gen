\section{Algebraic closure}
\begin{dfn}
    Let $K$ be a field.
    $L$ is an algebraic closure of $K$ 
    if there exists an algebraic field extension 
    $\io : K \to L$ and $L$ is algebraically closed.
\end{dfn}

\begin{prop}[Existence of algebraic closure of Fields \cite{atiyah}]
    \link{algebraic_closure_of_fields}
    Let $K$ be a field. 
    Then there exists an algebraic closure of $K$.
\end{prop}
\begin{proof}
    We build $L$ by induction, 
    at each step making an algebraic extension to a greater field
    containing roots of irreducible polynomials in the previous field.
    Without loss of generality we just do this for $K$.

    Let $F$ be the set of irreducible polynomials $f \in K[x]$.
    Consider $K[x_f]_{f \in F}$, 
    where each $x_f$ is a variable and the ideal 
    $\f{a} = \sum_{f \in F} K f(x_f) $.
    We want to quotient by this ideal,
    but we also want this to result in a field.
    Thus we try to find a maximal ideal containing it.
    By an application of Zorn, all proper ideals are 
    contained in a maximal ideal, 
    hence it suffices to show it is proper.
    Clearly it is non-trivial 
    (it contains $x_f - 1$ for $f = x - 1$).
    Assume for a contradiction that it is the whole ring,
    then there is a finite set $S$ such that
    \[1 = \sum_{f \in S} c_f f(x_f)\]
    where $S$ is finite and each $c_f \in K$.
    By the existance of splitting fields over base field $K$, 
    we take the splitting field $K'$ of $\prod_{f \in S} f$.
    Then from $K'[x_f]_{f \in S}$ 
    we can evaluate at the roots $r_f$ for each $f$:
    \[1 = \sum_{f \in S} c_f f(r_f) = \sum_{f \in S} 0 = 0\]
    a contradiction.

    Let $\f{m}$ be the maximal ideal containing $\f{a}$ and let 
    $K(1) = K[x_f]_{f \in F} / \f{m}$.
    There is an obvious ring morphism $K \to K(1)$. 
    If $f \in K[x]$ is irreducible then $f(x_f) \in \f{m}$ so
    $0 = \bar{f(x_f)} = f(\bar{x_f})$ in $K(1)$.
    Hence $f$ has a root $x_f$ in $K(1)$.

    By the above construction, 
    we have a chain $K : \N \to \Mod{\Si_\RNG}$ and so we can take 
    \linkto{direct_limit_of_chains}{the direct limit}; 
    call this $L$.
    We obtain for free that $L$ is a field and there is a ring morphism 
    $K \to L$ that commutes with all the morphisms in the chain.
    If $f \in L[x]$ then there exists a $\be \in \N$ and 
    $b \in K(\be)[x]$ such that $b \mapsto f$ 
    in the map $K(\be)[x] \to L[x]$.
    Then $b$ has finite degree $n$, 
    thus completely splits in $K(\be + n)$.
    Hence $b$ splits in $L$, 
    $f$ splits in $L$ so $L$ is algebraically closed.

    To show that $L$ is an algebraic extension let $a \in L$.
    Then there exists a $\be \in \N$ and $b \in K(\be)$ such that 
    $b \mapsto a$ in the map $K(\be) \to L$. 
    Since $b$ is algebraic over $K$ 
    (each extension is finite hence algebraic)
    $a$ is algebraic over $K$.
\end{proof}

\begin{prop}[Basic facts about algebraic closures]
    \link{alg_closures_props}
    Suppose $K$ has an algebraic closure $\Om$. 
    Then 
    \begin{enumerate}
        \item If $M$ is an algebraically closed field extending $K$ then 
        there exists a field extension $\Om \to M$ 
        such that the diagram commutes
        \begin{cd}
            K \ar[r] \ar[dr]&\Om \ar[d, dashed]\\
            &M
        \end{cd}
        \item If $K \to L$ is an algebraic field extension then 
        there exists a field extension $L \to \Om$ 
        such that the diagram commutes
        \begin{cd}
            K \ar[r] \ar[d]&\Om\\
            L \ar[ru, dashed]
        \end{cd}
    \end{enumerate}
\end{prop}
\begin{proof}~
    \begin{enumerate}
        \item The map $K \to \Om$ is algebraic and so we can consider $\Om$ as 
        $K(S)$ for some minimal generating set $S \subs \Om$.
        $M$ is algebraically closed thus it splits all minimal polynomials of 
        $a \in S$ over $K$, thus by the embedding theorem (Galois theory), %
        we have an embedding 
        $\Om \to M$ that commutes with the diagram.
        \item If $K \to L$ is algebraic then we can find a generating set 
        $S \subs L$ such that $K(S) = L$.
        $\Om$ is algebraically closed thus it splits all minimal polynomials of 
        $a \in S$ over $K$, thus by the embedding theorem (Galois theory), %
        we have an embedding 
        $L \to \Om$ that commutes with the diagram.
    \end{enumerate}
\end{proof}

\begin{prop}[Algebraic closures of isomorphic fields are isomorphic]
    \link{alg_closures_of_iso_are_iso}
    Suppose $K_0 \iso K_1$. 
    Then any algebraic closures of these fields are (non-canonically) isomorphic.
\end{prop}
\begin{proof}
    Call the algebraic closures $\Om_0$ and $\Om_1$:
    \begin{cd}
        K_0 \ar[r, "\io_0"] \ar[d, "\si", "\sim"{swap}]     & \Om_0\\
        K_1 \ar[r, "\io_1"]                                 & \Om_1
    \end{cd}
    By the \linkto{alg_closures_props}{properties of algebraic closures}
    there exist
    $f : \Om_0 \to \Om_1$ and $g : \Om_1 \to \Om_0$ 
    that individually commute with the above diagram.
    We claim that the composition $g \circ f$ is a surjection.

    The composition fixes $K_0$ because 
    $\io_0 = g \circ \io_1 \circ \si = g \circ f \circ \io_0$.
    Fix $a \in \Om_0$.
    Let $\La$ be the set of roots of the minimal polynomial $\min(a,K_0)$.
    $g \circ f$ is a bijection on $\La$: let $\la \in \La$, then
    $g \circ f (\min(a,K_0))(\la) = \min(a, K_0)(g \circ f(\la))$ because 
    $g \circ f$ fixes $K_0$.
    Thus $g \circ f (\La) \subs \La$. 
    It is a therefore a bijection on $\La$ as 
    $\La$ is finite and $g \circ f$ is injective (it is a field embedding).
    Thus there exists an element that maps to $a$ under the composition.
    Hence
    \[\Om_1 = g \circ f (\Om_1) \subs f(\Om_0)\]
    Thus $f$ is surjective and injective (as it is a field embedding).
    Hence $f$ is an isomorphism.
\end{proof}