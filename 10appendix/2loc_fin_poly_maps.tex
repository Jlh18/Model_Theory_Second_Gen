\section{Locally finite fields and polynomial maps}

\begin{dfn}[Locally finite]
    We say that a field is locally finite if for any finite subset,
    the minimal subfield `$K(S)$' containing the subset is finite.
\end{dfn}

\begin{dfn}[Polynomial map]
    Let $K$ be a field and $n$ a natural.
    Let $f : K^n \to K^n$ such that for each $a \in K^n$,
    \[f(a) = (f_1(a), \dots, f_n(a))\] for 
    $f_1, \dots, f_n \in K[x_1, \dots, x_n]$.
    Then we call $f$ a polynomial map over $K$.
\end{dfn}

\begin{lem}[Equivalences for locally finite over prime characteristic 
        \cite{stack0}]
    \link{locally_finite_over_prime}
    Let $K$ be a field of characteristic $p$ a prime.
    Then the following are equivalent:
    \begin{enumerate}
        \item $K$ is locally finite.
        \item $\F_p \to K$ is algebraic.
        \item $K$ embeds into an algebraic closure of $\F_p$.
    \end{enumerate}
    In particular, the algebraic closure of a finite field is locally finite.
\end{lem}
\begin{proof}~
    \begin{enumerate}
        \item[{}]$1.\implies 2.$ We show the contrapositive.
            Suppose there exists $a \in K$ such that $a$
            is not algebraic over $\F_p$. 
            Then $\F_p(a)$ is a isomorphic to the field of fractions
            of a polynomial ring over $K$ and so is infinite.
            Hence $K$ is not locally finite.
        \item[{}]$2. \implies 1.$ We show by induction that $K$ is locally 
            finite.  
            Let $S$ be a finite subset of $K$.
            If $S$ is empty then $\F_p(S) = \F_p$ and so it is finite.
            If $S = S' \cup {s}$ and $\F_p(S')$ is finite,
            then $s \in K$ is algebraic so by some Galois theory we can write%
            \[\F_p(S')[x] / (\min(s, \F_p(S'))) \iso \F_p(S)\]
            Where the left hand side is finite because it is a finite 
            dimensional vector space over a finite field.
            Hence $K$ is locally finite.
        \item[{}]$2. \implies 3.$ If $\F_p \to K$ is algebraic then 
            it can be written as $\F_p(S)$ 
            for some set $S$ algebraic over $\F_p$. 
            Any algebraic closure of $\F_p$ splits each $a \in S$ hence by the 
            embedding theorem (Galois Theory)%
            $K$ embeds into the algebraic closure.
        \item[{}]$3. \implies 2.$ If $a \in K$ then $a$ can be embedded into 
            the algebraic closure of $\F_p$ by assumption.
            Then it is algebraic over $\F_p$ thus $a$ is algebraic over $\F_p$.
    \end{enumerate}
    Any finite field is an algebraic extension over $\F_p$ where $p$ is its
    prime characteristic.
    Hence its algebraic closure is an algebraic extension over $\F_p$ and so 
    it is locally finite.
\end{proof}

\begin{cor}[Ax-Grothendieck for algebraic closure of finite fields]
    \link{appendix_algebraic_closure_ax_groth}
    If $\Om$ is an algebraic closure of a finite field
    then any injective polynomial map over $\Om$ is surjective.
\end{cor}
\begin{proof}
    Suppose for a contradiction 
    $f : \Om^n \to \Om^n$ is an injective polynomial map that is not surjective.
    Then there exists $b \in \Om^n$ such that $b \notin f(\Om^n)$.
    Writing $f = (f_1, \dots, f_n)$ for $f_i \in \Om[x_1, \dots, x_n]$
    we can find $A \subs \Om^n$, 
    the set of all the coefficients of all of the $f_i$.
    We want to find a contradiction by showing that $f$ is surjective on
    the subfield $K$ generated by $A \cup \set{b}$, 
    which contains $b$.
    $A \cup \set{b}$ is finite and \linkto{locally_finite_over_prime}{
        $\Om$ is locally finite},
    thus $K$ is finite.
    The restriction $\res{f}{K^n}$ is injective and has image inside $K^n$
    since each polynomial has coefficients in $K$ and is evaluated at 
    an element of $K^n$.
    Hence $\res{f}{K^n}$ is an injective endomorphism of a finite set thus 
    is surjective,
    a contradiction.
\end{proof}