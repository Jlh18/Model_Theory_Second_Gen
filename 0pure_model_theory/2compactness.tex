\subsection{The Compactness Theorem}

Read ahead for the statement of \linkto{compactness}{the Compactness Theorem}.
The first two parts of the theorem are easy to prove.
This chapter focuses on proving the final part.

\begin{dfn}[Witness property]
    Given a signature $\Si$ and a $\Si$-theory $T$, 
    we say that $\Si$ satisfies the witness property when
    for any $\Si$-formula $\phi$ with 
    \linkto{form_fin_var}{exactly one free variable} $v$,
    there exists $c \in \const{\Si}$ such that 
    $T \modelsi \brkt{\exists v, \phi(v)} \to \phi(c)$.
\end{dfn}
This says that if for all $\Si$-model $\MM$ of $T$,
there exists an element $a \in \MM$ such that $\MM \model{\Si} \phi(m)$
then there exists a constant symbol $c$ of $\Si$ 
such that $\phi(\mmintp{c})$ is true.

\begin{dfn}[Maximal theory]
    A $\Si$ theory $T$ is $\Si$-maximal if for any 
    $\Si$-formula $\phi$,
    if $\phi$ is a $\Si$-sentences then
    $\phi \in T$ or $\NOT \phi \in T$.
\end{dfn}

\begin{prop}[Maximum property]\link{max_prop}
    Given a $\Si$-maximal and finitely consistent theory $T$ and
    a $\Si$-sentence $\phi$, 
    \[ T \model{\Si} \phi \text{ if and only if } \phi \in T
    \text{ if and only if } \NOT \phi \notin T
    \text{ if and only if } \nodel{\Si} \NOT \phi\]   
\end{prop}
\begin{proof}
    First note that by maximality and finite consistency
    if $\phi, \NOT \phi \in T$ then we have a finite 
    \linkto{inconsistent} subset 
    $\set{\phi,\NOT \phi} \subs T$, which is false. 
    Hence
    \[\phi \in T \iff \NOT \phi \notin T\]
    We prove the first if and only if and deduce the third by 
    replacing $\phi$ with $\NOT \phi$.
        \begin{forward}
            Suppose $T \model{\Si} \phi$.
            Since $T$ is $\Si$-maximal, 
            we have $\phi \in T$ or $\NOT \phi \in T$.
            If $\NOT \phi \in T$ then we have a finite subset 
            $\set{\phi,\NOT \phi} \subs T$.
            Hence $T$ is \linkto{inconsistent}{not finitely consistent},  
            thus the second case is false.
        \end{forward}
        \begin{backward}
            Suppose $\phi \in T$. 
            Case on $T \model{\Si} \phi$ or $T \nodel{\Si} \phi$.
            If $T \nodel{\Si} \phi$ then there exists 
            $\NN$ a $\Si$-model of $T$ such that 
            $N \nodel{\Si} \phi$.
            But $\NN \model{\Si} \phi$ since $\phi \in T$.
            Thus the second case is false.
        \end{backward}
\end{proof}

\begin{nttn}[Ordering signatures]
    We write $\Si \leq \Si(*)$ for two signatures if
    $\const{\Si} \subs \const{\Si(*)}$,
    $\func{\Si} \subs \func{\Si(*)}$ and $\rel{\Si} \subs \rel{\Si(*)}$.
\end{nttn}

For the sake of formality we include the following two lemmas, 
neither of which are particularly inspiring or significant,
but they do allow us to move freely between signatures.
\begin{lem}[Moving models down signatures]
    \link{move_down_mod}
    Given two signatures such that 
    $\Si \leq \Si(*)$ and
    $\NN$ a $\Si(*)$-structure we can make 
    $\MM$ a $\Si$-structure such that
    \begin{enumerate}
        \item $\carrier{\MM} = \carrier{\NN}$
        \item They have the same interpretation on $\Si$.
        \item For any $\Si$-formula $\phi$ with free variables indexed by $S$
            and any $a \in \MM^S$
            \[\MM \model{\Si} \phi(a) \iff \NN \model{\Si(*)} \phi(a)\]
        \item If $T$ is a $\Si$-theory and $T(*)$ is a $\Si(*)$-theory 
            such that $T \subs T(*)$ and $\NN$ a $\Si(*)$-model of $T(*)$,
            then $\MM$ is a $\Si$-model of $T$.
    \end{enumerate}
    Technically the new structure is not $\NN$,
    but for convenience we will write 
    $\NN$ to mean either of the two and let subscripts involving 
    $\Si$ and $\Si(*)$ describe which one we mean.
\end{lem}
\begin{proof}
    We let the carrier set be the same and 
    define $\intp{\Si}{\MM}$ by restriction:
    \begin{itemize}
        \item $\intp{\const{\Si}}{\MM}$ is
        the restriction of $\intp{\const{\Si(*)}}{\NN}$ to $\const{\Si}$
        \item $\intp{\func{\Si}}{\MM}$ is
        the restriction of $\intp{\func{\Si(*)}}{\NN}$ to $\func{\Si}$
        \item $\intp{\rel{\Si}}{\MM}$ is
        the restriction of $\intp{\rel{\Si(*)}}{\NN}$ to $\rel{\Si}$
    \end{itemize}
    We will need that for any $\Si$-term $t$ with variables indexed by $S$,
    the \link{move_down_mod_int_of_terms} interpretation of terms is equal:
    $\subintp{\Si}{\MM}{t} = \subintp{\Si(*)}{\NN}{t}$.
    Indeed:
    \begin{itemize}
        \item If $t$ is a constant then 
        $\subintp{\Si}{\MM}{t} = \subintp{\Si}{\MM}{c} =
        \subintp{\Si(*)}{\NN}{c} = \subintp{\Si(*)}{\NN}{t}$
        \item If $t$ is a variable then 
        $\subintp{\Si}{\MM}{t} = \id{{\MM}} =
        \id{{\NN}} = \subintp{\Si(*)}{\NN}{t}$
        \item If $t$ is $f(s)$ then by induction
        $\subintp{\Si}{\MM}{t} = \subintp{\Si}{\MM}{f}(\subintp{\Si}{\MM}{s}) =
        \subintp{\Si(*)}{\NN}{f}(\subintp{\Si(*)}{\NN}{s}) = 
        \subintp{\Si(*)}{\NN}{t}$
    \end{itemize}
    
    Let $\phi$ be a $\Si$-formula with variables
    indexed by $S \subs \N$.
    Let $a$ be in $\MM^S$.
    Case on $\phi$ to show that 
    $\MM \model{\Si} \phi(a) \iff \NN \model{\Si(*)} \phi(a)$:
    \begin{itemize}
        \item {If $\phi$ is $\top$ then both satisfy $\phi$.}
        \item If $\phi$ is $t = s$ then
        \[
            \MM \model{\Si} \phi(a) \iff \subintp{\Si}{\MM}{t} = 
            \subintp{\Si}{\MM}{s}
            \iff \subintp{\Si(*)}{\MM}{t} = \subintp{\Si(*)}{\MM}{s} 
            \iff \NN \model{\Si(*)} \phi(a)
        \]
            Since the interpretation 
            of terms are equal from above.
        \item If $\phi$ is $r(t)$ then by how we defined 
        $\subintp{\Si}{\MM}{r}$ and since interpretation 
        of terms are equal
        \[
            \MM \model{\Si} \phi(a) \iff \subintp{\Si}{\MM}{t}(a) 
            \in \subintp{\Si}{\MM}{r}
            \iff \subintp{\Si(*)}{\NN}{t}(a) \in 
            \subintp{\Si(*)}{\NN}{r}
            \iff \NN \model{\Si(*)} \phi(a) 
        \]
        \item If $\phi$ is $\NOT \psi$ then using the induction hypothesis
            \[
                \MM \model{\Si} \phi(a) \iff \MM \nodel{\Si} \psi(a)
                \iff \NN \nodel{\Si(*)} \psi(a)
                \iff \NN \model{\Si(*)} \phi(a) 
            \]
        \item If $\phi$ is $\psi \OR \chi$ then using the induction hypothesis
            \begin{align*}
                \MM \model{\Si} \phi(a) \iff \MM \model{\Si} \psi(a) 
                \text{ or } \MM \model{\Si} \chi(a)
                \iff \NN \model{\Si(*)} \psi(a) \text{ or } 
                \NN \model{\Si(*)} \chi(a)
                \iff \NN \model{\Si(*)} \phi(a) 
            \end{align*}
        \item If $\phi$ is $\forall v, \psi$ then
        \begin{align*}
            \MM \model{\Si} \phi(a) &\iff \forall b \in {\MM}, 
            \MM \model{\Si} \psi(a,b)\\
                &\iff \forall b \in {\MM}, \NN \model{\Si(*)} \psi(a,b)
                & \text{by the induction hypothesis}\\
                &\iff \forall b \in {\NN}, \NN \model{\Si(*)} \psi(a,b)
                & \text{by the induction hypothesis}\\
                & \iff \NN \model{\Si(*)} \phi(a) 
        \end{align*}
    \end{itemize}
    Hence $\MM \model{\Si} \phi(a) \iff \NN \model{\Si(*)} \phi(a)$.

    Suppose $T \subs T(*)$ are respectively $\Si$ and $\Si(*)$-theories and 
    $\NN \model{\Si(*)} T(*)$.
    If $\phi \in T \subs T(*)$ then by the previous part, 
    $\NN \model{\Si(*)} \phi$ implies $\MM \model{\Si} \phi$.
    Hence $\MM \model{\Si} T$.
\end{proof}

\begin{lem}[Moving models and theories up signatures] 
    \link{move_up_mod}
    Suppose $\Si \leq \Si(*)$.
    \begin{enumerate}
        \item Suppose $\MM$ is a $\Si$-model of $\Si$-theory $T$.
            Then if there exists $\MM(*)$ a $\Si(*)$-structure 
            whose carrier set is the same as $\MM$
            and whose interpretation agrees with $\intp{\Si}{\MM}$ 
            on constants, functions and relations of $\Si$,
            then $\MM(*)$ is a $\Si(*)$-model of $T$.
        \item Suppose $T$ is a $\Si$-theory and $\phi$ is a 
            $\Si$-sentence such that $T \model{\Si} \phi$.
            Then $T \model{\Si(*)} \phi$.
    \end{enumerate}
    Again, if we have constructed such a $\MM(*)$ from $\MM$
    we tend to just refer to it as $\MM$ and let subscripts involving 
    $\Si$ and $\Si(*)$ describe which one we mean.
\end{lem}
\begin{proof}~
    \begin{enumerate}
        \item Suppose $\MM \model{\Si} T$.
        Let $\phi \in T$.
        To show that 
        $\MM(*) \model{\Si(*)} \phi$ we
        first we prove a useful claim:
        if $t \in \term{\Si}$ with no variables then 
        $\subintp{\Si(*)}{\MM(*)}{t} = \subintp{\Si}{\MM}{t}$.
        Case on what $t$ is:
        \begin{itemize}
            \item If $t$ is a constant symbol $c$ in $\const{\Si}$, 
                then since $\intp{\Si(*)}{\MM(*)} = 
                    \intp{\Si}{\MM}$ on $\Si$,
                \[\subintp{\Si(*)}{\MM(*)}{t} = 
                    \subintp{\Si(*)}{\MM(*)}{c} =
                    \subintp{\Si}{\MM}{c} =
                    \subintp{\Si}{\MM}{t}\]
            \item If $t$ is a variable then it has one variable,
            thus false.
            \item If $t$ is $f(s)$ then since 
            $\intp{\Si(*)}{\MM(*)} = 
            \intp{\Si}{\MM}$ on $\Si$,
            \[
                \subintp{\Si(*)}{\MM(*)}{t} =
                \subintp{\Si(*)}{\MM(*)}{f(s)} =
                \subintp{\Si(*)}{\MM(*)}{f}(\subintp{\Si(*)}{\MM(*)}{s}) =
                \subintp{\Si}{\MM}{f}(\subintp{\Si}{\MM}{s}) =
                \subintp{\Si}{\MM}{f(s)} =
                \subintp{\Si}{\MM}{t}
            \]
        \end{itemize}
        Case on what $\phi$ is ($\phi$ has no variables):
        \begin{itemize}
            \item {If $\phi$ is $\top$ then it is satisfied.}
            \item If $\phi$ is $t = s$, 
                then by the claim above,
                \[
                    \subintp{\Si(*)}{\MM(*)}{t} = 
                    \subintp{\Si}{\MM}{t} = 
                    \subintp{\Si}{\MM}{s} = 
                    \subintp{\Si(*)}{\MM(*)}{s}
                \]
            \item If $\phi$ is $r(t)$,
                then by the claim above
                and the fact that relations are interpreted the same way,
                \[
                    \subintp{\Si(*)}{\MM(*)}{t} = 
                    \subintp{\Si}{\MM}{t} \in
                    \subintp{\Si}{\MM}{r} = 
                    \subintp{\Si(*)}{\MM(*)}{r}
                \]
            \item If $\phi$ is $\NOT \psi$ 
                then using the induction hypothesis on $\psi$,
                \[
                    \MM \model{\Si} \phi \; \iff \;
                    \MM \nodel{\Si} \psi \; \iff \;
                    \MM(*) \nodel{\Si(*)} \psi \; \iff \;
                    \MM(*) \model{\Si(*)} \phi
                \]
            \item If $\phi$ is $\psi \OR \chi$
                then using the induction hypothesis on $\psi$ and $\chi$,
                \[
                    \MM \model{\Si} \phi \iff 
                    \MM \model{\Si} \psi \text{ or } \MM \model{\Si} \chi 
                    \iff \MM(*) \model{\Si(*)} \psi \text{ or } 
                    \MM(*) \model{\Si(*)} \chi 
                    \iff \MM(*) \model{\Si(*)} \phi
                \]
            \item If $\phi$ is $\forall v, \psi(v)$ and 
                $a \in {\MM(*)} = {\MM}$
                then using the induction hypothesis on $\psi$,
                $\MM \model{\Si} \psi(a) \implies \MM(*) \model{\Si(*)} \psi(a)$.
                Hence $\MM(*) \model{\Si(*)} \phi$.
        \end{itemize}
        Thus $\MM(*)$ is a $\Si(*)$-model of $T$.

        \item Suppose $T \model{\Si} \phi$.
        If $\MM(*) \model{\Si(*)} T$ then by 
        \linkto{move_down_mod}{moving $\MM(*)$ down to $\Si$},
        we have a corresponding $\MM \model{\Si} T$ 
        whose carrier set is the same as $\MM(*)$.
        Hence $\MM \model{\Si} \phi$.
        Naturally the models agree on interpretation
        of constants, functions and relations of $\Si$ and so 
        $\MM(*) \model{\Si(*)} \phi$ by the previous part.
    \end{enumerate}
\end{proof}

The following lemma is the bulk of the proof of the compactness theorem.
\begin{lem}[Henkin construction]
    \link{henkin}
    Let $\Si$ be a signature.
    Let $0 < \ka$ be a cardinal such that $|\const{\Si}| \leq \ka$.
    If a $\Si$-theory $T$ 
    \begin{itemize}
        \item has the witness property
        \item is $\Si$-maximal
        \item is finitely consistent
    \end{itemize}
    then it has a non-empty $\Si$-model $\MM$ such that $|\MM| \leq \ka$.
\end{lem}
\begin{proof}
    Without loss of generality $\Si$ is non-empty, 
    since we can add a constant symbol to $\Si$, 
    \linkto{move_up_mod}{see $T$ as a theory 
    in that signature}, and then 
    \linkto{move_down_mod}{take the model we make of $T$ back down} to 
    being a $\Si$-model of $T$.
    \textbf{The $\Si$-structure}: 
    Consider quotienting $\const{\Si}$ by the equivalence relation
    $c \sim d$ if and only if $T \modelsi c = d$.
    Let $\pi : \const{\Si} \to \const{\Si} / \sim $.
    This defines a non-empty $\Si$-structure $\MM$ in the following way:
    \begin{enumerate}
        \item We let the carrier set be the image of the quotient.
        We let the constant symbols be interpreted as their equivalence classes:
        $\intp{\const{\Si}}{\MM} = \pi$.
        We now have the desired cardinality for $\MM$:\footnote{From 
            this point onwards when it is obvious what we mean 
            we write $\MM$ for $\carrier{\MM}$.} 
        $|{\MM}| \leq |\const{\Si}| \leq \ka$
        and $\const{\Si}$ is non-empty so $\MM$ is non-empty.
        \item To interpret functions we must use the witness property.
        Given $f \in \func{\Si}$ and $\pi(c) \in \MM^{n_f}$
        ($\pi$ is surjective),
        we obtain by the witness property some $d \in \const{\Si}$
        such that \[T \modelsi (\exists v, f(c) = v) \to (f(c) = d)\]
        We define $\mmintp{f}$ to map $\pi(c) \mapsto \pi(d)$.
        
        To show that $\mmintp{f}$ is well-defined,
        let $c_0,c_1 \in (\const{\Si})^{n_f}$ 
        such that $\pi(c_0) = \pi(c_1)$ and
        suppose $\pi(d_0),\pi(d_1)$ are their images.
        It suffices that $\pi(d_0) = \pi(d_1)$, i.e. $T \modelsi d_0 = d_1$.
        Indeed, let $\Si$-structure $\NN$ be a model of $T$.
        Then
        \[
            \NN \modelsi (\exists v, f(c_0) = v) 
            \implies 
            \NN \modelsi f(c_0) = d_0
        \]
        Similarly $f(c_1) = d_1$. Hence 
        \[\modintp{\NN}{d_0} = \modintp{\NN}{f}(\modintp{\NN}{c_0})
        = \modintp{\NN}{f}(\modintp{\NN}{c_1}) = \modintp{\NN}{d_1}\]
        and $N \modelsi d_0 = d_1$.
        Hence $T \modelsi d_0 = d_1$.
        \item Let $r \in \rel{\Si}$. 
        We define $\mmintp{r} := 
        \set{\pi(c) \st c \in (\const{\Si})^{m_r} \AND T \modelsi r(c)}$
    \end{enumerate}
    Hence $\MM$ is a $\Si$-structure.
    We want to show that $\MM$ is a $\Si$-model of $T$.

    \textbf{Terms}: to show that $\MM$ satisfies formulas of $T$ 
    we first need that interpretation of terms is working correctly. 
    Claim: if $t \in \term{\Si}$ with variables indexed by $S$,
    $d \in \const{\Si}$ and
    $c$ is in $(\const{\Si})^S$
    then $T \modelsi t(c) = d$ 
    if and only if $\mmintp{t}(\mmintp{c}) = \mmintp{d}$.
    i.e. $\MM \model{\Si} t(c) = d$.
    Case on what $t$ is:
    \begin{itemize}
        \item If $t$ is a constant symbol,
        $T \modelsi t(c) = d$ if and only if 
        $T \modelsi t = d$ 
        if and only if $\pi(t) = \pi(d)$
        if and only if $\mmintp{t} = \mmintp{t}(\mmintp{c}) = \mmintp{d}$.
        \item Suppose $t \in \var{\Si}$,
        then it suffices to show that $T \modelsi c = d$ 
        if and only if $\mmintp{c} = \mmintp{d}$, 
        which we have already done above.
        \item With the induction hypothesis, 
        suppose $t = f(s)$, 
        where $f \in \func{\Si}$ and $s \in (\term{\Si})^{n_f}$.
        \begin{forward} 
            If we can find 
            $e = (e_1, \dots e_{n_f}) \in (\const{\Si})^{n_f}$ 
            such that each 
            $T \modelsi s_i (c) = e_i$
            and $\mmintp{f(e)} = \mmintp{d}$,
            then we have 
            $\mmintp{s_i}(\mmintp{c})
            = \mmintp{e_i}$ and so
            \[
                \mmintp{t}(\mmintp{c}) = \mmintp{(f(s))}(\mmintp{c})
                = \mmintp{f}(\mmintp{s}(\mmintp{c}))
                = \mmintp{f}(\mmintp{e})
                = \mmintp{f(e)} = \mmintp{d}
            \]
            Indeed, using the witness property $n_f$
            times we can construct $e$.
            Suppose we have by induction
            $e_1,\dots, e_{i-1} \in \const{\Si}$
            such that for each $j<i$, 
            they satisfy
            \[
                T \modelsi \exists x_{j+1}, \dots, \exists x_{n_f}, 
                f(e_1, \dots, e_j, x_{j+1},\dots, x_{n_f}) = d 
                \AND e_j = s_j (c)
            \]
            For each $i$ let $\phi_i$ be the formula
            \[
                \exists x_{i+1}, \dots, \exists x_{n_f}, 
                f(e_1, \dots, e_{i-1},v,x_{i+1},\dots, x_{n_f}) = d 
                \AND v = s_i (c)
            \]
            with a single variable $v$.
            Then by the witness property,
            there exists an $e_i \in \const{\Si}$ such that
            $T \modelsi \exists v, \phi_i(v) \to \phi_i(e_i)$.
            To complete the induction we show that $T \modelsi \phi_i(e_i)$.
            Then we will be done since each $\phi_i(e_i)$ 
            will give us $T \modelsi s_i (c) = e_i$
            and the last $\phi_{n_f}$ gives $\mmintp{f(e)} = \mmintp{d}$.

            To this end, let $\NN$ be a model of $T$.
            Then $\NN \modelsi \exists v, \phi_i(v) \to \phi_i(e_i)$.
            By assumption $T \modelsi t(c) = d$ so 
            $\nnintp{t}(\nnintp{c}) = \nnintp{d}$,
            hence 
            \[
                \modintp{\NN}{f}(
                    \modintp{\NN}{s_1}(c),\dots,\modintp{\NN}{s_{n_f}}(c)) = 
                \modintp{\NN}{d}
            \]
            Taking $v$ to be $s_i(c)$ and $x_{i+k}$ to be $s_{i+k}(c)$
            we have that 
            $\NN \modelsi \exists v, \phi_i(v)$.
            Hence $\NN \modelsi \phi_i(e_i)$.
            Thus we have $T \modelsi \phi_i(e_i)$.
        \end{forward}

        \begin{backward} 
            Note that each $\mmintp{(s_i(c))} = \mmintp{e_i}$ 
            for some $e_i \in \const{\Si}$ 
            since $\pi$ is surjective.
            By the induction hypothesis for each $i$ we have 
            $T \model{\Si} s_i(c) = e_i$.
            Hence 
            \[
                \mmintp{f(e)}
                = \mmintp{f}(\mmintp{e}) = \mmintp{f}(\mmintp{(s(c))}) 
                = \mmintp{f}(\mmintp{s}(\mmintp{c})) 
                = \mmintp{t}(\mmintp{c}) = \mmintp{d}
            \]
            Hence $\pi(f(e)) = \pi(d)$ and 
            $T \modelsi f(e) = d$.
            It follows that 
            $T \modelsi t(c) = d$.
        \end{backward}
    \end{itemize}
    Thus $T \modelsi t(c) = d \iff \mmintp{t}(\mmintp{c}) = \mmintp{d}$.

    \textbf{Formulas}:
    now we can show that $\MM \modelsi T$. 
    Since for all
    $\phi \in T$ we have $T \modelsi \phi$,
    it suffices to show that for all $\Si$-sentences
    $\phi, T \modelsi \phi $ 
    implies $\MM \modelsi \phi$.
    We prove a stronger statement
    which will be needed for the induction:
    for all $\Si$-formulas $\phi$ with variables indexed by $S$ and 
    $c \in (\const{\Si})^S$,
    \[
        T \modelsi \phi(c) \iff \MM \modelsi \phi(\mmintp{c})
    \]
    We case on what $\phi$ is:
    \begin{itemize}
        \item Case $\phi$ is $\top$:
        all $\Si$-structures satisfy $\top$.
        \item Case $\phi$ is $t = s$: 
            \begin{forward} 
                Apply the witness property to 
                $t(c) = v$,
                obtaining $d \in \const{\Si}$ such that
                $T \modelsi (\exists v, t(c) = v) \to t(c) = d$.
                Since clearly $T \modelsi \exists v, t(c) = v$
                (take any model 
                and it has an interpretation of $t(c)$) 
                we have $T \modelsi t(c) = d$.
                Also by assumption 
                $T \modelsi t(c) = s(c)$
                so it follows that $T \modelsi s(c) = d$.
                Using the claim from before for terms, 
                we obtain
                $\mmintp{t}(\mmintp{c}) = \mmintp{d} = 
                \mmintp{s}(\mmintp{c})$.
                Hence $\MM \modelsi t(\mmintp{c}) = s(\mmintp{c})$.
            \end{forward}

            \begin{backward}
                If $\MM \modelsi t(\mmintp{c}) = s(\mmintp{c})$ 
                then since $\pi$ is surjective,
                there exists $d \in \const{\Si}$ such that 
                \[\mmintp{t}(\mmintp{c}) = \mmintp{s}(\mmintp{c})= \mmintp{d}\]
                Using the claim for terms
                we obtain $T \model{\Si} t(c) = d$ and 
                $T \model{\Si} s(c) = d$.
                It follows that $T \model{\Si} t(c) = s(c)$.
            \end{backward}

        \item Case $\phi$ is $r(t)$:
            \begin{forward}
                Suppose $T \model{\Si} r(t(c))$.
                By induction, 
                apply the witness property 
                $m_r$ times to the formulas
                \[\exists x_{i + 1}, \dots, 
                \exists x_{m_r}, r(
                    \dots, e_{i-1}, v , x_{i+1},\dots
                    ) \AND v = t_i(c)\]
                each time obtaining $e_i \in \const{\Si}$ satisfying the formula.
                The result is $T \modelsi r(e)$ and each
                $T \model{\Si} t_i(c) = e_i$.
                Using the claim for terms and 
                how we interpreted relations in $\MM$ this implies
                $\mmintp{t}(\mmintp{c}) = \mmintp{e} \in \mmintp{r}$,
                and hence $\MM \modelsi r(t(c))$.
            \end{forward}

            \begin{backward}
                Suppose $\MM \modelsi r(t(c))$.
                Since $\pi$ is surjective, 
                there exists $e \in \const{\Si}$ 
                such that $\mmintp{e} = \mmintp{t}(\mmintp{c}) \in \mmintp{r}$.
                Using the claim for terms again
                we obtain $T \model{\Si} t(c) = e$ and 
                using how $\MM$ interprets relations,
                $T \model{\Si} r(e)$.
                It follows that $T \model{\Si} r(t(c))$.
            \end{backward}
        
        \item Case $\phi$ is $\NOT \chi$:
            Using \linkto{max_prop}{the maximal property of $T$} 
            for the first $\iff$
            and the induction hypothesis for the second $\iff$ we have
            \[T \model{\Si} \NOT \chi(c) \; \iff \;  
            T \nodel{\Si} \chi(c) \; \iff \; \MM \nodel{\Si} \chi(c) \; \iff \;
            \MM \model{\Si} \NOT \chi(c)\]

        \item Case $\phi$ is $\chi_0 \OR \chi_1$
            \begin{align*}
                \MM \model{\Si} \chi_0(\mmintp{c}) \OR \chi_1(\mmintp{c}) 
                    &\;\iff\; \MM \model{\Si} \chi_0(\mmintp{c}) 
                    \text{ or } \MM \model{\Si} \chi_1(\mmintp{c}) \\
                    &\;\iff\; T \model{\Si} \chi_0(c) \text{ or } 
                    T \model{\Si} \chi_1(c) 
                    \text{\quad by the induction hypothesis}
            \end{align*}
            Hence it suffices to show that 
            \[T \model{\Si} \chi_0(c) \text{ or } T \model{\Si} \chi_1(c) 
            \;\iff\; T \model{\Si} \chi_0(c) \OR \chi_1(c)\]
            \begin{forward}
                Suppose $T \model{\Si} \chi_0(c)$ or 
                $T \model{\Si} \chi_1(c)$.
                For $\NN$ a $\Si$-model of $T$,
                \[
                    \NN \model{\Si} \chi_0(\nnintp{c}) \text{ or } 
                    \NN \model{\Si} \chi_1(\nnintp{c}) 
                    \implies \NN \model{\Si} \chi_0(\nnintp{c}) 
                    \OR \chi_1(\nnintp{c})
                \]
                Thus $T \model{\Si} \chi_0(c) \OR \chi_1(c)$.
            \end{forward}

            \begin{backward}
                Suppose $T \model{\Si} \chi_0(c) \OR \chi_1(c)$.
                For $\NN$ a $\Si$-model of $T$,
                \[
                    \NN \model{\Si} \chi_0(\nnintp{c}) 
                    \OR \chi_1(\nnintp{c}) 
                    \implies
                    \NN \model{\Si} \chi_0(\nnintp{c}) \text{ or } 
                    \NN \model{\Si} \chi_1(\nnintp{c}) 
                \]
                Thus $T \model{\Si} \chi_0(c)$ or $T \model{\Si} \chi_1(c)$.
            \end{backward}

        \item Case $\phi$ is $\forall v, \chi(v)$
            \begin{forward}
                Let $d \in {\MM}$, 
                then since $\pi$ surjective $\exists e \in \const{\Si}$
                such that $\pi(e) = d$.
                Since $T \model{\Si} \forall v, \chi(c,v)$ 
                it follows that $T \model{\Si} \chi(c,e)$
                and by induction $\MM \model{\Si} \chi(\mmintp{c},d)$.
                Hence $\MM \model{\Si} \forall v, \chi(\mmintp{c},v)$.
            \end{forward}

            \begin{backward}
                We show the contrapositive.
                If $T \nodel{\Si} \forall v, \chi(c) (v)$, 
                then by \linkto{max_prop}{the maximal property of $T$},
                $T \model{\Si} \exists v, \NOT \chi(c,v)$.
                Applying the witness property to $\NOT \chi(c,v)$,
                there exists $e \in \const{\Si}$ such that 
                \[
                    T \model{\Si} (\exists v, \NOT \chi(c) (v)) \to 
                    (\NOT \chi(c) (v))
                    \quad 
                    \implies
                    \quad T \model{\Si} \NOT \chi(c) (v)
                \]
                Thus $T \nodel{\Si} \chi(c) (v)$ by 
                \linkto{max_prop}{the maximal property of $T$}
                and $\MM \nodel{\Si} \chi(\mmintp{c}) (v)$ 
                by the induction hypothesis.
                Hence $\MM \nodel{\Si} \forall v, \chi(\mmintp{c}) (v)$.
            \end{backward}
    \end{itemize}
    Thus $T \modelsi \phi \iff \MM \modelsi \phi$ and we are done.
\end{proof}

\begin{prop}[Giving Theories the Witness Property]
    \link{make_wit}
    Suppose $\Si(0)$-theory 
    $T(0)$ is finitely consistent.
    Then there exists a signature 
    $\Si(*)$ and $\Si(*)$-theory $T(*)$ such that 
    \begin{enumerate}
        \item $\const{\Si(0)} \subs \const{\Si(*)}$
        and they share the same function and relation symbols.
        \item $|\const{\Si(*)}| = |\const{\Si(0)}| + \aleph_0$
        \item $T(0) \subs T(*)$
        \item $T(*)$ is finitely consistent
        \item Any $\Si(*)$-theory $T'$ such that $T(*) \subs T'$ 
        has the witness property
    \end{enumerate}
\end{prop}
\begin{proof}
    Again without loss of generality $\Si(0)$ is non-empty.
    We want to define $\Si(i),T(i)$, 
    for each $i \in \N$.    
    By induction, 
    we assume we have $\Si(i)$ non-empty a signature and 
    $T(i)$ a $\Si$-theory such that 
    \begin{enumerate}
        \item $\const{\Si(0)} \subs \const{\Si(i)}$
        and they share the same function and relation symbols.
        \item $|\const{\Si(i)}| = |\const{\Si(0)}| + \aleph_0$
        \item $T(0) \subs T(i)$
        \item $T(i)$ is finitely consistent
    \end{enumerate}
    Let 
    \[
        W(i) := \set{\phi \in \form{\Si(i)} 
        \st \phi \text{ has exactly one free variable}}
    \]
    We construct $\Si(i+1)$ 
    by adding constant symbols $c_\phi$ for each $\phi \in W(i)$ and 
    keeping the same function and relation symbols of $\Si(i)$:
    \[
        \const{\Si(i+1)} :=
        \const{\Si(i)} \sqcup \set{c_\phi
        \st \phi \in W(i)}
    \]
    We create a witness formula $w(\phi)$ for each formula $\phi \in W$:
    \begin{align*}
        w : W(i) &\to \form{\Si(i+1)}\\
        \phi &\mapsto ((\exists v, \phi(v)) \to \phi(c_\phi))
    \end{align*}
    Then let \[T(i+1) := T(i) \cup w(W(i))\]
    Certainly
    $T(i+1)$ is a $\Si(i+1)$-theory such that $T(0) \subs T(i+1)$,
    $\const{\Si(0)} \subs \const{\Si(i+1)}$ 
    where the function and relation symbols are unchanged.
    Since $W(i)$ is countibly infinite, 
    $|\const{\Si(i+1)}| = |\const{\Si(i)}| + \aleph_0 = 
    |\const{\Si(0)}| + \aleph_0 $.
    We need to check that $T(i+1)$ is finitely consistent.
    Take a finite subset of $T(i+1)$.
    It is a union of two finite sets 
    $\De_T \subs T(i)$ and $\De_w \subs w(W(i))$.
    Since $T(i)$ is finitely consistent there exists a $\Si(i)$-model
    $\MM(i)$ of $\De_T$.
    Let $\MM(i+1)$ be defined to have carrier set ${\MM(i)}$.
    There exists some $b \in \MM(i) = \MM(i+1)$ since $\Si(i)$ is non-empty.
    To define interpretation for $\MM(i+1)$, 
    let $c \in \const{\Si(i+1)}$:
    \[
        \subintp{\Si(i+1)}{\MM(i+1)}{c} := 
        \begin{cases}
            \subintp{\Si(i)}{\MM(i)}{c} \quad &\text{when }  
            c \in \const{\Si(i)} \\
            a \quad &\text{when } c = c_\phi \text{ and } 
            \exists a \in {\MM(i)}, \MM(i) \model{\Si(i)} \phi(a)\\
            b \quad &\text{when } c = c_\phi \text{ and } 
            \forall a \in {\MM(i)}, \MM(i) \nodel{\Si(i)} \phi(a)
        \end{cases}
        \]
    Then $\MM(i+1)$ is a well defined $\Si(i+1)$-structure.
    We check is is a $\Si(i+1)$-model of $\De_T \cup \De_w$.
    Since 
    $\intp{\Si(i+1)}{\MM(i+1)}$ agrees with 
    $\intp{\Si(i)}{\MM(i)}$ for constants, 
    functions and relations from $\MM(i)$ - a $\Si(i)$-model
    of $\De_T$ -
    \linkto{move_up_mod}{it is a $\Si(*)$-model of $\De_T$}.
    If $\psi \in \De_w$ then it is 
    $\exists v, \phi(v) \to \phi(c_\phi)$ for some 
    $\phi \in W(i)$.
    Supposing that $\MM(i+1) \model{\Si(i+1)} \exists v, \phi(v)$
    it suffices to show $\MM(i+1) \model{\Si(i+1)} \phi(c_\phi)$.
    Then there exists $a \in \MM(i + 1) = \MM(i)$ such that 
    $\MM(i+1) \model{\Si(i+1)} \phi(a)$.
    \linkto{move_down_mod}{Hence} $\MM(i) \model{\Si(i)} \phi(a)$
    and so $c_\phi$ is interpreted as $a$ in $\MM(i+1)$.
    Hence $\MM(i + 1) \model{\Si(i)} \phi(c_\phi)$.
    Thus the induction is complete.

    Let $\Si(*)$ 
    be the signature such that its function and relations 
    are the same as $\Si(0)$
    and $\const{\Si(*)} = \bigcup_{i \in \N} \const{\Si(i)}$.
    Then 
    \[
        |\const{\Si(*)}| = |\bigcup_{i \in \N} \const{\Si(i)}| = 
        \aleph_0 \times (\aleph_0 + \const{\Si(0)}) = 
        \aleph_0 + \const{\Si(0)}
    \]

    Let $T(*) = \bigcup_{i \in \N} T(i)$.
    Any finite subset of $T(*)$ is a subset of some $T(i)$, 
    hence has a non-empty $\Si(i)$-model $\MM$.
    Checking the relevant conditions for
    \linkto{move_up_mod}{moving models up signatures},
    we have
    $\MM(*)$ a $\Si(*)$-model of the finite subset
    (by interpreting the new constant symbols as the 
    element of the non-empty carrier set.).
    Hence $T(*)$ is finitely consistent.
    
    If $T'$ is a $\Si(*)$-theory such that $T(*) \subs T'$,
    and $\phi$ is a $\Si(*)$-formula of exactly one variable.
    There exists an $i \in \N$ such that $\phi \in \form{\Si(i)}$.
    Since $c_\phi \in \Si(i+1)$ satisfies 
    $T(i) \model{\Si(i+1)} \brkt{\exists v, \phi(v)} \to \phi(c_\phi)$,
    by \linkto{move_up_mod}{moving the logical consequence up to $\Si(*)$,}
    we have $T(i) \model{\Si(*)} \brkt{\exists v, \phi(v)} \to \phi(c_\phi)$.
    If $\NN$ is a $\Si(*)$-model of $T'$
    then it is a $\Si(*)$-model of $T(i)$,
    then $\NN \model{\Si(*)} \brkt{\exists v, \phi(v)} \to \phi(c_\phi)$.
    Hence $T' \model{\Si(*)} \brkt{\exists v, \phi(v)} \to \phi(c_\phi)$,
    satisfying the witness property.
\end{proof}

\begin{lem}[Adding Formulas to Consistent Theories]
    \link{adding_formulas}
    If $T$ is a finitely consistent $\Si$-theory 
    and $\phi$ is a $\Si$-sentence then at least one of
    $T \cup \set{\phi}$ or $T \cup \set{\NOT \phi}$ is finitely consistent.
\end{lem}
\begin{proof}
    We show that for any finite $\De \subs T \cup \set{\phi}$ and 
    for any finite $\De_\NOT \subs T \cup \set{\NOT \phi}$,
    one of $\De$ or $\De_\NOT$ is consistent. 
    The finite subset
    \[(\De \setminus \set{\phi}) \cup 
    (\De_\NOT \setminus \set{\NOT \phi}) \subs T\]
    is consistent by finite consistency of $T$.
    Let $\MM$ be the model of 
    $(\De \setminus \set{\phi}) \cup 
    (\De_\NOT \setminus \set{\NOT \phi})$.
    Case on whether $\MM \model{\Si} \phi$ or not.
    In the first case $\MM \model{\Si} \De$
    and in the second $\MM \model{\Si} \De_\NOT$.
    Hence $T \cup \set{\phi}$ or $T \cup \set{\NOT \phi}$ 
    is finitely consistent.
\end{proof}

\begin{ex}
    Find a signature $\Si$, a consistent $\Si$-theory $T$
    and $\Si$-sentence $\phi$ such that  
    $T \cup \set{\phi}$ and 
    $T \cup \set{\NOT \phi}$ are both consistent.
\end{ex}
    
\begin{prop}[Extending a finitely consistent theory to a maximal theory (Zorn)]
    \link{make_max}
    Given a finitely consistent $\Si$-theory $T(0)$
    there exists a $\Si$-theory $T(*)$ such that 
    \begin{enumerate}
        \item $T(0) \subs T(*)$
        \item $T(*)$ is finitely consistent.
        \item $T(*)$ is $\Si$-maximal.
    \end{enumerate}
\end{prop}
\begin{proof}
    We use Zorn's Lemma.
    Let 
    \[Z := \set{T \in \theory{\Si} \st T 
    \text{ finitely consistent and } T(0) \subs T}\]
    be ordered by inclusion.
    Let $T(0) \subs T(1) \subs \cdots $ be a chain.
    Then $\bigcup_{i \in \N} T(i)$ is a $\Si$-theory
    such that any finite subset is a subset of some $T(i)$,
    hence is consistent by finite consistency of $T(i)$.
    Zorn's lemma implies there exists a $T(*) \in Z$
    that is maximal (in the order theory sense).
    Since $T(*)$ is in $Z$, 
    we have that it is a finitely consistent $\Si$-theory containing $T(0)$.
    
    To show that it is $\Si$-maximal we take a $\Si$-sentence $\phi$.
    By \linkto{adding_formulas}{the previous result,} 
    $T(*) \cup \set{\phi}$ or 
    $T(*) \cup \set{\NOT \phi}$ is finitely consistent.
    Hence 
    $T(*) \cup \set{\phi} = T(*)$ or 
    $T(*) \cup \set{\NOT \phi} = T(*)$ by (order theoretic) maximality,
    so $\phi \in T(*)$ or $\NOT \phi \in T(*)$.
\end{proof}

\begin{nttn}[Cardinalities of signatures and structures]
    Given a signature $\Si$, 
    we write
    $|\Si| := |\const{\Si}| + |\func{\Si}| + |\rel{\Si}|$
    and call this the cardinality of the signature $\Si$.
\end{nttn}

\begin{prop}[The compactness theorem]
    \link{compactness}
    If $T$ is a $\Si$-theory, 
    then the following are equivalent:
    \begin{enumerate}
        \item $T$ is finitely consistent.
        \item $T$ is consistent
        \item For any infinite cardinal $\ka$ 
        such that $|\Si| \leq \ka$, 
        there exists a non-empty $\Si$-model of 
        $T$ with cardinality $\leq \ka$.
    \end{enumerate}
\end{prop}
\begin{proof}
    3. implies 2. and 2. implies 1. are both obvious. 
    A proof of 1. implies 3. follows.

    Suppose an
    $\Si(0)$-theory $T(0)$ is finitely consistent.
    Let $\ka$ be an infinite cardinal such that 
    $|\Si(0)| \leq \ka$. Then
    $|\const{\Si(0)}| \leq |\Si(0)| \leq \ka$.
    We wish to find a $\Si$-model of $T$ with cardinality $\leq \ka$.
    We have shown that \linkto{make_wit}{there exists a signature 
    $\Si(1)$ and $\Si(1)$-theory $T(1)$} such that 
    \begin{enumerate}
        \item $\const{\Si(0)} \subs \const{\Si(1)}$
        and they share the same function and relation symbols.
        \item $|\const{\Si(1)}| = |\const{\Si(0)}| + \aleph_0$
        \item $T(0) \subs T(1)$
        \item $T(1)$ is finitely consistent.
        \item Any $\Si(1)$-theory $T$ such that $T(1) \subs T$ 
        has the witness property.
    \end{enumerate}
    $T(1)$ is finitely consistent so
    \linkto{make_max}{there exists a $\Si(1)$-theory $T(2)$} such that 
    \begin{enumerate}[resume]
        \item $T(1) \subs T(2)$
        \item $T(2)$ is finitely consistent.
        \item $T(2)$ is $\Si(1)$-maximal.
    \end{enumerate}
    Furthermore, $T(2)$ has the witness property due to point 5.
    Since $T(2)$ has the witness property, is $\Si(1)$-maximal
    and finitely consistent,
    $T(2)$ has a non-empty $\Si(1)$-model $\MM$ such that $|\MM| \leq \ka$
    \linkto{henkin}{by Henkin Construction}.
    $\MM \model{\Si(1)} T(0)$ since $T(0) \subs T(1) \subs T(2)$.
    We can \linkto{move_down_mod}{move $\MM$ down to $\Si(0)$,}
    obtaining $\MM \model{\Si(0)} T(0)$.
\end{proof}