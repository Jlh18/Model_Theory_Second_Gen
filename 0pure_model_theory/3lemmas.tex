\subsection{The Category of Structures}
\begin{dfn}[$\Si$-morphism, $\Si$-embedding, $\Si$-isomorphism]
    \link{partial_morph_dfn}
    Given $\Si$ a signature, 
    $\MM, \NN$ both $\Si$-structures, $A \subs \carrier{\MM}$
    and $\io : A \to \carrier{\NN}$,
    we call $\io$ a partial $\Si$-morphism from $\MM$ to $\NN$ when 
    \begin{itemize}
        \item For all $c \in C$ (such that $\mmintp{c} \in A$), 
        \[\io(\mmintp{c}) = \modintp{\NN}{c}\]
        \item For all $f \in F$ and all $a \in M^{n_f}$
        (such that $\mmintp{f}(a) \in A$), 
        \[\io \circ \mmintp{f}(a) = \modintp{\NN}{f} \circ \io(a)\]
        \item For all $r \in R$, for all $a \in M^{m_r} \cap A^{m_r}$,
        \[a \in \mmintp{r} \implies \io (a) \in \modintp{\NN}{r}\]
    \end{itemize}
    If in addition for relations we have
    \[a \in \mmintp{r} \limplies \io (a) \in \modintp{\NN}{r} \quad 
    \text{and} \quad \io \text { is injective,} \]
    then $\io$ is called a partial $\Si$-embedding 
    (the word extension is often used interchangably with embedding). 

    In the case that $A = \carrier{\MM}$
    we write $\io : \MM \to \NN$ and call $\io$ a $\Si$-morphism.
\end{dfn}
The notion of morphisms here will be the same as 
morphisms in the algebraic setting.
For example in the signature of monoids (groups), 
preserving interpretation of constant symbols 
says the identity is sent to the identity 
and preserving interpretation of function symbols
says the multiplication is preserved.

\begin{dfn}[Elementary Embedding]
    A partial $\Si$-embedding $\io : A \to \NN$ (for $A \subs \MM$)
    is elementary if for any $\Si$-formula $\phi$
    with variables indexed by $S$
    and $a \in A^S$,
    \[
        \MM \model{\Si} \phi(a) \quad \iff \quad \NN \model{\Si} \phi(\io(a))
    \]
\end{dfn}

The following is exactly what we expect - that terms are well behaved 
with respect to morphisms.
\begin{lem}[$\Si$-morphisms commute with interpretation of terms]
    \link{morph_comm_term_intp}
    Given a $\Si$-morphism $\io : \MM \to \NN$, 
    we have that for any $\Si$-term $t$ with variables indexed by $S$ and 
    $a \in \MM^S$,
    \[\io(\mmintp{t}(a)) = \nnintp{t}(\io(a))\]
\end{lem}
\begin{proof}
    We case on what $t$ is:
    \begin{itemize}
        \item If $t = c \in \const{\Si}$ then 
            \[
                \io(\mmintp{t}(a)) = \io(\mmintp{c}) = 
                \nnintp{c} = \nnintp{t}(\io(a))
            \]
        \item If $t = v \in \var{\Si}$ then 
            \[
                \io(\mmintp{t}(a)) =
                \io(\mmintp{v}(a)) = \io(a) = 
                \nnintp{v}(\io(a)) = \nnintp{t}(\io(a))
            \]
        \item If $t = f(s)$ then
            \[
                \io(\mmintp{t}(a)) = \io \circ \mmintp{f}(\mmintp{s}(a)) = 
                \nnintp{f} \circ \io(\mmintp{s}(a)) = 
                \nnintp{f} \circ \nnintp{s}(\io(a)) =
                \nnintp{t}(\io(a))
            \]
            Using the induciton hypothesis in the penultimate step.
    \end{itemize}
\end{proof}

It is worth knowing that the set of $\Si$-structures form a category:
\begin{dfn}[The category of $\Si$-structures]
    \link{category_of_structures}
    Given a signature $\Si$, 
    we have $\Mod{\Si}$ - a category where objects are $\Si$-structures 
    and morphisms are $\Si$-morphisms.
    
    Clearly for any $\MM$, 
    the identity exists and is a $\Si$-morphism.
    We show that composition of morphisms are morphisms.
    Furthermore, composition of embeddings are embeddings
    and composition of elementary embeddings are elementary.
    Thus we could also define morphisms between objects to be embeddings
    or elementary embeddings and obtain a subcategory.

    Hence we inherit a notion of isomorphism of $\Si$-structures
    from category theory.
\end{dfn}
\begin{proof}
    Let $\io_1 : \MM_0 \to \MM_1$ and 
    $\io_2 : \MM_1 \to \MM_2$ be $\Si$-morphisms.
    We show that the composition is a $\Si$-morphism:
    \begin{itemize}
        \item If $c \in \const{\Si}$ then 
            \[
                \io_1 \circ \io_0 (\modintp{\MM_0}{c}) = 
                \io_1 (\modintp{\MM_1}{c}) = \modintp{\MM_2}{c}
            \]
        \item If $f \in \func{\Si}$ and 
            $a \in {\MM_0}^{n_f}$ then
            \[
                \io_1 \circ \io_0 \circ \modintp{\MM_0}{f}(a) = 
                \io_1 \circ \modintp{\MM_1}{f} \circ \io_0 (a) = 
                \modintp{\MM_2}{f} \circ \io_1 \circ \io_0 (a)
            \]
        \item If $r \in \rel{\Si}$ and 
            $a \in {\MM_0}^{m_f}$ then
            \[
                a \in \modintp{\MM_0}{r} \implies \io_1 (a) \in \modintp{\MM_1}{r} 
                \implies \io_2 \circ \io_1 (a) \in \modintp{\MM_2}{r}
            \]
    \end{itemize}
    To show that embeddings compose to be embeddings we note that 
    the composition of injective functions is injective 
    and if $r \in \rel{\Si}$ and 
    $a \in {\MM_0}^{m_f}$ then
    \[
        a \in \modintp{\MM_0}{r} \iff \io_1 (a) \in \modintp{\MM_1}{r} 
        \iff \io_2 \circ \io_1 (a) \in \modintp{\MM_2}{r}
    \]
    To show that composition of elementary embeddings are elementary,
    let $\phi \in \form{\Si}$ and $a$ in ${\MM_0}$ be chosen suitably.
    Then
    \[
        \MM_0 \modelsi \phi(a) \iff \MM_1 \modelsi \phi(\io_1 (a))
        \iff \MM_2 \modelsi \phi(\io_2 \circ \io_1 (a))
    \]
\end{proof}

\begin{eg}
    Given the category of structures from the 
    \linkto{dfn_rings}{signature of rings}, 
    we can take the subcategory whose objects are models 
    of the theory of rings (namely rings), 
    hence producing the category of rings.
    Similarly taking the subcategory whose objects are models of the 
    theory of fields (namely fields) produces the category of fields.
\end{eg}

\begin{prop}[Embeddings Preserve Satisfaction of Quantifier Free Formulas]
    \link{emb_preserve_sat_of_quan_free}
    Given $\io : \MM \to \NN$ a $\Si$-embedding and
    $\phi$ a $\Si$-formula with variables indexed by $S$,
    and $a \in \MM^S$,
    \begin{enumerate}
        \item If $\phi$ is $\top$ then it is satisfied by both.
        \item If $\phi$ is $t = s$ then 
            $\MM \model{\Si} \phi(a)  \iff \NN \model{\Si} \phi(\io(a))$.
        \item If $\phi$ is $r(s)$ then 
            $\MM \model{\Si} \phi(a)  \iff \NN \model{\Si} \phi(\io(a))$.
        \item If $\phi$ is $\NOT \chi$ and 
            $\MM \model{\Si} \chi(a)  \iff \NN \model{\Si} \chi(\io(a))$
            then \[\MM \model{\Si} \phi(a)  \iff \NN \model{\Si} \phi(\io(a))\]
        \item If $\phi$ is $\chi_0 \OR \chi_1$ and 
            $\MM \model{\Si} \chi_i (a)  \iff \NN \model{\Si} \chi_i (\io(a))$
            then \[\MM \model{\Si} \phi(a)  \iff \NN \model{\Si} \phi(\io(a))\]
    \end{enumerate}
    Thus from the above we can immediately conclude by induction
    that if $\phi$ is a quantifier free $\Si$-formula, 
    \[\MM \model{\Si} \phi(a)  \iff \NN \model{\Si} \phi(\io(a))\]
    Note that our original result is stronger than this since we didn't
    assume the formula to be quantifier free.
\end{prop}
\begin{proof}~
    \begin{enumerate}
        \item Trivial.
        \item If $\phi$ is $t = s$ then
            \begin{align*}
                \MM \model{\Si} \phi(a) 
                & \iff \mmintp{t}(a) = \mmintp{s}(a)\\
                & \iff \io(\mmintp{t}(a)) = \io(\mmintp{s}(a)) 
                & \text{by injectivity} \\
                & \iff \nnintp{t}(\io(a)) = \nnintp{s}(\io(a)) 
                & \text{\linkto{morph_comm_term_intp}{
                    morphisms commute with}} \\
                & \iff \NN \model{\Si} \phi(\io(a))
                & \text{\linkto{morph_comm_term_intp}{interpretation of terms}}
            \end{align*}
        \item If $\phi$ is $r(s)$ then
            \begin{align*}
                \MM \model{\Si} \phi(a) 
                & \iff a \in \mmintp{r}& \\
                & \iff \io(a) \in \nnintp{r}
                & \text{embeddings property}\\
                & \iff \NN \model{\Si} \phi(\io(a))
            \end{align*}
        \item If $\phi$ is $\NOT \chi$ and 
            $\MM \model{\Si} \chi(a)  \iff \NN \model{\Si} \chi(\io(a))$ then
            \[
                \MM \model{\Si} \phi(a) 
                \iff \MM \nodel{\Si} \chi(a) 
                \iff \NN \nodel{\Si} \chi(\io(a))
                \iff \NN \model{\Si} \phi(\io(a))
            \]
        \item If $\phi$ is $\chi_0 \OR \chi_1$ and
            $\MM \model{\Si} \chi_i (a)  \iff \NN \model{\Si} \chi_i (\io(a))$
            \[
                \MM \model{\Si} \phi(a)
                \iff \MM \model{\Si} \chi_0(a) 
                \text{ or } \MM \model{\Si} \chi_1(a)
                \iff \NN \model{\Si} \chi_0(\io(a)) 
                \text{ or } \NN \model{\Si} \chi_1(\io(a))
                \iff \NN \model{\Si} \phi(\io(a))
            \]
    \end{enumerate}
\end{proof}

\begin{dfn}[Universal Formula, Universal Sentence]
    \link{universal_formula_def}
    A $\Si$-formula is universal if it can be 
    built inductively by the following two constructors:
    \begin{itemize}
        \item[$\vert$] If $\phi$ is a quantifier free $\Si$-formula 
        then it is a universal $\Si$-formula.
        \item[$\vert$] If $\phi$ is a universal $\Si$-formula then
        $\forall v, \phi(v)$ is a universal $\Si$-formula.
    \end{itemize}
    In other words universal $\Si$-formulas are formulas that start with 
    a bunch of `for alls' followed by a quantifier free formula.
\end{dfn}

\begin{prop}[Embeddings preserve satisfaction of universal formulas downwards]
    \link{emb_preserve_sat_of_forall_down}
    Given $\io : \MM \to \NN$ a $\Si$-embedding and
    $\phi$ a universal $\Si$-formula with 
    variables indexed by $S$ ($\chi$ is quantifier free).
    For any $a \in \MM^S$
    \[
        \NN \modelsi \phi(\io(a)) \quad \implies \quad
        \MM \model{\Si} \phi(a)
    \]
    By taking the contrapositive we can show that embeddings
    preserve satisfaction of `existential' $\Si$-formulas upwards.
\end{prop}
\begin{proof}
    We induct on $\phi$:
    \begin{itemize}
        \item If $\phi$ is a quantifier free then since 
        \linkto{emb_preserve_sat_of_quan_free}{embeddings preserve 
        satisfaction of quantifier free formulas}, 
        $\NN \modelsi \phi(\io(a)) \implies
        \MM \model{\Si} \phi(a)$.
        \item If $\phi$ is $\forall v_i, \psi$ with $S$ indexing the variables
        of $\psi$. 
        Let $T := S \setminus \set{i}$
        Assuming the inductive hypothesis: 
        for any $a \in {\MM}^T$ and 
        $b \in {\MM}$,
        \[\NN \modelsi \psi(\io(a),\io(b)) \quad \implies \quad
        \MM \model{\Si} \psi(a,b)\]
        Then for any $a \in {\MM}^T$
        \begin{align*}
            &\quad \NN \modelsi \phi(\io(a))\\
            &\implies \forall b \in {\MM}, 
            \NN \modelsi \psi(\io(a),\io(b)) \\
            &\implies \forall b \in {\MM}, 
            \MM \modelsi \psi(\io(a),\io(b)) & \text{by the induction}\\
            &\implies \MM \model{\Si} \phi(a)
        \end{align*}
    \end{itemize}
\end{proof}

\begin{prop}[Isomorphisms are Elementary]
    \link{iso_imp_elem_equiv}
    If two $\Si$-structures $\MM$ and $\NN$ 
    are $\Si$-isomorphic then the isomorphism is elementary.
\end{prop}
\begin{proof}
    Let $\io : \MM \to \NN$ be a $\Si$-isomorphism.
    We case on what $\phi$ is:
    \begin{itemize}
        \item If $\phi$ is quantifier free, 
            then each case is follows from applying
            \linkto{emb_preserve_sat_of_quan_free}{
                embeddings preserve satisfaction of quantifier free formulas.}
        \item If $\phi$ is $\forall v, \chi(v)$ then
            \begin{forward}
                Let $b \in \const{\NN}$ then
                $\io^{-1}(b) \in \const{\MM}$ is well defined by surjectivity.
                Hence $\MM \modelsi \chi(\io^{-1}(b),a)$ and so
                $\NN \modelsi \chi(b,\iota(a))$ by the induction hypothesis.
                Hence $\NN \model{\Si} \phi(\io(a))$.
            \end{forward}
            \begin{backward}
                The same.
            \end{backward}
    \end{itemize}
\end{proof}

\subsection{Vaught's Completeness Test}
Read ahead to the statement of \linkto{vaught_test}{Vaught's Completeness Test}.

\begin{dfn}[Finitely modelled, infinitely modelled]
    A $\Si$-theory $T$ 
    is finitely modelled when
    there exists a $\Si$-model of $T$ 
    with finite carrier set.
    It is infinitely modelled when
    there exists a $\Si$-model of $T$ 
    with infinite carrier set.
\end{dfn}
Finitely modelled is \emph{not} the same as finitely consistent.

\begin{prop}[Infinitely modelled theories have arbitrary large models]
    \link{inf_mod_theory_has_inf_mod}
    Given $\Si$ a signature, 
    $T$ a $\Si$-theory that is infinitely modelled,
    and a cardinal $\ka$ such that $|\const{\Si}| + \aleph_0 \leq \ka$, 
    there exists $\MM$ a $\Si$-model of $T$ such that 
    $\ka = |{\MM}|$.
\end{prop}
\begin{proof}
    Enrich only the signature's constant symbols to create $\Si(*)$ 
    a signature such that 
    $\const{\Si(*)} = \const{\Si} \cup \set{c_\al \st \al \in \ka}$.
    Let $T(*) = T \cup \set{c_\al \neq c_\be \st \al,\be \in \ka \AND \al \ne \be}$
    be a $\Si(*)$-theory.
    
    Using \linkto{compactness}{the compactness theorem}, 
    it suffices to show that $T(*)$ is finitely consistent.
    Take a finite subset of $T(*)$. 
    This is the union of a finite subset $\De_T \subs T$, 
    and a finite subset of 
    $\De_{\ka} \subs 
    \set{c_\al \neq c_\be \st \al,\be \in \ka \AND \al \ne \be}$.
    Let $\MM$ be the $\Si$-model of $T$ with infinite cardinality.
    We want to make 
    $\MM$ a ${\Si(*)}$-model of $\De_T \cup \De_\ka$ 
    by interpreting the new symbols of $\set{c_\al \st \al \in \ka}$
    in a sensible way.
    
    Since $\De_\ka$ is finite, 
    we can find a finite subset $I \subset \ka$ 
    that indexes the constant symbols appearing in $\De_\ka$. 
    Since $\MM$ is infinite and $I$ is finite,
    we can find distinct elements of $\MM$
    to interpret the elements of
    $\set{c_\al \st \al \in I}$. 
    Interpret the rest of the new constant symbols however,
    for example let them all be sent to the same element,
    then $\MM \model{\Si*} \De_T \cup \De_\ka$.
    Thus $T(*)$ is finitely consistent hence consistent.
    
    Using 
    \linkto{compactness}{the third equivalence of $T(*)$ being consistent},
    there exists
    $\MM$ a $\Si(*)$-model of $T(*)$ with $|{\MM}|\leq \ka$.
    If $|{\MM}| < \ka$ 
    then there would be $c_\al, c_\be$ that are interpreted as equal,
    hence $\MM \model{\Si(*)} c_\al = c_\be$ and $\MM \nodel{\Si(*)} c_\al = c_\be$, 
    a contradiction.
    Thus $|{\MM}| = \ka$.
    \linkto{move_down_mod}{Move $\MM$ down a signature}
    to make it a $\Si$-model of $T$.
    This doesn't change the cardinality of $\MM$,
    so we have a $\Si$-model of $T$ with cardinality $\ka$.
\end{proof}

\begin{dfn}[Elementary equivalence]
    Let $\MM$, $\NN$ be $\Si$-structures.
    They are elementarily equivalent if for any $\Si$-sentence $\phi$,
    $\MM \model{\Si} \phi$ if and only if $\NN \model{\Si} \phi$.
    We write $\MM \equiv_\Si \NN$.
\end{dfn}

\begin{lem}[Not a consequence is consistent]
    \link{not_consequence}
    Let $T$ be a $\Si$-theory
    and $\phi$ is a $\Si$-sentence
    then $T \nodel{\Si} \phi$
    if and only if $T \cup \set{ \NOT \phi}$ is consistent.
    Furthermore, $T \nodel{\Si} \NOT \phi$
    if and only if $T \cup \set{\phi}$ is consistent.
\end{lem}
\begin{proof}
    For the first statement:
    \begin{forward}
        Unfolding $T \nodel{\Si} \phi$,
        we have that there exists a $\Si$-model $\MM$ of $T$ 
        such that $\MM \nodel{\Si} \phi$.
        Hence $\MM \modelsi \NOT \phi$ and we are done.
    \end{forward}
    The backward proof is straightward.

    For the second statement, 
    apply the first to $\NOT \phi$ and obtain 
    $T \nodel{\Si} \NOT \phi$
    if and only if $T \cup \set{\NOT \NOT \phi}$ is consistent.
    Note that for any $\Si$-structure $\MM$, 
    $\MM \model{\Si} \NOT \NOT \phi$ 
    if and only if $\MM \model{\Si} \phi$.
    This completes the proof.
\end{proof}

\begin{prop}[Vaught's Completeness Test]
    \link{vaught_test}
    Suppose that $\Si$-theory $T$ is consistent, 
    not finitely modelled, and $\ka$-categorical 
    for some cardinal satisfying 
    $|\const{\Si}| + \aleph_0 \leq \ka$.
    Then $T$ is maximal (i.e. complete).
\end{prop}
\begin{proof}
    Suppose not: If $T$ is not maximal then there exists
    $\Si$-formula $\phi$ such that $T \nodel{\Si} \phi$ and 
    $T \nodel{\Si} \NOT \phi$.
    These imply $T \cup \set{\NOT \phi}$
    and $T \cup \set{\phi}$ are both
    \linkto{not_consequence}{consistent}.
    Let $\MM_\NOT$ and $\MM$ be models of  
    $T \cup \set{\NOT \phi}$ and $T \cup \set{\phi}$
    respectively.
    Then each are models of $T$ so they are infinite
    and so $T \cup \set{\NOT \phi}$ and $T \cup \set{\phi}$
    are infinitely modelled.

    Since we have $\ka$ such that $|\const{\Si}| + \aleph_0 \leq \ka$, 
    \linkto{inf_mod_theory_has_inf_mod}{
        there exists $\NN_\NOT, \NN$} respectively $\Si$-models of 
        $T \cup \set{\NOT \phi}$ and $T \cup \set{\phi}$
        such that 
        $\ka = |{\NN_\NOT}| = |{\NN}|$.
    Since $T$ is $\ka$-categorical
    $\NN$ and $\NN_\NOT$ are isomorphic
    \linkto{iso_imp_elem_equiv}{
        by an elementary $\Si$-embedding.}
    As $\phi$ has no free variables this implies that
    $\NN \model{\Si} \phi$ and $\NN \model{\Si} \NOT \phi$, 
    a contradiction.
\end{proof}

\subsection{Elementary embeddings and diagrams of models}
\begin{prop}[Tarski-Vaught Elementary Embedding Test]
    \link{tarski_vaught}
    Let $\io : \MM \to \NN$ be a $\Si$-embedding, 
    then the following are equivalent:
    \begin{enumerate}
        \item $\io$ is elementary 
        \item For any 
            $\phi \in \form{\Si}$ with free variables indexed by $S$,
            any $i \in S$ 
            and any $a \in ({\MM})^{S \setminus \set{i}}$,
            \[\forall b \in {\MM}, \NN \model{\Si} \phi(\io(a),\io(b)) 
            \quad \implies \quad 
            \forall c \in {\NN}, \NN \model{\Si} \phi(\io(a),c),\]
            which we call the Tarski-Vaught condition.
        \item For any 
            $\phi \in \form{\Si}$ with free variables indexed by $S$,
            any $i \in S$ 
            and any $a \in ({\MM})^{S \setminus \set{i}}$,
            \[\exists c \in {\NN}, \NN \model{\Si} \phi(\io(a),c)
            \quad \implies \quad
            \exists b \in {\MM}, \NN \model{\Si} \phi(\io(a),\io(b))\]
            This is essentially the contrapositive of the previous statement,
            and is included because it is more commonly 
            version of the statement.
    \end{enumerate}
\end{prop}
\begin{proof}
    We only show the first two statements are equivalent and
    leave the third as an exercise.
    \begin{forward}
        First show that $\MM \model{\Si} \forall v, \phi(a,v)$.
        Let $b \in {\MM}$, 
        then by assumption $\NN \model{\Si} \phi(\io(a),\io(b))$,
        which is implies $\MM \model{\Si} \phi(a,b)$
        as $\io$ is an elementary embedding.
        Thus we indeed have $\MM \model{\Si} \forall v, \phi(a,v)$
        which in turn implies $\NN \model{\Si} \forall v, \phi(\io(a),v)$
        and we are done.
    \end{forward}

    \begin{backward}
        We case on what $\phi$ is, 
        though most of the work was already done before.
        \begin{itemize}
            \item If $\phi$ is quantifier free, 
            then each case follows from applying
            \linkto{emb_preserve_sat_of_quan_free}{
                embeddings preserve satisfaction of quantifier free formulas.}

            \item The backwards implication follows from applying 
                \linkto{emb_preserve_sat_of_forall_down}{
                embeddings preserve satisfaction of universal formulas 
                downwards.}

                For the forwards implication 
                we use the Tarski-Vaught condition
                (so far $\io$ just needed to be a $\Si$-embedding)
\begin{align*}
    \MM \model{\Si} \forall v, \psi (a,v) &
        \implies \forall b \in {\MM}, \MM \model{\Si} \psi(a,b)\\
        &\implies \forall b \in {\MM}, 
            \NN \model{\Si} \psi(\io(a),\io(b))
            & \text{ by the induction hypothesis}\\
        &\implies \forall c \in {\NN}, \NN \model{\Si} \psi(\io(a),c)
            & \text{ by the Tarski-Vaught condition}\\
        &\implies \NN \model{\Si} \phi
\end{align*}
        \end{itemize}
    \end{backward}
\end{proof}

Again we have a technical detail which is not really worth spending too much
time one. 
It is a sensible justification for arguments based on creating new signatures.
\begin{lem}[Moving Morphisms Down Signatures]
    \link{move_down_morph}
    Suppose $\Si \leq \Si(*)$.
    If $\io : \MM \to \NN$ is a $\Si(*)$-morphism then 
    \begin{enumerate}
        \item $\io$ can be made into a $\Si$-morphism.
        \item If $\io$ is an embedding then it remains an embedding.
        \item If $\io$ is an elementary embedding then it remains elementary. 
    \end{enumerate}
\end{lem}
\begin{proof}
    \begin{enumerate}
        \item \linkto{move_down_mod}{
                Move $\MM$ and $\NN$ down to being $\Si$ structures
                (by picking $T(*) = T = \nothing$).}
            We show that the same set morphism 
            $\io : {\MM} \to {\NN}$
            is a $\Si$-morphism.
            \begin{itemize}
                \item If $c \in \const{\Si}$ then since moving structures down
                    signatures preserves interpretation on the lower signature,
                    and since $\io$ is a $\Si(*)$ embedding,
                    \[
                        \io(\subintp{\Si}{\MM}{c}) 
                        = \io(\subintp{\Si(*)}{\MM}{c})
                        = \subintp{\Si(*)}{\NN}{c}
                        = \subintp{\Si}{\NN}{c}
                    \]
                \item If $f \in \func{\Si}$ and $a \in ({\MM})^{n_f}$ 
                    then similarly
                    \[
                        \io \circ \subintp{\Si}{\MM}{f}(a)
                        = \io \circ \subintp{\Si(*)}{\MM}{f}(a)
                        = \subintp{\Si(*)}{\NN}{f}(\io(a))
                        = \subintp{\Si}{\NN}{f}(\io(a))
                    \]
                \item If $r \in \rel{\Si}$ and $a \in ({\MM})^{m_r}$
                    then 
                    \[
                        a \in \subintp{\Si}{\MM}{r}(a)
                        = \subintp{\Si(*)}{\MM}{r}
                        \implies \io(a)  \in \subintp{\Si(*)}{\NN}{r}
                        = \subintp{\Si}{\NN}{r}
                    \]
            \end{itemize}
        \item If we also have that it is an embedding in $\Si(*)$, 
            then injectivity is preserved as it is a property of set morphisms. 
            Given $r \in \rel{\Si}$ and $a \in ({\MM})^{m_r}$,
            \[
                \io(a)  \in \subintp{\Si}{\NN}{r}
                = \subintp{\Si(*)}{\NN}{r}
                \implies 
                a \in \subintp{\Si(*)}{\MM}{r}(a)
                = \subintp{\Si}{\MM}{r}
            \]
        \item If we also have that $\io$ is elementary in $\Si(*)$ 
            then we use the \linkto{tarski_vaught}{Tarski-Vaught Test}:
            let $\phi \in \form{\Si}$ have free variables indexed by $S$,
            let $i \in S$ 
            and let $a \in ({\MM})^{S \setminus \set{i}}$.
            Then due to the construction in
            \linkto{move_down_mod}{moving $\MM$ and $\NN$ down a signature} 
            we have that for any $b \in {\NN}$, 
            \[\NN \model{\Si} \phi (\io(a),\io(b)) 
            \iff \NN \model{\Si(*)} \phi (\io(a),\io(b))\]
            and similarly for $\MM$.
            Hence 
            \begin{align*}
                &\forall b \in {\NN}, 
                \NN \model{\Si} \phi(\io(a),\io(b)) \\
                \implies &\forall b \in {\NN}, 
                \NN \model{\Si(*)} \phi(\io(a),\io(b)) \\
                \implies &\forall c \in {\MM}, 
                \MM \model{\Si(*)} \phi(a,c) 
                & \text{$\io$ is elementary in $\Si(*)$}\\
                \implies &\forall c \in {\MM}, \MM \model{\Si} \phi(a,c)
            \end{align*}
            Hence $\io$ is elementary in $\Si$.
    \end{enumerate}
\end{proof}

\begin{nttn}
    Let $A$ be a set and $\Si$ be a signature,
    enriching only the constant symbols of $\Si$ we can create a signature 
    $\Si(A)$ such that 
    \[
        \const{\Si(A)} := 
        \const{\Si} \cup \set{c_a \st a \in A}
    \]
\end{nttn}

\begin{dfn}[Diagram and the Elementary Diagram of a Structure]
    Let $\MM$ be a $\Si$-structure,
    we move $\MM$ up to the signature $\Si(\MM)$
    by interpreting each new constant symbol $c_a$ as $a$.
    ($\MM$ satisfies the conditions of our lemma for
    \linkto{move_up_sig}{moving models up signatures} 
    by choosing $T = \nothing$).
    Thus we may treat $\MM$ as a $\Si(\MM)$ structure.
    We define the atomic diagram of $\MM$ over $\Si$:
    \begin{itemize}
        \item[$\vert$] If $\phi$ is an atomic
        $\Si(\MM)$-sentence such that $\MM \model{\Si(\MM)} \phi$,
        then $\phi \in \atdiag{\Si}{\MM}$.
        \item[$\vert$] If $\phi \in \atdiag{\Si}{\MM}$ then 
        $\NOT \phi \in \atdiag{\Si}{\MM}$.
    \end{itemize}
    We define the elementary diagram of $\MM$ over $\Si$ as 
    \[
        \eldiag{\Si}{\MM} := 
        \set{\phi \in \Si(\MM)\text{-sentences } \st \MM \model{\Si(\MM)} \phi}
    \]

    The elementary diagram of $\MM$ is a maximal $\Si(\MM)$-theory 
    with $\MM$ as a model of it.
    It is not the same as the set of all $\Si$-sentences satisfied by $\MM$,
    known as the theory of $\MM$ in $\Si$.
\end{dfn}

\begin{prop}[Models of the elementary diagram
        are elementary extensions]
    \link{elem_ext_equiv_eldiag_model}
    Given $\MM$ a $\Si$-structure and 
    $\NN$ a $\Si(\MM)$-structure such that
    $\NN \model{\Si(\MM)} \atdiag{\Si}{\MM}$,
    we can make $\NN$ into a $\Si$-structure and find
    a $\Si$-embedding from $\MM$ to $\NN$.
    Furthermore if
    $\NN \model{\Si(\MM)} \eldiag{\Si}{\MM}$ then 
    the embedding is elementary.

    Conversely, given an elementary $\Si$-embedding from 
    $\MM$ into a $\Si$-structure $\NN$, 
    we can move $\NN$ up to being a $\Si(\MM)$ structure such that 
    $\NN \model{\Si(\MM)} \eldiag{\Si}{\MM}$.
\end{prop}
\begin{proof}
    \begin{forward}
        Suppose $\NN \model{\Si(\MM)} \atdiag{\Si}{\MM}$.
        Firstly we work in $\Si(\MM)$ to define the embedding:
        \linkto{move_up_sig}{move $\MM$ up a signature}
        by taking the same interpretation as used in the 
        definition of $\Si(\MM)$: 
        \[\intp{\Si(\MM)}{\MM} : c_a \mapsto a\]
        and preserving the same interpretation for symbols of $\Si$.
        This makes $\intp{\const{\Si(\MM)}}{\MM}$ surjective.
        Thus we write elements of ${\MM}$ as 
        $\subintp{\Si(\MM)}{\MM}{c}$, 
        for some $c \in \const{\Si(\MM)}$.
        
        Next we define the $\Si(\MM)$-morphism 
        $\io : \MM \to \NN$ such that 
        $\io : \subintp{\Si(\MM)}{\MM}{c} \to \subintp{\Si(\MM)}{\NN}{c}$.
        To check that $\io$ is well defined, 
        take $c,d \in \const{\Si(\MM)}$ such that 
        $\subintp{\Si(\MM)}{\MM}{c} = \subintp{\Si(\MM)}{\MM}{d}$.
        \begin{align*}
            &\implies \MM \model{\Si(\MM)} c = d \\
            &\implies c = d \in \atdiag{\Si}{\MM} \\
            &\implies \NN \model{\Si(\MM)} c = d \\
            &\implies \subintp{\Si(\MM)}{\NN}{c} = \subintp{\Si(\MM)}{\NN}{d}
        \end{align*}
        Thus $\io$ is well defined.
        In fact doing `not' gives us injectivity in the same way:
        Take $c,d \in \const{\Si(\MM)}$ such that 
        $\subintp{\Si(\MM)}{\MM}{c} \ne \subintp{\Si(\MM)}{\MM}{d}$.
        \begin{align*}
            &\implies \MM \model{\Si(\MM)} c \ne d \\
            &\implies c \ne d \in \atdiag{\Si}{\MM} \\
            &\implies \NN \model{\Si(\MM)} c \ne d \\
            &\implies \subintp{\Si(\MM)}{\NN}{c} \ne \subintp{\Si(\MM)}{\NN}{d}
        \end{align*}
        Thus $\io$ is injective.
        To check that $\io$ is a $\Si(\MM)$-morphism, 
        we check interpretation of functions and relations.
        Let $f \in \func{\Si(\MM)} = \func{\Si}$ and 
        $c \in (\const{\Si(\MM)})^{n_f}$.
        $\intp{\const{\Si(\MM)}}{\MM}$ is surjective thus we can find 
        $d \in \const{\Si\brkt{\MM}}$ such that $\MM \model{\Si(\MM)} f(c) = d$.
        Hence $f(c) = d \in \atdiag{\Si}{\MM}$.
        Hence $\NN \model{\Si(\MM)} f(c) = d$.
        \begin{align*}
            \io \circ \subintp{\Si(\MM)}{\MM}{f}(\subintp{\Si(\MM)}{\MM}{c}) 
            &= \io(\subintp{\Si(\MM)}{\MM}{d}) \\
            &= \subintp{\Si(\MM)}{\NN}{d}\\
            &= \subintp{\Si(\MM)}{\NN}{f}(\subintp{\Si(\MM)}{\NN}{c})\\
            &= \subintp{\Si(\MM)}{\NN}{f} \circ \io(\subintp{\Si(\MM)}{\MM}{c})
        \end{align*}
        Let $r \in \rel{\Si(\MM)} = \rel{\Si}$ and 
        $c \in (\const{\Si(\MM)})^{m_r}$.
        \begin{align*}
            \subintp{\Si(\MM)}{\MM}{c}\in \subintp{\Si(\MM)}{\MM}{r} 
            &\implies \MM \model{\Si(\MM)} r(c)\\
            &\implies r(c) \in \atdiag{\Si}{\MM}\\
            &\implies \NN \model{\Si(\MM)} r(c) \\
            &\implies \io(\subintp{\Si(\MM)}{\MM}{c}) = 
            \subintp{\Si(\MM)}{\NN}{c} \in \subintp{\Si(\NN)}{\NN}{r} 
        \end{align*}
        To show that $\io$ is an embedding it remains to show 
        the backward implication for relations.
        Let $r \in \rel{\Si(\MM)} = \rel{\Si}$ and 
        $c \in (\const{\Si(\MM)})^{m_r}$.
        \begin{align*}
            \subintp{\Si(\MM)}{\MM}{c} \notin \subintp{\Si(\MM)}{\MM}{r} 
            &\implies \MM \nodel{\Si(\MM)} r(c)\\
            &\implies \NOT r(c) \in \atdiag{\Si}{\MM}\\
            &\implies \NN \nodel{\Si(\MM)} r(c) \\
            &\implies \io(\subintp{\Si(\MM)}{\MM}{c}) = 
            \subintp{\Si(\MM)}{\NN}{c} \notin \subintp{\Si(\NN)}{\NN}{r} 
        \end{align*}
        Assume furthermore that $\NN \model{\Si(\MM)} \eldiag{\Si}{\MM}$.
        We show that the embedding is elementary.
        Let $\phi$ be a $\Si(\MM)$-formula
        with variables indexed by $S$
        and $a \in ({\MM})^S$.
        Let $c \in (\const{\Si})^S$ be such that 
        $\subintp{\Si(\MM)}{\MM}{c} = a$.
        \begin{align*}
            \MM \model{\Si(\MM)} \phi(a) 
                &\implies \phi(c) \in \eldiag{\Si}{\MM}\\
                &\implies \NN \model{\Si(\MM)} \phi(c)\\
                &\implies \NN \model{\Si(\MM)} \phi(\io(a))
        \end{align*}
        Similarly,
        \begin{align*}
            \MM \nodel{\Si(\MM)} \phi(a) &\implies \NOT \phi(c) 
                \in \eldiag{\Si}{\MM}\\
            &\implies \NN \model{\Si(\MM)} \NOT \phi(c)\\
            &\implies \NN \nodel{\Si(\MM)} \phi(\io(a))
        \end{align*}
        Hence $\io$ is an elementary embedding.
        \linkto{move_down_morph}{Moving $\io : \MM \to \NN$ 
        down} to being a $\Si$-morphism of $\Si$-structures
        completes the proof.
    \end{forward}

    \begin{backward}
        Sketch: Suppose $\io : \MM \to \NN$ is an elementary embedding.
        Make $\MM$ and $\NN$ into $\Si(\MM)$-structures by 
        $\intp{\Si(\MM)}{\MM}: c_a \to a$ and 
        $\intp{\Si(\MM)}{\NN}: c_a \to \io(a)$,
        where $a \in {\MM}$.
        Show that $\io$ is still an elementary embedding
        when moved up to $\Si(\MM)$.
        Then for any $\phi \in \eldiag{\Si}{\MM}$,
        $\MM \model{\Si(\MM)} \phi$ and so by the embedding being elementary
        $\NN \model{\Si(\MM)} \phi$.
        Hence $\NN \model{\Si(\MM)} \eldiag{\Si}{\MM}$.
    \end{backward}
\end{proof}

\subsection{Universal axiomatization}
\begin{dfn}[Axiomatization, universal theory, universal axiomatization]
    A $\Si$-theory $A$ is an axiomatization of a 
    $\Si$-theory $T$ if for all $\Si$-structures $\MM$,
    \[\MM \model{\Si} T \iff \MM \model{\Si} A\]

    If $A$ is a set of universal $\Si$-sentences 
    is called a universal $\Si$-theory.
    We are interested in universal axiomatizations of theories.
\end{dfn}

\begin{lem}[Lemma on constants]
    \link{lemma_on_const}
    Suppose $\const{\Si} \subs \const{\Si(*)}$, 
    $T \in \theory{\Si}$, $\phi \in \form{\Si}$
    with variables indexed by $n \in \N$.
    Suppose there exists a list of constant symbols not from $\Si$, 
    i.e. $c \in (\const{\Si(*)}\setminus \const{\Si})^n$ 
    such that $T \model{\Si(*)} \phi(c)$.
    Then \[
        T \model{\Si} \forall v, \phi(v)
    \]
\end{lem}
\begin{proof}
    If $n = 0$ then the result is clear as there are no quantifiers.
    Suppose $n \ne 0$.
    We prove the contrapositive.
    Suppose $T \nodel{\Si} \forall v, \phi(v)$
    then there exists $\MM$ a $\Si$-model of $T$
    such that $\MM \nodel{\Si} \forall v, \phi(v)$.
    Thus there exists $a \in \MM^n$ 
    such that $\MM \nodel{\Si} \phi(a)$.

    We \linkto{move_up_mod}{move $\MM$ up a signature} 
    by extending the interpretation 
    to the new constant symbols:
    if $d \in \const{\Si(*)} \setminus \const{\Si}$ then
    \[
        \subintp{\Si(\MM)}{\MM}{d} := 
        \begin{cases}
            a_i &, \text{ if for some } 0 \leq i < n \text{ we have }d = c_i,\\
            a_0 &, \text{ otherwise} 
        \end{cases}
    \]
    Then $\MM$ is a $\Si(*)$-model of $T$ such that 
    $\MM \nodel{\Si(*)} \phi(a)$, 
    which by construction is equivalent to 
    $\MM \nodel{\Si(*)} \phi(c)$.
\end{proof}

\begin{nttn}{Universal consequences of $T$}
    Let $T$ be a $\Si$-theory, then 
    \[
        T_\forall := 
        \set{\phi \text{ universal $\Si$-sentences} \st T \modelsi \phi}
    \]
    is called the set of universal consequences of $T$.
\end{nttn}

\begin{prop}[Universal axiomatizations make substructures models]
    \link{universal_axiomatizations_make_subs_mods}
    $T$ a $\Si$-theory has a universal axiomatization if and only if
    for any $\Si$-model $\NN$ of $T$ and any $\Si$-embedding 
    from some $\Si$-structure $\MM \to \NN$,
    $\MM$ is a $\Si$-model of $T$.
\end{prop}
\begin{proof}
    \begin{forward}
        Suppose $A$ is a universal axiomatization of $T$,
        $\NN$ is a $\Si$-model of $T$ and $\MM \to \NN$ is a 
        $\Si$-embedding.
        Let $\phi \in T$. Then
        $\NN \model{\Si} T$ implies $\NN \model{\Si} A$
        by definition of $A$.
        $\NN \model{\Si} A$ implies $\MM \model{\Si} A$
        since \linkto{emb_preserve_sat_of_forall_down}{
            embeddings preserve the satisfaction of quantifier free formulas
            downwards}.
        Finally $\MM \model{\Si} A$ implies $\MM \model{\Si} T$
        by definition of $A$.
    \end{forward}

    \begin{backward}
        We show that $T_\forall$ is a universal axiomatization of $T$.
        Let $\MM \model{\Si} T$ and let $\phi \in T_\forall$.
        Then by definition of $T_\forall$, $T \model{\Si} \phi$.
        Hence $\MM \model{\Si} \phi$ and any $\Si$-model of $T$ is
        a $\Si$-model of $T_\forall$.

        Suppose $\MM \modelsi T_\forall$.
        We first show that $T \cup \atdiag{\Si}{\MM}$ is consistent.
        By the \linkto{compactness}{compactness theorem} 
        it suffices to show that for any subset 
        $\De$ of $\atdiag{\Si}{\MM}$,
        $T \cup \De$ is consistent.
        Write $\De  = \set{\psi_1, \dots, \psi_n}$.
        Let $\psi = \bigwedge_{1 \leq i \leq n} \psi_i$.
        We can find $S$ that indexes the constant symbols in 
        $\const{\Si(\MM)} \setminus \const{\Si}$ that appear in $\psi$ 
        (in the same way as we made indexing sets of the variables).
        Then we can create $\phi \in \form{\Si}$ 
        with variables indexed by $S$ such that 
        $\phi(c) = \psi$, 
        where $c$ is a list of constant symbols in 
        $\const{\Si(\MM)} \setminus \const{\Si}$ indexed by $S$.
        Since $\De \subs \atdiag{\Si}{\MM}$ we have
        $\forall i, \MM \model{\Si} \psi_i$.
        Hence $\MM \modelsi \phi(c)$.
        Then $\MM \modelsi \exists v,  \phi(v)$ and so
        $\MM \nodel{\Si} \forall v,  \NOT \phi(v)$.

        Since each $\psi_i$ is from the the atomic diagram of $\MM$ 
        they are all quantifier free.
        Thus $\phi$ is a quantifier free $\Si$-formula and 
        $\forall v,  \NOT \phi(v)$ is universal.
        Hence $T \nodel{\Si} \forall v, \NOT \phi(v)$ 
        by the definition of $T_\forall$.
        By \linkto{lemma_on_const}{the lemma on constants}
        this implies that $T \nodel{\Si(\MM)} \NOT \phi(c)$.
        Hence there exists a $\Si(\MM)$-model of $T \cup \phi(c)$.
        Then it follows that this is also a $\Si(\MM)$-model of $T \cup \De$.
        Thus $T \cup \De$ is consistent so
        $T \cup \atdiag{\Si}{\MM}$ is consistent.

        Thus there exists $\NN$ a $\Si$-model of 
        $T \cup \atdiag{\Si}{\MM}$.
        This is a model of $\atdiag{\Si}{\MM}$ thus by 
        \linkto{elem_ext_equiv_eldiag_model}{there 
        is a $\Si(\MM)$-embedding $\MM \to \NN$}.
        We \linkto{move_down_morph}{make this a $\Si$-embedding}, 
        hence using the theorem's hypothesis $\MM$ is a $\Si$-model of $T$.
    \end{backward}
\end{proof}

The following result has doesn't come up at all until much later,
but is included here as another demonstration of the lemma on constants in use.
It appears as an exercise in the second chapter of Marker's book \cite{marker}.

\begin{cor}[Amalgamation]
    \link{amalgamation}
    Let $\AA$, $\MM$ and $\NN$ be $\Si$-structures,
    and suppose we have \linkto{partial_morph_dfn}{partial} 
    elementary $\Si$-embeddings
    $\io_\MM : A \to \MM$ and 
    $\io_\NN : A \to \NN$, for $\nothing \ne A \subs \AA$.
    Then there exists a common elementary extension $\PP$ of $\MM$ and $\NN$
    such that the following commutes:
    \begin{cd}
        \MM \ar[r] &\PP \\
        A \ar[u, "\io_\MM"] \ar[r, "\io_\NN"] &\NN \ar[u]
    \end{cd}
    $\PP$ is the `amalgamation' of $\MM$ and $\NN$.
\end{cor}
\begin{proof}
    We show first that the theory $\eldiag{\Si}{\MM} \cup \eldiag{\Si}{\NN}$
    is consistent as a $\Si(\MM,\NN)$-theory, where $\const{\Si(\MM,\NN)}$ is 
    defined to be 
    \[
        \set{c_a \st a \in A} \cup 
        \set{c_a \st a \in \MM \setminus \io_\MM(A)}
        \cup \set{c_a \st a \in \NN \setminus \io_\NN(A)}
    \]
    where terms and formulas from $\Si(A), \Si(\MM), \Si(\NN)$ 
    are interpreted in the natural way: 
    the constants $c_{\io_\MM(a)} \mapsto c_a$ (similarly for $\NN$).
    For the rest of the proof we identify 
    $\const{\Si(\MM)}$ with
    $\const{\Si(A)} \cup \set{c_a \st a \in \MM \setminus \io_\MM(A)}$
    (similarly with $\NN$). 

    By the \linkto{compactness}{compactness theorem} it suffices to show that 
    for any finite subset $\De \subs \eldiag{\Si}{\NN}$,
    $\eldiag{\Si}{\MM} \cup \De$ is consistent.
    Let $\phi$ be the $\Si(\MM)$-formula and $a \in \NN^\star$ be such 
    that \footnote{Take out all the finitely many constants appearing from 
        $\NN \setminus \io_{\NN}(A)$ in 
        $\De$ and make them into a tuple $a$, 
        replacing them with free variables.
        What remains is a finite set of $\Si(A)$-formulas, 
        which are naturally also $\Si(\MM)$-formulas. 
        We take the `and' of all of them to be $\phi$.}
    \[\phi(a) = \bigand{\psi \in \De}{} \psi\]
    $\phi(a)$ is naturally a $\Si(\MM,\NN)$-sentence such that 
    $\NN \model{\Si(\MM,\NN)} \phi(a)$.
    
    Suppose for a contradiction $\eldiag{\Si}{\MM} \cup \De$ is inconsistent.
    Then any $\Si(\MM,\NN)$-model of 
    $\eldiag{\Si}{\MM}$ is not a model of $\De$,
    which implies it does not satisfy $\phi(a)$.
    Hence
    \[\eldiag{\Si}{\MM} \model{\Si(\MM,\NN)} \NOT \phi(a)\]
    By the \linkto{lemma_on_const}{lemma on constants} applied to 
    $\Si(\MM) \leq \Si(\MM,\NN)$, $\eldiag{\Si}{\MM}$ and 
    $a \in \const{\Si(\MM,\NN)} \setminus \const{\Si(\MM)}$ we have
    \[\eldiag{\Si}{\MM} \model{\Si(\MM)} \forall v, \NOT \phi(v)\]
    Noting that $\MM$ is a $\Si$-model of its elementary diagram, 
    and \linkto{move_down_mod}{moving $\MM$} down a signature we have that 
    \[
        \MM \model{\Si(\MM)} \forall v, \NOT \phi(v) \implies 
        \MM \model{\Si(A)} \forall v, \NOT \phi(v) \implies 
    \]
    Since $A \to \MM$ and $A \to \NN$ are partial elementary $\Si$-embeddings
    (and thus naturally $\Si(A)$-embeddings) 
    we have that 
    $\AA \model{\Si(A)} \forall v, \NOT \phi(v)$ and so 
    $\NN \model{\Si(A)} \forall v, \NOT \phi(v)$.
    \linkto{move_up_mod}{Move this up} 
    to $\Si(\MM,\NN)$ we have a contradiction, by choosing 
    $v$ to be $a$:
    $\NN \model{\Si(\MM,\NN)} \NOT \phi(a)$, 
    but we remarked before that $\NN \model{\Si(\MM,\NN)} \phi(a)$.

    Hence $\eldiag{\Si}{\MM} \cup \eldiag{\Si}{\NN}$
    is consistent as a $\Si(\MM,\NN)$-theory. 
    Let $\PP$ be a $\Si(\MM,\NN)$-model of this
    (and naturally a $\Si(\MM)$ or a $\Si(\NN)$ structure).
    Then \linkto{elem_ext_equiv_eldiag_model}{there exist elementary 
    $\Si(\MM)$ and $\Si(\NN)$-extensions}
    $\la_\MM : \MM \to \PP$ and $\la_\NN : \NN \to \PP$ such that 
    $\la_\MM(\subintp{\Si(\MM)}{\MM}{c_m}) = \subintp{\Si(\MM)}{\PP}{c_m}$ 
    for each constant symbol $c_m$ for $m \in \MM \setminus \io_\MM(A)$ 
    and 
    $\la_\MM(\subintp{\Si(\MM)}{\MM}{c_a}) = \subintp{\Si(\MM)}{\PP}{c_a}$ 
    for each constant symbol $c_a$ for $a \in A$
    (similarly with $\NN$).

    Naturally, we can move everything down to $\Si(A)$.
    Thus for any $a \in A$
    \[
        \la_\MM \circ \io_\MM(a) = \la_\MM (\subintp{\Si(A)}{\MM}{c_a})
        = \la_\MM (\subintp{\Si(\MM)}{\MM}{c_a}) = 
        \subintp{\Si(\MM)}{\PP}{c_a} = \subintp{\Si(A)}{\PP}{c_a}
    \]
    By symmetry we have 
    $\la_\MM \circ \io_\MM(a) = \subintp{\Si(A)}{\PP}{c_a} =
    \la_\NN \circ \io_\NN(a)$.
\end{proof}

\subsection{The L\"{o}wenheim-Skolem Theorems}
The results in this subsection aren't used until much later. %? when?
It is worth skipping for now, 
but the material can be covered with the foundations made so far.

\begin{prop}[Upward L\"{o}wenheim-Skolem Theorem]
    \link{upwards_lowenheim_skolem}
    If $\MM$ is an infinite $\Si$-structure 
    and $\ka$ a cardinal such that 
    $|{\MM}|+ |\const{\Si}| \leq \ka$,
    there exists a $\Si$-structure $\NN$ with cardinality 
    $\ka$ as well as an elementary 
    $\Si$-embedding from $\MM$ to $\NN$.
\end{prop}
\begin{proof}
    \linkto{move_up_sig}{$\MM$ is naturally a $\Si(\MM)$-structure},
    and $\MM \model{\Si(\MM)} \eldiag{\Si}{\MM}$.
    Thus $\eldiag{\Si}{\MM}$ is a 
    $\Si(\MM)$-theory with an infinite model $\MM$,
    \linkto{inf_mod_theory_has_inf_mod}{
        hence it has a $\Si(\MM)$-model $\NN$ of cardinality $\ka$}.
    Making $\NN$ a $\Si$-structure, 
    \linkto{elem_ext_equiv_eldiag_model}{
        we obtain a $\Si$-embedding from $\MM$ to $\NN$}.
\end{proof}

\begin{dfn}[Skolem Functions]
    We say that a $\Si$-theory $T$ 
    \emph{has built in Skolem functions} when for any $\Si$-formula $\phi$
    that is not a sentence, 
    with free variables indexed by $S$,
    there exists $f \in \func{\Si}$ such that $n_f = k$ and 
    \[T \model{\Si} \bigforall{i \in S}{} w_i, 
        \brkt{\exists v, \phi(v,w) \to \phi(f(w),w)},\]
    Note that $w$ can be length $0$, 
    in which case $f$ has arity $0$ 
    and so would be interpreted as a constant map.
    We would have 
    \[T \model{\Si}  
        \exists v, \phi(v) \to \phi(f)\]
\end{dfn}

\begin{prop}[Skolemization]
    \link{skolemization}
    Let $T(0)$ be a $\Si(0)$-theory, 
    then there exists $T$ a $\Si$ theory such that
    \begin{enumerate}
        \item $|\func{\Si}| = |\func{\Si(0)}| + \aleph_0$
        \item $\func{\Si(0)} \subs \func{\Si}$, 
            and they share the same constant and relation symbols
        \item $T(0) \subs T$
        \item $T$ has built in Skolem functions
        \item All models of $T(0)$ can be moved up to being models of $T$ 
            with interpretations agreeing on $\Si$.
    \end{enumerate} 
    We call $T$ the Skolemization of $T(0)$.
\end{prop}
\begin{proof}
    Similarly to the \linkto{make_wit}{Witness Property proof}, 
    we define $\Si(i), T(i)$ for each $i \in \N$.
    Suppose by induction that we have $T(i) \in \theory{\Si}$, 
    such that 
    \begin{enumerate}
        \item $|\func{\Si(i)}| = |\func{\Si(0)}| + \aleph_0$
        \item $\func{\Si(0)} \subs \func{\Si(i)}$
            and they share the same constant and relation symbols
        \item $T(0) \subs T(i)$
        \item All models of $T(0)$ can be moved up to being models of $T(i)$
            with interpretations agreeing on $\Si$
    \end{enumerate} 
    Then define $\Si(i+1)$ such that only the function symbols are enriched:
    \[
        \func{\Si(i+1)} := \func{\Si(i)} \cup 
        \set{f_{\phi} \st \phi \in \form{\Si(i)} 
        \text{ and $\phi$ is not a sentence}}
    \]
    extending the arity $n_\star$ to by having 
    $n_{f_\phi} = |S| - 1$, where $S$ is indexes the free variables of $\phi$.
    There are countably infinite $\Si(i)$-formulas, 
    thus $|\func{\Si(i)}| = |\func{\Si(0)}| + \aleph_0$.

    Define $\Psi : \form{\Si(i)} \to \form{\Si(i+1)}$ mapping 
    \[
        \phi \mapsto \forall w, (\exists v, \phi(v,w)) \to (\phi(f_\phi (w),w)),
    \] 
    where $w$ is a list of variables of the suitable length. 
    We then define 
    \[
        T(i+1) := T(i) \cup \Psi(\form{\Si(i)})
    \]
    which is a $\Si(i+1)$-theory because the image of $\Psi$ has only 
    $\Si$-sentences.
    Note that $T(0) \subs T(i) \subs T(i+1)$.

    Let $\MM(0)$ be a $\Si(0)$-model of $T(0)$, 
    then we have $\MM(i)$ a $\Si(i)$-model of $T(i)$ 
    with the same carrier set and same
    interpretation on $\Si(0)$ as $\MM(0)$.
    Let $\MM(i+1)$ have the same carrier set as $\MM(0)$.
    To extend interpretation to $\Si(i+1)$,
    we first deal with the case where ${\MM(i)}$ is empty by simply
    interpreting all new function symbols as the empty function.
    Otherwise we have a $c \in {\MM(0)}$. 
    For $f_\phi \in \Psi(\form{\Si(i)})$ define
    \begin{align*}
        \modintp{\MM(i+1)}{f_\phi} : 
        {\MM(i+1)}^{n_{f_\phi}} &\to {\MM(i+1)}\\
        a &\mapsto \begin{cases}
            b &, \text{ if } \exists b \in {\MM}, 
            \MM(i) \model{\Si(i)} \phi(b,a)\\
            c &, \text{ otherwise}
        \end{cases}\\
    \end{align*}
    Then by construction, 
    $\MM(i+1) \model{\Si(i+1)} \Psi(\form{\Si(i)})$.
    By checking the conditions on \linkto{move_up_sig}{
        moving $\MM(i)$ up to $\MM(i+1)$},
    we can also conclude
    $\MM(i+1) \model{\Si(i+1)} T(i)$.
    Hence $\MM(i+1) \model{\Si(i+1)} T(i+1)$.
    
    Let $\Si(*)$ 
    be the signature such that its constants and relations agree with $\Si(0)$
    and $\func{\Si(*)} = \bigcup_{i \in \N} \func{\Si(i)}$.
    Then 
    \[
        |\func{\Si(*)}| = |\bigcup_{i \in \N} \func{\Si(i)}| = 
        \aleph_0 \times (\aleph_0 + \func{\Si(0)}) = 
        \aleph_0 + \func{\Si(0)}
    \]

    Let $T(*) = \bigcup_{i \in \N} T(i)$.
    We show that $T(*)$ has built in Skolem functions.
    Let $\phi$ be a $\Si(*)$-formula that is not a $\Si$-sentence.
    Then $\phi \in \form{\Si(i)}$ for some $i \in \N$. 
    Thus $\Psi(\phi) \in T(i+1) \subs T(*)$,
    hence 
    \[T(*) \model{\Si(*)} \forall w, (\exists v, \phi(v,w)) 
    \to (\phi(f_\phi (w),w))\]
    Thus $T(*)$ has built in Skolem functions.
    
    If $\MM \model{\Si} T$ then let ${\MM(*)} = {\MM}$ 
    and define the interpretation such that for all $i \in \N$, 
    and $f \in T(i)$, 
    $\subintp{\Si(*)}{\MM(*)}{f} = \subintp{\Si(i)}{\MM(i)}{f}$.
    Since all interpretations agree upon intersection this is well-defined.
    To show that $\MM(*)$ is a $\Si(*)$-model of $T(*)$,
    let $\phi$ be in $T(*)$; 
    there is some $i \in \N$ such that $\phi \in T(i)$.
    Using our lifted $\MM(i)$ from before we have 
    $\MM(i) \model{\Si(i)} \phi$.
    By checking the conditions on \linkto{move_up_sig}{
        moving $\MM(i)$ up to $\MM(*)$},
    we can also conclude
    $\MM(*) \model{\Si(*)} \phi$
    (by taking the $\Si(*)$ theory $\set{\phi}$).
    Hence
    $\MM(*) \model{\Si(*)} T(*)$.
\end{proof}

\begin{dfn}[Theory of a Structure]
    We define the theory of a $\Si$-structure $\MM$ to be
    \[\Theory_{\MM} := \set{\phi \in \form{\Si} \st 
    \phi \text{ is a $\Si$-sentence and } \MM \model{\Si} \phi}\]
\end{dfn}

\begin{prop}[Downward L\"{o}wenheim-Skolem Theorem]
    Let $\NN$ be an infinite $\Si(0)$-structure and $M(0) \subs {\NN}$.
    Then there exists a $\Si(0)$-structure $\MM$ such that 
    \begin{itemize}
        \item $M(0) \subs {\MM} \subs {\NN}$
        \item $|{\MM}| \leq |M(0)| + |\func{\Si(0)}| + \aleph_0$
        \item The inclusion $\subs : \MM \to \NN$ is an elementary embedding.
    \end{itemize}
\end{prop}
\begin{proof}
    We first take the \linkto{skolemization}{Skolemization} 
    of $\Theory_{\NN}$ and call the new signature and theories 
    $\Si$ and $T$.
    Since $\NN \model{\Si(0)} \Theory_{\NN}$, 
    we can move it up to being a $\Si$-structure so that 
    $\NN \model{\Si} T$.

    We want to create the carrier set of $\MM$,
    it has to be big enough so that interpreted functions are closed on $\MM$.
    Given $M(i)$ such that $|M(i)| \leq |M(0)| + |\func{\Si}| + \aleph_0$, 
    we inductively define $M(i+1)$:
    \[
        M(i+1) := M(i) \cup \set{\subintp{\Si}{\NN}{f}(a) 
        \st f \in \func{\Si} \AND a \in {M(i)}^{n_f}}
    \]
    Then \begin{align*}
        |M(i+1)| &\leq |M(i)| + |\func{\Si}| \times |M(i)^{n_f}|\\
        &\leq |M(i)| + |\func{\Si}| \times (|M(i)| + \aleph_0)\\
        &\leq |M(0)| + |\func{\Si}| + \aleph_0 + |\func{\Si}| 
        \times (|M(0)| + |\func{\Si}| + \aleph_0)\\
        &\leq |M(0)| + |\func{\Si}| + \aleph_0
    \end{align*}
    Then ${\MM} := \bigcup_{i} M(i)$ and 
    $|{\MM}| \leq |M(i)| \times \aleph_0 =
    |M(0)| + |\func{\Si}| + \aleph_0 
    \leq |M(0)| + |\func{\Si(0)}| + \aleph_0$.

    We first interpret function symbols, 
    which will give us a way to interpret constant symbols. 
    For $f \in \func{\Si}$ and $a \in ({\MM})^{n_f}$,
    define $\subintp{\Si}{\MM}{f}(a) = \subintp{\Si}{\NN}{f}(a)$.
    This is well-defined as there exists 
    $i \in \N$ such that $a \in (M(i))^{n_f}$,
    \[\subintp{\Si}{\MM}{f}(a) \in M(i+1) \subs {\MM}\]
    Then to interpret constant symbols,
    we consider for each $c \in \const{\Si}$ the formula $v = c$.
    Since $T$ has built in Skolem functions and $\NN \model{\Si} T$, 
    there exists $f$ with arity $n_f = 0$ such
    $\NN \model{\Si} (\exists v, v = c) \to f = c$.
    Since $\NN \model{\Si} \exists v, v = c$, 
    we have $\subintp{\Si}{\NN}{f} = \subintp{\Si}{\NN}{c}$.
    Since $\subintp{\Si}{\NN}{f} = \subintp{\Si}{\MM}{f} : 
        ({\MM})^0 \to {\MM}$ 
    we can define $\subintp{\Si}{\MM}{c} = \subintp{\Si}{\NN}{c} = 
        \subintp{\Si}{\NN}{f} = \subintp{\Si}{\MM}{f} \in {\MM}$.
    
    Lastly define the interpretation of relations as 
    $\subintp{\Si}{\MM}{r} = ({\MM})^{m_r} \cap \subintp{\Si}{\NN}{r}$.

    By construction the inclusion $\subs$ is a $\Si$-embedding. 
    We check that it is elementary using the third equivalent condition in the
    \link{tarski_vaught}{Tarski Vaught Test}:
    let $\phi \in \form{\Si}$ with free variables indexed by $S$,
    $i \in S$ and
    $a \in ({\MM})^{S \setminus \set{i}}$.
    Suppose $\exists c \in {\NN}, \NN \model{\Si} \phi(a,c)$.
    $T$ has built in Skolem functions,
    and $\NN \model{\Si} T$. 
    Hence there exists $f \in \func{\Si}$ such that
    \[
        \NN \model{\Si} (\exists v, \phi(a,v)) \to \phi(a,f(a)) 
    \]
    We can deduce $\NN \model{\Si} \phi(a,f(a))$.
    Noting that $\subintp{\Si}{\MM}{f}(a) = \subintp{\Si}{\NN}{f}(a)$ 
    completes the Tarski Vaught Test. 
    Hence $\subs$ is an elementary $\Si$-embedding.
    
    We \linkto{move_down_morph}{move $\subs : \MM \to \NN$ down a signature} 
    since by 
    \linkto{skolemization}{Skolemization} we have 
    $\Si(0) \leq \Si$.
    Then $\subs : \MM \to \NN$ is an elementary $\Si$-embedding.
\end{proof}
