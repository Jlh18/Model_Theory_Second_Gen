\section{Monsters}

\subsection{Properties of the monster model}
\begin{nttn}
    We write $\Si(\dots,r)$ to mean the signature 
    with an extra relation symbol thrown in with the rest.
\end{nttn}

\begin{dfn}[$\ka$-monster \cite{hodges}]
    Let $\ka$ be a cardinal and $r$ 
    a relation symbol not in the signature $\Si$.
    We say a $\Si$-structure $\MM$ is $\ka$-monsterous
    (also called big or splendid by Hodges) if it satisfies:
    for any subset $A \subs \MM$ 
    of cardinality $< \ka$ and any $\Si(A,r)$ structure $\NN$ that 
    such that $\NN$ and $\MM$ are elementarily equivalent \emph{in $\Si$}, 
    there is an interpretation of $\MM$ as a $\Si(A,r)$-structure 
    such that $\NN$ and $\MM$ are elementarily equivalent \emph{in $\Si(r)$}.
    \[\MM \equiv_\Si \NN \rightsquigarrow \MM \equiv_{\Si(r)} \NN\]

    We may also say that $\MM$ is a $\ka$-monster.
\end{dfn}

We want monsters to have the properties of being saturated, 
homogeneous and universal, 
which we define soon.
We show that monsters are 
saturated and that saturated structures are homogeneous and universal.

\begin{cd}
    & & \text{homogeneous}\\
    \text{monsterous} \ar[r]& \text{saturated} \ar[ur] \ar[dr]& \\
    & & \text{universal}\\
\end{cd}

\begin{ex}
    Let $\phi$ be a $\Si$-formula with $n$ free variables.
    There exists a $\Si$-formula that means 
    `there exists a unique tuple $x = (x_1,\dots,x_n)$ such that $\phi(x)$'.
    We write this as 
    \[\exists ! x, \phi(x)\]
\end{ex}

\begin{lem}[Monsters are saturated]
    Let the $\Si$-structure $\MM$ be $\ka$-monsterous.
    Then it is $\ka$-saturated.
\end{lem}
\begin{proof}
    Let $A$ be a subset of $\MM$ with cardinality $< \ka$.
    Let $p \in S_n(\Th_\MM(A))$ be a maximal $n$-type.
    Since \linkto{finite_realisation_and_embeddings}{$p$ 
        is realised by $b \in \NN^n$ in some elementary 
        $\Si(A)$-extension $\NN$}
    we can make $\NN$ a $\Si(A,r)$-structure that interprets 
    $r$ as the singleton set $\set{b}$.
    We have $\MM \equiv_\Si \NN$ as $\NN$ is an elementary extension.
    Since $\MM$ is $\ka$-monsterous and $\MM \equiv_\Si \NN$ 
    we have an interpretation of $r$ in $\MM$
    such that $\MM \equiv_{\Si(A,r)} \NN$.
    Hence 
    \[\NN \model{\Si(A,r)} \exists ! x, r(x) \implies 
    \MM \model{\Si(A,r)} \exists ! x, r(x) \]
    Taking this unique element $a \in \MM^n$ we see that 
    \[  
        p = \subintp{A}{\NN}{\tp}(b) = 
        \subintp{A}{\NN}{\tp}(\io(a)) 
        = \subintp{A}{\MM}{a}
    \]
    The last equality using the fact that $\NN$ is an elementary 
    $\Si(A)$-extension of $\MM$. 
    (Note that we might not have that it is an elementary 
    $\Si(A,r)$-extension)
    Hence $p$ is realised in $\MM$ and $\MM$ is $\ka$-saturated.
\end{proof}

\begin{dfn}[$\ka$-saturation (strong $\ka$-homogeneity)]
    \link{ka_saturation_dfn}
    Let $\ka$ be a cardinal and $\MM$ be a $\Si$-structure. 
    $\MM$ is $\ka$-saturated if for subset $A \subs \MM$
    such that $\abs{A} < \ka$
    every $n \in \N$ and every $p \in S_n(\Theory_\MM(A))$,
    $p$ is realised in $\MM$.

    Equivalently $\MM$ is $\ka$-saturated
    if for any $A \subs \NN$ in any $\Si$-structure $\NN$ 
    satisfying $\abs{A} < \ka$ either of the equivalent
    following hold:
    \begin{itemize}
        \item For any partial elementary embedding $f : A \to \MM$ and 
            any $b \in \NN$, 
            $f$ can be extended to a partial elementary $\Si$-embedding 
            $A \cup \set{b} \to \MM$.
        \item If $\MM \equiv_{\Si(A)} \NN$ then for any $b \in \NN$ we have 
            $\MM \equiv_{\Si(A,c)} \NN$ for some constant symbol $c$ that 
            is interpreted as $b$ in $\NN$.
    \end{itemize}

    The latter two definitions are equivalent due to the fact that 
    $\MM \equiv_{\Si(A)} \NN$ if and only if 
    there is a way to interpret symbols from $A$ in $\MM$ such that 
    \[\MM \model{\Si(A)} \phi \iff \NN \model{\Si(A)} \phi\] 
    for any $\Si(A)$-sentence $\phi$, 
    which is equivalent to the existance of a partial embedding $A \to \MM$. 
    The strong version implies the weak version by taking $\NN$ to be 
    $\MM$. 
    It remains to prove the equivalence with the first definition:
\end{dfn}
\begin{proof}
    \begin{forward}
        Let $f : A \to \MM$ be a partial elementary embedding,
        where $A \subs \NN$ and $\abs{A} < \ka$; let $b \in \NN$.
        We show that $\MM \equiv_{\Si(A,c)} \NN$
        for a constant symbol $c$ that is interpreted as $b$ in $\NN$
        which gives a way to extend $f$ over $A \cup \set{b}$.

        By \linkto{amalgamation}{amalgamation} there exists $\LL$ and 
        commuting elementary $\Si(A)$-embeddings $\io_\MM,\io_\NN$into 
        $\LL$ from $\MM$ and $\NN$.
        First we show that $\subintp{A}{\LL}{\tp}(\io_\NN(b))$ is consistent 
        with $\Theory_\MM(A)$.
        It suffices to show \linkto{compactness_for_types}{finite consistency:} 
        let $\De$ be a finite subset of $\Theory_\MM(A)$. 
        The formula
        \[\exists x, \bigand{\psi \in \De}{} \psi(x)\]
        is satisfied in $\LL$ by taking $\io_\NN(b)$ hence is satisfied in 
        $\MM$ as $\io_\MM$ is $\Si(A)$-elementary.
        Thus $\subintp{A}{\LL}{\tp}(\io_\NN(b))$ is consistent 
        with $\Theory_\MM(A)$.

        Hence, as $\MM$ is $\ka$-saturated we can find $a \in \MM$ such that 
        \[
            \subintp{A}{\LL}{\tp}(\io_\NN(b)) = 
            \subintp{A}{\MM}{\tp}(a)
        \]

        Add a constant symbol $c$ to the signature such that 
        $\nnintp{c} = b$ and $\mmintp{c} = a$.
        Then $x = c$ is in both types and so $\io_\NN(b) = \io_\MM (a)$.
        Finally we have 
        \[\NN \equiv_{\Si(A,c)} \LL \equiv_{\Si(A,c)} \MM\]
        since $\NN \to \LL$ and $\MM \to \MM$ are elementary.
    \end{forward}

    \begin{backward}
        If $p \in S_n(\Theory_\MM(A))$ then \linkto{}{it is realised 
        in some elementary $\Si$-extension} $\io : \MM \to \NN$
        by $b \in \NN$.
        Hence we have a partial elementary $\Si$-embedding 
        $\io^{-1} : \io(\MM) \to \MM$ that can be extended to have domain 
        $\io^{-1} \cup \set{b}$ by assumption.
        Hence we have the image of $b$ under this map as the element of $\MM$ 
        realising the type $p$.
    \end{backward}
\end{proof}

A special case of the second definition of 
saturation comes in the form of homogeneity.
\begin{dfn}[$\ka$-homogeneity]
    Let $\ka$ be a cardinal and $\MM$ a $\Si$-structure.
    $\MM$ is $\ka$-homogeneous 
    if for any $A \subs \MM$, 
    any \linkto{partial_morph_dfn}{partial elementary embedding} 
    $f : A \to \MM$ and any $b \in \MM$, 
    if $\abs{A} < \ka$ then 
    $f$ can be extended to a partial elementary embedding 
    $A \cup \set{b} \to \MM$.
\end{dfn}

\begin{dfn}[$\ka$-universality]
    Let $\MM$ be a $\Si$-structure.
    Then $\MM$ is a $\ka$-universal $\Si$-structure if 
    any $\Si$-structure $\NN$ such that $\abs{\NN} < \ka$ and 
    $\NN \equiv_\Si \MM$ has an elementary 
    embedding into $\MM$.
\end{dfn}

\begin{nttn}
    We replace $\ka$ with $\ka^+$ in all the previous definitions
    when the $<$ is upgraded to $\leq$.
    Note that if a structure is $\ka^+$-something then it is also 
    $\ka$-something and if $\la \leq \ka$ and it is $\ka$-something 
    then the structure is also $\la$-something.
\end{nttn}

\begin{lem}[Saturated structures are universal \cite{marker}]
    Let $\MM$ be a $\ka$-saturated $\Si$-structure.
    Then $\MM$ is a $\ka^+$-universal $\Si$-structure.
\end{lem}
\begin{proof}
    Let $\NN_\be$ be a $\Si$-structure such that 
    $\abs{\NN_\be} = \be \leq \ka$ and $\NN \equiv_\Si \MM$.
    Then enumerate $\NN_\be = \set{n_\ga \st \ga \leq \be}$
    and define nested subsets $\NN_\al := \set{n_\ga \st \ga \leq \al}$
    for each $\al \leq \be$.
    We then use \linkto{transfinite_induction}{transfinite induction} to define 
    partial elementary embeddings $f_\al : \NN_\al \to \MM$ 
    for $\al \leq \be$ such that they mutually agree upon restriction.

    If $\al$ is a limit ordinal and for each $\ga \leq \al$ we have 
    a well-defined partial elementary embedding $f_\ga : \NN_\ga \to \MM$ 
    that agrees upon restriction
    then $f_\al := \bigcup_{\ga < \al} f_\ga$ is well-defined
    and agrees with the others upon restriction.
    In the case that $\ga = 0$ and $f_\ga$ 
    is the empty map it is a partial elementary embedding 
    because the only formulas we need to verify are sentences 
    (the domain is empty).
    Since $\NN \equiv_\Si \MM$ we immediately have that it is a 
    partial embedding.
    In the case where $0 < \ga$, $f_\ga$ is clearly elementary.

    To define $f_{\al+1}$ using $f_\al$ we first define 
    \[\Ga = \set{\phi(v,f_\al(a)) \in \form{\Si(f_\al(\NN_\al))} \st
    \exists a \in \NN^\star, \NN \model{\Si} \phi(n_\al,a))}\]
    $\Ga$ is finitely consistent with $\Theory_\MM(f_\al(\NN_\al))$:
    for any finite subset $\De \subs \Ga$,
    we can take the conjuction,
    \[\exists v, \bigand{\phi(v,f_\al(a)) \in \De}{} \phi(v,a)\]
    Note that this is satisfied by $\NN$ therefore as 
    $f_\al$ is a partial elementary $\Si$-embedding:
    \[\MM \model{\Si} \exists v, 
    \bigand{\phi(v,f_\al(a)) \in \De}{} \phi(v,f_\al(a))\]
    Hence $\MM \model{\Si} \De(b)$ for some $b \in \MM$.
    By \linkto{compactness_for_types}{compactness for types}
    $\Ga$ is consistent with $\Theory_\MM(f_\al(\NN_\al))$.
    This implies that it 
    \linkto{extend_to_maximal_type_zorn}{can be extended} to a maximal $1$-type 
    $p \in S_1(\Theory_\MM(\NN_\al))$.
    By $\ka$-saturation, $\Ga(v)$ is realised by some 
    $b \in \MM$.
    Define $f_{\al + 1} : n_\al \mapsto b$ and agreeing with $f_\al$ otherwise.
    By definition $f_{\al + 1}$ is a partial elementary $\Si$-embedding.

    Hence $f_\be : \NN_\be \to \MM$ is an elementary $\Si$-embedding.
\end{proof}

\subsection{Finite structures are monsters}
\begin{lem}[Finite structures in a finite signature are 
        characterised by one formula]
    \link{finite_struc_sig_are_charac_by_formula}
    Let $\Si$ be a signature with finitely many constant, 
    function and relation symbols.
    If $\MM$ is a finite $\Si$-structure then there exists a `characterising'
    $\Si$-sentence $\phi$ such that for any $\Si$-structure $\NN$,
    \[\NN \model{\Si} \phi \iff \NN \iso_\Si \MM\]
\end{lem}
\begin{proof}
    For convenience we write $\const{\Si} = C$, $\form{\Si} = F$
    and $\rel{\Si} = R$.
    Let $n \in \N$ be the cardinality of $\MM$ and index it:
    \[\MM = \set{a_1,\dots,a_n}\]

    We define $\phi$:
    \begin{align*}
        \phi := &\bigexists{i = 1}{n} x_i,\\
        & \bigand{i < j}{} x_i \ne x_j &\text{at least $n$ elements}\\
        & \AND \forall x, \bigor{i = 1}{n} x = x_i 
        &\text{at most $n$ elements}\\
        & \AND \bigand{c \in C, \mmintp{c} = a_i}{} c = x_i
        &\text{constants interpreted like in $\MM$}\\
        & \AND \bigand{f \in F, \mmintp{f}(a_{k_1},\dots,a_{k_n}) = a_i}{} 
        f(x_{k_1},\dots,x_{k_n}) = x_i
        &\text{functions interpreted like in $\MM$}\\
        & \AND \bigand{r \in R,
        (a_{k_1},\dots,a_{k_m}) \in \mmintp{r}}{}
        r(x_{k_1},\dots,x_{k_n})\\
        & \AND \bigand{r \in R,
        (a_{k_1},\dots,a_{k_m}) \notin \mmintp{r}}{}
        \NOT r(x_{k_1},\dots,x_{k_n}) 
        &\text{relations interpreted like in $\MM$}
    \end{align*}
    This is something we can write down since there the signature and the 
    structure are both finite.
    
    Suppose $\NN \model{\Si} \phi$.
    Then there exist $b_i \in \NN$ from satisfaction of $\phi$.
    Then taking the map $a_i \mapsto b_i$ gives a map $\MM \to \NN$.
    Unfolding the relevant parts of $\phi$ 
    shows that this is a $\Si$-isomorphism.
    Conversely if $\NN$ and $\MM$ are isomorphic then taking the images of 
    each $a_i$ under the isomorphism $\MM \to \NN$ 
    we obtain elements of $\NN$ that satisfy $\phi$.
\end{proof}

\begin{cor}[Elementarily equivalence with a finite structure is isomorphism]
    \link{fin_elem_equiv_imp_iso}
    If $\MM$ is a finite $\Si$-structure and $\NN$ is another $\Si$-structure
    such that $\MM \equiv_\Si \NN$, then $\MM \iso_\Si \NN$.
\end{cor}
\begin{proof}
    Suppose $\Si$-structure $\MM$ has cardinality $n \in \N$.
    Let $\NN$ be another $\Si$-structure such that $\NN \equiv_\Si \MM$.
    Taking the formula 
    \begin{align*}
        &\bigexists{i = 1}{n} x_i,\\
        & \bigand{i < j}{} x_i \ne x_j &\text{at least $n$ elements}\\
        & \AND \forall x, \bigor{i = 1}{n} x = x_i 
        &\text{at most $n$ elements}
    \end{align*}
    which is satisfied by $\MM$, 
    we see that it is also satisfied by $\NN$ by elementary equivalence.
    Thus $\abs{\NN} = \abs{\MM}$ and so $\NN$ is also finite.

    Since both structures are finite there are only finitely many 
    bijections $\io : \MM \to \NN$.
    For a contradiction suppose that none of these are $\Si$-isomorphisms,
    then for each $\io$ there exists a constant, function, or relation symbol
    that is not preserved by $\io$.
    Taking the set of these we have a finite signature $\Si_{\text{fin}}$
    with $\Si_{\text{fin}}$-structures $\MM$ and $\NN$.
    Hence we have a 
    \linkto{finite_struc_sig_are_charac_by_formula}{characterising formula} 
    $\phi$ for $\MM$ in $\Si_{\text{fin}}$.
    By elementary equivalence $\NN \model{\Si} \phi$ and so 
    $\NN \iso_\Si \MM$.
\end{proof}

\begin{cor}
    Finite $\Si$-structures are monsters.
\end{cor}
\begin{proof}
    Let $\MM$ be a finite $\Si$-structure.
    Let $r$ be a relation symbol and suppose $\NN$ is a $\Si(r)$-structure 
    such that $\NN \equiv_{\Si} \MM$.
    Since $\MM$ is finite, 
    this implies \linkto{fin_elem_equiv_imp_iso}{$\NN \iso_\Si \MM$}.
    Defining $\mmintp{r}$ as the image of $\nnintp{r}$ under the isomorphism
    makes $\MM$ a $\Si(r)$-structure such that $\NN \iso_\Si(r) \MM$.
    Since \linkto{iso_imp_elem_equiv}{isomorphisms are elementary embeddings}
    we have $\MM \equiv_{\Si(r)} \NN$.
\end{proof}

\subsection{Existence of monsterous extensions}
\begin{prop}
    \link{existence_of_monster}
    If $\MM$ is a $\Si$-structure there exists a $\ka$-monsterous elementary 
    $\Si$-extension of $\MM$ for any cardinal $\ka$.
\end{prop}
\begin{proof}
    Omitted for now; see \cite{hodges}.
\end{proof}
