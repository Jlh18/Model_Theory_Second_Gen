\section{Basics}
These first two sections follow Marker's book on Model Theory 
\cite{marker} with more 
emphasis on where things are happening, i.e. what signature we are working in,
and some more general statements such as working with embeddings rather than 
subsets.

\subsection{Signatures}
\begin{dfn}[First order language]
    We assume we have a tuple
    $\LL = (\CC,\FF,\RR,\VV,\set{\NOT,\OR,\forall,=,\top})$ such that 
    \begin{itemize}
        \item $|\CC|,|\FF|,|\RR|$ each sufficiently large 
            (say they have cardinality $\aleph_5$ or something).
        \item $|\VV| = \aleph_0$. 
            We index $\VV = \set{v_0, v_1, \dots }$ using $\N$.
        \item $\CC$,$\FF$,$\RR$, $\VV$, $\set{\NOT,\OR,\forall,=}$ 
            do not overlap.
    \end{itemize}
    We call $\LL$ the language and only really 
    use it to get symbols to work with.
    Whenever we introduce new symbols to create larger signatures, 
    we are pulling them out of this box.
\end{dfn}

\begin{dfn}[Signature]
    In a language $\LL$, 
    a tuple $\Si = (C,F,n_\star ,R, m_\star)$ is a signature\footnote{
        Many authors call $\Si$ the language, 
        but I have chosen this way of defining things instead.
    } when 
    \begin{itemize}
        \item $C \subs \CC$. 
            We call $C$ the set of constant symbols.
        \item $F \subs \FF$ and 
            $n_{\star} : F \to \N$, 
            which we call the function arity. 
            We call $F$ the set of function symbols.
        \item $R \subs \RR$
            and $m_{\star} : R \to \N$,
            which we call the relation arity.
            We call $T$ the set of relation symbols.
    \end{itemize}
    Given a signature $\Si$, we may refer to its constant, 
    function and relation symbol sets as $\const{\Si},\func{\Si},\rel{\Si}$.
    We will always denote function arity using $n_\star$ 
    and relation arity using $m_\star$.
\end{dfn}

\begin{eg}
    The \linkto{dfn_rings}{signature of rings} 
    will be used to define the theory of rings, 
    the theory of integral domains, the theory of fields, and so on.
    The signature of \linkto{infinite_infinite_classes}{binary relations} 
    will be used to define the theory of partial orders
    with the interpretation of the relation as $<$, 
    to define the theory of equivalence relations with the 
    interpretation of the relation as $~$,
    and to define the theory $\ZFC$ with the relation interpreted as $\in$.
\end{eg}

\begin{dfn}[$\Si$-terms]
    Given $\Si$ a signature, its set of terms
    $\term{\Si}$ is inductively defined using three constructors:
    \begin{itemize}
        \item[$\vert$] If $c \in \const{\Si}$
            then $c$ is a $\Si$-term.
        \item[$\vert$] If $v_i \in \VV$, 
            $v_i$ is a $\Si$-term.
        \item[$\vert$] Given $f \in \func{\Si}$ and a $n_f$-tuple of $\Si$-terms
            $t \in (\term{\Si})^{n_f}$
            then the symbol $f(t)$ is a $\Si$-term.
    \end{itemize}
    The terms $v_i$ are called variables and will be 
    referred to as elements of $\var{\Si}$.
\end{dfn}

\begin{eg}
    In the signature of rings, 
    \linkto{terms_in_RNG_are_polynomials}{terms will be multivariable 
        polynomials over $\Z$} 
    since they are sums and products of constant symbols $0, 1$ and 
    variable symbols.
    In the \linkto{infinite_infinite_classes}{signature of binary relations} 
    there are no constant or function symbols so the only terms are variables.
\end{eg}

\begin{dfn}[$\Si$-formula, free variable]
    Given $\Si$ a signature, 
    its set of $\Si$-formulas $\form{\Si}$ is inductively defined:
    \begin{itemize}
        \item[$\vert$] $\top$ is a $\Si$-formula
        \item[$\vert$] Given $t, s \in \term{\Si}$, 
        the string $(t = s)$ is a $\Si$-formula
        \item[$\vert$] Given $r \in \rel{\Si}, t \in (\term{\Si})^{m_r}$, 
        the string $r(t)$ is a $\Si$-formula \vspace{1em}
        \item[$\vert$] Given $\phi \in \form{\Si}$, 
        the string $(\NOT \phi)$ is a $\Si$-formula 
        \item[$\vert$] Given $\phi, \psi \in \form{\Si}$, the string 
        $(\phi \lor \psi)$ is a $\Si$-formula
        \item[$\vert$] Given $\phi \in \form{\Si}$ 
        and $v_i \in \var{\Si}$, we take the replace all occurrences of
        $v_i$ with an unused symbol such as 
        $z$ in the string $(\forall v_i, \phi)$
        and call this a $\Si$-formula.
    \end{itemize}
    Shorthand for some strings in $\form{\Si}$ include 
    \begin{itemize}
        \item $\bot := \NOT \top$
        \item $\phi \land \psi := \NOT((\NOT\phi) \lor (\NOT\psi))$
        \item $\phi \to \psi := (\NOT\phi) \lor \psi$
        \item $\exists v, \phi := \NOT(\forall v, \NOT\phi)$
    \end{itemize}
    
    The symbol $z$ is meant to be a `bounded variable', 
    and will not be considered when we want to evaluate
    variables in formulas.
    Not bound variables are called `free variables'.
\end{dfn}
\begin{rmk}
    There are two different uses of the symbol `$=$' from now on, 
    and context will allow us to tell them apart.
    Similarly, logical symbols might be used in our `higher language' 
    and will not be confused with symbols from formulas.

    Formulas should be thought of as propositions with some bits loose, 
    namely the free variables, since it does not make any sense to ask if 
    $x = x \OR x = a$ without saying what $x$ you are taking
    (where $x$ is a variable as $a$ is a constant symbol, say).
    When there are no free variables we get what intuitively looks like a 
    proposition, and we will call these particular formulas sentences.
\end{rmk}

For the sake of learning the theory the following two lemmas should be skipped.
They are essentially algorithms that tell us what the variables in terms 
and what the free variables in formulas are.
We include them for formality.
\begin{prop}[Terms have finitely many variables]
    \link{term_fin_var}
    For any term $t \in \term{\Si}$ there exists a finite subset 
    $S \subs \N$ indexing the variables $v_i$ occuring in $t$.
\end{prop}
\begin{proof}
    If there exists a finite subset $T$ of $\N$ such that if $v_i$ occurs in $t$
    then $i \in T$
    then we can take the intersection of all such
    sets and have the finite set $S$ we're interested in.
    
    We prove existence of such a 
    $T$ using the inductive definition of $t$:
    \begin{itemize}
        \item If $t = c$ a constant symbol, 
        then $T = \nothing$ satisfies the above.
        \item If $t = v_i$ a variable symbol, 
        then $T = \set{i}$ satisfies the above.
        \item If $t = f(t_0, \dots, t_{n_f})$, 
        then by our induction hypothesis we have 
        a $T_i$ satisfying the condition for each $t_i$. 
        Then $\cup_{i} T_i$ satisfies the condition for $t$.
    \end{itemize}
\end{proof}

\begin{prop}[Formulas have finitely many \emph{free} variables]
    \link{form_fin_var}
    Given $\phi \in \form{\Si}$,
    there exists a finite $S \subs \N$ 
    indexing the \emph{free} variables $v_i$ occuring in $\phi$.
\end{prop}
\begin{proof}
    Like in the \linkto{term_fin_var}{terms case}, 
    we only need to show that there exists a $T$
    If $v_i$ occurs freely in $\phi$ then $i \in T$.
    We induct on what $\phi$ is, noting that until the last case there are
    no quantifiers being considered so the variables in question are free:
    \begin{itemize}
        \item If $\phi$ is $\top$ then it has no variables.
        \item If $\phi$ is $t = s$, 
        then we have $S_t, S_s$ indexing the (free) variables of $t$ and $s$ 
        \linkto{term_fin_var}{by the previous proposition},
        and so we can pick $T = S_t \cup S_s$.
        \item If $\phi$ is $r(t_0, \dots, t_{m_r})$, 
        then for each $t_i$ we have $S_i$ indexing the variables of $t_i$.
        Hence we can pick $T = \cup_i S_i$.
        \item If $\phi$ is $\NOT \psi$, 
        then by the induction hypothesis we have $T$ 
        satisfying the above conditions for $\psi$. 
        Pick this $T$ for $\phi$.
        \item If $\phi$ is $\psi \lor \chi$
        then by the induction hypothesis we have 
        $T_{\psi}, T_{\chi}$ satisfying the above conditions for $\psi$ and $\chi$.
        We take $T$ to be the union of indexing sets for $\psi$ and $\chi$.
        \item If $\phi$ is $\forall v_i, \psi$ with $v_i$ substituted for $z$,
        then by the induction hypothesis we have 
        $T_{\psi}$ satisfying the above conditions for $\psi$.
        Take $T = T_{\psi} \setminus \{v_i\}$. 
        In fact taking $T_{\psi}$ itself works as well.
    \end{itemize}
\end{proof}

\begin{nttn}[Substituting Terms for Variables]
    If a $\Si$-formula $\phi$ has a free variable $v_i$ 
    then to remind ourselves of the variable we can write 
    $\phi = \phi(v_i)$ instead.

    If $\phi$ has $S$ indexing its free variables and $t \in (\term{\Si})^S$, 
    then we write $\phi(t)$ to mean $\phi$ with 
    $t_i$ substituted for each $v_i$.
    We can show by induction on terms and formulas that this is still a 
    $\Si$-formula.
\end{nttn}

\begin{dfn}[$\Si$-structure, interpretation]
    Given a signature $\Si$, a set $M$ and interpretation functions 
    \begin{itemize}
        \item $\intp{\const{\Si}}{\MM} : \const{\Si} \to M$
        \item $\intp{\func{\Si}}{\MM} : \func{\Si} \to (M^{n_\star} \to M)$
        \item $\intp{\rel{\Si}}{\MM} : \rel{\Si} \to \PP(M^{m_{\star}})$
    \end{itemize}
    we say that $\MM := (M, \intp{\Si}{\MM})$ is a $\Si$-structure. 
    The latter functions two are dependant types since the powers
    $n_\star, m_\star$ 
    depend on the function and relation symbols given.
    The class (or set or whatever) of $\Si$-structures is denoted $\struc{\Si}$.
    Given only the $\Si$-structure $\MM$, 
    we call its underlying carrier set $M$ as $\carrier{\MM}$.

    We write $\intp{\Si}{\MM}$ to represent any of the three interpretation 
    functions when the context is clear.
    Given $c \in \const{\Si}, f \in \func{\Si}, r \in \rel{\Si}$, 
    we might write the `interpretations' 
    of these symbols as any of the following
    \[
        \subintp{\const{\Si}}{\MM}{c} = \subintp{\Si}{\MM}{c} = \modintp{\MM}{c}
        \quad \quad 
        \subintp{\func{\Si}}{\MM}{f} = \subintp{\Si}{\MM}{f}  = \modintp{\MM}{f}
        \quad \quad
        \subintp{\rel{\Si}}{\MM}{r} = \subintp{\Si}{\MM}{r}  = \modintp{\MM}{r}
    \]
\end{dfn}
The structures in a signature will become the models which we are interested in,
in particular structures will be a models of theories.
For example $\Z$ is a structure in the signature of rings, 
and models the theory of rings but not the theory of fields.
In the signature of binary relations, 
$\N$ with the usual ordering $\leq$ is a structure that models of 
the theory of partial orders but not the theory of equivalence relations.

\begin{dfn}[Interpretation of terms]
    Given a signature $\Si$, 
    a $\Si$-structure $\MM$ and a $\Si$-term $t$,
    let $S$ be the 
    \linkto{term_fin_var}{unique set indexing the variables of $t$}.
    Then there exists a unique induced map 
    $\subintp{T}{\MM}{t} : \carrier{\MM}^S \to \carrier{\MM}$, 
    that commutes with the interpretation of constants and
    functions\footnote{See this more precisely stated in the proof}.
    We then refer to this map as \emph{the} interpretation of the term $t$.
    This in turn defines a dependant $\Pi$-type  
    \[
        \intp{T}{\MM}: 
        \term{\Si} \to (\carrier{\MM}^{S_\star} \to \carrier{\MM})
    \]
\end{dfn}
\begin{proof}
    To define a map $\subintp{T}{\MM}{t} : M^S \to M$ for each
    $t$ we use the inductive definition of $t \in \term{\Si}$.
    If $M$ is empty we define $\subintp{T}{\MM}{t}$ as the empty function.
    Otherwise let $a \in M^S$:
    \begin{itemize} 
        \item If $t = c \in \const{\Si}$ then define 
        $\subintp{T}{\MM}{t} : a \mapsto \subintp{C}{\MM}{c}$, 
        the constant map.
        This type checks since $S = \nothing$ 
        therefore $\subintp{T}{\MM}{t} : M^0 \to M$.
        \item If $t = v_i \in \var{\Si}$
        then define $\subintp{T}{\MM}{t} : a \mapsto a$, the identity.
        This type checks since $|S| = 1$.
        \item If $t = f(s)$ for some $f \in \func{\Si}$ and 
        $s \in (\var{\Si})^{n_f}$ then define 
        $\subintp{T}{\MM}{t} : 
        a \mapsto \subintp{F}{\MM}{f}(\subintp{T}{\MM}{s}(a))$.
        This type checks since $s$ has the same number of variables as $t$.
    \end{itemize}
    By definition, 
    this map commutes with the interpretation of constants and functions, i.e.
    \begin{center}
        \begin{tikzcd}[column sep = tiny]
        \const{\Si} \ar[r, "\subs"] \ar[rd, "\intp{\const{\Si}}{\MM}", swap]
        & \term{\Si} \ar[d, "\intp{T}{\MM}"]
        &&
        \prod_{f \in \func{\Si}} \brkt{\term{\Si}}^{n_f}
        \ar[rrr, bend left=15]
        \ar[rrrdd, bend right=15]& 
        s \in \brkt{\term{\Si}}^{n_f} \ar[r, mapsto] \ar[rd, mapsto]& f(s) & 
        \term{\Si} \ar[dd, "\intp{T}{\MM}"]
        \\
        & (\MM^{S_\star} \to \MM) 
        &&
        && \subintp{\func{\Si}}{\MM}{f}(\subintp{T}{\MM}{s}(\star))\\
        &&&&&&(\MM^{S_\star} \to \MM) 
    \end{tikzcd}
    \end{center}
    The map is clearly unique.
\end{proof}
Where there is no ambiguity, 
we write $\subintp{T}{\MM}{t} = \modintp{\MM}{t}$.
Furthermore, if we have a tuple 
$t \in (\term{\Si})^k$,
then we write $\subintp{T}{\MM}{t}:= 
(\mmintp{t_0},\cdots,\mmintp{t_k})$
    
\begin{dfn}[Sentences and satisfaction]
    Let $\Si$ be a signature and $\phi$ a $\Si$-formula.
    Let \linkto{form_fin_var}{$S \subs \N$ index the free variables of $\phi$}.
    We say $\phi \in \form{\Si}$ is a $\Si$-sentence when $S$ is empty.

    If $\Si$ has no constant symbols then $\nothing$ is a $\Si$-structure
    by interpreting functions and relations as $\nothing$.
    Then given a $\Si$-sentence (not any $\Si$-formula) $\phi$ we 
    want to define $\nothing \model{\Si} \phi$
    using the inductive definition of $\phi$:
    \begin{itemize}
        \item If $\phi$ is $\top$ then $\nothing \model{\Si} \phi$.
        \item $\phi$ cannot be $t = s$ since this contains a constant
            symbol or a variable.
            \vspace{1em}
        \item If $\phi$ is 
            $\NOT \psi$ for some $\psi \in \form{\Si}$, 
            then $\nothing \model{\Si} \phi$ when $\nothing \nodel{\Si} \psi$
        \item If $\phi$ is $(\psi \OR \chi)$, 
            then $\nothing \model{\Si} \phi$ when 
            $\nothing \model{\Si} \psi$ or $\nothing \model{\Si} \chi$.
        \item If $\phi$ is of the form
            $\forall v, \psi$, then $\nothing \model{\Si} \phi$.
    \end{itemize}

    Let $\MM$ be a $\Si$-structure with non-empty carrier set.
    Then given $a \in (\carrier{\MM})^S$, 
    we want to define $\MM \model{\Si} \phi(a)$:
    \begin{itemize}
        \item If $\phi$ is $\top$ then $\MM \model{\Si} \phi$.\footnote{
            We can omit the $a$ when there are no free variables.
            Formally this $a$ is the unique element in $\MM^\nothing$
            given by the empty set.}
        \item If $\phi$ is $t = s$ then
            $\MM \model{\Si} \phi(a)$ when 
            $\modintp{\MM}{t}(a) = \modintp{\MM}{s}(a)$.
            \item If $\phi$ is $r(t)$, 
            where $r \in \rel{\Si}$ and 
            $t \in (\term{\Si})^{m_r}$,
            then $\MM \model{\Si} \phi(a)$ when 
            $\modintp{\MM}{t}(a) \in \modintp{\MM}{r}$.
            \vspace{1em}
        \item If $\phi$ is 
            $\NOT\psi$ for some $\psi \in \form{\Si}$, 
            then $\MM \model{\Si} \phi(a)$ when $\MM \nodel{\Si} \psi(a)$
        \item If $\phi$ is  $(\psi \lor \chi)$, 
            then $\MM \model{\Si} \phi(a)$ when 
            $\MM \model{\Si} \psi(a)$ or $\MM \model{\Si} \chi(a)$.
        \item If $\phi$ is 
            $(\forall v, \psi(a)) \in \form{\Si}$,
            then $\MM \model{\Si} \phi(a)$ 
            if for any $b \in \carrier{\MM}$,   
            $\MM \model{\Si} \psi(a)(b)$.
    \end{itemize}
    We say $\MM$ satisfies $\phi(a)$.
\end{dfn}
\begin{rmk}
    Any $\Si$-structure $\MM$ satisfies $\top$ 
    and does not satisfy $\bot$.
    The empty set satisfies things of the form $\forall, \dots$ 
    but not $\exists, \dots$, as we would expect.
    Note that for $c$ a tuple of constant symbols
    $\MM \model{\Si} \phi(c)$ is the same thing as 
    $\MM \model{\Si} \phi(\mmintp{c})$.
\end{rmk}

\subsection{Theories and Models}
\begin{dfn}[$\Si$-theory]
    $T$ is an $\Si$-theory when it is a subset of $\form{\Si}$
    such that all elements of $T$ are $\Si$-sentences.
    We denote the set of $\Si$-theories as $\theory{\Si}$.
\end{dfn}

\begin{dfn}[Models] 
    \link{consistent}
    Given an $\Si$-structure $\MM$ and $\Si$-theory $T$, 
    we write $\MM \model{\Si} T$ and say
    \emph{$\MM$ is a $\Si$-model of $T$} when 
    for all $\phi \in T$ we have $\MM \model{\Si} \phi$.
\end{dfn}

\begin{eg}[The empty signature and theory]
    $\Si_\nothing = (\nothing, \nothing, n_\star, \nothing, m_\star)$ 
    is the empty signature.
    (We pick the empty functions for $n_\star, m_\star$.)
    The empty $\Si_\nothing$-theory is given by $\nothing$.
    Notice that any set is a $\Si_\nothing$-structure and moreover
    a $\Si_\nothing$-model of the empty $\Si_\nothing$-theory.
\end{eg}

\begin{eg}
    In the signature of rings, 
    \linkto{terms_in_RNG_are_polynomials}{the rings axioms} will be the theory
    of rings and structures satisfying the theory will be rings.
    The \linkto{missing_link}{theory of $\ZFC$} %?$? MISSING LINK
    consists of the $\ZFC$ axioms and a model of $\ZFC$ would be thought of as 
    the `class of all sets'.
\end{eg}

\begin{dfn}[Consequence]
    Given a $\Si$-theory $T$ 
    and a $\Si$-sentence $\phi$,
    we say $\phi$ is a consequence of $T$
    and say $T \model{\Si} \phi$ 
    when for all non-empty $\Si$-models $\MM$ of $T$, 
    we have $\MM \model{\Si} \phi$.
\end{dfn}
\begin{rmk}
    We have to be a bit careful when we go from something like
    $\MM \model{\Si} \phi(a)$ to deducing something about $T$.
    This is because there might not exist a $\Si$-constant $c$ 
    such that $\modintp{\MM}{c} = a$,
    it only makes sense to write $T \model{\Si} \phi$ if $\phi$
    is a \emph{sentence}.
\end{rmk}

\begin{ex}[Logical consequence]
    Let $T$ be a $\Si$-theory and $\phi$ and $\psi$ be $\Si$-sentences.
    Show that the following are equivalent:
    \begin{itemize}
        \item $T \model{\Si} \phi \to \psi$
        \item $T \model{\Si} \phi$ implies $T \model{\Si} \psi$.
    \end{itemize}
\end{ex}

\begin{dfn}[Consistent theory]
    A $\Si$-theory $T$ is consistent if either of the following equivalent 
    definitions hold:
    \begin{itemize}
        \item
            There does not exists a 
            $\Si$-sentence $\phi$ such that
            $T \model{\Si} \phi$ and $T \model{\Si} \NOT \phi$.
        \item There exists 
            a non-empty $\Si$-model of $T$. 
    \end{itemize}
    Thus the definition of consistent is intuitively 
    `$T$ does not lead to a contradiction'.

    A theory $T$ is finitely consistent if all 
    finite subsets of $T$ are consistent.
    This will turn out to be another equivalent definition,
    given by the \linkto{compactness}{compactness theorem}.
\end{dfn}
\begin{proof} 
    We show that the two definitions are equivalent.
    \begin{forward}
        Suppose no non-empty model exists.
        Take $\phi$ to be the $\Si$-sentence $\top$.
        Since the only model of $T$ is the empty $\Si$-structure.
        Hence all non-empty $\Si$-models of $T$ satisfy $\top$ and $\bot$
        (there are none) so
        $T \model{\Si} \top$ and $T \model{\Si} \bot$.
    \end{forward}
    \begin{backward}
        Suppose $T$ has non-empty $\Si$-model $\MM$
        and $\MM \model{\Si} \phi$ and $\MM \model{\Si} \NOT \phi$.
        This implies $\MM \model{\Si} \phi$ and $\MM \nodel{\Si} \phi$,
        a contradiction.
    \end{backward}
\end{proof}