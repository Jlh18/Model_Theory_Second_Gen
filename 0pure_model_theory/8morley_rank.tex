\section{Morley Rank}
\subsection{Saturation}

\begin{prop}[Elementarily equivalent structures have a common extension 
        \cite{poizat}]
    \link{elem_equiv_struc_has_common_ext}
    Let $\NN_i$ be $\Si$-structures, 
    for each $i \in I$ a non-empty set.
    Suppose that for each $i, j \in I$, 
    $\NN_i$ is elementarily equivalent to $\NN_j$.
    Then there exists $\MM$ a $\Si$-structure and elementary $\Si$-embeddings 
    $\io_i : \NN_i \to \MM$ for each $i \in I$.
\end{prop}
\begin{proof}
    We show that 
    $\bigcup_{i \in I} \eldiag{\Si}{\NN_i}$ is consistent
    as a $\Si(*):=\Si(\bigcup_{i \in I} ({\NN_i}))$ theory.
    This would give us 
    \linkto{elem_ext_equiv_eldiag_model}{elementary $\Si(*)$-embeddings}
    from each $\NN_i$ into some non-empty model $\MM$,
    and we would then be done (by moving this embedding down to $\Si$).

    By \linkto{compactness}{compactness} it suffices to show that 
    each finite subset of $\bigcup_{i \in I} \eldiag{\Si}{\NN_i}$ is consistent.
    The finite subset can be written in the form
    \[\bigcup_{i \in S} \De_i\]
    where $S \subs I$ is a finite subset 
    and each $\De_i \subs \eldiag{\Si}{\NN_i}$ is finite.
    Let $0$ denote the element of $I$ (non-empty). 
    We show that 
    \[\NN_0 \model{\Si(*)} \bigcup_{i \in S} \De_i\]
    By definition of the elementary diagram,
    each $\NN_i$ is a $\Si(\NN_i)$-model of $\De_i$. 
    We can write $\De_i$ as $\Ga_i(a_i)$ 
    where $\Ga_i$ is a set of $\Si$-formulas
    with at most $n$ variables and $a_i \in {\NN_i}^n$.
    Then using elementary equivalence
    \begin{align*}
        &\NN_i \model{\Si(\NN_i)} \bigand{\phi \in \De_i}{} \phi
        \implies \NN_i \model{\Si} \bigand{\phi \in \Ga_i}{} \phi(a_i)\\
        \implies &\NN_i \model{\Si} \bigexists{j = 1}{n} v_j, 
        \bigand{\phi \in \Ga_i}{} \phi(v)
        \implies \NN_0 \model{\Si} \bigexists{j = 1}{n} v_j, 
        \bigand{\phi \in \Ga_i}{} \phi(v)\\
        \implies &\exists b_i \in {\NN_0}^n, \NN_0 \model{\Si} 
        \bigand{\phi \in \Ga_i}{} \phi(b_i)
    \end{align*}
    We proceed to use this fact to define interpretation on $\NN_0$ as 
    a $\Si(*)$ structure.

    If $\NN_0$ were empty then any sentence is satisfied by it, 
    hence by elementary equivalence every $\NN_i$ satisfies all sentences.
    Hence all the structures are empty and we can find a common extension,
    namely the empty set.
    Otherwise let $c \in {\NN_0}$.
    We only need to update interpretation of constant symbols from 
    ${\NN_i}$ for $i \ne 0$.
    Let $d \in {\NN_i}$.
    If $d$ appears in the tuple $a_i$ from above, 
    we interpret it as the corresponding term in the tuple $b_i$ such that 
    $\modintp{\NN_0}{a_i} = b_i$.
    Otherwise intperpret $d$ as $c$.
    Clearly for each $i \in S$ and each $\phi \in \Ga_i$
    \[\NN_0 \model{\Si} \phi(b_i) \implies 
    \NN_0 \model{\Si(*)} \phi(b_i) \implies 
    \NN_0 \model{\Si(*)} \phi(a_i)\]
    and so 
    \[\NN_0 \model{\Si(*)} \bigcup_{i \in S} \Ga_i(a_i)
    \implies \NN_0 \model{\Si(*)} \bigcup_{i \in S} \De_i\]
\end{proof}

\begin{dfn}[More general version of the theory of a structure]
    Let $\MM$ be a $\Si$-structure and let $A \subs {\MM}$.
    Move $\MM$ up to $\Si(A)$ in the obvious way. 
    The theory of $\MM$ over $A$ is defined by
    \[\Theory_\MM(A):= 
    \set{\phi \in \Si(A)\text{-sentences} \st \MM \model{\Si(A)} \phi}\]
    Note that if $\MM$ is non-empty then 
    $\Theory_\MM(A)$ is a consistent and complete $\Si(A)$-theory
    as it is modelled by $\MM$ 
    and any formula is either satisfied by $\MM$ or not.
    Note also that the theory of $\MM$ over 
    ${\MM}$ is the elementary diagram.
\end{dfn}

\begin{lem}[Types over $\Theory_\MM(A)$ are realised in extensions]
    \link{types_of_models_realised_in_extensions}
    Let $\MM$ be a $\Si$-structure, 
    $A \subs \MM$ and $p \in S_n(\Theory_\MM(A))$.
    Then there exists an elementary $\Si(A)$-embedding $\MM \to \NN$ such that 
    $p$ is realised in $\NN$.
\end{lem}
\begin{proof}
    By definition of the Stone space, 
    $p$ is consistent with $\Theory_\MM(A)$ and so there exists $\RR$
    a $\Si(A)$-model of $\Theory_\MM(A)$ that realises $p$.
    Since $\Theory_\MM(A)$ is complete, 
    and both $\MM$ and $\RR$ are $\Si(A)$-models of it,
    $\MM$ and $\RR$ are elementarily equivalent.
    \linkto{elem_equiv_struc_has_common_ext}{Hence they have a common
        elementary extension $\NN$.}
    Since $\RR$ realises $p$ and the extension $\RR \to \NN$ is elementary 
    $\NN$ realises $p$.
    Hence we have an elementary $\Si(A)$-extension of $\MM$ that realises $p$.
\end{proof}

\begin{lem}[Types are preserved downwards in elementary embeddings]
    \link{technical_lemma_types_preserved_downwards}
    Suppose $\io : \MM \to \NN$ is an elementary $\Si$-embedding
    and $A \subs \MM$ is a finite subset.
    Consider $p \in F(\Si(\io(A)),n)$.
    For convinience we write this as $q(A)$, 
    where $q \in F(\Si, n + m)$,
    $m$ is the cardinality of $A$.
    Then 
    \[q(\io(A)) \in S_n(\Theory_{\NN}(\io(A)))
    \implies q(A) \in S_n(\Theory_{\MM}(A))\]
\end{lem}
\begin{proof}
    Suppose $q(\io(A)) \in S_n(\Theory_{\NN}(\io(A)))$.
    Then there exists $\RR$ a $\Si(\io(A))$-model of $\Theory_\NN(\io(A))$ 
    that realises $q(\io(A))$. 
    It suffices to show that $\RR \model{\Si(A)} \Theory_\MM(A)$ and 
    realises $q(A)$.
    Take the interpretation of a constant symbol $a \in A$ as the 
    $\modintp{\RR}{\io(a)}$, 
    the interpretation in $\Si(\io(A))$.
    Then for any $\phi \in \form{\Si}$ (with $m$ free variables) 
    and $a \in \RR^m$
    \[\RR \model{\Si(A)} \phi(A)(a) \iff 
    \RR \model{\Si(\io(A))} \phi(\io(A))(a)\]

    Let $\phi(A) \in \Theory_\MM(A)$ such that $\phi \in \form{\Si}$.
    Then $\MM \model{\Si(A)} \phi(A)$ hence $\MM \model{\Si} \phi(A)$
    and as the embedding is elementary $\NN \model{\Si} \phi(\io(A))$
    and $\NN \model{\Si(\io(A))} \phi(\io(A))$.
    Hence $\phi(\io(A)) \in \Theory_\NN(A)$ and 
    $\RR \model{\Si(\io(A))} \phi(\io(A))$ and 
    $\RR \model{\Si(A)} \phi(A)$.
    Thus $\RR \model{\Si(A)} \Theory_\MM(A)$
    
    There exists $a \in \RR^n$ such that $\RR \model{\Si(\io(A))} q(\io(A))(a)$.
    If $\phi \in q$ then $\RR \model{\Si(\io(A))} \phi(\io(A))(a)$
    and $\RR \model{\Si(A)} q(A)(a)$.
\end{proof}

\begin{dfn}[Embedding Chain, Elementary Chain \cite{marker}]
    Given $(I,\leq)$ a non-empty linear order and a functor $M: I \to \Mod{\Si}$
    that sends each $\al \in I$ to a $\Si$-structure $\MM(\al)$
    and each $\al \leq \be$ in $I$ to a 
    $\Si$-embedding $\lift{\al}{\be} : \MM(\al) \to \MM(\be)$, called a lift.
    Then we call the functor $M$ an embedding chain of $\Si$-structures.

    Furthermore if $M$ only results in elementary $\Si$-embeddings
    then $M$ is an elementary chain.
\end{dfn}

\begin{prop}[$\om$-saturated elementary extensions \cite{poizat}]
    \link{om_sat_elem_ext_of_models}
    Every $\Si$-structure $\MM$ has an 
    \linkto{om_saturation_dfn}{$\om$-saturated} elementary extension.
\end{prop}
\begin{proof}
    Let $\MM$ be a $\Si$-structure.
    We will inductively create an elementary chain of $\Si$-structures
    $\MM = \MM(0) \to \MM(1) \to \cdots$ such that for each $\al \in \N$,
    every finite subset $A \subs \MM(\al)$, 
    every $n \in \N$ and every $p(A) \in S_n(\Theory_{\MM(\al)}(A))$,
    $p(\lift{\al}{\al + 1} A)$ is realised in $\MM(\al + 1)$.
    \linkto{direct_limit_of_chains}{Taking the direct limit} 
    gives us a $\Si$-structure $\NN$ and elementary $\Si$-embeddings 
    $\io_\al : \MM(\al) \to \NN$ for each $\al \in \N$ that commute with 
    the lifts from the chain.
    In particular we will have that $\NN$ is an elementary $\Si$-extension
    of $\MM$.
    For any finite subset $A \subs \NN$, 
    any $n \in \N$ and any $p(A) \in S_n(\Theory_{\NN}(A))$,
    since $A$ is finite there exists $\al \in \N$ such that 
    ${\io_\al}^{-1}(A) \subs \MM(\al)$ bijects with $A$.
    \linkto{technical_lemma_types_preserved_downwards}{As types are preserved
        downwards in elementary embeddings} 
    $p(A) \in S_n(\Theory_{\NN}(A))$ implies 
    $p(\io_\al^{-1} A) \in S_n(\Theory_{\MM(\al)}(\io_\al^{-1} A))$.
    By construction we see that
    $p$ is realised in $\MM(\al + 1)$
    thus $p$ is realised in $\NN$ as $\io_{\al + 1}$ is elementary.
    This $\NN$ is $\om$-saturated.

    Furthermore, 
    if $A \subs \MM$, $n \in \N$ and $p(A) \in S_n(\Theory_{\MM})(A)$,
    then $p(\lift{0}{1} A)$ is realised in $\MM(1)$.
    Hence $p(\io_0 A)$ is realised in $\NN$ as $\io_{1}$ is elementary.

    It suffices to build $\MM(1)$ from $\MM$.
    Let $A \subs \MM$, $n \in \N$ and $p_{A,n} \in S_n(\Theory_\MM (A))$.
    \linkto{types_of_models_realised_in_extensions}{$p_{A,n}$ 
        is realised in some $\Si(A)$-extension $\MM(p_{A,n})$.}
    As $\MM(p_{A,n})$ extends $\MM$, 
    it is naturally a $\Si(\MM)$-model of $\eldiag{\Si}{\MM}$.
    We create such a $\Si(\MM)$-structure for each $A$, $n$ and $p_{A_n}$.
    Consider the set 
    \[\set{\MM(p_{A,n}) \st A \subs \MM, 
    n \in \N, p_{A,n} \in S_n(\Theory_\MM (A))}\]
    Any two structures in the set are elementarily equivalent since 
    they would both model $\eldiag{\Si}{\MM} = \Theory_\MM(\MM)$ 
    which is a complete $\Si(\MM)$-theory.
    Thus \linkto{elem_equiv_struc_has_common_ext}{there 
        exists a common $\Si(\MM)$-extension $\MM(1)$}
    of all the structures in the set. 
    In particular there exists 
    an elementary $\Si(\MM)$-embedding $\lift{0}{1} : \MM \to \MM(1)$
    (for the existance of one intermediate structure take 
    $A = \nothing, n = 0$ and \linkto{make_max}{extend 
    $\Theory_\MM(\nothing)$ to a maximal $\Si$-theory}).

    \begin{cd}
        &&&\MM(1)&&&\\
        \MM(\dots)\ar[urrr, shorten >= 10pt]&\MM(\dots)
        \ar[urr, shorten >= 5pt]&\MM(\dots)\ar[ur]
        &\MM(p_{A,n})\ar[u]
        &\MM(\dots)\ar[ul]&\MM(\dots)
        \ar[ull, shorten >= 5pt]&\MM(\dots)
        \ar[ulll, shorten >= 10pt]\\
        &&&\MM \ar[ulll]\ar[ull]\ar[ul]\ar[u]\ar[ur]\ar[urr]\ar[urrr]&&&
    \end{cd}

    To show that $\MM(1)$ has the desired property let
    $A \subs \MM$, 
    $n \in \N$ and $p \in S_n(\Theory_{\MM}(A))$.
    By design there exists $\MM(p)$ that realises $p$.
    Since $\MM(1)$ is an elementary $\Si(\MM)$-extension of $\MM(p)$
    (we can turn this into a $\Si(A)$-embedding for things to make sense)
    we have that 
    $p$ is realised in $\MM(1)$.
\end{proof}

\begin{lem}[Disjunctive normal form]
    \link{disjunctive_normal_form_0}
    Let $\phi$ be a quantifier free $\Si$-formula with variables indexed
    by $S$.
    Then there exist atomic $\Si$-formulas
    $f_{ij}$ such that
    for any $\Si$-structure $\MM$
    \[ 
        \MM \model{\Si} \bigforall{s \in S}{}v_s, \phi(v) 
        \IFF \bigor{i \in I}{} 
        \bigand{j \in J_{i}}{} f_{ij}(v)
    \]
    An immediate improvement to this is that there exist $\Si$-formulas
    $f_{ij}, g_{ij}$ of the form $s = t$ or $r(t)$ 
    (equality of terms or a relation) such that
    for any $\Si$-structure $\MM$
    \[ 
        \MM \model{\Si} \bigforall{s \in S}{}v_s, \phi(v) 
        \IFF \bigor{i \in I}{} 
        \brkt{\bigand{j \in J_{i0}}{} f_{ij}(v)\AND 
        \bigand{j \in J_{i1}}{} \NOT g_{ij}(v)}
    \]
\end{lem}
\begin{proof}
    We induct on $\phi$:
    \begin{itemize}
        \item If $\phi$ is $\top$ 
            then we take a single `or' $I = \set{0}$ and the empty `and' for 
            $J_{0}$.
        \item If $\phi$ is $t = s$ then
            let $I = J_{0} = \set{0}$ and take $f_{00}$ to be $t = s$.
        \item If $\phi$ is $r(t)$ then 
            let $I = J_{0} = \set{0}$ and take $f_{00}$ to be $r(t)$.
        \item If $\phi$ is $\NOT \psi$ and there exist the relevant things for 
            $\psi$ that satisfy
            \[ 
                \MM \model{\Si} \bigforall{s \in S}{}v_s, \psi(v) 
                \IFF \bigor{i \in I}{} 
                \bigand{j \in J_{i}}{} f_{ij}(v)
            \]
            for all $\Si$-structures $\MM$ and for all $a \in {\MM}^S$.
            We can negate and rearrange this statement to give us what we 
            want for $\phi$.
            The gist of the induction goes as follows:
            \begin{align*}
                &\NOT \bigor{i \in I}{} 
                \bigand{j \in J_{i}}{} f_{ij}(v) \IFF 
                \bigand{i \in I}{} 
                \bigor{j \in J_{i}}{} \NOT f_{ij}(v)\\
                \IFF &\brkt{\bigor{j \in J_0}{} \NOT f_{0j}} \AND
                \brkt{\bigand{0 \ne i \in I}{} 
                \bigor{j \in J_{i}}{} \NOT f_{ij}(v)} \\
                 \IFF 
                &\bigor{k_0 \in J_0}{} 
                \sqbrkt{\NOT f_{0k_0} \AND \bigand{0 \ne i \in I}{} 
                \bigor{j \in J_{i}}{} \NOT f_{ij}(v)}
                \text{ (induction on $J_0$)}\\
                \IFF & \bigor{k_0 \in J_0}{} 
                \sqbrkt{\NOT f_{0k_0} \AND \bigor{k_1 \in J_1}{} 
                \brkt{\NOT f_{1k_1} \AND \bigand{i \ne 0, 1}{} 
                \bigor{j \in J_{i}}{} \NOT f_{ij}(v)}} \\
                \IFF & \bigor{k_0 \in J_0}{} 
                \bigor{k_1 \in J_1}{} 
                \brkt{\NOT f_{0k_0} \AND \NOT f_{1k_1} 
                \AND \bigand{i \ne 0, 1}{} 
                \bigor{j \in J_{i}}{} \NOT f_{ij}(v)} \\
                \IFF & \dots \text{ induction on $I$} 
                \IFF \bigor{k_0 \in J_0}{} \dots \bigor{}{}
                \sqbrkt{\bigand{}{}g_{ij}}
            \end{align*}
        \item If $\phi$ is $\chi_0 \OR \chi_1$ 
            and there exist the relevant things 
            for $\chi_0$ and $\chi_1$, then we can simply take the or of
            the two formulas found and obtain what we want.
    \end{itemize}
\end{proof}

\begin{lem}[Type of an element from the model is isolated]
    \link{type_of_an_element_from_the_model}
    Let $T$ be a complete $\Si$-theory
    and $\MM$ be a non-empty $\Si$-model of $T$. 
    \linkto{elems_of_stone_space_are_types_of_elements}{Any 
        type over $\MM$ is of the form}\footnote{Here we have $\nothing$ in 
            $\subintp{\nothing,1}{\NN}{\tp}(a)$ as we are working in the theory 
            $\eldiag{\Si}{\MM}$, 
            which is in the language $\Si_E(\MM)$,
            thus we don't need to add symbols from $\MM$.}
        $\subintp{\nothing,1}{\NN}{\tp}(a)$, 
    where $\NN$ is an
    \linkto{om_sat_elem_ext_of_models}{$\om$-saturated elementary extension} 
        of $\MM$ and $a \in \NN$.
    Then $a$ is in the image of $\MM$ if and only if 
    $\tp(a)$ is an \linkto{dfn_isolated_point}{isolated point} 
    in $S_1(\MM)$.
    Thus $\MM$ bijects with the isolated points of $S_1(\MM)$
    by taking $c \mapsto \set{\tp(io(c))}$ and the 
    \linkto{dfn_derived_set}{derived set} is
    \[  
        S_1(\MM)' = S_1(\MM) \setminus \bigcup_{c \in \MM} [x = c]
        = \bigcap_{c \in \MM} [x \ne c] 
    \]
\end{lem}
\begin{proof}
    \begin{forward}
        Suppose $a$ is in the image of $\MM$.
        Then we have $c \in \MM$ such that $a = \nnintp{c}$.
        Hence $x = c$ is an element of $\tp(a)$ and so $\tp(a) \in [x = c]$.
        If $q \in [x = c]$ then as $q$ is realised by some $b \in \NN$, 
        $b = \nnintp{c} = a$, 
        which implies $q = \tp(a)$.
        Thus the open set $[x = c]$ is a singleton and the point is isolated.
        Thus $c \mapsto \tp(\io(c))$ has image a subset of the isolated points.
        It is injective since for $c, d \in \MM$, 
        if $\tp(\io(c)) = \tp(\io(d))$ then 
        $x = c \in \tp(\io(c)) = \tp(\io(d))$
        and so $\io(d) = \io(c)$ which implies $c = d$ as $\io$ is injective.
    \end{forward}
    \begin{backward}
        On the other hand, suppose there is an isolated point $p$ in $S_1(\MM)$,
        then $\set{p}$ is open. 
        This open set is equal to $\bigcup_{\phi \in I} [\phi]$
        as a finite intersection of clopen sets is clopen,
        hence $\psi \in I$ such that $p \in [\psi]$, 
        whence $[\psi] = \set{p}$.
        As $\psi \in p$ (which is consistent with $T$) we have that 
        $\exists x, \psi(x)$ is realised in some model of $T$ 
        (any elementary extension of $\MM$ is a model of $T$). 
        Since $T$ is complete
        this implies $T \model{\Si_E} \exists x, \psi(x)$
        In particular $\MM$ is a model of $T$ and so there exists 
        $a \in \MM$ such that $\MM \model{\Si_E} \psi(a)$.
        Thus $\NN \model{\Si_E} \psi(\io(a))$ so $\psi \in \tp(\io(a))$
        and $\tp(\io(a)) \in [\psi]$.
        Since there is only one element in this set $p = \tp(\io(a))$.
        Thus $c \mapsto \set{\tp(\io(c))}$ is surjective.
    \end{backward}
\end{proof}

\begin{eg}[Infinite infinite equivalence classes revisited]
    \link{infinite_infinite_classes1}
    Consider again the theory $T$ of 
    \linkto{infinite_infinite_classes}{infinite infinite equivalence classes},
    and $\MM$ a non-empty $\Si_E$-model of $T$. 
    We wish to classify all the types over $\MM$, i.e. the elements of 
    $S_1(\MM) := S_1(\eldiag{\Si}{\MM})$.
    We know that
    \linkto{elems_of_stone_space_are_types_of_elements}{
        any type over $\MM$ is of the form}
        $\subintp{\nothing,1}{\NN}{\tp}(a)$, 
    where $\NN$ is the 
    \linkto{om_sat_elem_ext_of_models}{$\om$-saturated elementary extension} 
        of $\MM$ and $a \in \NN$.
    We show that there are three cases:
    \begin{enumerate}
        \item The element $a$ is in the image of $\MM$ if and only if 
            $\tp(a)$ is an \linkto{dfn_isolated_point}{isolated point} 
            in $S_1(\MM)$.
            Thus $\MM$ bijects with the isolated points of $S_1(\MM)$
            by taking $c \mapsto \set{\tp(io(c))}$.
        \item The element $a$ is not in the image but it equivalent under the 
            relation to something
            in the image if and only if $\tp(a)$ is isolated in $S_1(\MM)'$,
            the \linkto{dfn_derived_set}{derived set} of the Stone space.
        \item Otherwise it is an isolated point 
            in $S_1(\MM)''$ the second derived set. 
            There is exactly one such type.
    \end{enumerate}
    \[  
        S_1(\MM) = \set{\tp(a) \st a \in \MM} \sqcup 
        \set{\tp(a) \st a \notin \MM \AND \exists c \in \MM, a \sim c} \sqcup
        \set{\tp(a) \st \forall c \in \MM, a \nsim c}
    \]
    Hence $\abs{\MM} \leq \abs{\MM} + \abs{\MM} + 1 \leq \abs{S_1(\MM)}$.
    By $(1.)$, $\MM$ bijects with a subset of $S_1(\MM)$ and so 
    $\abs{\MM} = \abs{S_1(\MM)}$.
\end{eg}
\begin{proof}
    (1.) This is covered in \linkto{type_of_an_element_from_the_model}{the 
        lemma}.
    
    (2.)
    Suppose $a$ is not in the image of $\MM$ but is equivalent to 
    an element of $c \in \MM$.
    Then 
    \[\tp(a) \in \bigcap_{d \in \MM} [x \ne d] \cap [x \sim c]\]
    Notice that we cannot make the intersection a formula because 
    this would then be quantifying over $\NN$ instead of $\MM$,
    also this intersection is infinite as $\MM$ is infinite.
    To show that this is the only element of this set, 
    suppose $\tp(b)$ also satisfies 
    \[(\forall d \in \MM, x \ne d \in \tp(b)) \text{ and } x \sim c \in \tp(b)\]
    then let $\phi \in QFF(\Si(\MM),1)$.
    We case on $\phi$ to show that $\phi \in \tp(a) \iff \phi \in \tp(b)$.
    By \linkto{infinite_infinite_classes}{quantifier elimination of $T$} and the 
    \linkto{quant_elim_for_types}{types version of quantifier elimination}
    it suffices to show that $\qftp(a) = \qftp(b)$.
    (for each step we only show $\phi \in \qftp(a) \implies \phi \in \qftp(b)$
    the other direction is the same)
    \begin{itemize}
        \item If $\phi = \top$ then it is trivial.
        \item If $\phi$ is $r = s$ then as the only $\Si_E(\MM)$-terms 
            that exist are constants $d_i \in \MM$ or variables $x_i$, 
            we case on $r$ and $s$.
            If both are constants $d_0, d_1$ then since $\phi$ is in $\qftp(a)$,
            $\nnintp{d_0} = \nnintp{d_1}$ and so $\phi$ is in $\qftp(b)$.
            If one is a constant $d$ and the other a variable $x$ then 
            $\nnintp{d} = \nnintp{x}(a) = a$ which is a contradiction as 
            $x \ne d \in \qftp(a)$.
            If both are variables then as $\phi$ has at most one variable it 
            is $x = x$ which is clearly in $\tp(b)$.
        \item If $\phi$ is $r \sim s$ then we again case on what the terms are.
            If they are constants $d_0,d_1$ 
            we have $\nnintp{d_0} \nnintp{\sim} \nnintp{d_1}$ 
            and so $\phi$ is in $\tp(b)$.
            If one is a constant $d$ and the other a variable $x$ then by 
            the fact that $x \ne d \in \qftp(a) \cap \qftp(b)$ 
            \[\nnintp{d} \nnintp{\sim} \nnintp{x}(a) = a 
            \nnintp{\sim} c \nnintp{\sim} b\]
            Hence $d \sim x \in \qftp(b)$.
            If both are variables then as above it is $x \sim x$ which is in 
            $\tp(b)$.
        \item If $\phi$ is $\NEG \psi$ then $\NN \model{\Si_E(\MM)} \phi(a)$
            implies $\NN \nodel{\Si_E(\MM)} \psi(a)$ which by the induction 
            hypothesis tells us $\NN \nodel{\Si_E(\MM)} \psi(b)$, 
            whence $\phi(b) \in \qftp(b)$.
        \item If $\phi$ is $\psi \OR \chi$ then 
            $\NN \model{\Si_E(\MM)} \psi(a)$ or 
                $\NN \model{\Si_E(\MM)} \chi(a)$
            and by the induction hypothesis 
            $\NN \model{\Si_E(\MM)} \psi(b)$ or 
            $\NN \model{\Si_E(\MM)} \chi(b)$
            hence $\phi \in \qftp(b)$.
    \end{itemize}
    Thus (though $a$ is not unique) $\tp(a)$ is the unique type satisfying 
    this characterisation.
    Hence the intersection of the derived set and 
    $[x \sim c]$ is $\set{\tp(a)}$,
    as any $p$ in the derived set is (as we proved above) \emph{not} due 
    to some element from $\MM$.
    Thus these are isolated in the derived set.

    We prove a lemma:
    if $\psi$ is a formula of one variable
    then $[\psi] \cap S_1(\MM)' = [x \sim d] \cap S_1(\MM)'$ for some 
    $d \in \MM$ or 
    $S_1(\MM)' \subs [\psi]$.
    As $T$ has \linkto{infinite_infinite_classes}{quantifier elimination} 
    we can assume $\psi$ is quantifier free.
    The \linkto{disjunctive_normal_form_0}{disjunctive normal form} of $\psi$
    gives us 
    \[[\psi] = \bigcup_{i \in I} \bigcap_{j \in J_{i}} [f_{ij}(v)]\]
    There exists an $i$ such that $\bigcap_{j \in J_{i}} [f_{ij}(v)]$ 
    contains $p$ hence 
    \[  
        [\psi] \cap S_1(\MM)' =  \bigcap_{j \in J_{i}} [f_{ij}(v)] 
            \cap \bigcap_{c \in \MM} [x \ne c]
    \]
    If $f_{ij}$ is of the form $r = s$ then if $r$ and $s$ are both constant
    symbols or both variable symbols it is true and we can remove it,
    otherwise it is $x = c$ for some $c \in \MM$, which implies that $p$
    contains $x = c$ and $x \ne c$, a contradiction.
    If $f_{ij}$ is of the form $r \ne s$ then if $r$ and $s$ are both constant
    symbols or both variable symbols it is either true and we can remove it,
    or false and we have a contradiction;
    otherwise it is $x \ne c$ for some $c \in \MM$, 
    which is already given in the second intersection and we can remove it.
    If $f_{ij}$ is $r \nsim s$ then $r \ne s$ is also in $p$ so by the previous
    point we can remove it.
    Thus the only remaining case is when the $f_{ij}$ are $r \sim s$,
    again we can remove the cases where $r$ and $s$ are both constants or 
    variables and we are left with $x \sim c$ for some $c \in \MM$.
    If there is more than one of these we can remove them as they either 
    contradict one another (via transitivity) or they are redundant information
    and can be removed.
    Hence we are left with either $[\psi] \cap S_1(\MM)' = [x \sim d] 
    \cap S_1(\MM)'$ for some $d \in \MM$ or 
    $[\psi] \cap S_1(\MM)' = S_1(\MM)'$.
    In the second case $S_1(\MM)' \subs [\psi]$.

    Using the lemma we show that for any isolated point 
    $\tp(a)$ in the derived set $S_1(\MM)'$,
    there exists $d \in \MM$ such that $a \sim d$.
    If it is isolated
    \[\set{\tp(a)} = \bigcup_{\phi \in I} [\phi] \cap S_1(\MM)'\] 
    which implies 
    there is some $\psi$ such that 
    \[  
        \set{\tp(a)} = [\psi] \cap S_1(\MM)'
    \]
    By the lemma, either $\set{\tp(a)} = [x \sim d] 
    \cap \bigcap_{c \in \MM} [x \ne c]$ for some $d \in \MM$ or 
    $\set{\tp(a)} = \bigcap_{c \in \MM} [x \ne c]$.
    In the second case we have a contradiction as for two elements $d, e$in 
    distinct equivalence classes of $\MM$
    $\bigcap_{d \in \MM} [x \ne d] \cap [x \sim c]$ and 
    $\bigcap_{e \in \MM} [x \ne e] \cap [x \sim c]$ contain distinct types
    and are both subsets of $\bigcap_{c \in \MM} [x \ne c]$, 
    which is a singleton set.
    Thus there exists $d \in \MM$ such that $x \sim d \in \tp(a)$.

    (3.) There is the final case where $a$ is 
    not equivalent to anything from $\MM$ (therefore not in the image). 
    By the above two cases 
    we have that $\tp(a)$ is not isolated in $S_1(\MM)$ or $S_1(\MM)'$.
    We show that there is exactly one such type, 
    which implies $S_1(\MM)''$ consists of one isolated point.
    $\MM$ has infinitely many equivalence classes 
    and so the set $\Ga := \set{ x \nsim c \st c \in \MM}$ is finitely 
    realised in $\MM$, 
    \linkto{finite_realisation_and_embeddings}{hence consistent}
    with the elementary diagram of $\MM$, 
    which implies it can be 
    \linkto{extend_to_maximal_type_zorn}{extended to a maximal type}.
    Hence there exists a type $\tp(a)$ such that 
    $a$ is not equivalent to any thing from $\MM$.
    There is only one:
    Suppose $\tp(a)$ and $\tp(b)$ both satisfy the above.
    Let $\phi \in \tp(a)$.
    By the lemma $[\psi] \cap S_1(\MM)' = [x \sim d] \cap S_1(\MM)'$ 
    for some $d \in \MM$
    or $S_1(\MM)' \subs [\psi]$.
    The first case is false since it implies 
    $\tp(a) \in [\psi] \cap S_1(\MM)' \subs [x \sim d]$ 
    but $x \nsim d$ is in $\tp(a)$ by assumption.
    Thus $\tp(b) \in S_1(\MM)' \subs [\psi]$ and so $\psi \in \tp(b)$.
    The other direction is the same and so $\tp(a) = \tp(b)$ and 
    this point is unique.
\end{proof}

\subsection{Morley Rank}
This subsection follows Marker's \cite{marker} material again.

\begin{dfn}[Definable]
    \link{definable_set}
    Let $\MM$ be a $\Si$-structure and $A \subs \MM$.
    We say $X \subs \MM^0$ is $\Si(A)$-definable if $X$ is non-empty.
    In the non-degenerate case:
    $X \subs {\MM}^n$ is $\Si(A)$-definable over $\MM$ if there exists a 
    $\Si(A)$-formula in $n$ free variables such that 
    \[X = \set{a \in {\MM}^n \st \MM \model{\Si} \phi(a)}\]

    For a $\Si(A)$-formula $\phi$ with $n$ free variables we use $\phi(\MM)$ 
    to denote $\set{a \in \MM^n \st \MM \model{\Si(\MM)} \phi(a)}$,
    the set defined by $\phi$.
    
    Note that if $\phi$ is a sentence then 
    $\phi(\MM) := \set{\nothing}$ when $\phi$ is 
    satisfied by $\MM$ and it is empty otherwise.
    For now we only concern outselves with the case where $A = \MM$.
\end{dfn}

%? Example: definable subsets of a model of ZFC are precisely `classes' of ZFC!

\begin{dfn}[Morley rank with respect to a structure (not necessarily saturated)]
    Let $\MM$ be a $\Si$-structure, 
    let $\al$ be an ordinal.
    By transfinite induction of $\al$
    we define what it means for any $\Si(\MM)$-formula $\phi$ to satisfy 
    $\al \leq \MR{\MM}{\phi}$:
    \begin{itemize}
        \item If $\phi(\MM)$ is non-empty then $0$ is in $R$.
        \item If $\al$ is a non-zero limit ordinal and for each $\be < \al$
            and $\phi$, 
            $\be \leq \MR{\MM}{\phi}$ then $\al \leq \MR{\MM}{\phi}$.
        \item Suppose $\al + 1$ is a successor ordinal. 
            If there exists for each 
            $n \in \N$ a $\Si(\MM)$-formula $\psi_n$ satisfying 
            $\al \leq \MR{\MM}{\psi_n}$ such that the $\psi_n(\MM)$
            are pairwise disjoint subsets of $\phi(\MM)$, 
            then $\al \leq \MR{\MM}{\phi}$.
            \footnote{The relations $\al \leq \MR{\MM}{\psi_n}$ 
            is already defined by induction.
            Each $\psi_n$ has the same number of variables as $\phi$
            since we require $\psi_n(\MM) \subs \phi(\MM)$.}
    \end{itemize}

    We then define $\MR{\MM}{\phi} \in \set{-\infty,\infty} \cup \ord$, 
    the Morley rank of $\phi$:
    \begin{itemize}
        \item If for each ordinal $\al$, 
            $\al \nleq \MR{\MM}{\phi}$ then we take 
            $\MR{\MM}{\phi} := -\infty$.
        \item If for all ordinals $\al$, 
            $\al \leq \MR{\MM}{\phi}$ then we take $\MR{\MM}{\phi} := \infty$.
        \item If there exists an ordinal $\al$ 
            such that $\al \leq \MR{\MM}{\phi}$ but 
            $\al + 1 \nleq \MR{\MM}{\phi}$, 
            then $\MR{\MM}{\phi} := \al$.
    \end{itemize}
\end{dfn}
\begin{proof}
    To show that the second part of the definition is well-defined
    we need a lemma:
    By induction on $\al$ we show that if $\be \leq \al$,
    then for any $\Si(\MM)$-formula $\phi$,
    $\al \leq \MR{\MM}{\phi}$ implies $\be \leq \MR{\MM}{\phi}$.
    \begin{itemize}
        \item If $\al = 0$ then it is vaccuously true.
        \item Suppose $\al$ is a non-zero limit ordinal.
            If $\al \leq \MR{\MM}{\phi}$, $\be < \al$ and 
            $\phi \in \form{\Si(\MM)}$ then by definition of 
            $\al \leq \MR{\MM}{\phi}$ for non-zero limit ordinals
            we have $\be \leq \MR{\MM}{\phi}$.
        \item Suppose $\al$ satisfies the condition. 
            We show that $\al + 1$ satisfies the condition as well.
            Let $\phi$ be a $\Si(\MM)$-formula.
            Suppose $\al + 1 \leq \MR{\MM}{\phi}$.
            If $\be \leq \al + 1$ then either $\be = \al$ or $\be < \al$.
            It suffices to show the first case - $\al \leq \MR{\MM}{\phi}$ - 
            as by the induction hypothesis this implies that for any $\be < \al$
            we also have $\be \leq \MR{\MM}{\phi}$, 
            which covers the second case.
            By definition we have $\Si(\MM)$-formulas $\psi_n$ for each natural
            $n$ with $\psi_n(\MM)$ disjoint subset of $\phi(\MM)$ and 
            $\al \leq \MR{\MM}{\psi_n}$.
            This is not quite what we want so we 
            show $\al \leq \MR{\MM}{\phi}$ by induction on $\al$ once again.
            \begin{itemize}
                \item If $\al = 0$ then since $\psi_0(\MM) \subs \phi(\MM)$
                    and $\psi_0(\MM) \ne \nothing$ by the fact that 
                    $0 \leq \MR{\MM}{\psi_n}$, 
                    we have that $\phi(\MM) \ne \nothing$ and so 
                    $0 \leq \MR{\MM}{\phi}$.
                \item If $\al$ is a non-zero ordinal and all $\be < \al$
                    satisfy $\be \leq \MR{\MM}{\phi}$ then clearly 
                    $\al \leq \MR{\MM}{\phi}$.
                \item If $\al = \be + 1$ 
                    then by the original induction hypothesis, 
                    we have for each $n \in \N$, 
                    \[\al \leq \MR{\MM}{\psi_n} \implies 
                    \be \leq \MR{\MM}{\psi_n}\]
                    Thus $\al \leq \MR{\MM}{\phi}$.
            \end{itemize}
    \end{itemize}

    To show then that Morley rank is well-defined: 
    if the first and second cases don't hold,
    we can find the minimal element $\al \in \ord$ such that 
    $\al \nleq \MR{\MM}{\phi}$
    as \linkto{basic_facts_ordinals}{$\ord$ is well-ordered}.
    This is not a limit ordinal, since if $\be$ less than it satisfy
    $\be \leq \MR{\MM}{\phi}$ and it it a limit ordinal then 
    $\al \leq \MR{\MM}{\phi}$, a contradiction.
    Thus it must be a successor and so we can find its predecessor.
    This is unique: suppose for a contradiction that $\al < \be$ satisfy 
    \[\al \leq \MR{\MM}{\phi}, \al + 1 \nleq \MR{\MM}{\phi}, 
    \be \leq \MR{\MM}{\phi}, \be + 1 \nleq \MR{\MM}{\phi}\] 
    then \linkto{less_than_and_succ_of_ord}{$\al + 1 < \be$ or $\al + 1 = \be$},
    so using the lemma above for the first case,
    we have in either case $\al + 1 \leq \MR{\MM}{\phi}$,
    which is a contradiction.
    Thus $\al = \be$ and this ordinal is unique.
\end{proof}
\begin{rmk}
    We tacitely order $\set{-\infty,\infty} \cup \ord$ by the usual ordering on 
    $\ord$ together with $-\infty \leq$ everything $\leq \infty$.
    We can then compare the Morley rank of two formulas by $\leq$.
\end{rmk}

\begin{lem}[Morley rank for elementary extensions between saturated structures]
    \link{Morley_rank_elem_ext_lem}
    If $\MM$ and $\NN$ are both $\om$ saturated $\Si$-structures and 
    $\MM \to \NN$ is an elementary $\Si$-extension, 
    then for any $\Si(\MM)$-formula $\phi$,
    \[\MR{\MM}{\phi} = \MR{\NN}{\phi}\]
    (where in the second case $\phi$ is considered to be a $\Si(\NN)$-formula).
\end{lem}
\begin{proof}
    Again it suffices to show by induction that for each $\al \in \ord$, 
    given $\MM$ and $\NN$ and $\phi$ as above, we have
    \[\al \leq \MR{\MM}{\phi} \quad \iff \quad \al \leq \MR{\NN}{\phi}\]

    The $0$ case: \begin{align*}
        & 0 \leq \MR{\MM}{\phi}\\
        \iff & \MM \model{\Si(\MM)} \exists v, \phi(v)\\
        \iff & \NN \model{\Si(\MM)} \exists v, \phi(v)\\
        \iff & 0 \leq \MR{\NN}{\phi}
    \end{align*} 

    The non-zero limit ordinal case is standard.

    Successor:
    With the induction hypothesis for $\al$, 
    suppose $\al + 1 \leq \MR{\MM}{\phi}$
    then there exist $\psi_n$ for each $n \in \N$ such that 
    $\al \leq \MR{\MM}{\psi_n}$, 
    $\psi_n(\MM) \subs \phi(\MM)$ and these sets are pairwise disjoint.
    By the induction hypothesis we have for each $n$ that 
    $\al \leq \MR{\NN}{\psi_n}$. 
    Since the embedding is elementary
    \begin{align*}
        &\psi_n(\MM) \subs \phi(\MM)\\
        \implies & \MM \model{\Si} \forall v, \psi_n \to \phi\\
        \implies & \NN \model{\Si} \forall v, \psi_n \to \phi\\
        \implies & \psi_n(\NN) \subs \phi(\NN)
    \end{align*}
    Similarly 
    \begin{align*}
        &\psi_n(\MM) \cap \psi_{l}(\MM) = \nothing\\
        \implies & \MM \model{\Si} \forall v, \NOT (\psi_n \AND \phi_{n+1})\\
        \implies & \NN \model{\Si} \forall v, \NOT (\psi_n \AND \phi_{n+1})\\
        \implies & \psi_n(\NN) \cap \psi_{l}(\NN) = \nothing
    \end{align*}
    Hence $\al + 1 \leq \MR{\NN}{\phi}$.

    For the other direction assume the induction hypothesis and suppose 
    $\al + 1 \leq \MR{\NN}{\phi}$.
    First replace $\phi$ with a $\Si$-formula $\phi(v,a)$ at some 
    $a \in \MM^{i}$.
    For each $n \in \N$
    there exist $\Si$-formulas $\psi_n$ and $c_n \in \N^{i_n}$
    such that $\al \leq \MR{\NN}{\psi_n(v,c_n)}$ and $\psi_n(\NN, c_n)$
    are disjoint subset of $\phi(\NN,a)$.
    Since the extension is elementary $\mmintp{\tp}(a) = \nnintp{\tp}(a)$
    and by \linkto{infty_equivalence_01}{$\infty$-equivalence} for each $n$
    there exist $d_0 \in \MM^{i_0},\dots,d_n \in \MM^{i_n}$ such that 
    \[
        \nnintp{\tp}(a,c_0,\dots,c_n) = \mmintp{\tp}(a,d_0,\dots,d_n)
        = \nnintp{\tp}(a, d_0,\dots,d_n)
    \]
    Similarly to before, for each $n \in \N$
    \begin{align*}
        &\psi_n(\NN,c_n) \subs \phi(\NN,a)\\
        \iff & \NN \model{\Si} \forall v, \psi_n(v,c_n) \to \phi(v,a)\\
        \iff & \forall v, \psi_n(v,c_n) \to \phi(v,a) \in 
        \nnintp{\tp}(a,c_n) = \mmintp{\tp}(a,d_n)\\
        \iff & \MM \model{\Si} \forall v, \psi_n(v,d_n) \to \phi(v,a)\\
        \iff & \psi_n(\MM,d_n) \subs \phi(\MM,a)
    \end{align*}
    They are disjoint:
    \begin{align*}
        &\psi_n(\NN,c_n) \cap \psi_l(\NN,c_l) = \nothing\\
        \iff & \NN \nodel{\Si} \exists v, \psi_n(v,c_n) \AND \psi_l(v,c_l)\\
        \iff & \exists v, \psi_n(v,w_n) \AND \psi_l(v,w_l) \notin 
        \tp(a,c_n,c_l) = 
        \tp(b,d_n,d_l)\\
        \iff & \MM \nodel{\Si} \exists v, \psi_n(v,d_n) \AND \psi_l(v,d_l)\\
        \iff & \psi_n(\MM,d_n) \cap \psi_l(\MM,d_l) = \nothing
    \end{align*}
    Thus the induction is complete.
\end{proof}

\begin{cor}[Saturated elementary extensions give equal Morley rank]
    \link{sat_elem_ext_give_eq_Morley}
    Let $\AA,\MM,\NN$ be $\Si$-structures such that $\MM$ and $\NN$ are 
    $\om$-saturated elementary $\Si$-extensions of $\AA$. 
    Then for any $\Si(\AA)$-formula $\phi$, 
    \[\MR{\MM}{\phi} = \MR{\NN}{\phi}\]
\end{cor}
\begin{proof}
    Let $\LL$ be the \linkto{amalgamation}{amalgamation} of $\MM$ and $\NN$,
    and let $\Om$ be an 
    \linkto{om_sat_elem_ext_of_models}{$\om$-saturated elementary extension} 
    of $\LL$.
    The \linkto{category_of_structures}{composition} 
    of elementary extensions is elementary and so 
    $\MM \to \Om$ and $\NN \to \Om$ is elementary.
    Hence by the \linkto{Morley_rank_elem_ext_lem}{previous lemma}
    we have that 
    \[\MR{\MM}{\phi} = \MR{\Om}{\phi} = \MR{\NN}{\phi}\]
\end{proof}

\begin{prop}[Subset implies inequality of Morley rank]%?
    \link{implication_subset_inequality}
    Let $\phi$ and $\psi$ be two $\Si(\MM)$-formulas
    and $\NN$ be an elementary extension of $\MM$.
    If $\psi(\NN) \subs \phi(\NN)$ then
    \[\MR{\NN}{\psi} \leq \MR{\NN}{\phi}\]

    Hence if $X$ is defined by two formulas then the Morley 
    rank of the two formulas in some $\NN$ is the same.
    \[
        \phi(\NN) = X = \psi(\NN) 
        \quad \implies \quad 
        \MR{\NN}{\phi} = \MR{\NN}{\psi}
    \]
\end{prop}
\begin{proof}
    %Note that $\psi(\NN) \subs \phi(\NN)$ (see important footnote\footnote{
    %    $\psi(\NN) \subs \phi(\NN)$
    %    is not true when the sets of free variables differ.
    %    For example if $\psi$ is a sentence then it defines a point but if 
    %    $\phi$ were $x = x$ then it would define all of $\NN$.)
    %    }).
    Induct on $\al \in \set{-\infty} \cup \ord$ to show that 
    \[\al \leq \MR{\NN}{\psi} \implies \al \leq \MR{\NN}{\phi}\]

    The $-\infty$ case is clear.
    The $0$ case follows from noting that if 
    $0 \leq \MR{\NN}{\psi}$ then $\nothing \ne \psi(\NN) \subs \phi(\NN)$.
    The non-zero limit case is the same as usual.
    The successor case follows from noting that the
    $\psi_i$ given for each $i \in \N$ define subsets of $\psi(\NN)$ and 
    therefore subsets of $\phi(\NN)$.

    Now that we have that it is true for all ordinals the $\infty$ case also
    follows.

    The `hence' uses anti-symmetry of the ordering on 
    $\set{-\infty,\infty} \cup \ord$.
\end{proof}

\begin{dfn}[Morley rank in saturated extensions]
    Let $\MM$ be a $\Si$-structure and $\phi$ a $\Si(\MM)$-formula.
    Then $\MR{}{\phi} := \MR{\NN}{\phi}$ for $\NN$ any $\om$-saturated extension 
    of $\MM$.
    This is unique with respect to the choice of $\NN$ since 
    \linkto{sat_elem_ext_give_eq_Morley}{saturated elementary extensions 
    give equal Morley rank}.
    Suppose $X = \phi(\MM)$ is the subset of $\MM^n$ defined by $\phi$.
    Then we write $\MR{}{X} = \MR{}{\phi}$ to be the Morley rank of the
    $\Si(\MM)$-definable set.
    It is unique with respect to the choice of $\phi$ since
    \linkto{implication_subset_inequality}{equal 
        $\Si(\MM)$-definable sets implies equal Morley rank}.
\end{dfn}
Note that for an $\om$-saturated structure $\MM$ and $\phi$ a $\Si(\MM)$-formula
\[\MR{}{\phi} = \MR{\MM}{\phi}\]
since $\MM$ is an $\om$-saturated extension of itself.

\begin{prop}[Basic facts about Morley rank of a definable set]
    \link{basic_facts_morley_rank_of_dfnbl_set}
    Let $X$ and $Y$ be $\Si(\MM)$-definable sets in $\MM^n$ a $\Si$-structure.
    \begin{enumerate}
        \item If $X \subs Y$ then $\MR{}{X} \leq \MR{}{Y}$.
        \item $\MR{}{X} = 0$ if and only if $X$ is finite and non-empty.
        \item $\MR{}{X \cup Y} = \max \set{\MR{}{X}, \MR{}{Y}}$.
    \end{enumerate}
\end{prop}
\begin{proof}
    $(1.)$ Follows from the \linkto{implication_subset_inequality}{
        previous lemma}.

    $(2.)$ First note that $0 \leq \MR{}{X}$ if and only if the 
    formula defining $X$ is satisfied if and only if $X$ is non-empty.
    It remains to show that $1 \nleq \MR{}{X}$ if and only if 
    $X$ is finite.
    \begin{forward}
        If $X$ were infinite, we can let $\phi$ be the formula defining $X$ and 
        define formulas $\phi \AND x = a$ for each $a \in X$,
        hence having infinitely many satisfiable formulas corresponding to
        disjoint subsets of $X$.
        Hence $1 \leq \MR{}{X}$, a contradiction.
    \end{forward}

    \begin{backward}
        If $1 \leq \MR{}{X}$ then we would have infinitely many disjoint
        subsets of $X$, which implies $X$ is infinite, a contradiction.
    \end{backward}

    $(3.)$ It suffices to show that for all $X$ and $Y$ 
    $\Si(\MM)$-definable subsets of $\MM$
    if $\MR{}{X} \leq \MR{}{Y}$ then $\MR{}{Y} = \MR{}{X \cup Y}$.
    By the first part we have that $\MR{}{Y} \leq \MR{}{X \cup Y}$,
    thus we need only to show the other inequality by induction on 
    $\al \in \set{-\infty} \cup \ord$:
    \[\al \leq \MR{}{X \cup Y} \implies \al \leq \MR{}{Y}\]
    For $\al = - \infty$ then it is clear.
    For $\al = 0$ suppose $0 \leq \MR{}{X \cup Y}$. 
    Then $X \cup Y$ is non empty.
    Suppose for a contradiction $Y$ is empty then $X$ must be non-empty
    and we have that $\MR{}{Y} < \MR{}{X}$, a contradiction.
    Hence $0 \leq \MR{}{Y}$.

    The non-zero limit ordinal case is trivial as usual.
    Suppose it is true for $\al$ and suppose $\al + 1 \leq \MR{}{X \cup Y}$.
    Then there exist disjoint subsets $S_i \subs X \cup Y$ such that 
    $\al \leq \MR{}{S_i}$.
    Consider the subsets $S_i \cap X \subs X$ and $S_i \cap Y \subs Y$.
    If infinitely many $S_i \cap Y$ satisfy $\al \leq \MR{}{S_i \cap Y}$
    then we are done.
    On the other hand if only finitely many do so then by induction
    \[\al \leq \MR{}{S_i} = \max \set{\MR{}{S_i \cap X}, \MR{}{S_i \cap Y}}\]
    Hence infinitely many satisfy $\al \leq \MR{}{S_i \cap X}$.
    Thus $\al + 1 \leq \MR{}{X} \leq \MR{}{Y}$ and the induction is complete.
    
    As usual the $\infty$ case follows from the result given by the induction.
\end{proof}
Naturally parts $2.$ and $3.$ also have analogues in the formula version of 
Morley rank.

\subsection{Morley degree}
\begin{nttn}
    Let $A$ be a set.
    Then $A^{<\om}$ is the set 
    \[A^{<\om} := \bigsqcup_{n \in \N} A^n\]
    To be explicit we mean a disjoint union of the sets of maps 
    (a.k.a $n$-tuples) $n \to A$.

    For $a = (a_1,\dots,a_n) \in A^n \subs A^{<\om}$ 
    and $b \in A$ we write $a;b$ to mean
    $a$ with $b$ appended to it: 
    \[(a_1,\dots,a_n,b) \in A^{n+1} \subs A^{<\om}\]
\end{nttn}

The following lemma has been adapted to be suitable without pre-requisites.
For the general version see Proof Wiki \cite{wiki1}.
\begin{lem}[K\"{o}nig's Tree]
    \link{konigs_tree}
    Partially order $2^{<\om}$ by $t \leq s$ 
    if and only if $s$ restricts to the map
    $t$ on the domain domain of $t$.
    Let $T \subs 2^{<\om}$ be such that 
    \begin{itemize}
        \item For any $s \in T$ and $t \in 2^{<\om}$ if $t \leq s$ 
            then $t \in T$. 
            We call this property $T$ being a `binary tree'.
        \item $T$ is countably infinite.
    \end{itemize}
    Then there exists an `infinite branch' $b \in 2^\N$ 
    (i.e. $b : \N \to 2$) such that 
    for each $n \in \N$ the restriction $\res{b}{n}$ is in $T$.
\end{lem}
\begin{proof}
    By induction we construct for each $n \in \N$, $t_n \in 2^n$ such that 
    \begin{itemize}
        \item $t_n \in T$
        \item for each $i < n$, $t_{i + 1} \in D(t_i)$
        \item for each $i \leq n$, $D(t_i)$ is infinite
    \end{itemize}
    We take $t_0$ to be the empty map. 
    $T$ is non-empty (it is countably infinite), 
    the empty map is in $2^\nothing$, 
    and anything is equal to the empty map upon restriction to $\nothing$.
    Thus applying the binary tree property we have $t_0 \in T$.
    For the same reason $D(t_0) = T$ is infinite.

    Assuming by induction that we have defined $t_0, \dots, t_n$
    we assume for a contradiction that both $D(t_n;0)$ and $D(t_n;1)$ are
    finite.
    Then clearly 
    \[D(t_n) \subs \set{t_n;0\,,\, t_n;1} \cup D(t_n;0) \cup D(t_n;1)\]
    and so $D(t_n)$ is finite, a contradiction.
    Hence we pick one that has infinitely many descendents to be $t_{n+1}$.
    Taking any descendent and restricting its domain to $2^{n+1}$ we see that 
    $t_{n+1} \in T$; restricting $t_{n+1}$ to $2^{n}$ we obtain $t_{n}$.
    Hence the induction is complete.

    We take $b = \bigcup_{n \in \N} t_n$ to be the infinite branch.
    By construction each finite restriction lies in $T$.
\end{proof}

\begin{dfn}[$\al$-minimality]
    \link{al_minimality_dfn}
    Let $\M$ be an $\om$-saturated $\Si$-structure and $\al$ an ordinal.
    We say a $\Si(\M)$-formula $\phi$ is $\al$-minimal if 
    \[\MR{}{\phi} = \al \text{ and there does not exist } 
    \psi \in \form{\Si(\M)} \text{ such that } 
    \MR{}{\phi \AND \psi} = \MR{}{\phi \AND \NOT \psi} = \al\]
\end{dfn}

The following lemma gives us the existence of Morley degree,
a way to specify how many disjoint subsets of a definable set we can produce.
This also gives us a way to split up any formula into $\al$-minimal formulas
which are easier to work with.
\begin{lem}[Morley degree decomposition \cite{marker}]
    \link{morley_degree_lem}
    Let $\M$ be an $\om$-saturated $\Si$-structure.
    Let $\phi$ be a $\Si(\M)$-formula.
    Suppose that $\MR{}{\phi} = \al \in \ord$.
    Then there exist $\al$-minimal formulas $\psi_1,\dots,\psi_d$ 
    defining disjoint subsets of $\phi(\M)$
    each with $\MR{}{X_i} = \al$.
    Futhermore if $X_1,\dots,X_n$ are 
    pairwise disjoint subsets of $\phi(\M)$ 
    each with $\MR{}{X_i} = \al$ then $n \le d$.
\end{lem}
\begin{proof}
    We build a binary tree $T \subs 2^{<\om}$ and apply 
    \linkto{konigs_tree}{K\"{o}nig's tree lemma}.
    Inductively we define for each $n \in \N$ a finite `binary tree' 
    $T_n \subs 2^{\leq n}$ and $\Phi_n : T_n \to \form{\Si(\M)}$
    such that \begin{itemize}
        \item for each $i < n$, $T_{i+1} \cap 2^{\leq i} = T_i$
        \item for each $i < n$, $\res{\Phi_{i+1}}{2^{\leq i}} = \Phi_i$
        \item for each $t \in T_n$, $\MR{}{\Phi_n(t)} = \al$
    \end{itemize}
    We take $T_0 = \set{\nothing}$ and have $\Phi_n : \nothing \mapsto \phi$.

    By induction, supposing that we have defined $T_i$ and $\Phi_i$ for 
    each $i \leq n$, we define $T_{n+1}$ as
    \begin{align*}
        T_{n+1} = T_n &\cup 
        \set{t;0 \st t \in T_n \text{ is } \al-\text{minimal}}\\
        &\cup \set{t;1 \st t \in T_n \text{ is } \al-\text{minimal}}
    \end{align*}
    Then we find (using the axiom of choice) for each $t;0 \in T_{n+1}$
    a $\Si(\M)$-formula $\chi_{n+1}$ such that 
    \[
        \MR{}{\Phi_n(t) \AND \chi_{n+1}} = 
        \MR{}{\Phi_n(t) \AND \NOT \chi_{n+1}} = \al
    \]
    We define $\Phi_{n+1}$ such that it agrees upon restriction with $\Phi_n$
    and for each $t \in T_n \cap 2^{n}$,
    \[t;0 \mapsto \Phi_{n+1}(t) \AND \chi_{n+1}
    \text{ and } t;1 \mapsto \Phi_{n+1}(t) \AND \NOT \chi_{n+1}\]
    By assumption
    \[\MR{}{\Phi_{n+1}(t;0)} = \MR{}{\Phi_{n}(t) \AND \chi_{n+1}} = \al\]
    Hence $\Phi_{n+1}$ satisfies the conditions we required.
    It then makes sense to define 
    \[  
        T := \bigcup_{n \in \N} T_n \quad \text{ and } \quad
        \Phi := \bigcup_{n \in \N} \Phi_n : T \to \form{\Si(\M)} 
    \]
    $T$ is a binary tree:
    if $s \in T \cap 2^{n}$ and we have $t < s \in T$ then $0 < n$
    and so it must be that $s$ is created due to the induction step.
    Thus we have $\res{s}{s^{n-1}} \in T$ and by induction 
    $t = \res{s}{s^m} \in T$, where $m$ is power of $2$ which $t$ lies in.

    Suppose for a contradiction $T$ is infinite.
    Then 
    \[\abs{T} \leq \abs{2^{<\om}} = \abs{\bigsqcup_{n \in \N} 2^n} 
    \linkto{infinite_union_of_finite}{= \aleph_0}\]
    Hence $T$ is countably infinite and we can apply 
    \linkto{konigs_tree}{K\"{o}nig's tree lemma},
    giving us an `infinite brach' $b : \N \to 2$ such that for each $n \in \N$
    we have $\res{b}{n} \in T_n \subs T$.
    From this we obtain for each $n \in \N$ a formula $\Phi_n(\res{b}{n})$
    and define a $\Si(\M)$-formula
    \[\psi_n := \Phi_n(\res{b}{n}) \AND \NOT \Phi_{n+1}(\res{b}{n+1})\]
    These will define infinitely many disjoint subsets of $\phi(\M)$
    with Morley rank $\al$.
    Without loss of generality $\res{b}{n+1} = \res{b}{n};0$, 
    hence for each $n$:
    \begin{align*}
        \psi_n(\M) &
        = \brkt{\Phi_n(\res{b}{n}) \AND \NOT \Phi_{n+1}(\res{b}{n+1})}(\M)\\
        &= \brkt{\Phi_n(\res{b}{n}) \AND 
        (\NOT \Phi_{n}(\res{b}{n}) \OR \NOT \chi_{n+1})}(\M)\\
        &= \brkt{\Phi_n(\res{b}{n}) \AND \NOT \chi_{n+1}}(\M)
    \end{align*}
    \linkto{basic_facts_morley_rank_of_dfnbl_set}{Computing} 
    the Morley rank is then straightforward:
    \[\MR{}{\psi_n} = \MR{}{\Phi_n(\res{b}{n}) \AND \NOT \chi_{n+1}} = \al\]
    which follows from the definition of $\chi_{n+1}$.
    Furthermore to see that any two distinct $\psi_n$ define disjoint 
    subsets of $\phi$, let $n < m$, then 
    \[\Phi_{n}(\res{b}{n})(\M) \subs \Phi_{n+1}(\res{b}{n+1})(\M) \subs 
    \dots \Phi_{m}(\res{b}{m})(\M)\]
    Hence 
    $\NOT \Phi_{m}(\res{b}{m})(\M) \subs \NOT \Phi_n(\res{b}{n})(\M)$ and
    \[
        \psi_m(\M) \subs (\NOT \Phi_{m+1}(\res{b}{m+1}))(\M)
        \subs (\NOT \Phi_{n}(\res{b}{n}))(\M)
    \]
    the last of which does not intersect with $\psi_n(\M)$.
    Hence they are disjoint.
    They are subsets of $\phi(\M)$ since they are subset of 
    $\Phi_n(\res{b}{n})$ which by induction are subset of $\phi(\M)$.
    Hence $\phi$ has Morley rank $\al + 1$ which is a contradiction.
    Thus $T$ is finite.

    We consider the set of terminal nodes 
    $N = \set{t \in T \st \forall s \in T, t \nless s}$.
    Since $T$ is finite $N$ is also finite;
    we take $d$ to be the cardinality of $N$.
    Consider $\Phi(N)$, the image of $N$.
    There are three points to note about this: 
    \begin{itemize}
        \item For each $t \in N$, $\Phi(t)$ is $\al$-minimal.
        \item The members of $\Phi(N)$ define pairwise disjoint subsets of 
            $\phi(\M)$.
        \item Thus $\phi(\M)$ is equal to a disjoint union
            $\brkt{\bigor{t \in N}{} \Phi(t)}(\M) = 
            \bigsqcup_{t \in \N} \Phi(t)(\M)$
    \end{itemize}
    We complete the proof using these points first, 
    then come back and prove them.

    Let $\Si(\M)$-formulas 
    $\om_1,\dots,\om_n$ have Morley rank $\al$ and 
    define pairwise disjoint subsets of $\phi(\M)$.
    We need to show that $n \leq d$.
    Suppose for a contradiction $d < n$.
    Suppose for another contradiction 
    that for all $i \leq n$ there exists $t \in N$ such that 
    $\al \leq \MR{}{\Phi(t) \AND \om_i}$ 
    (hence $\al = \MR{}{\Phi(t) \AND \om_i}$).
    Then there exist $1 \leq i < j \leq n$ such that 
    \[\MR{}{\om_i \AND \Phi(t)} = \MR{}{\om_j \AND \Phi(t)} = \al\]
    Then 
    \[\al = \MR{}{\om_j \AND \Phi(t)} 
    \leq \MR{}{(\NOT \om_i) \AND \Phi(t)} \leq \MR{}{\Phi(t)} = \al\]
    This implies $\Phi(t) \notin N$, a contradiction.
    Then there must be some $i \leq n$ such that for all $t \in N$
    $\MR{}{\Phi(t) \AND \om_i} < \al$.
    However 
    \[
        \om_i(\M) = (\om_i \AND \phi)(\M) = 
        \brkt{\om_i \AND \bigor{t \in N}{} \Phi(t)}(\M)
        = \brkt{\bigor{t \in N}{} \om_i \AND \Phi(t)}(\M)
    \]
    Hence $\MR{}{\om_i} \linkto{basic_facts_morley_rank_of_dfnbl_set}{=} 
    \max_{t \in \N} \MR{}{\om_i \AND \Phi(t)} < \al$, a contradiction.
    Hence $n \leq d$.

    We show the three facts from before.
    The first is clear from the definition of $\N$.
    Let $s$ and $t$ be distinct elements of $N$. 
    Let $n \in \N$ be the maximal natural such that $\res{s}{n} = \res{t}{n}$.
    If $s \in 2^n$ then $s < t$ which is a contradiction.
    Thus $s, t \notin 2^n$.
    As $n$ is maximal, $\res{s}{n+1} \ne \res{t}{n+1}$ and so 
    $\Phi(\res{s}{n+1})(\M)$ is disjoint with $\Phi(\res{t}{n+1})(\M)$.
    $\Phi(s)(\M) \subs \Phi(\res{s}{n+1})(\M)$ and similarly with $t$ hence 
    $\Phi(s)(\M)$ and $\Phi(t)(\M)$ are disjoint.

    Clearly $\brkt{\bigor{t \in N}{} \Phi(t)}(\M) \subs \phi(\M)$.
    For the other direction note that for any $t \in T$ such that 
    $\exists s \in T, t < s$, 
    \[\Phi(t)(\M) = \Phi(t;0)(\M) \sqcup \Phi(t;1)(\M)\]
    Hence each element of $T$ (in $2^n$) 
    is either in $N$ or defines a set that is 
    a disjoint union due two elements in $2^{n+1} \cap T$.%
    As $T$ is finite we can find the maximal $n \in \N$ such that 
    $\exists t \in 2^n \cap T, a \in \Phi(t)(\M)$.
    If $t \in N$ then we are done.
    Suppose $t \notin N$ then $a \in \Phi(t;0)(\M) \sqcup \Phi(t;1)(\M)$
    and so $\exists s \in 2^{n+1} \cap T, a \in \Phi(s)(\M)$,
    contradicting maximality of $t$.
    Hence $a \in \bigsqcup_{t \in N} \Phi(t)(\M)$ and 
    \[\phi(\M) = \brkt{\bigor{t \in N}{} \Phi(t)}(\M) = 
    \bigsqcup_{t \in \N} \Phi(t)(\M)\]
\end{proof}

\begin{dfn}[Morley degree]
    Let $\phi$ be a $\Si(\MM)$-formula for a $\Si$-structure $\MM$.
    We natually interpret $\phi$ as a $\Si(\M)$-formula
    where $\M$ is an \linkto{om_sat_elem_ext_of_models}{
        $\om$-saturated elementary $\Si$-extension} of $\MM$.
    We define the Morley degree of $\phi$, which we write as $\mdeg (\phi)$:
    If $\MR{}{\phi} \in \ord$ then
    $\MR{}{\phi}$ is the minimal $d \in \N$ such that 
    if $X_1,\dots,X_n$ are 
    pairwise disjoint subsets of $\phi(\M)$ 
    each with $\MR{}{X_i} = \MR{}{\phi}$ then $n \le d$.
    The existence of such a $d$ comes from the \link{morley_degree_lem}{lemma}.
    If $\MR{}{\phi} = - \infty$ then $\mdeg(\phi) := -\infty$.
    If $\MR{}{\phi} = \infty$ then $\mdeg(\phi) := \infty$.

    We also write $\mdeg(\phi(\MM)):= \mdeg{(\phi)}$ for the Morley degree of 
    a $\Si(\MM)$-definable subset of $\MM$.
\end{dfn}
Note that Morley degree for formulas with ordinal Morley rank will always 
be natural numbers greater than $0$, 
since the formula itself defines a (trivially pairwise disjoint) subset.

\subsection{Morley rank and degree for types}
\begin{dfn}[Morley rank and degree for types]
    \link{morley_rank_for_types_dfn}
    Let $\MM$ be a $\Si$-structure with $A \subs \MM$.
    Let $p \in S_n(\Theory_\MM(A))$.
    Then we define the Morley rank of $p$ to be 
    \[\MR{}{p} := \min \set{\MR{}{\phi} \st \phi \in p 
    \text{ with exactly $n$ free variables}}\]
    which is well-defined since ordinals are 
    \linkto{basic_facts_ordinals}{well-ordered}.

    Note that for any $p$ (by choice) there exists $\phi \in p$ with exactly $n$ 
    free variables such that $\MR{}{\phi} = \MR{}{p}$,
    we call $\phi$ the rank representative.

    We also define Morley degree of $p$:
    If $\MR{}{p} = -\infty$ then $\mdeg(p) = -\infty$
    and if $\MR{}{p} = \infty$ then $\mdeg(p) = \infty$.
    Otherwise we take 
    \[
        \mdeg(p) := 
        \min \set{\mdeg(\phi) \st \phi \in p 
        \text{ and } \MR{}{\phi} = \MR{}{p} \text{ and } 
        \phi \text{ has exactly $n$ free variables}}
    \]
\end{dfn}

\begin{lem}[$\al$-minimal rank representative of a type]
    \link{smallest_rank_rep}
    Suppose $\M$ is an $\om$-saturated strongly minimal $\Si$-structure,
    $A \subs \M$ and $p \in S_n(\Theory_\M(A))$.
    Let $\al = \MR{}{p}$.
    Then there exists an $\al$-minimal rank representative $\phi$ of $p$.
\end{lem}
\begin{proof}
    Suppose no such $\phi$ exists, then starting with any rank representative 
    $\phi_0$ and using maximality of $p$
    there exists $\psi_0 \in p$ such that 
    \[\MR{}{\phi_0 \AND \psi_0} = \MR{}{\phi_0 \AND \NOT \psi_0} = \al\]
    We define $\phi_1$ as $\phi_0 \AND \psi_0$ 
    which by assumption is not $\al$-minimal and proceed by induction 
    to create $\phi_i$ and $\psi_i$.
    Now $\phi_i \AND \NOT \psi_i$ are infinitely 
    many formulas with Morley rank $\al$ defining disjoint subsets of $\phi_0$,
    so $\al + 1 \leq \MR{}{\phi_0}$, a contradiction.
\end{proof}

The following lemma tells us we can also go the other way.
Given a formula we can find a type with equal Morley rank 
containing the formula.
This uses the Morley degree decomposition of a formula.
\begin{lem}[Formulas are represented by types \cite{tent}]
    \link{formulas_rep_by_types}
    Let $A \subs \M$, for $\M$ an $\om$-saturated $\Si$-structure.
    If $\phi$ is a $\Si(A)$-formula with $\MR{}{\phi} \in \ord$, 
    then there exists a type $p \in S_n(\Theory_\M(A))$ such that 
    $\MR{}{\phi} = \MR{}{p}$ and $\phi \in p$.

    Hence 
    \[\MR{}{\phi} = \max \set{\MR{}{q} \st \phi \in q \in S_n(\Theory_\M(A))}\]
\end{lem}
\begin{proof}
    We first show that this holds when $\phi$ is $\al$-minimal,
    where $\al := \MR{}{\phi}$.
    We take 
    \[
        p := \set{\psi \in \linkto{dfn_types_on_theories}{F(\Si(A),n)} 
        \st \MR{}{\phi \AND \NOT \psi}} < \al
    \]
    as our $n$-type.
    To show that it is finitely consistent with $\Theory_\M(A)$
    let $\De$ be a finite subset of $p$,
    it suffices to show that the conjunction 
    $\Psi := \bigand{\psi \in \De}{} \psi$
    has Morley rank at least $\al$.
    \[
        \MR{}{\phi \AND \NOT \Psi} = 
        \MR{}{\bigor{\psi \in \De}{} \phi \AND \NOT \psi} 
        \linkto{basic_facts_morley_rank_of_dfnbl_set}{=} 
        \max_{\psi \in \De} (\MR{}{\phi \AND \NOT \psi} ) < \al
    \]
    Hence the conjunction $\Phi$ is in $p$.
    Supposing $\MR{}{\Psi} < \al$ implies
    $\MR{}{\phi \AND \Psi} < \al$ and 
    \[
        \MR{}{\phi} \linkto{basic_facts_morley_rank_of_dfnbl_set}{=} 
        \max (\MR{}{\phi \AND \Psi}, \MR{}{\phi \AND \NOT \Psi})
        < \al
    \]
    which is a contradiction.
    Hence $\al \leq \MR{}{\Psi}$ and so $p$ is finitely consistent.
    To show that it is maximal, suppose $\psi \notin p$. Then 
    \[\al \leq \MR{}{\phi \AND \NOT \psi} \leq \MR{}{\phi} = \al\]
    and by $\al$-minimality of $\phi$ we cannot have 
    $\al \leq \MR{}{\phi \AND \psi}$.
    Hence $\NOT \psi \in p$.

    Clearly $\phi \in p$, hence $\MR{}{p} \leq \MR{}{\phi}$.
    Furthermore if $\psi \in p$ then $\al \nleq \MR{}{\phi \AND \NOT \psi}$
    hence 
    \[
        \MR{}{\phi} \linkto{basic_facts_morley_rank_of_dfnbl_set}{=} 
        \max (\MR{}{\phi \AND \psi}, \MR{}{\phi \AND \NOT \psi})
        = \MR{}{\phi \AND \psi} \leq \MR{}{\psi}
    \]
    hence $\MR{}{p} = \MR{}{\psi}$.

    Now we can do the case without the $\al$-minimal hypothesis.
    Write $\al \in \ord$ for the Morley rank and 
    $d \in \N$ the Morley degree of $\phi$.
    \linkto{morley_degree_lem}{There exist $\al$-minimal formulas}
    $\psi_1,\dots,\psi_d$ that partition $\phi(\M)$,
    each with $\MR{}{\psi_i} = \al$.
    By the first part we have for $\psi_1$ a maximal type 
    $p \in S_n(\Theory_\M(A))$ such that $\psi_1 \in p$ and 
    $\MR{}{p} = \MR{}{\psi_1} = \al$.
    In fact $\phi \in p$ as well since $\psi_1(\M) \subs \phi(\M)$
    so we have found the type that we required.

    The `hence' follows immediately.
\end{proof}

%? Where to put?
\subsection{Constructable sets}
\begin{dfn}[Constructable]
    Let $\MM$ be a $\Si$-structure.
    The set of constructable subsets of $\MM^n$ are defined by:
    \begin{itemize}
        \item[$\vert$] If $\phi$ is an atomic $\Si(\MM)$-formula with up to 
            $n$ free variables then $\phi(\MM) \subs K^n$ is constructable.
        \item[$\vert$] If $X \subs \MM^n$ is constructable 
            then $\MM^n \setminus X$ is constructable. 
        \item[$\vert$] If $X,Y \subs \MM^n$ are constructable then $X \cup Y$
            is constructable. 
    \end{itemize}
    Thus these are `finite boolean combinations' of sets 
    $\Si(\MM)$-definable by atomic 
    formulas, i.e. sets $\Si(\MM)$-definable by a quantifier free formula.
\end{dfn}

\begin{prop}[Constructable is definable]
    \link{definable_is_constructable_01}
    Let $T$ be a $\Si$-theory with quantifier elimination and 
    $\MM$ be a $\Si$-model of $T$.
    Then subsets of $\MM^n$ are constructable if and only if they are 
    $\Si(\MM)$-definable by a quantifier free formula if and only if they 
    are $\Si(\MM)$-definable.
\end{prop}
\begin{proof}
    We only show the first equivalence as the second is trivial.
    \begin{forward}
        Suppose $X \subs \MM^n$ is constructable.
        Then we induct on $X$:
        \begin{itemize}
            \item If $X$ is defined by an atomic formula then 
                there is nothing to show.
            \item If $X$ is $\MM^n \setminus Y$ and $Y$ is constructable and 
                by induction $Y$ is $\Si(\MM)$-definable 
                then we take the negation of its defining formula.
            \item If $X$ is $Y \cup Z$, both constructable and 
                by induction $Y,Z$ are both $\Si(\MM)$-definable 
                then we take the `or' of their defining formulas.
        \end{itemize}
    \end{forward}
    \begin{backward}
        Suppose $X$ is $\Si(\MM)$-definable by the quantifier free formula 
        $\phi(v,b)$ with $b \in \MM^m$ for some $m$.
        We induct on what $\phi$ is:
        \begin{itemize}
            \item If $\phi$ is atomic then $X$ is constructable.
            \item If $\phi$ is $\NOT \psi$ and $\psi(\MM,b)$
                is constructable then $X = \MM^n \setminus \psi(\MM,b)$
                hence it is constructable.
            \item If $\phi$ is $\psi \OR \chi$ then $X$ is the union of 
                the sets $\psi(\MM,b)$ and $\chi(\MM,b)$ which are 
                constructable by induction. 
                The union of constructable is constructable.
        \end{itemize}
    \end{backward}
\end{proof}

%\begin{eg}[Infinite infinite equivalence classes re-revisited]
%    \link{infinite_infinite_classes2}
%    Consider again again the theory $T$ of 
%    \linkto{infinite_infinite_classes}{infinite infinite equivalence classes},
%    and $\MM$ a non-empty $\Si_E$-model of $T$. 
%    As an exercise we try to compute the Morley rank of some 
%    $\Si_E(\MM)$-formulas.
%
%    Say we are interested in computing the Morley rank of $x=x$.
%    We shown that $2 \leq \MR{\MM}{x = x}$ by first taking 
%    $\psi_i$ to be $x \sim a_i$ such that the $a_i$ are mutually dissimilar
%    (for each $i \ne k$, $a_i \nsim a_k$).
%    These $\psi_i$ should have Morley rank $1$ or greater: take 
%    $\psi_{ij}$ to be $x = b_{ij}$ where the $b_{ij}$ are all similar to $a$
%    but mutually distinct ($b_{ij} \sim a_i$ and $b_{ij} \ne b_{ik}$).
%    These formulas define singleton sets, hence they have Morley rank $0$.
%    Hence the $1 \leq \MR{\MM}{\psi_i}$ and $2 \leq \MR{\MM}{x = x}$.
%
%    I conjecture that $3 \nleq \MR{\MM}{x = x}$.
%    This requires some knowledge of the disjoint 
%    non-empty $\Si(\MM)$-definable subsets of $\MM$.
%    By quantifier elimination%?
%    $\Si(\MM)$-definable 
%    is constructable so we first consider sets defined by atomic
%    formulas (the `generators').%?%?
%
%    If the atomic formula is true regardless of what we evaluate at:
%    $x = x, c = c, x \sim x, c \sim c, c \sim d$ 
%    ($c$ and $d$ represent constants that are interpreted distinct but similar 
%    in $\MM$,
%    $c$ and $e$ represent constants that are dissimilar.)
%    then the set they define is $\MM$.
%    If the atomic formula is false regardless of what we pick:
%    $c = d, c \sim e$
%    then they define the empty set.
%    The remaining cases are $x = c$ which corresponds to a singleton set
%    and $x \sim c$ which gives an infinite equivalence class.
%
%    Now suppose $3 \leq \MR{\MM}{x = x}$.
%    We then have for each $i \in \N$, $\psi_i$ such that 
%    $\psi_i(\MM)$ are disjoint subsets of $\MM$
%    and $2 \leq \MR{\MM}{\psi_i}$.
%    Each $\psi_i(\MM)$ then has disjoint subsets $\chi_{ij}(\MM)$ such that 
%    $1 \leq \MR{\MM}{\chi_{ij}}$.
%    By disjunctive normal form%?
%    we have that each $\chi_{ij}(\MM)$ is a finite union of intersections of 
%    atomically defined sets and their negations.
%    We can ignore all instances of `true' and `false' atomic formulas 
%    since 
%    \[X \cap \nothing = \nothing, X \cup \nothing = X, X \cap \MM = X,
%    X \cup \MM = \MM\]
%    and $\chi_{ij}$ must be non-empty and a proper subset of $\MM$.
%    Hence it is a finite union of intersections of 
%    singleton sets, equivalence classes and their complements. 
%    \[\chi_{ij}(\MM) = \bigcup_{\al \in I} \bigcap_{\be \in J_i} \om_{\al\be}\]
%    Since $1 \leq \MR{\MM}{\chi_{ij}}$, $\chi_{ij}(\MM)$ 
%    \linkto{basic_facts_morley_rank_of_dfnbl_set}{must be infinite},
%    thus there exists $\al \in I$ such that 
%    $\bigcap_{\be \in J_i} \om_{\al\be}$ is infinite,
%    ruling out the presence of singleton sets (but not their complements).
%    What does this all mean??%?
%\end{eg}
%
%%?Where to put?
%\begin{lem}[Morley rank for formulas at elements of the same type]
%    Let $\MM$ be a $\om$-saturated $\Si$-structure.
%    Let $\phi(v,w)$ be a $\Si$-formula.
%    Let $a, b \in \MM^m$ satisfy 
%    \[\subintp{\nothing}{\MM}{\tp}(a) = \subintp{\nothing}{\MM}{\tp}(b)\]
%    then $\MR{\MM}{\phi(v,a)} = \MR{\MM}{\phi(v,b)}$, 
%    where $\phi(v,a),\phi(v,b)$ 
%    are seen as $\Si(\MM)$-formulas with $n$ variables.
%\end{lem}
%\begin{proof}
%    It suffices to show by induction on $\al$ that for all $\al \in \ord$,
%    for any $\Si$-formula $\phi(v,w)$, if $a, b \in \MM^m$ satisfy 
%    \[\subintp{\nothing}{\MM}{\tp}(a) = \subintp{\nothing}{\MM}{\tp}(b)\]
%    then 
%    $\al \leq \MR{\MM}{\phi(v,a)} \iff \al \leq \MR{\MM}{\phi(v,b)}$.
%
%    If $\al = 0$ then since 
%    $\subintp{\nothing}{\MM}{\tp}(a) = \subintp{\nothing}{\MM}{\tp}(b)$
%    we have 
%    \[\MM \model{\Si} \exists v, \phi(v,a) 
%        \iff \MM \model{\Si} \exists v, \phi(v,b)\]
%    Hence 
%    \begin{align*}
%        &0 \leq \MR{\MM}{\phi(v,a)} \quad \iff \quad 
%        \phi(v,a)(\MM) \ne \nothing\\
%        \iff \quad & \MM \model{\Si} \exists v, \phi(v,a) \quad \iff \quad
%        \MM \model{\Si} \exists v, \phi(v,b) \\
%        \iff \quad & \phi(v,b)(\MM) \ne \nothing \quad \iff 
%        \quad 0 \leq \MR{\MM}{\phi(v,a)}
%    \end{align*}
%    
%    If $\al$ is a non-zero limit ordinal and all $\be$ less than $\al$ satisfy
%    the condition then 
%    \begin{align*}
%        &\al \leq \MR{\MM}{\phi(v,a)}\\
%         \iff 
%        &\forall \be < \al, \be \leq \MR{\MM}{\phi(v,a)}\\
%        \iff & \forall \be < \al, \be \leq \MR{\MM}{\phi(v,b)}\\
%        \iff & \al \leq \MR{\MM}{\phi(v,b)}
%    \end{align*}
%
%    For the succesor ordinal case by symmetry it suffices to show that if 
%    $\al$ satisfies the condition and $\al + 1 \leq \MR{\MM}{\phi(v,a)}$
%    then $\al + 1 \leq \MR{\MM}{\phi(v,b)}$.
%    As $\al + 1 \leq \MR{\MM}{\phi(v,a)}$, 
%    we have for each $n \in \N$ a $\Si(\MM)$-formula $\psi_n(v,c_n)$ 
%    (where $\psi_n \in \form{\Si}$ and $c_n \in \MM^{i_n}$) satisfying 
%    $\al \leq \MR{\MM}{\psi_n(v,c_n)}$ such that the $\psi_n(\MM,c_n)$
%    are pairwise disjoint subsets of $\phi(\MM)$.
%    For $n \in \N$, since $\MM$ is $\om$-sturated we can apply 
%    the `forth' property of 
%    \linkto{infty_equivalence_01}{$\infty$-equivalence} for 
%    $\MM$ with itself $i_0 + \dots + i_n$ times, 
%    using the fact that the types of $a$ and $b$ are equal in the first step:
%    \[\exists d_0 \in \MM^{i_n},\dots,
%    \exists d_n \in \MM^{i_n}, 
%    \subintp{\nothing}{\MM}{\tp}(a,c_0,\dots,c_n) = 
%    \subintp{\nothing}{\MM}{\tp}(b,d_0,\dots,d_n)\]
%    In particular we have for each $n$
%    \[
%        \subintp{\nothing}{\MM}{\tp}(c_n) = 
%        \subintp{\nothing}{\MM}{\tp}(d_n)
%    \]
%    and by the induction hypothesis 
%    \[\al \leq \MR{\MM}{\psi_n(v,c_n)} \text{ if and only if }
%    \al \leq \MR{\MM}{\psi_n(v,d_n)}\]
%    It remains to show that the $\psi_n(\MM,d_n)$ 
%    are disjoint subsets of $\phi(\MM,b)$.
%    Indeed, for each $n \in \N$
%    \begin{align*}
%        &\psi_n(\MM,c_n) \subs \phi(\MM,a)\\
%        \iff & \MM \model{\Si} \forall v, \psi_n(v,c_n) \to \phi(v,a)\\
%        \iff & \forall v, \psi_n(v,c_n) \to \phi(v,a) \in 
%        \subintp{\nothing}{\MM}{\tp}(a,c_n) = 
%        \subintp{\nothing}{\MM}{\tp}(b,d_n)\\
%        \iff & \MM \model{\Si} \forall v, \psi_n(v,d_n) \to \phi(v,b)\\
%        \iff & \psi_n(\MM,d_n) \subs \phi(\MM,b)
%    \end{align*}
%    To show that they are disjoint:
%    \begin{align*}
%        &\psi_n(\MM,c_n) \cap \psi_l(\MM,c_l) = \nothing\\
%        \iff & \MM \nodel{\Si} \exists v, \psi_n(v,c_n) \AND \psi_l(v,c_l)\\
%        \iff & \exists v, \psi_n(v,w_n) \AND \psi_l(v,w_l) \notin 
%        \tp(a,c_n,c_l) = \tp(b,d_n,d_l)\\
%        \iff & \MM \nodel{\Si} \exists v, \psi_n(v,d_n) \AND \psi_l(v,d_l)\\
%        \iff &\psi_n(\MM,d_n) \cap \psi_l(\MM,d_l) = \nothing
%    \end{align*}
%    Thus the induction is complete.
%\end{proof}
%