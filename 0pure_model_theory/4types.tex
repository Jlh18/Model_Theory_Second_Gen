\section{Types}
This section mainly follows material from Tent and Ziegler's book \cite{tent}.
\subsection{Types on theories}
\begin{dfn}[$F(\Si,n)$ and formulas consistent with a theory]
    \link{dfn_types_on_theories}
    Let ${v_1,\dots v_n}$ be variables,
    $T$ be a $\Si$-theory.
    Let $F(\Si,n)$ be the set of $\Si$-formulas with at most 
    $v_1,\dots v_n$ as their free variables.
    For any $c \in \const{\Si}^n$, $p \subs F(\Si,n)$
    we write 
    \[p(c) = \set{\phi(c) \st \phi \in p}\]
    and if $\MM$ is a $\Si$-structure with $a \in {\MM}^n$ we write 
    \[\MM \model{\Si} p(a)\]
    to mean for every $\phi \in p$, $\MM \model{\Si} \phi(a)$.

    We say $p \subs F(\Si,n)$ is consistent with $T$ if there exists 
    $\MM \model{\Si} T$ and $a \in {\MM}^n$
    such that $\MM \modelsi p(a)$. 

    We say $p \subs F(\Si,n)$ is finitely consistent with $T$ 
    if for any finite subset $\De \subs p$, 
    $\De$ is consistent with $T$.

    We say $p$ is a maximal if for any $\phi \in F(\Si,n)$,
    $\phi \in p$ or $\NOT \phi \in p$.
\end{dfn}

\begin{lem}[Equivalent definition of consistency]
    \link{equiv_def_of_consistent_with_theory}
    Let $T$ be a $\Si$-theory and $p$ be a subset of $F(\Si,n)$.
    Let $c_1,\dots,c_n$ be new constant symbols and let $\Si(c)$ be 
    the signature with these added constant symbols.
    The following are equivalent:
    \begin{itemize}
        \item $T \cup p(c)$ is consistent in $\Si(c)$.
            (Where $p(c)$ is the formulas of $p$ with the variables substituted 
            by $c_1,\dots,c_n$.)
        \item $p$ is consistent with $T$.
    \end{itemize}
\end{lem}
\begin{proof}
    \begin{forward}
        Suppose we have a $\Si(c)$-structure 
        $\MM \model{\Si(c)} T \cup p(c)$.
        Then by taking the images of the interpretation of each $c_i$ in $\MM$
        we obtain $a = \modintp{\MM}{c} \in {\MM}^n$ such that 
        $\MM \model{\Si(c)} p(a)$.
        \linkto{move_down_mod}{Moving this down to $\Si$} preserves 
        satisfaction of $p(a)$  as elements of $p(a)$ are $\Si$-formulas
        with values in $\MM$ (and $T$ for the same reason):
        \[\MM \model{\Si} T \cup p(a)\]
        and we have what we want.
    \end{forward}

    \begin{backward}
        Suppose we have $\MM \model{\Si} T$ and $a \in {\MM}^n$
        such that $\MM \modelsi p(a)$.
        We can make $\MM$ a $\Si(c)$-structure such that 
        everything from $\Si$ is interpreted in the same way 
        and each constant symbol $c_i$ is interpreted as $a_i$.
        Thus \linkto{move_up_mod}{$\MM \model{\Si(c)} T$} and for any 
        $\phi(c) \in p(c)$,
        \[\MM \model{\Si} \phi(a) \implies \MM \model{\Si(c)} \phi(a)
        \implies \MM \model{\Si(c)} \phi(c)\]
        as $c$ is interpreted as $a$.
        Hence $\MM \model{\Si(c)} T \cup p(c)$ 
        and $T \cup p(c)$ is consistent in $\Si(c)$.
    \end{backward}
\end{proof}

\begin{lem}[Compactness for types]
    \link{compactness_for_types}
    Let $T$ be a $\Si$-theory and $p$ be a subset of $F(\Si,n)$.
    Let $c_1,\dots,c_n$ be new constant symbols and let $\Si(c)$ be 
    the signature with these added constant symbols.
    The following are equivalent:
    \begin{itemize}
        \item $T \cup p(c)$ is consistent in $\Si(c)$.
            (Where $p(c)$ is the formulas of $p$ with the variables substituted 
            by $c_1,\dots,c_n$.)
        \item $p$ is consistent with $T$.
        \item $p$ is finitely consistent with $T$.
    \end{itemize}
\end{lem}
\begin{proof}
    We use \linkto{equiv_def_of_consistent_with_theory}{$1. \iff 2.$} 
    to show $2. \iff 3.$

    ($2. \iff 3.$) \begin{align*}
        &p \text{ consistent with } T \\
        &\iff T \cup p(c) \text{ consistent in } \Si(c) 
        \quad \text{by (1. $\iff$ 2.)}\\
        &\iff \text{for any finite } \De(c) \subs p(c), 
        T \cup \De(c) \text{ consistent in } \Si(c) 
        \quad \text{by \linkto{compactness}{compactness}}\\
        &\iff \text{for any finite } \De \subs p, 
        T \cup \De(c) \text{ consistent in } \Si(c) \\
        &\iff \text{for any finite } \De \subs p, 
        \De \text{ consistent with } T
        \quad \text{by (1. $\iff$ 2.)}
    \end{align*}
\end{proof}

\begin{dfn}[Types on a theory]
    Let $T$ be a $\Si$-theory.
    Any subset $p \subs F(\Si,n)$ that is consistent with $T$
    is called an $n$-type on $T$.
\end{dfn}

\begin{dfn}[Stone space of a theory]
    Let $T$ be a $\Si$-theory.
    Let the stone space of $T$, 
    $S_n(T)$ be the set of all maximal $n$-types on $T$.
    (The signature of the $n$-types of the on $T$ is implicit, 
    given by the signature of $T$.)
    We give a topology on $S_n(T)$ by specifying an open basis;
    $U \subs S_n(T)$ is an element of the basis when there exists 
    $\phi \in F(\Si,n)$ such that 
    \[U = [\phi]_T := \set{p \in S_n(T) \st \phi \in p}\]
\end{dfn}

\begin{prop}[Extending to maximal $n$-types (Zorn)]
    \link{extend_to_maximal_type_zorn}
    Any $n$-type can be extended to a maximal $n$-type.
\end{prop}
\begin{proof}
    Let $T$ be a theory and $p$ be a $n$-type.
    Order by inclusion the set
    \[ 
        Z = \set{q \in S_n(T) \st q \text{ is an } n\text{-type and }
        p \subs q}
    \]
    This is non-empty as it contains $p$.
    Let $p_0 \subs p_1 \subs \dots$ be a chain in $Z$.
    Then $m = \bigcup_{i \in \N} p_i$ is finitely consistent with $T$
    (by taking large enough $i$)
    and so is consistent with $T$. 
    By Zorn we have the existence of a maximal element $q$ in $Z$.
    To show that $q$ is a maximal $n$-type let $\phi \in F(\Si,n)$. 
    As $q$ is consistent with $T$ there exists a $\Si$-structure 
    $\MM \model{\Si} T$ and $a \in {\MM}^n$ such that 
    $\MM \model{\Si} q(a)$. 
    In the case that $\MM \model{\Si} \phi(a)$
    we have $q \cup \set{\phi}$ is consistent with $T$ and so by 
    maximality $\phi \in q$.
    In the other case $q \cup \set{\NOT \phi}$ is consistent and so
    $\NOT \phi \in q$.
\end{proof}

\begin{prop}[Basic facts about the basis elements]
    \link{basic_facts_basis_elems}
    Let $T$ be a $\Si$-theory, $\phi, \psi \in F(\Si,n)$.
    \begin{itemize}
        \item $(\NOT \phi) \in p$ if and only if $p \notin [\psi]_T$.
        \item $[\phi]_T = [\psi]_T$ if and only if $\phi$ and $\psi$ are 
            equivalent modulo $T$.

        The basis elements are closed under Boolean operations
        \item $[\top]_T = S_n(T)$
        \item $[\NOT \phi]_T = S_n(T) \setminus [\phi]_T$
        \item $[\phi \OR \psi]_T = [\phi]_T \cup [\psi]_T$
        \item $[\bot]_T = \nothing$
        \item $[\phi \AND \psi]_T = [\phi]_T \cap [\psi]_T$
    \end{itemize}
\end{prop}
\begin{proof}~
    \begin{itemize}
        \item Suppose $(\NOT \phi) \in p$.
            Then if $p \in [\phi]_T$ then since $p$ is consistent with $T$
            there exists a model $\MM$ and $a$ from 
            $\MM$ such that
            $\MM \modelsi \phi(a)$ and $\MM \nodel{\Si} \phi(a)$, 
            a contradiction.
            For the other direction, 
            $p \notin [\psi]_T$ and so $ \psi \notin p$ and
            by maximality $\NOT \phi \in p$.
        \item \begin{forward}
            Suppose for a contradiction
            $T \nodel{\Si} \forall v, \brkt{\phi \iff \psi}$.
            then \linkto{not_a_consequence}{there 
            exists $\MM \model{\Si} T$} and $a \in {\MM}^n$ such that 
            $\MM \modelsi \phi \AND \NOT \psi$ or 
            $\MM \modelsi (\NOT \phi) \AND \psi$.
            In the first case we have that $\set{\phi, \NOT \psi}$ 
            is consistent with $T$ and so can be 
            \linkto{extend_to_maximal_type_zorn}{extended
            to a maximal $n$-type $p$}.
            Thus $p \in [\phi]_T = [\psi]_T$ and $p \notin [\psi]_T$, 
            a contradiction.
            \end{forward}
            \begin{backward}
                Suppose $T \model{\Si} \forall v, \brkt{\phi \iff \psi}$.
                Let $p \in [\phi]_T$.
                It suffices to show that $p \in [\psi]_T$.
                Since $p$ is consistent with $T$ there exists a $\Si$-structure
                $\MM \model{\Si} T$ and $a \in {\MM}^n$ such that 
                $\MM \model{\Si} p(a)$.
                By assumption $\MM \model{\Si} \brkt{\phi \iff \psi}(a)$
                and $p \in [\phi]_T$ so 
                $\MM \model{\Si} \psi(a)$.
                Suppose $p \notin \psi$, then $\NOT \psi \in p$ hence 
                we have a contradiction.
            \end{backward}
        \item For any maximal $n$-type $p$,
            either $\top$ or $\bot$ is in $p$ and in the latter case we have a 
            contradiction as $p$ is consistent with $T$.
        \item $[\NOT \phi]_T = S_n(T) \setminus [\phi]_T$ follows from the 
            first point.
        \item $p \in [\phi \OR \psi]_T$ if and only if 
            $(\phi \OR \psi) \in p$. 
            Suppose $\phi \notin p$ then by maximality 
            $(\NOT \phi) \in p$ and so $p \in [\psi]_T$. 
            In the other case $p \in [\phi]_T$.
            For the other direction $p \in [\phi]_T \cup [\psi]_T$
            implies $\phi \in p$ or $\psi \in p$.
            In the first case we have $\MM \model{\Si} T$ such that 
            $\MM \model{\Si} \phi$.
            Then $\MM \model{\Si} \phi \OR \psi$ and so $(\phi \OR \psi) \in p$.
    \end{itemize}
    We omit the last two parts.
\end{proof}

\begin{prop}[Properties of the Stone space]
    \link{properties_of_stone_space}
    Let $T$ be a theory.
    \begin{itemize}
        \item Elements of the basis of $S_n(T)$ are clopen.
        \item $S_n(T)$ is Hausdorff.
        \item $S_n(T)$ is compact.
    \end{itemize}
\end{prop}
\begin{proof}~
\begin{itemize}
    \item By maximality of each $p$ the complement of $U$ is also 
        in the open basis:
        \[  
            \set{p \in S_n(T) \st \phi \notin p}
            = \set{p \in S_n(T) \st (\NOT \phi) \in p}
        \]
        Hence each element of the basis is clopen.
    \item Let $p,q \in S_n(T)$ and suppose $p \ne q$. 
        By maximality and the fact that $\form{\Si}$ is non-empty we can assume
        without loss of generality that there is $\phi \in p \setminus q$.
        Again by maximality $(\NOT \phi) \in q$, 
        and so $p \in [\phi], q \in [\NOT \phi]_T$.
        These opens are disjoint:
        if $r \in [\phi]_T \cap [\NOT \phi]_T$ then as 
        $r$ is consistent with $T$, there exists 
        $\MM \model{\Si} T$ such that 
        $\MM \model{\Si} \phi$ and $\MM \model{\Si}(\NOT \phi)$
        a contradiction. 
    \item Let $C$ be a collection of closed sets with finite intersection
        property.
        Then each closed set can be written as an intersection of
        basis elements (a finite union of closed sets is still a basis element
        since \linkto{basic_facts_basis_elems}{$[\phi]\cup[\psi]_T = 
            [\phi \OR \psi]_T$}):
        \[C = \set{\bigcap_{\phi \in \al}[\phi]_T \st \al \in I}\]
        Let 
        \[\Ga = \set{\phi \st \phi \in \al \in I} \quad 
        \text{ and } \quad [\Ga]_T = \set{[\phi]_T \st \phi \in \al \in I}\]
        Then the intersection any finite subset of $[\Ga]_T$ is non-empty as it
        contains a finite intersection of elements in $C$.
        Thus for any finite subset $\De \subs \Ga$ 
        there exists $p \in S_n(T)$ such that $\De \subs p$,
        and as $p$ is consistent with $T$ so is $\De$.
        Hence $\Ga$ is finitely consistent with $T$ and by 
        \linkto{compactness_for_types}{$\Ga$} is consistent with $T$.
        \linkto{extend_to_maximal_type_zorn}{Extending $\Ga$ to a 
        maximal $n$-type} $q$ gives us $\phi \in q$ for every $\phi \in \Ga$.
        Hence for all $\al \in I$ and for all $\phi \in \al$, $p \in [\phi]_T$
        and the intersection of $C$ is non-empty.
\end{itemize}\end{proof}

Stone space is meant to have a geometric interpretation as 
\linkto{prime_spec_zariski_top}{$\spec(\MM[x_1,\dots,x_n])$}
when $\MM$ is an algebraically closed field.
We will this in a few results.

\subsection{Types on structures}
\begin{dfn}[Realisation]
    Let $\MM$ be a $\Si$-structure and $A \subs {\MM}$.
    %Let $v = (v_1, \dots, v_n)$ be a list of variables.
    %Then the `set of $\Si(A)$-formulas in $n$ variables' is
    %\[F(\Si(A),n) = \set{\phi \in \form{\Si(A)} \st
    %\text{ its free variables are } v_1, \dots, v_n}\]
    %We can also see this as 
    %\[\set{\phi(v, b_\phi) \st \exists m \in \N, phi \in \form{\Si} 
    %\text{ with free variables } v, w_1, \dots, w_m \text{ and }
    %b_\phi \in A^m}\]
    %Any subset of this - usually denoted as $p$ - is called an $n$-type
    %on $\MM$ over $A$.
    Let $p$ be a subset of $F(\Si(A),n)$
    (we will often be considering the $n$-types on $\eldiag{\Si}{\MM}$,
    a special case of this where $A = {\MM}$).
    Let $\NN$ be a $\Si(A)$-structure.
    \begin{itemize}
        \item $p$ is realised in $\NN$ by $a \in {\NN}^n$ over $A$ if
            \[\NN \model{\Si(A)} p(a)\]
            We also just say $p$ is realised in $\NN$.
            If $p$ is not realised in $\NN$
            then we say $\NN$ omits $p$.
        \item $p$ is finitely realised in $\NN$ 
            over $A$ if for each finite subset $\De \subs p$ 
            there exists $a \in {\NN}^n$
            such that $\De$ is realised in $\MM$ by $a$.
        %\item $p$ is maximal (`complete' in other sources) if it is 
        %    finitely realised in $\NN$ over $A$ and for any $\phi \in F(\Si(A),n)$,
        %    $\phi \in p$ or $\NOT \phi \in p$.
    \end{itemize}
\end{dfn}

\begin{lem}[Finite realisation and embeddings]
    \link{finite_realisation_and_embeddings}
    Let $\MM$ be a $\Si$-structure, 
    $A$ a subset of ${\MM}$ and $p$ a subset of $F(\Si(A),n)$.
    Then the following are equivalent 
    \begin{itemize}
        \item $p$ is consistent with $\eldiag{\Si}{\MM}$ 
            (i.e. it is an $n$-type over $\eldiag{\Si}{\MM}$).
        \item There exists an elementary embedding $\MM \to \NN$ 
            and a $b \in {\NN}^n$ such that $p$ 
            is realised by $b$ in $\NN$.
        \item There exists an elementary embedding $\MM \to \NN$ 
            such that $p$ is finitely realised in $\NN$.
        \item $p$ is finitely realised in $\MM$.
    \end{itemize}
    The elementary embeddings can be seen as both $\Si$-embeddings or 
    $\Si(A)$-embeddings for any subset $A \subs \MM$.
\end{lem}
\begin{proof}
    $(1. \implies 2.)$ If there exists $\NN$ and a
    $b \in {\NN}^n$ such that 
    $\NN \model{\Si(\MM)} \eldiag{\Si}{\MM}$ and 
    $\NN \model{\Si(\MM)} p(b)$.
    Then since \linkto{elem_ext_equiv_eldiag_model}{models of the 
    elementary diagram correspond to elementary extensions}, 
    there exists an elementary $\Si(\MM)$-embedding $\MM \to \NN$ 
    and $b \in {\NN}^n$ such that $p$ 
    is realised by $b$ in $\NN$.
    (This can be moved down to being a $\Si(A)$-embedding for any 
    subset $A \subs \MM$.)

    $(2. \implies 3.)$ Let $\De \subs p$ be finite.
    Then for the same embedding into $\NN$ we can see that $\De$ 
    is realised by $b$ in $\NN$.

    $(3. \implies 4.)$ Let $\De \subs p$ be finite.
    Then by assumption there exists an elementary $\Si(A)$-embedding 
    $\io : \MM \to \NN$ and $b \in {\NN}^n$ 
    such that  
    $\NN \model{\Si(A)} \De(b)$.
    Choose the constant symbols $c_1,\dots,c_m$ from $\Si(\MM)$ such that
    $\modintp{\NN}{c_i} = b_i$ for each $i$.
    We take $a = (\mmintp{c_1},\dots,\mmintp{c_n})$ to realise $\De$.
    Note that since embeddings commute with interpretation of constants,
    $\io(a) = \io(\mmintp{c}) = \nnintp{c} = b$ and since the 
    embedding is elementary 
    \[\NN \model{\Si(A)} \phi(b) 
    \implies \NN \model{\Si(A)} \phi(\io(a)) 
    \implies \MM \model{\Si(A) \phi(a)}\]

%    $(1. \implies 4.)$ By \linkto{compactness_for_types}{compactness for types},
%    $2.$ is equivalent to $p$ 
%    being finitely consistent with the elementary diagram.
%    We show that this is equivalent to $p$ being finitely realised on $\MM$.
%
%    \begin{forward}
%        Let $\De \subs p$ be finite.
%        Then by assumption there exists
%        $\NN$ and a $b \in {\NN}^n$ 
%        such that 
%        $\NN \model{\Si(\MM)} \eldiag{\Si}{\MM}$ and 
%        $\NN \model{\Si(\MM)} \De(b)$ 
%        (we \linkto{move_down_mod}{move $\NN$ down to $\Si$}).
%        Choose the constant symbols $c_1,\dots,c_m$ from $\Si(\MM)$ such that
%        $\modintp{\NN}{c_i} = b_i$ for each $i$.
%        We take $a = (\mmintp{c_1},\dots,\mmintp{c_n})$ to realise $\De$.
%        \linkto{elem_ext_equiv_eldiag_model}{models of the 
%        elementary diagram correspond to elementary extensions},
%        we have $\io : \MM \to \NN$ an elementary embedding.
%        Note that since embeddings commute with interpretation of constants,
%        $\io(a) = \io(\mmintp{c}) = \nnintp{c} = b$ and since the 
%        embedding is elementary 
%        \[\NN \model{\Si(\MM)} \phi(b) 
%        \implies \NN \model{\Si(\MM)} \phi(\io(a)) 
%        \implies \MM \model{\Si(\MM) \phi(a)}\]
%        (To be precise we can move this down to $\Si(A)(a)$ such that
%        $\MM \model{\Si(A)(a)} \phi(a)$.)
%    \end{forward}

    ($4. \implies 1.$)
        By \linkto{compactness_for_types}{compactness for types},
        it suffices to show that $p$ 
        is finitely consistent with the elementary diagram.
        Let $\De \subs p$ be finite.
        Then by assumption there is $a \in {\MM}^n$ such that 
        $\MM \model{\Si(A)(a)} \De(a)$ and so $\MM \model{\Si(\MM)} \De(a)$.
        Clearly $\MM$ is a model of its elementary diagram.
        Hence we have satisfied the conditions for $2$.
\end{proof}

\begin{dfn}[Type of an element]
    Let $\MM$ be a $\Si$-structure containing
    $A \subs {\MM}$ and $a \in {\MM}^n$
    Then 
    \[\subintp{A,n}{\MM}{\tp}(a) := 
    \set{\phi \in F(\Si(A),n) \st \MM \model{\Si(A)} \phi(a)}\]
    is the type of $a$ in $\MM$ over $A$.
    One can verify that this is a maximal $n$-type on $T$
    if $\MM$ is a model of $T$.
\end{dfn}

\begin{prop}
    \link{elems_of_stone_space_are_types_of_elements}
    A type $p$ of a $\Si$-theory $T$ 
    is realised by $a$ in an extension $\NN$ if and only if 
    $p = \subintp{\nothing,n}{\NN}{\tp(a)}$.
    Any element of the Stone space is the type of an element.
\end{prop}
\begin{proof}
    If $p$ is realised by $a$ in $\NN$ then 
    $\NN \model{\Si} p(a)$ hence $p \subs \subintp{\nothing,n}{\MM}{\tp}(a)$
    and by maximality of $p$ they are equal.

    Any element of the Stone space is realised in some 
    $\Si$-structure due to consistency
    hence it is the type of an element by the above.
\end{proof}

\begin{prop}[All maximal realised $n$-types are types of an element]
    %? Where to put?
    Let $\MM$ be a $\Si$-structure,
    $A$ a subset of ${\MM}$ and $p$ a subset of $F(\Si(A),n)$.
    Let $a \in {\MM}^n$.
    Then \begin{itemize}
        \item $p$ is a maximal $n$-type on $\eldiag{\Si}{\MM}$ 
            that is realised by $a \in {\MM}^n$ 
            if and only if $p = \subintp{A,n}{\MM}{\tp}(a)$.
        \item If $\MM \subs \NN$ is an elementary embedding then
            \[\subintp{A,n}{\MM}{\tp}(a) = \subintp{A,n}{\NN}{\tp}(a)\]
    \end{itemize}
\end{prop}
\begin{proof}~\begin{itemize}
    \item \begin{forward}
        As $p$ is realised by $a$, 
        $p \subs \subintp{A,n}{\MM}{\tp}(a)$.
        By maximality of $p$ any formula in $\subintp{A,n}{\MM}{\tp}(a)$ 
        is either in $p$
        or its negation is in $p$.
        If its negation is in $p \subs \subintp{A,n}{\MM}{\tp}(a)$ we have a
        contradiction as this would imply $\MM \model{\Si(A)} \phi(a)$ and
        $\MM \nodel{\Si(A)} \phi(a)$.
    \end{forward}
    \begin{backward}
        If $p = \subintp{A,n}{\MM}{\tp}(a)$ 
        then clearly $p$ is realised by $a$ and so it is consistent with 
        $\eldiag{\Si}{\MM}$ thus it is an $n$-type over $\eldiag{\Si}{\MM}$.
        For any $\phi \in F(\Si(A),n)$,
        $\MM \model{\Si(A)} \phi(a)$ or $\MM \nodel{\Si(A)} \phi(a)$.
        Hence $\phi$ or $\NOT \phi$ is in $p$ and so it is maximal.
    \end{backward}
    \item \[\phi \in \subintp{A,n}{\MM}{\tp}(a) \iff \MM \model{\Si(A)} \phi(a)
        \iff \NN \model{\Si(A)} \phi(a) \iff 
        \phi \in \subintp{A,n}{\NN}{\tp}(a)\]
\end{itemize}\end{proof}

\subsection{Quantifier free types}
Quantifier free types become important when we start making 
back and forth constructions.
All the results carry through, 
thus there isn't much to say until they are relevant.
Thus this section is worth skipping for now.
\begin{dfn}[$QFF(\Si,n)$ and formulas consistent with a theory]
    We define $QFF(\Si,n)$ to be the subset of $F(\Si,n)$ 
    of quantifier free formulas.

    For subsets $p \subs QFF(\Si,n) \subs F(\Si,n)$ 
    the definitions of consistency carry through and 
    we can apply compactness for types on these subsets too.
    The definition of maximality is the same but restricted to $QFF(\Si,n)$.

    Let $T$ be a $\Si$-theory.
    Any subset $p \subs QFF(\Si,n)$ that is consistent with $T$
    is called a quantifier free $n$-type on $T$.
    Note that any quantifier free $n$-type is an $n$-type.
\end{dfn}

\begin{dfn}[Quantifier free Stone space of a theory]
    Let $T$ be a $\Si$-theory.
    Let the stone space of $T$, 
    $S_n^{\qf}(T)$ be the set of all maximal quantifier free $n$-types on $T$.
    We give the same topology:
    $U \subs S_n(T)$ is an element of the basis when there exists 
    $\phi \in QFF(\Si,n)$ such that 
    \[U = [\phi]_T^{\qf} := \set{p \in S_n^{\qf}(T) \st \phi \in p}\]
\end{dfn}

\begin{prop}[Extending to maximal quantifier free $n$-types]
    Any quantifier free $n$-type on a theory 
    can be extended to a maximal quantifier free 
    $n$-type.
\end{prop}
\begin{proof}
    Any quantifier free $n$-type is an $n$-type, 
    hence \link{extend_to_maximal_type_zorn}{can be extended 
        to a maximal $n$-type}.
    The intersection of a maximal $n$-type with $QFF$ is a maximal 
    quantifier free $n$-type. 
    This intersection extends the quantifier free $n$-type and we are done.
\end{proof}

\begin{prop}[Properties of the Stone space]
    Let $T$ be a $\Si$-theory, $\phi, \psi \in QFF(\Si,n)$.
    Elementary properties:
    \begin{itemize}
        \item $(\NOT \phi) \in p$ if and only if $p \notin [\psi]_T^{\qf}$.
        \item $[\phi]_T = [\psi]_T^{\qf}$ if and only if $\phi$ and $\psi$ are 
            equivalent modulo $T$.
        \item $[\top]_T^{\qf} = S_n^{\qf}(T)$
        \item $[\NOT \phi]_T^{\qf} = S_n(T) \setminus [\phi]_T^{\qf}$
        \item $[\phi \OR \psi]_T^{\qf} = [\phi]_T^{\qf} \cup [\psi]_T^{\qf}$
        \item $[\bot]_T^{\qf} = \nothing$
        \item $[\phi \AND \psi]_T^{\qf} = [\phi]_T^{\qf} \cap [\psi]_T^{\qf}$
    \end{itemize}
    Topological properties:
    \begin{itemize}
        \item Elements of the basis of $S_n^{\qf}(T)$ are clopen.
        \item $S_n^{\qf}(T)$ is Hausdorff.
        \item $S_n^{\qf}(T)$ is compact.
    \end{itemize}
\end{prop}
\begin{proof}
    The proofs of these are exactly the same as before, 
    some of them can even be bypassed by using the previous results.
\end{proof}

\begin{dfn}[Quantifier free type of an element]
    Let $\MM$ be a $\Si$-structure, 
    $A \subs {\MM}$ and $a \in {\MM}^n$
    Then 
    \[\subintp{A,n}{\MM}{\qftp}(a) := 
    \set{\phi \in QFF(\Si(A),n) \st \MM \model{\Si(A)} \phi(a)}\]
    is the quantifier free type of $a$ in $\MM$ over $A$.
    One can verify that this is a maximal quantifier free $n$-type on $T$
    if $\MM$ is a model of $T$.
    We will often drop parts of the subscripts and superscripts when it is
    clear. 
    In fact the $n$ can be deduced by the length of $a$ 
    and serves only to explicitely spell things out.
\end{dfn}