\section{Quantifier elimination and model completeness}
Written whilst following section on algebraically closed fields.
\subsection{Quantifier elimination}

\begin{dfn}[Equivalence modulo a theory]
    We say two $\Si$-formulas $\phi$ and $\psi$ with 
    free variables indexed by $S$ are equivalent modulo a 
    $\Si$-theory $T$ if 
    \[T \model{\Si} \forall v, \brkt{\phi \iff \psi}\]
    where $v = (v_i)_{i \in S}$.
\end{dfn}

\begin{dfn}[Quantifier elimination]
    Let $T$ be a $\Si$-theory and $\phi$ a $\Si$-formula.
    We say the quantifiers of $\phi$ can be eliminated if there exists a
    quantifier free $\Si$-formula $\psi$ that is equivalent to $\phi$ 
    modulo $T$.
    We say $\phi$ is reduced to $\psi$.

    We say $T$ has quantifier elimination if the quantifiers of any 
    $\Si$-formula can be eliminated.
\end{dfn}

\begin{lem}[Deduction]
    \link{deduction}
    Let $T$ be a $\Si$-theory, $\De$ a finite $\Si$-theory and 
    $\psi$ a $\Si$-sentence.
    Then $T \cup \De \model{\Si} \psi$ if and only if 
    \[T \model{\Si} \brkt{\bigand{\phi \in \De}{} \phi} \to \psi\]
\end{lem}
\begin{proof}
    We first case on if $\De$ is empty or not.
    If it is empty then $T \cup \De \model{\Si} \psi$ if and only if
    $T \model{\Si} \psi$ if and only if $T \model{\Si} \top \to \psi$
    if and only if 
    \[T \model{\Si} \brkt{\bigand{\phi \in \De}{} \phi} \to \psi\]
    \begin{forward}
        Suppose $\MM \model{\Si} T$ then we need to show 
        $\MM \model{\Si} \brkt{\bigand{\phi \in \De}{} \phi} \to \psi$. 
        Indeed, suppose $\MM \model{\Si} \brkt{\bigand{\phi \in \De}{} \phi}$ 
        then by induction $\MM \model{\Si} T \cup \De$ and so by assumption
        that $T \cup \De \model{\Si} \psi$ we have 
        $\MM \model{\Si} \psi$. 
        Hence $\MM \model{\Si} \brkt{\bigand{\phi \in \De}{} \phi} \to \psi$.
    \end{forward}

    \begin{backward}
        Suppose $\MM \model{\Si} T \cup \De$ then 
        $\MM \model{\Si} T$ thus by assumption that 
        $T \model{\Si} \brkt{\bigand{\phi \in \De}{} \phi} \to \psi$
        we have $\MM \model{\Si} \brkt{\bigand{\phi \in \De}{} \phi} \to \psi$.
        By induction $\MM \model{\Si} \brkt{\bigand{\phi \in \De}{} \phi}$ 
        thus we have $\MM \model{\Si} \psi$.
    \end{backward}
\end{proof}

\begin{lem}[Proofs are finite]
    \link{proofs_are_finite}
    Suppose $T$ is a $\Si$-theory and $\phi$ a $\Si$-sentence such that 
    $T \model{\Si} \phi$. 
    Then there exists a finite subset $\De$ of $T$ such that 
    $\De \model{\Si} \phi$.
\end{lem}
\begin{proof}
    We show the contrapositive.
    Suppose for all finite subsets $\De$ of $T$, $\De \nodel{\Si} \phi$,
    then \linkto{not_consequence}{$\De \cup \set{\phi}$ is consistent} and
    by \linkto{compactness}{compactness} $T \cup \set{\phi}$ is consistent.
    Hence \linkto{not_consequence}{$T \nodel{\Si} \phi$}.
\end{proof}

\begin{prop}[Eliminating quantifiers of a formula]
    \link{elim_quant_of_form}
    Let $\Si$ be a signature such that $\const{\Si} \ne \nothing$.
    Suppose $T$ is a $\Si$-theory and $\phi$ is a $\Si$-formula
    with free-variables $v = (v_1,\dots,v_n)$.
    Then the quantifiers of $\phi$ 
    can be eliminated if and only if the following holds:
    for any two $\Si$-models $\MM,\NN $ of $T$ and any 
    $\Si$-structure $\AA$ that with $\Si$-embeddings into both $\MM$ and $\NN$ 
    ($\io_\MM, \io_\NN$),
    if $a \in \AA^n$ then 
    \[\MM \model{\Si} \phi(\io_\MM (a)) 
    \iff \NN \model{\Si} \phi(\io_\NN (a))\]
\end{prop}
\begin{proof}
    \begin{forward}
        Let $a \in \AA^n$.
        By assumption there exists $\psi \in \form{\Si}$ such that 
        $T \model{\Si} \forall v, \brkt{\phi(v) \IFF \psi(v)}$
        Then $\MM \model{\Si} \phi(\io_\MM(a))$ if and only if 
        $\MM \model{\Si} \psi(\io_\MM(a))$ if and only if 
        $\AA \model{\Si} \psi(a)$, 
        since \linkto{emb_preserve_sat_of_quan_free}{
            embeddings preserve the satisfaction of quantifier free formulas}.
        Similarly, this is if and only if $\NN \model{\Si} \phi(\io_\NN(a))$.
    \end{forward}

    \begin{backward}
        Let 
        \[
            \Ga := \set{\psi \text{ quantifier free } \form{\Si} \st
            T \model{\Si} \forall v, (\phi \to \psi)}
        \]
        and let $\Si(*)$ be such that 
        $\const{\Si(*)} = \const{\Si} \cup \set{d_1,\dots,d_n}$
        for some new constant symbols $d_i$ 
        (indexed according to the free-variables of $\phi$).
        We claim that 
        $T \cup \set{\psi(d) \st \psi \in \Ga} \model{\Si(*)} \phi(d)$.
        We first look at how this would complete the proof.
        If it is true then as 
        \linkto{proofs_are_finite}{proofs are finite}
        we have a finite subsets $\De \subs \Ga$ such that
        $T \cup \set{\psi(d) \st \psi \in \De} \model{\Si(*)} \phi(d)$.
        By \linkto{deduction}{deduction} we have
        \[T \model{\Si(*)} \brkt{\bigand{\psi \in \De}{} \psi(d)} \to \phi(d)\]
        and by the \linkto{lemma_on_const}{lemma on constants}
        \[
            T \model{\Si} \forall v, 
            \brkt{\bigand{\psi \in \De}{} \psi(v)} \to \phi(v) 
        \]
        where $\brkt{\bigand{\psi \in \De}{} \psi(v)}$ is quantifier free.
        By the definition of $\De$ we have the other implication as well:
        \[
            T \model{\Si} \forall v, 
            \brkt{\bigand{\psi \in \De}{} \psi(v)} \IFF \phi(v) 
        \]
        hence the result.

        Suppose for a contradiction 
        $T \cup \set{\psi(d) \st \psi \in \Ga} \nodel{\Si(*)} \phi(d)$.
        Then there exists a model
        $\MM$ of $T \cup \set{\psi(d) \st \psi \in \Ga}$ such that 
        $\MM \nodel{\Si} \phi(d)$.

        Suppose for a second contradiction that the $\Si(*)(\MM)$-theory
        $T \cup \atdiag{\Si(*)}{\MM} \cup \set{\phi(d)}$ is inconsistent.
        Then by \linkto{compactness}{compactness} 
        some subset
        $T \cup \De \cup \set{\phi(d)}$ is inconsistent, 
        where $\De \subs \atdiag{\Si(*)}{\MM}$ is finite.
        \linkto{not_a_consequence}{This implies}
            $T \cup \De \model{\Si(*)(\MM)} \NOT \phi(d)$.
        Hence by \linkto{deduction}{deduction} we have 
        \[T \model{\Si(*)(\MM)} 
            \brkt{\bigand{\psi(d) \in \De}{} \psi(d)} \to \NOT \phi(d)\]
        By the \linkto{lemma_on_const}{lemma on constants
        applied to $\const{\Si} \subs \const{\Si(*)(\MM)}$}
        \[T \model{\Si} 
            \forall v, 
            \sqbrkt{\brkt{\bigand{\psi(d) \in \De}{}\psi(v)} 
            \to \NOT \phi(v)}\]
        Taking the contrapositive, 
        \[T \model{\Si} 
            \forall v, 
            \sqbrkt{\phi(v) 
            \to \brkt{\bigor{\psi(d) \in \De}{}\NOT \psi(v)}}\]
        Hence $\bigor{\psi(d) \in \De}{}\NOT \psi(v) \in \Ga$
        and so $\MM \model{\Si(*)} \bigor{\psi(d) \in \De}{}\NOT \psi(v)$
        by definition of $\MM$.
        However each $\De \subs \atdiag{\Si(*)}{\MM}$ and so
        $\MM \model{\Si(*)} \bigand{\psi(d) \in \De}{}\psi(v)$, 
        a contradiction.
        Thus there exists a model
        \[\NN \model{\Si(*)(\MM)} T \cup 
            \atdiag{\Si(*)}{\MM} \cup \set{\phi(d)}\]
        Since $\NN \model{\Si(*)(\MM)} \atdiag{\Si(*)}{\MM}$ there exists a 
        $\Si(*)(\MM)$ morphism $\io : \MM \to \NN$.
        \linkto{move_down_morph}{Move this morphism down to $\Si$}, then
        by assumption with $\AA := \MM$, for any sentence $\chi$
        \[\MM \model{\Si} \chi
            \iff \NN \model{\Si} \chi\]
        Since $\NN \model{\Si(*)(\MM)} \phi(d)$ by the 
        \linkto{lemma_on_const}{lemma on constants} 
        $\NN \model{\Si} \forall v, \phi(v)$
        and so $\MM \model{\Si} \forall v, \phi(v)$.
        Which is a contradiction because
        $\MM \model{\Si(*)} \NOT \phi(d)$ and so 
        by the \linkto{lemma_on_const}{lemma on constants} 
        $\MM \model{\Si} \forall v, \NOT \phi(v)$.
        (We have
        $\MM \model{\Si} \phi(\modintp{\MM}{c}, \dots \modintp{\MM}{c})$
        and $\MM \nodel{\Si} \phi(\modintp{\MM}{c}, \dots \modintp{\MM}{c})$.)
    \end{backward}
\end{proof}

\begin{lem}[Sufficient condition for quantifier elimination]
    \link{condition_for_q_e}
    Let $T$ be a $\Si$-theory and suppose for any quantifier free $\Si$-formula
    $\psi$ with at least one free variable $w$, the quantifier of 
    $\forall w, \psi (w)$ can be eliminated.
    Then $T$ has quantifier elimination.
\end{lem}
\begin{proof}
    Induct on what $\phi$ is.
    \begin{itemize}
        \item If $\phi$ is $\top$, an equality or a relation then it is already 
            quantifier free.
        \item If $\phi$ is $\NOT \chi$ and there exists a quantifier free 
            $\Si$-formula $\psi$ such that 
            $T \model{\Si} \forall v, \chi \IFF \psi$.
            Then $T \model{\Si} \forall v, \NOT \chi \IFF \NOT \psi$.
            Hence $\phi$ can be reduced to $\NOT \psi$ which is quantifier free.
        \item If $\phi$ is $\chi_0 \OR \chi_1$ and there exist respective 
            reductions of these $\psi_0$ and $\psi_1$ then
            $\phi$ reduces to $\psi_0 \OR \psi_1$ which is quantifier free.
        \item If $\phi$ is $\forall w, \chi(w)$ and there exists quantifier free
            $\psi$ such that 
            \[T \model{\Si} \forall w, \bigforall{v \in S}{} v, 
                \brkt{\chi \IFF \psi}\]
            where $S$ indexes the rest of the free variables in 
            $\chi$ and $\psi$.
            Then we can show that 
            \[
                T \model{\Si} \bigforall{v \in S}{}
                \brkt{\phi \IFF (\forall w, \psi)}
            \]
            By assumption there exists $\om$ a quantifier free $\Si$-formula
            such that 
            \[
                T \model{\Si} \bigforall{v \in S}{}
                \brkt{\om \IFF (\forall w, \psi)}
            \]
            Hence $\phi$ can be reduced to $\om$.
    \end{itemize}
\end{proof}

\begin{cor}[Improvement: Sufficient condition for quantifier elimination]
    \link{improved_condition_for_q_e}
    If $T$ be a $\Si$-theory if
    for any quanfier free $\Si$-formula $\phi$ with at least one free variable
    $w$ (index the rest by $S$),
    for any $\MM, \NN$ $\Si$-models of $T$, for any $\Si$-structure $\AA$ 
    that embeds into $\MM$ and $\NN$ (via $\io_\MM,\io_\NN$) and 
    any $a \in ({\AA})^S$,
    \[\MM \model{\Si} \forall w, \phi(\io_{\MM}(a)) \implies 
    \NN \model{\Si} \forall w, \phi(\io_\NN (a))\]
    then $T$ has quantifier elimination.

    Equivalently we can use the statement 
    \[\MM \model{\Si} \exists w, \phi(\io_{\MM}(a)) \implies 
    \NN \model{\Si} \exists w, \phi(\io_\NN (a))\]
    by negating $\phi$.
\end{cor}
\begin{proof}
    To show that $T$ has quantifier elimination
    \linkto{condition_for_q_e}{it suffices to show that} 
    for any quanfier free $\Si$-formula $\phi$ with at least one free variable
    $w$ (index the rest by $S$), 
    the quantifiers of $\forall w, \phi$ can be eliminated.
    This is true \linkto{elim_quant_of_form}{if and only if} for any 
    $\MM, \NN$ $\Si$-models of $T$, 
    for any $\Si$-structure $\AA$ 
    that injects into $\MM$ and $\NN$ (via $\io_\MM,\io_\NN$) and 
    any $a \in (\AA)^S$,
    \[\MM \model{\Si} \forall w, \phi(\io_{\MM}(a)) \implies 
    \NN \model{\Si} \forall w, \phi(\io_\NN (a))\]
    By symmetry of $\MM$ and $\NN$ we only require one implication.
    Hence the proposition.
\end{proof}
\begin{rmk}
    For quantifier elimination it also suffices to show that for any 
    $\Si$-models $\MM$ of $T$, for any $\Si$-structure $\AA$ 
    that embeds into $\MM$ (via $\io_\MM$) and 
    any $a \in ({\AA})^S$,
    \[\MM \model{\Si} \forall w, \phi(\io(a)) \implies 
    \AA \model{\Si} \forall w, \phi(a)\]
    since \linkto{emb_preserve_sat_of_forall_down}{
        embeddings preserve satisfaction of universal formulas downwards}.
    Equivalently we can use
    \[\AA \model{\Si} \exists w, \phi(a) \implies 
    \MM \model{\Si} \exists w, \phi(\io(a))\]
\end{rmk}

\subsection{Back and Forth}
`Back and forth' is a technique used to determine 
elementary equivalence of models,
quantifier elimination of theories
and completeness of theories.
This section draws together work from Poizat \cite{poizat}, 
OLP \cite{openlogicproject},
and Pillay \cite{pillay}.
It is motivated by the \linkto{infinite_infinite_classes}{example at the end}, 
which should be looked at first.

\begin{dfn}[Substructure generated by a subset]
    Let $\MM$ be a $\Si$-structure.
    Let $A \subs \MM$.
    Then the following are equal:
    \begin{itemize}
        \item The set $\<A\>$ defined inductively: $A\subs \<A\>$;
        if $c \in \const{\Si}$ then 
        $\mmintp{c} \in \<A\>$; if $f \in \func{\Si}$ and 
        $\al \in \<A\>^{n_f}$ then $\mmintp{f}(\al) \in \<A\>$.
        \item $\bigcap \set{\NN \text{ substructure of } \MM \st A \subs \NN}$
    \end{itemize}
    and define a substructure of $\MM$.
    We say it is the `substructure of $\MM$ generated by $A$'.
    
    We say a substructure is finitely generated if there exists a finite set 
    $A$ such that it is equal to $\<A\>$.
\end{dfn}
\begin{proof}
    We note show that $\<A\>$ is a substructure of $\MM$ containing $A$:
    It contains the interpretations of constant symbols from $\MM$.
    By definition $\modintp{\<A\>}{f} := \mmintp{f}$ is well defined.
    Each relation $r$ is naturally interpreted as the intersection of relations
    on $\MM$ intersected with $\<A\>^{m_r}$.
    Hence $\bigcap \NN \subs \<A\>$.

    For the other direction note that if $a \in \<A\>$ then it is in $A$,
    $\mmintp{c}$
    or $\mmintp{f}(\al)$ for some $\al \in \<A\>^{n_f}$.
    If it is in $A$ then we are done.
    Any substructure of $\MM$ contains the $\mmintp{c}$ for each constant symbol
    hence the first case is fine.
    Any substructure of $\MM$ is closed under $\mmintp{f}$ and by induction
    $\al \in \NN^{n_f}$ for any substructure $\NN$.
    Hence $f(\al) \in \NN$ for any substructure.
    Thus $\<A\> \subs \bigcap \NN$ and we are done.
\end{proof}

\begin{prop}[Image of generators are generators of the image]
    \link{im_of_gen_are_gen_of_im_substructures}
    The image of a substructure generated by a subset is a substructure 
    generated by the image of a set.
    In particular,
    a finitely generated substructure has finitely generated image under a 
    $\Si$-morphism given by the image of the generators.
\end{prop}
\begin{proof}
    Let $\io : \<A\> \to \NN$ be a $\Si$-morphism.
    We show that $\<\io(A)\> = \io(\<A\>)$.
    If $b \in \<\io(A)\>$ then $b = \nnintp{c}$ or $b = \nnintp{f}(\io(\al))$ 
    for $al \in \<A\>$.
    Hence $b = \nnintp{c} = \io(\mmintp{c}) \in \io(\<A\>)$
    or 
    \[b = \nnintp{f}(\io(\al)) = \io(\mmintp{f}(\al)) \in \io(\<A\>)\]
    Thus $\<\io(A)\> \subs \io(\<A\>)$.
    The other direction is similar.
\end{proof}

\begin{dfn}[Partial isomorphisms]
    Let $\MM$ and $\NN$ be $\Si$-structures.
    A partial isomorphism from $\MM$ to $\NN$ is a $\Si$-isomorphism $p$
    with finitely generated domain in of $\MM$ 
    and codomain in $\NN$.
\end{dfn}

\begin{prop}[Equivalent definition of partial isomorphism]
    \link{equiv_def_partial_iso}
    Let $\MM$ and $\NN$ be $\Si$-structures. 
    Let $a \in \MM^n$ and $b \in \NN^n$.
    The following are equivalent:
    \begin{itemize}
        \item There exists a partial isomorphism $p : \<a\> \to \<b\>$ 
            such that $p(a) = b$.
        \item $\subintp{\nothing}{\MM}{\qftp}(a) = 
            \subintp{\nothing}{\NN}{\qftp}(b)$
    \end{itemize}
\end{prop}
\begin{proof}
    \begin{forward}
        We induct on terms to show that $\mmintp{t}(a) = \modintp{\<a\>}{t}(a)$
        for each term $t$:
        \begin{itemize}
            \item If $t$ is a constant symbol or a variable 
                then by definition of the 
                the substructure interpretation they are equal.
            \item If $t$ is $f(s)$ and we have the inductive hypothesis 
                $\mmintp{s}(a) = \modintp{\<a\>}{s}(a)$
                then by definition of the substructure interpretation
                \[
                    \mmintp{t}(a) = \mmintp{f}(\mmintp{s}(a))
                    = \mmintp{f}(\modintp{\<a\>}{s}(a))
                    = \modintp{\<a\>}{f}(\modintp{\<a\>}{s}(a))
                    = \modintp{\<a\>}{t}(a)
                \]
        \end{itemize}
    
    Let $\phi$ be a quantifier free $\Si$-formula with up to $n$ variables.
    We show by induction on $\phi$ that 
    \[
        \MM \model{\Si} \phi(a)
        \iff \<a\> \model{\Si} \phi(a)
    \]
    \begin{itemize}
        \item If $\phi$ is $\top$ it is trivial.
        \item If $\phi$ is $t = s$ then it is clear that
            \[
                \mmintp{t}(a) = \mmintp{s}(a)
                \iff \modintp{\<a\>}{t}(a) = \modintp{\<a\>}{s}(a)
            \]
            by what we showed for terms.
        \item If $\phi$ is $r(t)$ then 
            \[  
                (a_{i_1},\dots, a_{i_m}) \in \mmintp{r}
                \iff (a_{i_1},\dots, a_{i_m}) \in 
                \mmintp{r} \cap \<a\> = \modintp{\<a\>}{r}
            \]
        \item If $\phi$ is $\NOT \psi$ or $\psi \OR \chi$ then it is 
            clear by induction.
    \end{itemize}
    As $p$ is an $\Si$-isomorphism, for any quantifier free $\Si$-formula
    with up to $n$ variables,
    \[\MM \model{\Si} \phi(a) \iff \<a\> \model{\Si} \phi(a)
    \iff \<b\> \model{\Si} \phi(b)
    \NN \model{\Si} \phi(b)\]
    \end{forward}

    \begin{backward}
        Suppose $\subintp{\nothing}{\MM}{\qftp}(a) = 
        \subintp{\nothing}{\NN}{\qftp}(b)$.
        We define $p : \<a\> \to \NN$ by the following:
        if $\al \in \<a\>$ then one can write $\al$ as a term $t$ evaluated
        at $a$: $\al = \mmintp{t}(a)$; $p$ maps $a$ to $\nnintp{t}(b)$.
        To show that $p$ is well-defined, note that if two terms $t$ and $s$
        are such that $\mmintp{t}(a) = \mmintp{s}(a)$ then 
        $t = s$ is a formula in $\subintp{\nothing}{\MM}{\qftp}(a) = 
        \subintp{\nothing}{\NN}{\qftp}(b)$ and so 
        $\nnintp{t}(b) = \nnintp{s}(b)$.
        it is injective because if two terms $t$ and $s$
        are such that $\nnintp{t}(b) = \nnintp{s}(b)$ then 
        $t = s$ is a formula in $\subintp{\nothing}{\NN}{\qftp}(b) = 
        \subintp{\nothing}{\MM}{\qftp}(a)$ and so 
        $\mmintp{t}(a) = \mmintp{s}(a)$.

        By definition $p$ commutes with the interpretation of constant symbols,
        function symbols, and relations.
        Furthermore, for each $i$, $p(a_i) = b_i$ by taking the term to be a 
        variable and evaluating at $a_i$. 
        \linkto{im_of_gen_are_gen_of_im_substructures}{The 
            image of $p$ is $\<b\>$ as the image of $a$ is $b$}.
        Hence it is a partial isomorphism $\<a\> \to \<b\>$ 
        such that $p(a) = b$.
    \end{backward}
\end{proof}

\begin{prop}[Basic facts about partial isomorphisms]
    \link{basic_facts_partial_isomorphisms}
    Let $\MM$ and $\NN$ be $\Si$-structures.
    \begin{itemize}
        \item The inverse of a partial isomorphism is a partial isomorphism.
        \item The restriction of a partial isomorphism is a partial isomorphism.
        \item The composition of partial isomorphisms is a partial isomorphism.
    \end{itemize}
\end{prop}

\begin{dfn}[Partially isomorphic structures]
    Let $\MM$ and $\NN$ be $\Si$-structures.
    A partial isomorphism from $\MM$ to $\NN$ is said to have 
    the back and forth property if 
    \begin{itemize}
        \item (Forth) For each $a \in \MM$
            there exists a partial isomorphism $q$ such that 
            $q$ extends $p$ and $a \in \dom{p}$.
        \item (Back) For each $p \in I$ 
            there exists a partial isomorphism $q$ such that 
            $q$ extends $p$ and $b \in \codom{q}$.
    \end{itemize}

    We say $\MM$ and $\NN$ are back and forth equivalent 
    when all partial isomorphisms from $\MM$ to $\NN$ 
    have the back and forth property.
\end{dfn}

\begin{prop}[Equivalent definition of back and forth property]
    \link{equiv_def_back_and_forth}
    Let $\MM$ and $\NN$ be $\Si$-structures.
    Let $p : \<a\> \to \<b\>$ for $a \in \MM^n$ and $b \in \NN^n$
    be a partial isomorphism such that $p(a) = b$.
    It has the back and forth property if and only if the two conditions hold
    \begin{itemize}
        \item (Forth) For any $\al \in \MM$, 
            there exists $\be \in \NN$ such that 
            $\subintp{\nothing}{\MM}{\qftp}(a,\al) = 
            \subintp{\nothing}{\NN}{\qftp}(b,\be)$
        \item (Back) For any $\be \in \NN$, 
        there exists $\al \in \MM$ such that 
        $\subintp{\nothing}{\MM}{\qftp}(a,\al) = 
        \subintp{\nothing}{\NN}{\qftp}(b,\be)$
    \end{itemize}
\end{prop}
\begin{proof}
    \begin{forward}
        Suppose $p$ has the back and forth property.
        We only show `forth' as the `back' case is similar.
        Let $\al \in \MM$.
        By `forth' there exists $q$ 
        a partial isomorphism extending $p$ such that 
        $\al \in \dom(q)$.
        By \linkto{basic_facts_partial_isomorphisms}{restriction}
        and the fact that \linkto{image_of_generators}{the image of 
        generators generates the image},
        there exists $\be \in \NN$ such that 
        \[\res{q}{\<a,\al\> \to \<b,\be\>}\]
        is a local isomorphism.
        Using the \linkto{equiv_def_of_partial_iso}{the equivalent definition}
        we obtain ${\qftp}(a,\al) = {\qftp}(b,\be)$.
    \end{forward}

    \begin{backward}
        We show that $p$ has the `forth' property.
        Let $\al \in \MM$.
        By assumption there exists $\be \in \MM$ such that 
        \[\subintp{\nothing}{\MM}{\qftp}(a,\al) = 
        \subintp{\nothing}{\NN}{\qftp}(b,\be)\]
        Thus \linkto{equiv_def_of_partial_iso}{there exists 
        $q : \<a,\al\> \to \<b,\be\>$} such that $q(a) = b$ and $q(\al) = \be$.
        Hence $p$ is extended by $q$ with $\al$ in its domain.
    \end{backward}
\end{proof}

\begin{prop}[Quantifier elimination for types]
    \link{quant_elim_for_types}
    Let $T$ be a $\Si$-theory.
    $T$ has quantifier elimination if and only if for any $n \in \N$,
    any two $\Si$-models of $T$ and any $a \in \MM^n, b \in \NN^n$,
    if \[\subintp{\nothing}{\MM}{\qftp}(a) = 
    \subintp{\nothing}{\NN}{\qftp}(b)\]
    then \[\subintp{\nothing}{\MM}{\tp}(a) = 
    \subintp{\nothing}{\NN}{\tp}(b)\]
\end{prop}
\begin{proof}
    \begin{forward}
        Let $\phi \in {\tp}(a)$.
        By quantifier elimination there exists quantifier free $\psi$
        such that they are equivalent modulo $T$.
        Then $\MM \model{\Si} \psi(a)$ and 
        $\psi \in {\qftp}(a) = {\qftp}(b)$.
        Thus $\NN \model{\Si} \psi(b)$ and by equivalence modulo $T$
        $\NN \model{\Si} \phi(b)$.
        Hence $\phi \in {\tp}(b)$.
        The other inclusion is similar.
    \end{forward}

    \begin{backward}
        Let $n \in \N$. Define a map $f : S_n(T) \to S_n^{\qf}(T)$ 
        that takes a maximal $n$-type $p$ to $p \cap QFF(\Si,n)$.
        It is well-defined as the image is indeed a maximal $n$-type.
        It is a surjection as any quantifier free maximal 
        $n$ type is an $n$-type 
        and therefore 
        \linkto{extend_to_maximal_type_zorn}{can be extended to a maximal 
        $n$-type.}
        To show injectivity we note that
        \linkto{elems_of_stone_space_are_types_of_elements}{any two elements 
        of $S_n(T)$ can be written as types of elements}
        $\subintp{\nothing}{\MM}{\tp}(a)$ and  
        $\subintp{\nothing}{\NN}{\tp}(b)$.
        If their images are equal then 
        \[\subintp{\nothing}{\MM}{\qftp}(a) = 
        \subintp{\nothing}{\NN}{\qftp}(b)\]
        thus by assumption they are equal.

        To show that $f$ is continuous we show that elements of 
        the clopen basis have clopen preimage.
        Let $[\phi]_T^{\qf}$ be in the clopen basis of $S_n^{\qf}(T)$.
        Then $p \in [\phi]_T$ if and only if $\phi \in p$ if and only if 
        $\phi \in f(p)$ if and only if $f(p) \in [\phi]_T^{\qf}$.
        Hence the preimage is $[\phi]_T$ which is clopen.

        A continuous bijection between Hausdorff compact spaces is a 
        homeomorphism. 
        Hence for any $\phi \in F(\Si,n)$ the image of the clopen set generated
        by $\phi$ is clopen: there exists $\psi \in QFF(\Si,n)$
        such that $[\phi]_T = f^{-1}[\psi]_T^{\qf} = [\psi]_T$.
        $[\phi]_T = [\psi]_T$ \linkto{basic_facts_basis_elems}{if and only if} 
        they are equivalent modulo $T$.
        Thus we can eliminate quantifiers for any 
        $\phi \in F(\Si,n)$ for any $n$.
        Thus $T$ has quantifier elimination.
    \end{backward}
\end{proof}

\begin{lem}[Back and forth equivalence implies quantifier elimination for types]
    \link{back_and_forth_gives_quantifier_elimination_lem}
    Let $\MM$ and $\NN$ be $\Si$-structures.
    If $\MM$ and $\NN$ are back and forth equivalent and
    $a \in \MM^n$ and $b \in \NN^n$ are such that
    \[\subintp{\nothing}{\MM}{\qftp}(a) = 
    \subintp{\nothing}{\NN}{\qftp}(b)\]
    then \[\subintp{\nothing}{\MM}{\tp}(a) = 
    \subintp{\nothing}{\NN}{\tp}(b)\]
\end{lem}
\begin{proof}
    Let $\phi \in F(\Si,n)$.
    If $\phi$ is quantifier free then 
    $\MM \modelsi \phi(a) \iff \NN \modelsi \phi(b)$.
    By induction on formulas it suffices to show that if 
    $\phi$ is the formula $\forall v, \psi$ and 
    for any $\al \in \MM$ there exists $\be \in \NN$ such that 
    $\MM \modelsi \psi(a,\al) \iff \NN \modelsi \psi(b,\be)$, 
    then we have 
    $\MM \modelsi \forall v, \psi(a) \iff \NN \modelsi \forall v, \psi(b)$.

    By the \linkto{equiv_def_partial_iso}{
        equivalent definition of partial isomorphisms,}
    there exists $p : \<a\> \to \<b\>$ 
    a partial isomorphism in $p$ such that $p(a) = b$.
    Suppose $\MM \modelsi \forall v, \psi(a)$ and let $\be \in \NN$, 
    then $\MM \modelsi \forall v, \psi(a,\al)$.
    By `back' in 
    \linkto{equiv_def_back_and_forth}{
        the equivalent definition of the back and forth property}
    there exists $\al \in \MM$ such that 
        $\subintp{\nothing}{\MM}{\qftp}(a,\al) = 
        \subintp{\nothing}{\NN}{\qftp}(b,\be)$
    Hence $\NN \modelsi \forall v, \psi(a,\al)$.
    The other direction is similar.
\end{proof}

\begin{cor}[Back and forth implies elementary equivalence]
    \link{back_and_forth_implies_elem_equiv}
    Let $\MM$ and $\NN$ be $\Si$-structures.
    If $\MM$ and $\NN$ are back and forth equivalent
    then they are elementarily equivalent.
\end{cor}
\begin{proof}
    Let $\phi$ be a quantifier free $\Si$-formula with $0$ variables,
    i.e. a quantifier free sentence.
    As the empty set is a partial isomorphism.
    Thus by the \linkto{equiv_def_partial_iso}{equivalent 
        definition of a partial isomorphism,}
    \[\subintp{\nothing,0}{\MM}{\qftp}(\nothing) = 
    \subintp{\nothing,0}{\NN}{\qftp}(\nothing)\]
    By the fact that 
    \linkto{back_and_forth_gives_quantifier_elimination_lem}{back 
        and forth equivalence implies quantifier elimination for types},
    \[\subintp{\nothing,0}{\MM}{\tp}(\nothing) = 
    \subintp{\nothing,0}{\NN}{\tp}(\nothing)\]
    
    Thus for any $\Si$-sentence $\phi$, $\MM \modelsi \phi$ if and only if 
    $\phi \in \subintp{\nothing,0}{\MM}{\tp}(\nothing) = 
    \subintp{\nothing,0}{\NN}{\tp}(\nothing)$
    if and only if $\NN \modelsi \phi$.
\end{proof}

\begin{dfn}[$\om$-saturation]
    \link{om_saturation_dfn}
    Let $\MM$ be a $\Si$-structure. 
    $\MM$ is $\om$-saturated if for every finite subset $A \subs \MM$, 
    every $n \in \N$ and every $p \in S_n(\Theory_\MM(A))$,
    $p$ is realised in $\MM$.

    See the general version $\ka$-saturated \linkto{ka_saturation_dfn}{here}.
\end{dfn}

\begin{prop}[$\infty$-equivalence]
    \link{infty_equivalence_01}
    Let $\MM$ and $\NN$ be $\om$-saturated $\Si$-structures.
    If $a \in \MM^n$ and $b \in \NN^n$ satisfy
    \[\subintp{\nothing,n}{\MM}{\tp}(a) = 
    \subintp{\nothing,n}{\NN}{\tp}(b)\]
    then 
    \begin{itemize}
        \item (Forth) For any $\al \in \MM$ there exists $\be \in \NN$ such that
        \[\subintp{\nothing,n+1}{\MM}{\tp}(a,\al) = 
        \subintp{\nothing,n+1}{\NN}{\tp}(b,\be)\]
        \item (Back) For any $\be \in \NN$ there exists $\al \in \MM$ such that
        \[\subintp{\nothing,n+1}{\MM}{\tp}(a,\al) = 
        \subintp{\nothing,n+1}{\NN}{\tp}(b,\be)\]
    \end{itemize}
    If this property holds for any pair $a,b$ related by a partial isomorphism
    we say $\MM$ and $\NN$ are $\infty$-equivalent.
\end{prop}
\begin{proof}
    Let $\al \in \MM$ and consider 
    \[p(a,v) := \subintp{a,1}{\MM}{\tp}(\al) \in S_1(\Theory_\MM(a))\]
    Any formula in $p(a,v)$ can be written as a 
    $\Si$-formula $\phi(w,v)$ with variables $w$
    replaced with elements of $a$ 
    ($v$ represents a single variable to be replaced by $\al$). 
    Let 
    \[p(w,v) := \set{\phi(w,v) \st \phi(a,v) \in p(a,v)}\]
    We claim that 
    \[p(b,v) := \set{\phi(b,v) \st \phi \in p(w,v)} \in S_1(\Theory_\NN(b))\]
    To this end, we note that it is indeed 
    a maximal subset of $F(\Si(b),1)$ since for any $\phi(b) \in F(\Si(b),1)$
    \[\phi(a) \in p(a,v) \text{ or } \NOT \phi(a) \in p(a) \implies 
    \phi(b) \in p(b,v) \text{ or } \NOT \phi(b) \in p(b)\]
    We just need to show that it is consistent with $\Theory_\NN(b)$.

    By \linkto{compactness_for_types}{compactness for types} 
    and noting that $\NN$ is a $\Si(b)$-model of $\Theory_\NN(b)$,
    it suffices to show
    that for any finite subset $\De(w,v) \subs p(w,v)$ 
    there exists $\be \in \NN^m$ such that 
    $\NN \model{\Si(b)} \De(b,\be)$.
    \begin{align*}
        &\MM \model{\Si(a)} \bigand{\phi \in \De}{} \phi(a,\al)\\
        \implies &\MM \model{\Si} \exists v, \bigand{\phi \in \De}{} \phi(a,v)\\
        \implies &\brkt{\exists v, \bigand{\phi \in \De}{} \phi(a,v)} \in 
        \subintp{\nothing}{\MM}{\tp}(a) = 
        \subintp{\nothing}{\NN}{\tp}(b)\\
        \implies &\NN \modelsi \exists v, \bigand{\phi \in \De}{} \phi(b,v)\\
        \implies &\exists \be \in \NN, 
        \NN \modelsi \bigand{\phi \in \De}{} \phi(b,\be)\\
        \implies &\exists \be \in \NN, \NN \model{\Si(b)} \De(b,\be)
    \end{align*}
    Thus $p(b,v)\in S_1(\Theory_\NN(b))$ and since $\NN$ is $\om$-saturated
    $p(b,v)$ is realised in $\NN$ by some $\be$.
    Thus by maximality, $p(b,v) = \subintp{b,1}{\NN}{\tp}(\be)$.

    Finally, for $\phi(v,w) \in F(\Si,n+1)$
    \begin{align*}
        &\phi(v,w) \in \subintp{\nothing}{\MM}{\tp}(a,\al)
        \iff &\MM \model{\Si} \phi(a,\al) \iff \MM \model{\Si(a)} \phi(a,\al)\\
        \iff &\phi(a,v) \in \subintp{a,1}{\MM}{\tp}(\al) = p(a,v)\\
        \iff &\phi(b,v) \in p(b,v) = \subintp{b,1}{\MM}{\tp}(\be)\\
        \iff &\NN \model{\Si(b)} \phi(b,\be) \iff \NN \modelsi \phi(b,\be)\\
        \iff &\phi(w,v) \in \subintp{\nothing}{\NN}{\tp}(b,\be)
    \end{align*}
\end{proof}

\begin{prop}[Back and forth method for showing quantifier elimination]
    \link{om_sat_models_and_quantifier_elimination}
    Let $T$ be a $\Si$-theory.
    If $T$ has quantifier elimination then
    for any two $\om$-saturated $\Si$-models of $T$
    are back and forth equivalent.

    If any two $\Si$-models of $T$
    are back and forth equivalent then $T$ has quantifier 
    elimination.
    \footnote{We could also phrase this as $T$ has 
    quantifier elimination if and only if
    any two $\om$-saturated $\Si$-models of $T$
    are back and forth equivalent, but the saturation requirement becomes 
    redundant in one direction.}
\end{prop}
\begin{proof}
    \begin{forward}
        Let $p$ be a partial isomorphism from $\MM$ to $\NN$.
        By the \linkto{equiv_def_partial_iso}{
            equivalent definition of partial isomorphisms}
        there exists $a \in \MM^n$ and $b \in \NN^n$ such that 
        $p(a) = b$ and 
        \[\subintp{\nothing}{\MM}{\qftp}(a) = 
        \subintp{\nothing}{\NN}{\qftp}(b)\]
        By \linkto{quant_elim_for_types}{
            quantifier elimination for types}
        \[\subintp{\nothing}{\MM}{\tp}(a) = 
        \subintp{\nothing}{\NN}{\tp}(b)\]
        The models are $\om$-saturated, hence  
        \linkto{infty_equivalence_01}{by $\infty$-equivalence}
        for any $\al \in \MM$ there exists $\be \in \NN$ such that 
        \[\subintp{\nothing}{\MM}{\tp}(a,\al) = 
        \subintp{\nothing}{\NN}{\tp}(b,\be)\]
        Taking only the quantifier free elements,
        we obtain 
        \[\subintp{\nothing}{\MM}{\qftp}(a,\al) = 
        \subintp{\nothing}{\NN}{\qftp}(b,\be)\]
        and by the
        \linkto{equiv_def_back_and_forth}{
            equivalent definition of the back and forth property}
        we have that $p$ has the back and forth property.
    \end{forward}

    \begin{backward}
        Let $n \in \N$, $\MM$ and $\NN$ be models of $T$,
        $a \in \MM^n$ and $b \in \NN^n$.
        By \linkto{quant_elim_for_types}{
            quantifier elimination for types} it suffices to show that 
        if \[\subintp{\nothing}{\MM}{\qftp}(a) = 
        \subintp{\nothing}{\NN}{\qftp}(b)\]
        then \[\subintp{\nothing}{\MM}{\tp}(a) = 
        \subintp{\nothing}{\NN}{\tp}(b)\]

        This is satisfied as
        \linkto{back_and_forth_gives_quantifier_elimination_lem}{any 
        two models of $T$ are back and forth equivalent}.
    \end{backward}
\end{proof}

\begin{cor}[Back and forth condition for completeness]
    \link{back_and_forth_implies_completeness}
    Let $T$ be a $\Si$-theory.
    If any two models of $T$ are back and forth equivalent then 
    $T$ is complete.
\end{cor}
\begin{proof}
    If any two models are back and forth equivalent 
    \linkto{back_and_forth_implies_elem_equiv}{then any two non-empty models 
        are elementarily equivalent} (the non-empty is redundant information).
    Hence \linkto{equiv_def_completeness_0}{$T$ is complete}.
\end{proof}

We end this section with a nice example of all of this in action.
\begin{eg}[Infinite infinite equivalence classes]
    \link{infinite_infinite_classes}
    \[\Si_E := (\nothing,\nothing,n_f,\set{E},m_r)\]
    where $m_E = 2$ and $n_f$ is the empty function, 
    defines the signature of binary relations.
    We write for variables $x$ and $y$, 
    we write $x \sim y$ as notation for $E(x,y)$
    The theory of equivalence relations $\ER$ 
    is set set containing the following formulas:
    \begin{align*}
        &\text{Reflexivity - } \forall x, x \sim x\\
        &\text{Symmetry - } \forall x \forall y, x \sim y \to y \sim x\\
        &\text{Transitivity - } 
        \forall x \forall y \forall z, (x \sim y \AND y \sim z) \to x \sim z
    \end{align*}
    For $n \in \N_{>1}$ define 
    \begin{align*}
        &\phi_n := \bigexists{i = 1}{n} x_i, \bigand{i < j}{} x_i \nsim x_j\\
        &\psi_n := \forall x, \bigexists{i = 1}{n} x_i, 
            \bigand{i = 1}{n} \brkt{x \sim x_i} \AND 
            \bigand{i < j}{} \brkt{x_i \ne x_j}
    \end{align*}
    Show that the theory $T = \ER \cup {\phi_n, \psi_n}_{1 < i}$ has 
    quantifier elimination and is complete.
    (You may wonder if it is indeed a theory
    and what nasty induction must be done to 
    show that its formulas can be constructed.)
\end{eg}
\begin{proof}
    We first define the projection into the quotient:
    if $\MM \model{\Si_E} T$ and $a \in {\MM}$ then 
    \[\pi_\MM(a) := \set{b \in {\MM} \st \MM \model{\Si_E} a \sim b}\]
    If $A \subs \MM$ we write $\pi_\MM(A)$ to be the image
    \[\set{\pi_\MM(a) \st \exists a \in A}\]
    Note that the quotient is $\pi_\MM(\MM)$.

    Let $\MM, \NN$ be $\Si_E$-models of $T$
    and let $p$ be a partial isomorphism from $\MM$ to $\NN$.
    By \linkto{om_sat_models_and_quantifier_elimination}{the back and forth
        condition for quantifier elimination} and 
        \linkto{back_and_forth_implies_completeness}{the 
        back and forth condition for completeness}
    it suffices to show that $p$ has the back and forth property.
    
    We only show `forth'.
    Let $\al \in \MM$.
    Suppose $\pi_\MM(\al) \cap \dom p$ is empty.
    We can show that $\pi_\NN(\NN)$ is infinite whilst 
    $\pi_\NN(\codom p)$ is finite, 
    hence there exists $\be \in \NN$ such that 
    $\pi(\be) \in \pi_\NN(\NN) \setminus \pi_\NN(\codom p)$ is non-empty.
    Then define $q : \dom p \cup \set{\al} \to \codom p \cup \set{\be}$
    to agree with $p$ on its domain and send $\al$ to $\be$.
    Note that the domain and codomain of $q$ are substructures
    as the language only contains a relation symbol 
    (thus all subsets are substructures).
    We show that $q$ is an isomorphism.
    It is clearly bijective, and to be an embedding it just needs to preserve
    interpretation of the relation.
    Let $a,b \in \dom q$, if 
    both are in $\dom p$ then as $p$ is a partial isomorphism
    \[a \modintp{\MM}{\sim} b \iff p(a) \modintp{\NN}{\sim} p(b) \iff 
    q(a) \nnintp{\sim} q(b)\]
    Otherwise WLOG $a = \al$.
    If $b = \al$ then it is clear.
    If $b \in \dom p$ then by assumption $b \notin \pi_\MM(\al) = \pi(a)$
    hence $\NOT a \mmintp{\sim} b$.
    By construction 
    \[q(a) = q(\al) = \be \implies 
    \pi_\NN(q(a)) \notin \pi_\NN(\codom p) \quad \text{ and } \quad
    q(b) = p(b) \in \codom p\]
    hence $\NOT q(a) \nnintp{\sim} q(b)$.
    Thus $q$ is a local isomorphism extending $p$.

    Suppose $\pi_\MM(\al) \cap \dom p$ is non-empty,
    i.e. there exists $a \in \dom p$ such that $\al \mmintp{\sim} a$
    We can show that 
    $\pi_\NN(p(a))$ is infinite and $\codom p$ is finite
    hence there exists $\be \in \pi_\NN(p(a)) \setminus \codom p$.
    Then define $q : \dom p \cup \set{\al} \to \codom p \cup \set{\be}$
    to agree with $p$ on its domain and send $\al$ to $\be$.
    Again $p$ is clearly a bijection on substructures, 
    and we show that the relation is preserved.
    Let $b,c \in \dom q$. 
    If $b,c \in \dom p$ then it is clear as $p$ is an isomorphism,
    it is also clear if $b,c = \al$.
    Otherwise WLOG $c = \al$ and $b \in \dom p$.
    Then $c = \al \mmintp{\sim} a$ and
    by construction of $\be$ 
    \[q(c) = q(\al) = \be \nnintp{\sim} p(a)\]
    Noting $a \mmintp{\sim} b$ if and only if $p(a) \nnintp{\sim} p(b)$
    as $p$ is a partial isomorphism
    thus $c \mmintp{\sim} a \mmintp{\sim} b$ 
    if and only if $q(c) \nnintp{\sim} p(a) \nnintp{\sim} p(b) = q(b)$.
    Hence $q$ is a local isomorphism extending $p$.
    Thus $p$ has the `forth' property (and similarly the `back' property).
\end{proof} 

\begin{prop}[Countable back and forth equivalent structures are isomorphic]
    %? Where to put this?
    Let $\MM$ and $\NN$ be countably infinite $\Si$-structures.
    If $\MM$ and $\NN$ are back and forth equivalent then 
    $\MM$ and $\NN$ are isomorphic. 
\end{prop}
\begin{proof}
    Write $\MM = \set{a_i}_{i \in \N}$ and $\NN = \set{b_i}_{i \in \N}$.
    Inductively define partial isomorphisms $p_n$ for $n \in \N$:
    \begin{itemize}
        \item Take $p_0$ to be the empty function.
        \item If $n + 1$ is odd then ensure $a_{n/2}$ 
            is in the domain:
            by the `forth' property of $p$ there exists $p_{n+1}$ 
            extending $p_n$ such that $a_{n/2} \in \dom(p_{n+1})$.
        \item If $n + 1$ is even then ensure $b_{(n+1)/2}$ is in the codomain:
            by the `back' property of $p$ there exists $p_{n+1}$ 
            extending $p_n$ such that $b_{(n-1)/2} \in \codom(p_{n+1})$.
    \end{itemize}
    We claim that $p$, the union of the partial isomorphisms 
    $p_n$ for each $n \in \N$, is an isomorphism.
    Note that it is well-defined and has image $\NN$ as the $p_i$ are nested and
    for any $a_i \in \MM$ and $b_i \in \NN$, 
    $a_i \in \dom(p_{2i+1})$ and $b_i \in \dom(p_{2i+2})$.
    It is injective: if $a_i, a_j \in \MM$ and $p(a_i) = p(a_j)$ then 
    $p_{2i+2}(a_i) = p_{2i+2}(a_j)$ and so $a_i = a_j$ as $p_{2i+2}$ is a 
    partial isomorphism.
    One can show that it is an $\Si$-embedding.
\end{proof}

\subsection{Model completeness}%where to put this?
\begin{dfn}[Model Completeness]
    We say a $\Si$-theory $T$ is model complete when given two $\Si$-models
    of $T$ and a $\Si$-embedding $\io : \MM \to \NN$,
    the embedding is elementary.
\end{dfn}
\begin{rmk}
    \link{quantifier_elimination_implies_model_completeness}
    Any $\Si$-theory $T$ with quantifier elimination is model complete.
    If $\phi$ is a $\Si$-formula and $a \in ({\MM})^S$.
    Then given two $\Si$-models
    of $T$ and a $\Si$-embedding $\io : \MM \to \NN$ we can take
    $\psi$ a quantifier free formula such that 
    $T \modelsi \forall v, \phi \IFF \psi$.
    Since \linkto{emb_preserve_sat_of_quan_free}{embeddings preserve
    satisfaction of quantifier free formulas}
    \[
        \MM \modelsi \phi(a) \iff \MM \modelsi \psi(a)
        \iff \NN \modelsi \psi(\io (a)) \iff \NN \modelsi \phi(\io(a))
    \]
    Thus the extension is elementary.
\end{rmk}
