\documentclass{book}
\usepackage[left=1in,right=1in,tmargin = 25mm,bmargin = 25mm]{geometry}
\usepackage[left=1in,right=1in]{geometry}
\usepackage{subfiles}
\usepackage{amsmath, amssymb, stmaryrd, verbatim} % math symbols
\usepackage{amsthm} % thm environment
\usepackage{mdframed} % Customizable Boxes
\usepackage{hyperref,nameref,cleveref,enumitem} % for references, hyperlinks
\usepackage[dvipsnames]{xcolor} % Fancy Colours
\usepackage{mathrsfs} % Fancy font
\usepackage{tikz, tikz-cd, float} % Commutative Diagrams
\usepackage{perpage}
\usepackage{parskip} % So that paragraphs look nice
\usepackage{ifthen,xargs} % For defining better commands
\usepackage{anyfontsize}
\usepackage[T1]{fontenc}
\usepackage[utf8]{inputenc}
\usepackage{tgpagella}

% Bibliography 
\usepackage{url}
\usepackage{biblatex}

\addbibresource{refs.bib}

% % Misc
\newcommand{\brkt}[1]{\left(#1\right)}
\newcommand{\sqbrkt}[1]{\left[#1\right]}
\newcommand{\dash}{\text{-}}

% % Logic
\renewcommand{\implies}{\Rightarrow}
\renewcommand{\iff}{\Leftrightarrow}
\newcommand{\IFF}{\leftrightarrow}
\newcommand{\limplies}{\Leftarrow}
\newcommand{\NOT}{\neg\,}
\newcommand{\AND}{\, \land \,}
\newcommand{\OR}{\, \lor \,}
\newenvironment{forward}{($\implies$)}{}
\newenvironment{backward}{($\limplies$)}{}
% General way of making larger symbols with limits above and below
\makeatletter
\DeclareRobustCommand\bigop[1]{%
  \mathop{\vphantom{\sum}\mathpalette\bigop@{#1}}\slimits@
}
\newcommand{\bigop@}[2]{%
  \vcenter{%
    \sbox\z@{$#1\sum$}%
    \hbox{\resizebox{
      \ifx#1\displaystyle.7\fi\dimexpr\ht\z@+\dp\z@}{!}{$\m@th#2$}}% symbol size
  }%
}
\makeatother
\newcommand{\bigforall}[2]{\DOTSB\bigop{\forall}_{#1}^{#2}}
\newcommand{\bigexists}[2]{\DOTSB\bigop{\exists}_{#1}^{#2}}
\newcommand{\bigand}[2]{\DOTSB\bigop{\mbox{\Large$\land$}}_{#1}^{#2}}
\newcommand{\bigor}[2]{\DOTSB\bigop{\mbox{\Large$\lor$}}_{#1}^{#2}}

% % Sets
\DeclareMathOperator{\supp}{supp}
\newcommand{\set}[1]{\left\{#1\right\}}
\newcommand{\st}{\,|\,}
\newcommand{\minus}{\setminus}
\newcommand{\subs}{\subseteq}
\newcommand{\ssubs}{\subsetneq}
\DeclareMathOperator{\im}{Im}
\newcommand{\nothing}{\varnothing}
\newcommand\res[2]{{% we make the whole thing an ordinary symbol
  \left.\kern-\nulldelimiterspace 
  % automatically resize the bar with \right
  #1 % the function
  \vphantom{\big|} 
  % pretend it's a little taller at normal size
  \right|_{#2} % this is the delimiter
  }}

% % Greek 
\newcommand{\al}{\alpha}
\newcommand{\be}{\beta}
\newcommand{\ga}{\gamma}
\newcommand{\de}{\delta}
\newcommand{\ep}{\varepsilon}
\newcommand{\io}{\iota}
\newcommand{\ka}{\kappa}
\newcommand{\la}{\lambda}
\newcommand{\om}{\omega}
\newcommand{\si}{\sigma}

\newcommand{\Ga}{\Gamma}
\newcommand{\De}{\Delta}
\newcommand{\Th}{\Theta}
\newcommand{\La}{\Lambda}
\newcommand{\Si}{\Sigma}
\newcommand{\Om}{\Omega}

% % Mathbb
\newcommand{\N}{\mathbb{N}}
\newcommand{\M}{\mathbb{M}}
\newcommand{\Z}{\mathbb{Z}}
\newcommand{\Q}{\mathbb{Q}}
\newcommand{\R}{\mathbb{R}}
\newcommand{\C}{\mathbb{C}}
\newcommand{\F}{\mathbb{F}}
\newcommand{\V}{\mathbb{V}}
\newcommand{\U}{\mathbb{U}}

% % Mathcal
\renewcommand{\AA}{\mathcal{A}}
\newcommand{\BB}{\mathcal{B}}
\newcommand{\CC}{\mathcal{C}}
\newcommand{\DD}{\mathcal{D}}
\newcommand{\EE}{\mathcal{E}}
\newcommand{\FF}{\mathcal{F}}
\newcommand{\GG}{\mathcal{G}}
\newcommand{\HH}{\mathcal{H}}
\newcommand{\II}{\mathcal{I}}
\newcommand{\JJ}{\mathcal{J}}
\newcommand{\KK}{\mathcal{K}}
\newcommand{\LL}{\mathcal{L}}
\newcommand{\MM}{\mathcal{M}}
\newcommand{\NN}{\mathcal{N}}
\newcommand{\OO}{\mathcal{O}}
\newcommand{\PP}{\mathcal{P}}
\newcommand{\QQ}{\mathcal{Q}}
\newcommand{\RR}{\mathcal{R}}
\renewcommand{\SS}{\mathcal{S}}
\newcommand{\TT}{\mathcal{T}}
\newcommand{\UU}{\mathcal{U}}
\newcommand{\VV}{\mathcal{V}}
\newcommand{\WW}{\mathcal{W}}
\newcommand{\XX}{\mathcal{X}}
\newcommand{\YY}{\mathcal{Y}}
\newcommand{\ZZ}{\mathcal{Z}}

% % Mathfrak
\newcommand{\f}[1]{\mathfrak{#1}}

% % Mathrsfs
\newcommand{\s}[1]{\mathscr{#1}}

% % Category Theory
\newcommand{\obj}[1]{\mathrm{Obj}\left(#1\right)}
\newcommand{\Hom}[3]{\mathrm{Hom}_{#3}(#1, #2)\,}
\newcommand{\mor}[3]{\mathrm{Mor}_{#3}(#1, #2)\,}
\newcommand{\End}[2]{\mathrm{End}_{#2}#1\,}
\newcommand{\aut}[2]{\mathrm{Aut}_{#2}#1\,}
\newcommand{\CAT}{\mathbf{Cat}}
\newcommand{\SET}{\mathbf{Set}}
\newcommand{\TOP}{\mathbf{Top}}
%\newcommand{\GRP}{\mathbf{Grp}}
\newcommand{\RING}{\mathbf{Ring}}
\newcommand{\MOD}[1][R]{#1\text{-}\mathbf{Mod}}
\newcommand{\VEC}[1][K]{#1\text{-}\mathbf{Vec}}
\newcommand{\ALG}[1][R]{#1\text{-}\mathbf{Alg}}
\newcommand{\PSH}[1]{\mathbf{PSh}\brkt{#1}}
\newcommand{\map}[2]{ \yrightarrow[#2][2.5pt]{#1}[-1pt] }
\newcommand{\op}{^{op}}
\newcommand{\darrow}{\downarrow}
\newcommand{\LIM}[2]{\varprojlim_{#2}#1}
\newcommand{\COLIM}[2]{\varinjlim_{#2}#1}

% % Algebra
\newcommand{\iso}{\cong}
\newcommand{\nsub}{\trianglelefteq}
\newcommand{\id}[1]{\mathrm{id}_{#1}}
\newcommand{\inv}{^{-1}}
\DeclareMathOperator{\dom}{dom}
\DeclareMathOperator{\codom}{codom}
\DeclareMathOperator{\coker}{Coker}
\DeclareMathOperator{\spec}{Spec}

% % Analysis
\newcommand{\abs}[1]{\left\vert #1 \right\vert}
\newcommand{\norm}[1]{\left\Vert #1 \right\Vert}
\renewcommand{\bar}[1]{\overline{#1}}
\newcommand{\<}{\langle}
\renewcommand{\>}{\rangle}
\renewcommand{\check}[1]{\widecheck{#1}}

% % Galois
\newcommand{\Gal}[2]{\mathrm{Gal}_{#1}(#2)}
\DeclareMathOperator{\Orb}{Orb}
\DeclareMathOperator{\Stab}{Stab}
\newcommand{\emb}[3]{\mathrm{Emb}_{#1}(#2, #3)}
\newcommand{\Char}[1]{\mathrm{Char}#1}

% % Model Theory
\newcommand{\intp}[2]{
    \star_{\text{\scalebox{0.7}{$#1$}}}^{
    \text{\scalebox{0.7}{$#2$}}}}
\newcommand{\subintp}[3]{
    {#3}_{\text{\scalebox{0.7}{$#1$}}}^{
    \text{\scalebox{0.7}{$#2$}}}}
\newcommand{\modintp}[2]{#2^\text{\scalebox{0.7}{$#1$}}}
\newcommand{\mmintp}[1]{\modintp{\MM}{#1}}
\newcommand{\nnintp}[1]{\modintp{\NN}{#1}}
\newcommand{\const}[1]{{#1}_\mathrm{con}}
\newcommand{\func}[1]{{#1}_\mathrm{fun}}
\newcommand{\rel}[1]{{#1}_\mathrm{rel}}
\newcommand{\term}[1]{{#1}_\mathrm{ter}}
\newcommand{\struc}[1]{{#1}_\mathrm{str}}
\newcommand{\form}[1]{{#1}_\mathrm{for}}
\newcommand{\var}[1]{{#1}_\mathrm{var}}
\newcommand{\theory}[1]{{#1}_\mathrm{the}}
\newcommand{\carrier}[1]{{#1}_\mathrm{car}}
\newcommand{\model}[1]{\vDash_{#1}}
\newcommand{\nodel}[1]{\nvDash_{#1}}
\newcommand{\modelsi}{\model{\Si}}
\newcommand{\eldiag}[2]{\mathrm{ElDiag}(#1,#2)}
\newcommand{\atdiag}[2]{\mathrm{AtDiag}(#1,#2)}
\newcommand{\Theory}{\mathrm{Th}}
\newcommand{\unisen}[1]{{#1}_\mathrm{uni}}
\newcommand{\lift}[2]{\uparrow_{#1}^{#2}}
\newcommand{\fall}[2]{\downarrow_{#1}^{#2}}
\newcommand{\Mod}[1]{\M \mathbf{od}(#1)}
\DeclareMathOperator{\GRP}{GRP}
\newcommand{\RNG}{\mathrm{RNG}}
\newcommand{\ER}{\mathrm{ER}}
\DeclareMathOperator{\FLD}{FLD}
\DeclareMathOperator{\ID}{ID}
\DeclareMathOperator{\ZFC}{ZFC}
\DeclareMathOperator{\ACF}{ACF}
\newcommand{\BLN}{\mathrm{BLN}}
\newcommand{\PO}{\mathrm{PO}}
\DeclareMathOperator{\tp}{tp}
\DeclareMathOperator{\qftp}{qftp}
\DeclareMathOperator{\qf}{qf}
\DeclareMathOperator{\eqzero}{eqzero}
\newcommand{\MR}[2]{\mathrm{MR}^{#1}(#2)}
\DeclareMathOperator{\MD}{MD}
\DeclareMathOperator{\acl}{acl}
\DeclareMathOperator{\cl}{cl}
\DeclareMathOperator{\mdeg}{m.deg}
\DeclareMathOperator{\kdim}{k.dim}

% % Set theory
\DeclareMathOperator{\ord}{Ord}

% % Boolean algebra
\newcommand{\NEG}{\smallsetminus}
\newcommand{\upa}[1]{#1^{\uparrow}}

% % Field theory
\DeclareMathOperator{\tdeg}{t.deg}
\newcommand{\zmo}[2][p]{\Z/#1^{#2}\Z}

%% code from mathabx.sty and mathabx.dcl to get some symbols from mathabx
\DeclareFontFamily{U}{mathx}{\hyphenchar\font45}
\DeclareFontShape{U}{mathx}{m}{n}{
      <5> <6> <7> <8> <9> <10>
      <10.95> <12> <14.4> <17.28> <20.74> <24.88>
      mathx10
      }{}
\DeclareSymbolFont{mathx}{U}{mathx}{m}{n}
\DeclareFontSubstitution{U}{mathx}{m}{n}
\DeclareMathAccent{\widecheck}{0}{mathx}{"71}

% Arrows with text above and below with adjustable displacement
% (Stolen from Stackexchange)
\newcommandx{\yaHelper}[2][1=\empty]{
\ifthenelse{\equal{#1}{\empty}}
  % no offset
  { \ensuremath{ \scriptstyle{ #2 } } } 
  % with offset
  { \raisebox{ #1 }[0pt][0pt]{ \ensuremath{ \scriptstyle{ #2 } } } }  
}

\newcommandx{\yrightarrow}[4][1=\empty, 2=\empty, 4=\empty, usedefault=@]{
  \ifthenelse{\equal{#2}{\empty}}
  % there's no text below
  { \xrightarrow{ \protect{ \yaHelper[ #4 ]{ #3 } } } } 
  % there's text below
  {
    \xrightarrow[ \protect{ \yaHelper[ #2 ]{ #1 } } ]
    { \protect{ \yaHelper[ #4 ]{ #3 } } } 
  } 
}

% xcolor
\definecolor{darkgrey}{gray}{0.10}
\definecolor{lightgrey}{gray}{0.30}
\definecolor{slightgrey}{gray}{0.80}
\definecolor{softblue}{RGB}{30,100,200}

% hyperref
\hypersetup{
      colorlinks = true,
      linkcolor = {softblue},
      citecolor = {blue}
}

\newcommand{\link}[1]{\hypertarget{#1}{}}
\newcommand{\linkto}[2]{\hyperlink{#1}{#2}}

% Theorems

% % custom theoremstyles
\newtheoremstyle{definitionstyle}
{5pt}% above thm
{0pt}% below thm
{}% body font
{}% space to indent
{\bf}% head font
{\vspace{1mm}}% punctuation between head and body
{\newline}% space after head
{\thmname{#1}\thmnote{\,\,--\,\,#3}}

% % custom theoremstyles
\newtheoremstyle{propositionstyle}
{5pt}% above thm
{0pt}% below thm
{}% body font
{}% space to indent
{\bf}% head font
{\vspace{1mm}}% punctuation between head and body
{\newline}% space after head
{\thmname{#1}\thmnote{\,\,--\,\,#3}}

\newtheoremstyle{exercisestyle}%
{5pt}% above thm
{0pt}% below thm
{\it}% body font
{}% space to indent
{\scshape}% head font
{.}% punctuation between head and body
{ }% space after head
{\thmname{#1}\thmnote{ (#3)}}

\newtheoremstyle{remarkstyle}%
{5pt}% above thm
{0pt}% below thm
{}% body font
{}% space to indent
{\it}% head font
{.}% punctuation between head and body
{ }% space after head
{\thmname{#1}\thmnote{\,\,--\,\,#3}}

% % Theorem environments

\theoremstyle{definitionstyle}
\newmdtheoremenv[
    linewidth = 2pt,
    leftmargin = 20pt,
    rightmargin = 0pt,
    linecolor = darkgrey,
    topline = false,
    bottomline = false,
    rightline = false,
    footnoteinside = true
]{dfn}{Definition}
\newmdtheoremenv[
    linewidth = 2 pt,
    leftmargin = 20pt,
    rightmargin = 0pt,
    linecolor = darkgrey,
    topline = false,
    bottomline = false,
    rightline = false,
    footnoteinside = true
]{prop}{Proposition}
\newmdtheoremenv[
    linewidth = 2 pt,
    leftmargin = 20pt,
    rightmargin = 0pt,
    linecolor = darkgrey,
    topline = false,
    bottomline = false,
    rightline = false,
    footnoteinside = true
]{cor}{Corollary}
\newmdtheoremenv[
    linewidth = 2 pt,
    leftmargin = 20pt,
    rightmargin = 0pt,
    linecolor = darkgrey,
    topline = false,
    bottomline = false,
    rightline = false,
    footnoteinside = true
]{lem}{Lemma}


\theoremstyle{exercisestyle}
\newmdtheoremenv[
    linewidth = 0.7 pt,
    leftmargin = 20pt,
    rightmargin = 0pt,
    linecolor = darkgrey,
    topline = false,
    bottomline = false,
    rightline = false,
    footnoteinside = true
]{ex}{Exercise}
\newmdtheoremenv[
    linewidth = 0.7 pt,
    leftmargin = 20pt,
    rightmargin = 0pt,
    linecolor = darkgrey,
    topline = false,
    bottomline = false,
    rightline = false,
    footnoteinside = true
]{eg}{Example}
\newmdtheoremenv[
    linewidth = 0.7 pt,
    leftmargin = 20pt,
    rightmargin = 0pt,
    linecolor = darkgrey,
    topline = false,
    bottomline = false,
    rightline = false,
    footnoteinside = true
]{nttn}{Notation}

\theoremstyle{remarkstyle}
\newtheorem{rmk}{Remark}

% % footnotes
\renewcommand{\thempfootnote}{$\dagger$}
\MakePerPage{footnote}

% % tikzcd diagram 
\newenvironment{cd}{
    \begin{figure}[H]
    \centering
    \begin{tikzcd}
}{
    \end{tikzcd}
    \end{figure}
}

% tikzcd
% % Substituting symbols for arrows in tikz comm-diagrams.
\tikzset{
  symbol/.style={
    draw=none,
    every to/.append style={
      edge node={node [sloped, allow upside down, auto=false]{$#1$}}}
  }
}


\begin{document}
\title{Model Theory}
\author{JLH}
\maketitle

\tableofcontents

\chapter{Pure Model Theory}
\section{Basics}
These first two sections follow Marker's book on Model Theory 
\cite{marker} with more 
emphasis on where things are happening, i.e. what signature we are working in,
and some more general statements such as working with embeddings rather than 
subsets.

\subsection{Signatures}
\begin{dfn}[First order language]
    We assume we have a tuple
    $\LL = (\CC,\FF,\RR,\VV,\set{\NOT,\OR,\forall,=,\top})$ such that 
    \begin{itemize}
        \item $|\CC|,|\FF|,|\RR|$ each sufficiently large 
            (say they have cardinality $\aleph_5$ or something).
        \item $|\VV| = \aleph_0$. 
            We index $\VV = \set{v_0, v_1, \dots }$ using $\N$.
        \item $\CC$,$\FF$,$\RR$, $\VV$, $\set{\NOT,\OR,\forall,=}$ 
            do not overlap.
    \end{itemize}
    We call $\LL$ the language and only really 
    use it to get symbols to work with.
    Whenever we introduce new symbols to create larger signatures, 
    we are pulling them out of this box.
\end{dfn}

\begin{dfn}[Signature]
    In a language $\LL$, 
    a tuple $\Si = (C,F,n_\star ,R, m_\star)$ is a signature\footnote{
        Many authors call $\Si$ the language, 
        but I have chosen this way of defining things instead.
    } when 
    \begin{itemize}
        \item $C \subs \CC$. 
            We call $C$ the set of constant symbols.
        \item $F \subs \FF$ and 
            $n_{\star} : F \to \N$, 
            which we call the function arity. 
            We call $F$ the set of function symbols.
        \item $R \subs \RR$
            and $m_{\star} : R \to \N$,
            which we call the relation arity.
            We call $T$ the set of relation symbols.
    \end{itemize}
    Given a signature $\Si$, we may refer to its constant, 
    function and relation symbol sets as $\const{\Si},\func{\Si},\rel{\Si}$.
    We will always denote function arity using $n_\star$ 
    and relation arity using $m_\star$.
\end{dfn}

\begin{eg}
    The \linkto{dfn_rings}{signature of rings} 
    will be used to define the theory of rings, 
    the theory of integral domains, the theory of fields, and so on.
    The signature of \linkto{infinite_infinite_classes}{binary relations} 
    will be used to define the theory of partial orders
    with the interpretation of the relation as $<$, 
    to define the theory of equivalence relations with the 
    interpretation of the relation as $~$,
    and to define the theory $\ZFC$ with the relation interpreted as $\in$.
\end{eg}

\begin{dfn}[$\Si$-terms]
    Given $\Si$ a signature, its set of terms
    $\term{\Si}$ is inductively defined using three constructors:
    \begin{itemize}
        \item[$\vert$] If $c \in \const{\Si}$
            then $c$ is a $\Si$-term.
        \item[$\vert$] If $v_i \in \VV$, 
            $v_i$ is a $\Si$-term.
        \item[$\vert$] Given $f \in \func{\Si}$ and a $n_f$-tuple of $\Si$-terms
            $t \in (\term{\Si})^{n_f}$
            then the symbol $f(t)$ is a $\Si$-term.
    \end{itemize}
    The terms $v_i$ are called variables and will be 
    referred to as elements of $\var{\Si}$.
\end{dfn}

\begin{eg}
    In the signature of rings, 
    \linkto{terms_in_RNG_are_polynomials}{terms will be multivariable 
        polynomials over $\Z$} 
    since they are sums and products of constant symbols $0, 1$ and 
    variable symbols.
    In the \linkto{infinite_infinite_classes}{signature of binary relations} 
    there are no constant or function symbols so the only terms are variables.
\end{eg}

\begin{dfn}[$\Si$-formula, free variable]
    Given $\Si$ a signature, 
    its set of $\Si$-formulas $\form{\Si}$ is inductively defined:
    \begin{itemize}
        \item[$\vert$] $\top$ is a $\Si$-formula
        \item[$\vert$] Given $t, s \in \term{\Si}$, 
        the string $(t = s)$ is a $\Si$-formula
        \item[$\vert$] Given $r \in \rel{\Si}, t \in (\term{\Si})^{m_r}$, 
        the string $r(t)$ is a $\Si$-formula \vspace{1em}
        \item[$\vert$] Given $\phi \in \form{\Si}$, 
        the string $(\NOT \phi)$ is a $\Si$-formula 
        \item[$\vert$] Given $\phi, \psi \in \form{\Si}$, the string 
        $(\phi \lor \psi)$ is a $\Si$-formula
        \item[$\vert$] Given $\phi \in \form{\Si}$ 
        and $v_i \in \var{\Si}$, we take the replace all occurrences of
        $v_i$ with an unused symbol such as 
        $z$ in the string $(\forall v_i, \phi)$
        and call this a $\Si$-formula.
    \end{itemize}
    Shorthand for some strings in $\form{\Si}$ include 
    \begin{itemize}
        \item $\bot := \NOT \top$
        \item $\phi \land \psi := \NOT((\NOT\phi) \lor (\NOT\psi))$
        \item $\phi \to \psi := (\NOT\phi) \lor \psi$
        \item $\exists v, \phi := \NOT(\forall v, \NOT\phi)$
    \end{itemize}
    
    The symbol $z$ is meant to be a `bounded variable', 
    and will not be considered when we want to evaluate
    variables in formulas.
    Not bound variables are called `free variables'.
\end{dfn}
\begin{rmk}
    There are two different uses of the symbol `$=$' from now on, 
    and context will allow us to tell them apart.
    Similarly, logical symbols might be used in our `higher language' 
    and will not be confused with symbols from formulas.

    Formulas should be thought of as propositions with some bits loose, 
    namely the free variables, since it does not make any sense to ask if 
    $x = x \OR x = a$ without saying what $x$ you are taking
    (where $x$ is a variable as $a$ is a constant symbol, say).
    When there are no free variables we get what intuitively looks like a 
    proposition, and we will call these particular formulas sentences.
\end{rmk}

For the sake of learning the theory the following two lemmas should be skipped.
They are essentially algorithms that tell us what the variables in terms 
and what the free variables in formulas are.
We include them for formality.
\begin{prop}[Terms have finitely many variables]
    \link{term_fin_var}
    For any term $t \in \term{\Si}$ there exists a finite subset 
    $S \subs \N$ indexing the variables $v_i$ occuring in $t$.
\end{prop}
\begin{proof}
    If there exists a finite subset $T$ of $\N$ such that if $v_i$ occurs in $t$
    then $i \in T$
    then we can take the intersection of all such
    sets and have the finite set $S$ we're interested in.
    
    We prove existence of such a 
    $T$ using the inductive definition of $t$:
    \begin{itemize}
        \item If $t = c$ a constant symbol, 
        then $T = \nothing$ satisfies the above.
        \item If $t = v_i$ a variable symbol, 
        then $T = \set{i}$ satisfies the above.
        \item If $t = f(t_0, \dots, t_{n_f})$, 
        then by our induction hypothesis we have 
        a $T_i$ satisfying the condition for each $t_i$. 
        Then $\cup_{i} T_i$ satisfies the condition for $t$.
    \end{itemize}
\end{proof}

\begin{prop}[Formulas have finitely many \emph{free} variables]
    \link{form_fin_var}
    Given $\phi \in \form{\Si}$,
    there exists a finite $S \subs \N$ 
    indexing the \emph{free} variables $v_i$ occuring in $\phi$.
\end{prop}
\begin{proof}
    Like in the \linkto{term_fin_var}{terms case}, 
    we only need to show that there exists a $T$
    If $v_i$ occurs freely in $\phi$ then $i \in T$.
    We induct on what $\phi$ is, noting that until the last case there are
    no quantifiers being considered so the variables in question are free:
    \begin{itemize}
        \item If $\phi$ is $\top$ then it has no variables.
        \item If $\phi$ is $t = s$, 
        then we have $S_t, S_s$ indexing the (free) variables of $t$ and $s$ 
        \linkto{term_fin_var}{by the previous proposition},
        and so we can pick $T = S_t \cup S_s$.
        \item If $\phi$ is $r(t_0, \dots, t_{m_r})$, 
        then for each $t_i$ we have $S_i$ indexing the variables of $t_i$.
        Hence we can pick $T = \cup_i S_i$.
        \item If $\phi$ is $\NOT \psi$, 
        then by the induction hypothesis we have $T$ 
        satisfying the above conditions for $\psi$. 
        Pick this $T$ for $\phi$.
        \item If $\phi$ is $\psi \lor \chi$
        then by the induction hypothesis we have 
        $T_{\psi}, T_{\chi}$ satisfying the above conditions for $\psi$ and $\chi$.
        We take $T$ to be the union of indexing sets for $\psi$ and $\chi$.
        \item If $\phi$ is $\forall v_i, \psi$ with $v_i$ substituted for $z$,
        then by the induction hypothesis we have 
        $T_{\psi}$ satisfying the above conditions for $\psi$.
        Take $T = T_{\psi} \setminus \{v_i\}$. 
        In fact taking $T_{\psi}$ itself works as well.
    \end{itemize}
\end{proof}

\begin{nttn}[Substituting Terms for Variables]
    If a $\Si$-formula $\phi$ has a free variable $v_i$ 
    then to remind ourselves of the variable we can write 
    $\phi = \phi(v_i)$ instead.

    If $\phi$ has $S$ indexing its free variables and $t \in (\term{\Si})^S$, 
    then we write $\phi(t)$ to mean $\phi$ with 
    $t_i$ substituted for each $v_i$.
    We can show by induction on terms and formulas that this is still a 
    $\Si$-formula.
\end{nttn}

\begin{dfn}[$\Si$-structure, interpretation]
    Given a signature $\Si$, a set $M$ and interpretation functions 
    \begin{itemize}
        \item $\intp{\const{\Si}}{\MM} : \const{\Si} \to M$
        \item $\intp{\func{\Si}}{\MM} : \func{\Si} \to (M^{n_\star} \to M)$
        \item $\intp{\rel{\Si}}{\MM} : \rel{\Si} \to \PP(M^{m_{\star}})$
    \end{itemize}
    we say that $\MM := (M, \intp{\Si}{\MM})$ is a $\Si$-structure. 
    The latter functions two are dependant types since the powers
    $n_\star, m_\star$ 
    depend on the function and relation symbols given.
    The class (or set or whatever) of $\Si$-structures is denoted $\struc{\Si}$.
    Given only the $\Si$-structure $\MM$, 
    we call its underlying carrier set $M$ as $\carrier{\MM}$.

    We write $\intp{\Si}{\MM}$ to represent any of the three interpretation 
    functions when the context is clear.
    Given $c \in \const{\Si}, f \in \func{\Si}, r \in \rel{\Si}$, 
    we might write the `interpretations' 
    of these symbols as any of the following
    \[
        \subintp{\const{\Si}}{\MM}{c} = \subintp{\Si}{\MM}{c} = \modintp{\MM}{c}
        \quad \quad 
        \subintp{\func{\Si}}{\MM}{f} = \subintp{\Si}{\MM}{f}  = \modintp{\MM}{f}
        \quad \quad
        \subintp{\rel{\Si}}{\MM}{r} = \subintp{\Si}{\MM}{r}  = \modintp{\MM}{r}
    \]
\end{dfn}
The structures in a signature will become the models which we are interested in,
in particular structures will be a models of theories.
For example $\Z$ is a structure in the signature of rings, 
and models the theory of rings but not the theory of fields.
In the signature of binary relations, 
$\N$ with the usual ordering $\leq$ is a structure that models of 
the theory of partial orders but not the theory of equivalence relations.

\begin{dfn}[Interpretation of terms]
    Given a signature $\Si$, 
    a $\Si$-structure $\MM$ and a $\Si$-term $t$,
    let $S$ be the 
    \linkto{term_fin_var}{unique set indexing the variables of $t$}.
    Then there exists a unique induced map 
    $\subintp{T}{\MM}{t} : \carrier{\MM}^S \to \carrier{\MM}$, 
    that commutes with the interpretation of constants and
    functions\footnote{See this more precisely stated in the proof}.
    We then refer to this map as \emph{the} interpretation of the term $t$.
    This in turn defines a dependant $\Pi$-type  
    \[
        \intp{T}{\MM}: 
        \term{\Si} \to (\carrier{\MM}^{S_\star} \to \carrier{\MM})
    \]
\end{dfn}
\begin{proof}
    To define a map $\subintp{T}{\MM}{t} : M^S \to M$ for each
    $t$ we use the inductive definition of $t \in \term{\Si}$.
    If $M$ is empty we define $\subintp{T}{\MM}{t}$ as the empty function.
    Otherwise let $a \in M^S$:
    \begin{itemize} 
        \item If $t = c \in \const{\Si}$ then define 
        $\subintp{T}{\MM}{t} : a \mapsto \subintp{C}{\MM}{c}$, 
        the constant map.
        This type checks since $S = \nothing$ 
        therefore $\subintp{T}{\MM}{t} : M^0 \to M$.
        \item If $t = v_i \in \var{\Si}$
        then define $\subintp{T}{\MM}{t} : a \mapsto a$, the identity.
        This type checks since $|S| = 1$.
        \item If $t = f(s)$ for some $f \in \func{\Si}$ and 
        $s \in (\var{\Si})^{n_f}$ then define 
        $\subintp{T}{\MM}{t} : 
        a \mapsto \subintp{F}{\MM}{f}(\subintp{T}{\MM}{s}(a))$.
        This type checks since $s$ has the same number of variables as $t$.
    \end{itemize}
    By definition, 
    this map commutes with the interpretation of constants and functions, i.e.
    \begin{center}
        \begin{tikzcd}[column sep = tiny]
        \const{\Si} \ar[r, "\subs"] \ar[rd, "\intp{\const{\Si}}{\MM}", swap]
        & \term{\Si} \ar[d, "\intp{T}{\MM}"]
        &&
        \prod_{f \in \func{\Si}} \brkt{\term{\Si}}^{n_f}
        \ar[rrr, bend left=15]
        \ar[rrrdd, bend right=15]& 
        s \in \brkt{\term{\Si}}^{n_f} \ar[r, mapsto] \ar[rd, mapsto]& f(s) & 
        \term{\Si} \ar[dd, "\intp{T}{\MM}"]
        \\
        & (\MM^{S_\star} \to \MM) 
        &&
        && \subintp{\func{\Si}}{\MM}{f}(\subintp{T}{\MM}{s}(\star))\\
        &&&&&&(\MM^{S_\star} \to \MM) 
    \end{tikzcd}
    \end{center}
    The map is clearly unique.
\end{proof}
Where there is no ambiguity, 
we write $\subintp{T}{\MM}{t} = \modintp{\MM}{t}$.
Furthermore, if we have a tuple 
$t \in (\term{\Si})^k$,
then we write $\subintp{T}{\MM}{t}:= 
(\mmintp{t_0},\cdots,\mmintp{t_k})$
    
\begin{dfn}[Sentences and satisfaction]
    Let $\Si$ be a signature and $\phi$ a $\Si$-formula.
    Let \linkto{form_fin_var}{$S \subs \N$ index the free variables of $\phi$}.
    We say $\phi \in \form{\Si}$ is a $\Si$-sentence when $S$ is empty.

    If $\Si$ has no constant symbols then $\nothing$ is a $\Si$-structure
    by interpreting functions and relations as $\nothing$.
    Then given a $\Si$-sentence (not any $\Si$-formula) $\phi$ we 
    want to define $\nothing \model{\Si} \phi$
    using the inductive definition of $\phi$:
    \begin{itemize}
        \item If $\phi$ is $\top$ then $\nothing \model{\Si} \phi$.
        \item $\phi$ cannot be $t = s$ since this contains a constant
            symbol or a variable.
            \vspace{1em}
        \item If $\phi$ is 
            $\NOT \psi$ for some $\psi \in \form{\Si}$, 
            then $\nothing \model{\Si} \phi$ when $\nothing \nodel{\Si} \psi$
        \item If $\phi$ is $(\psi \OR \chi)$, 
            then $\nothing \model{\Si} \phi$ when 
            $\nothing \model{\Si} \psi$ or $\nothing \model{\Si} \chi$.
        \item If $\phi$ is of the form
            $\forall v, \psi$, then $\nothing \model{\Si} \phi$.
    \end{itemize}

    Let $\MM$ be a $\Si$-structure with non-empty carrier set.
    Then given $a \in (\carrier{\MM})^S$, 
    we want to define $\MM \model{\Si} \phi(a)$:
    \begin{itemize}
        \item If $\phi$ is $\top$ then $\MM \model{\Si} \phi$.\footnote{
            We can omit the $a$ when there are no free variables.
            Formally this $a$ is the unique element in $\MM^\nothing$
            given by the empty set.}
        \item If $\phi$ is $t = s$ then
            $\MM \model{\Si} \phi(a)$ when 
            $\modintp{\MM}{t}(a) = \modintp{\MM}{s}(a)$.
            \item If $\phi$ is $r(t)$, 
            where $r \in \rel{\Si}$ and 
            $t \in (\term{\Si})^{m_r}$,
            then $\MM \model{\Si} \phi(a)$ when 
            $\modintp{\MM}{t}(a) \in \modintp{\MM}{r}$.
            \vspace{1em}
        \item If $\phi$ is 
            $\NOT\psi$ for some $\psi \in \form{\Si}$, 
            then $\MM \model{\Si} \phi(a)$ when $\MM \nodel{\Si} \psi(a)$
        \item If $\phi$ is  $(\psi \lor \chi)$, 
            then $\MM \model{\Si} \phi(a)$ when 
            $\MM \model{\Si} \psi(a)$ or $\MM \model{\Si} \chi(a)$.
        \item If $\phi$ is 
            $(\forall v, \psi(a)) \in \form{\Si}$,
            then $\MM \model{\Si} \phi(a)$ 
            if for any $b \in \carrier{\MM}$,   
            $\MM \model{\Si} \psi(a)(b)$.
    \end{itemize}
    We say $\MM$ satisfies $\phi(a)$.
\end{dfn}
\begin{rmk}
    Any $\Si$-structure $\MM$ satisfies $\top$ 
    and does not satisfy $\bot$.
    The empty set satisfies things of the form $\forall, \dots$ 
    but not $\exists, \dots$, as we would expect.
    Note that for $c$ a tuple of constant symbols
    $\MM \model{\Si} \phi(c)$ is the same thing as 
    $\MM \model{\Si} \phi(\mmintp{c})$.
\end{rmk}

\subsection{Theories and Models}
\begin{dfn}[$\Si$-theory]
    $T$ is an $\Si$-theory when it is a subset of $\form{\Si}$
    such that all elements of $T$ are $\Si$-sentences.
    We denote the set of $\Si$-theories as $\theory{\Si}$.
\end{dfn}

\begin{dfn}[Models] 
    \link{consistent}
    Given an $\Si$-structure $\MM$ and $\Si$-theory $T$, 
    we write $\MM \model{\Si} T$ and say
    \emph{$\MM$ is a $\Si$-model of $T$} when 
    for all $\phi \in T$ we have $\MM \model{\Si} \phi$.
\end{dfn}

\begin{eg}[The empty signature and theory]
    $\Si_\nothing = (\nothing, \nothing, n_\star, \nothing, m_\star)$ 
    is the empty signature.
    (We pick the empty functions for $n_\star, m_\star$.)
    The empty $\Si_\nothing$-theory is given by $\nothing$.
    Notice that any set is a $\Si_\nothing$-structure and moreover
    a $\Si_\nothing$-model of the empty $\Si_\nothing$-theory.
\end{eg}

\begin{eg}
    In the signature of rings, 
    \linkto{terms_in_RNG_are_polynomials}{the rings axioms} will be the theory
    of rings and structures satisfying the theory will be rings.
    The \linkto{missing_link}{theory of $\ZFC$} %?$? MISSING LINK
    consists of the $\ZFC$ axioms and a model of $\ZFC$ would be thought of as 
    the `class of all sets'.
\end{eg}

\begin{dfn}[Consequence]
    Given a $\Si$-theory $T$ 
    and a $\Si$-sentence $\phi$,
    we say $\phi$ is a consequence of $T$
    and say $T \model{\Si} \phi$ 
    when for all non-empty $\Si$-models $\MM$ of $T$, 
    we have $\MM \model{\Si} \phi$.
\end{dfn}
\begin{rmk}
    We have to be a bit careful when we go from something like
    $\MM \model{\Si} \phi(a)$ to deducing something about $T$.
    This is because there might not exist a $\Si$-constant $c$ 
    such that $\modintp{\MM}{c} = a$,
    it only makes sense to write $T \model{\Si} \phi$ if $\phi$
    is a \emph{sentence}.
\end{rmk}

\begin{ex}[Logical consequence]
    Let $T$ be a $\Si$-theory and $\phi$ and $\psi$ be $\Si$-sentences.
    Show that the following are equivalent:
    \begin{itemize}
        \item $T \model{\Si} \phi \to \psi$
        \item $T \model{\Si} \phi$ implies $T \model{\Si} \psi$.
    \end{itemize}
\end{ex}

\begin{dfn}[Consistent theory]
    A $\Si$-theory $T$ is consistent if either of the following equivalent 
    definitions hold:
    \begin{itemize}
        \item
            There does not exists a 
            $\Si$-sentence $\phi$ such that
            $T \model{\Si} \phi$ and $T \model{\Si} \NOT \phi$.
        \item There exists 
            a non-empty $\Si$-model of $T$. 
    \end{itemize}
    Thus the definition of consistent is intuitively 
    `$T$ does not lead to a contradiction'.

    A theory $T$ is finitely consistent if all 
    finite subsets of $T$ are consistent.
    This will turn out to be another equivalent definition,
    given by the \linkto{compactness}{compactness theorem}.
\end{dfn}
\begin{proof} 
    We show that the two definitions are equivalent.
    \begin{forward}
        Suppose no non-empty model exists.
        Take $\phi$ to be the $\Si$-sentence $\top$.
        Since the only model of $T$ is the empty $\Si$-structure.
        Hence all non-empty $\Si$-models of $T$ satisfy $\top$ and $\bot$
        (there are none) so
        $T \model{\Si} \top$ and $T \model{\Si} \bot$.
    \end{forward}
    \begin{backward}
        Suppose $T$ has non-empty $\Si$-model $\MM$
        and $\MM \model{\Si} \phi$ and $\MM \model{\Si} \NOT \phi$.
        This implies $\MM \model{\Si} \phi$ and $\MM \nodel{\Si} \phi$,
        a contradiction.
    \end{backward}
\end{proof}
\subsection{The Compactness Theorem}

Read ahead for the statement of \linkto{compactness}{the Compactness Theorem}.
The first two parts of the theorem are easy to prove.
This chapter focuses on proving the final part.

\begin{dfn}[Witness property]
    Given a signature $\Si$ and a $\Si$-theory $T$, 
    we say that $\Si$ satisfies the witness property when
    for any $\Si$-formula $\phi$ with 
    \linkto{form_fin_var}{exactly one free variable} $v$,
    there exists $c \in \const{\Si}$ such that 
    $T \modelsi \brkt{\exists v, \phi(v)} \to \phi(c)$.
\end{dfn}
This says that if for all $\Si$-model $\MM$ of $T$,
there exists an element $a \in \MM$ such that $\MM \model{\Si} \phi(m)$
then there exists a constant symbol $c$ of $\Si$ 
such that $\phi(\mmintp{c})$ is true.

\begin{dfn}[Maximal theory]
    A $\Si$ theory $T$ is $\Si$-maximal if for any 
    $\Si$-formula $\phi$,
    if $\phi$ is a $\Si$-sentences then
    $\phi \in T$ or $\NOT \phi \in T$.
\end{dfn}

\begin{prop}[Maximum property]\link{max_prop}
    Given a $\Si$-maximal and finitely consistent theory $T$ and
    a $\Si$-sentence $\phi$, 
    \[ T \model{\Si} \phi \text{ if and only if } \phi \in T
    \text{ if and only if } \NOT \phi \notin T
    \text{ if and only if } \nodel{\Si} \NOT \phi\]   
\end{prop}
\begin{proof}
    First note that by maximality and finite consistency
    if $\phi, \NOT \phi \in T$ then we have a finite 
    \linkto{inconsistent} subset 
    $\set{\phi,\NOT \phi} \subs T$, which is false. 
    Hence
    \[\phi \in T \iff \NOT \phi \notin T\]
    We prove the first if and only if and deduce the third by 
    replacing $\phi$ with $\NOT \phi$.
        \begin{forward}
            Suppose $T \model{\Si} \phi$.
            Since $T$ is $\Si$-maximal, 
            we have $\phi \in T$ or $\NOT \phi \in T$.
            If $\NOT \phi \in T$ then we have a finite subset 
            $\set{\phi,\NOT \phi} \subs T$.
            Hence $T$ is \linkto{inconsistent}{not finitely consistent},  
            thus the second case is false.
        \end{forward}
        \begin{backward}
            Suppose $\phi \in T$. 
            Case on $T \model{\Si} \phi$ or $T \nodel{\Si} \phi$.
            If $T \nodel{\Si} \phi$ then there exists 
            $\NN$ a $\Si$-model of $T$ such that 
            $N \nodel{\Si} \phi$.
            But $\NN \model{\Si} \phi$ since $\phi \in T$.
            Thus the second case is false.
        \end{backward}
\end{proof}

\begin{nttn}[Ordering signatures]
    We write $\Si \leq \Si(*)$ for two signatures if
    $\const{\Si} \subs \const{\Si(*)}$,
    $\func{\Si} \subs \func{\Si(*)}$ and $\rel{\Si} \subs \rel{\Si(*)}$.
\end{nttn}

For the sake of formality we include the following two lemmas, 
neither of which are particularly inspiring or significant,
but they do allow us to move freely between signatures.
\begin{lem}[Moving models down signatures]
    \link{move_down_mod}
    Given two signatures such that 
    $\Si \leq \Si(*)$ and
    $\NN$ a $\Si(*)$-structure we can make 
    $\MM$ a $\Si$-structure such that
    \begin{enumerate}
        \item $\carrier{\MM} = \carrier{\NN}$
        \item They have the same interpretation on $\Si$.
        \item For any $\Si$-formula $\phi$ with free variables indexed by $S$
            and any $a \in \MM^S$
            \[\MM \model{\Si} \phi(a) \iff \NN \model{\Si(*)} \phi(a)\]
        \item If $T$ is a $\Si$-theory and $T(*)$ is a $\Si(*)$-theory 
            such that $T \subs T(*)$ and $\NN$ a $\Si(*)$-model of $T(*)$,
            then $\MM$ is a $\Si$-model of $T$.
    \end{enumerate}
    Technically the new structure is not $\NN$,
    but for convenience we will write 
    $\NN$ to mean either of the two and let subscripts involving 
    $\Si$ and $\Si(*)$ describe which one we mean.
\end{lem}
\begin{proof}
    We let the carrier set be the same and 
    define $\intp{\Si}{\MM}$ by restriction:
    \begin{itemize}
        \item $\intp{\const{\Si}}{\MM}$ is
        the restriction of $\intp{\const{\Si(*)}}{\NN}$ to $\const{\Si}$
        \item $\intp{\func{\Si}}{\MM}$ is
        the restriction of $\intp{\func{\Si(*)}}{\NN}$ to $\func{\Si}$
        \item $\intp{\rel{\Si}}{\MM}$ is
        the restriction of $\intp{\rel{\Si(*)}}{\NN}$ to $\rel{\Si}$
    \end{itemize}
    We will need that for any $\Si$-term $t$ with variables indexed by $S$,
    the \link{move_down_mod_int_of_terms} interpretation of terms is equal:
    $\subintp{\Si}{\MM}{t} = \subintp{\Si(*)}{\NN}{t}$.
    Indeed:
    \begin{itemize}
        \item If $t$ is a constant then 
        $\subintp{\Si}{\MM}{t} = \subintp{\Si}{\MM}{c} =
        \subintp{\Si(*)}{\NN}{c} = \subintp{\Si(*)}{\NN}{t}$
        \item If $t$ is a variable then 
        $\subintp{\Si}{\MM}{t} = \id{{\MM}} =
        \id{{\NN}} = \subintp{\Si(*)}{\NN}{t}$
        \item If $t$ is $f(s)$ then by induction
        $\subintp{\Si}{\MM}{t} = \subintp{\Si}{\MM}{f}(\subintp{\Si}{\MM}{s}) =
        \subintp{\Si(*)}{\NN}{f}(\subintp{\Si(*)}{\NN}{s}) = 
        \subintp{\Si(*)}{\NN}{t}$
    \end{itemize}
    
    Let $\phi$ be a $\Si$-formula with variables
    indexed by $S \subs \N$.
    Let $a$ be in $\MM^S$.
    Case on $\phi$ to show that 
    $\MM \model{\Si} \phi(a) \iff \NN \model{\Si(*)} \phi(a)$:
    \begin{itemize}
        \item {If $\phi$ is $\top$ then both satisfy $\phi$.}
        \item If $\phi$ is $t = s$ then
        \[
            \MM \model{\Si} \phi(a) \iff \subintp{\Si}{\MM}{t} = 
            \subintp{\Si}{\MM}{s}
            \iff \subintp{\Si(*)}{\MM}{t} = \subintp{\Si(*)}{\MM}{s} 
            \iff \NN \model{\Si(*)} \phi(a)
        \]
            Since the interpretation 
            of terms are equal from above.
        \item If $\phi$ is $r(t)$ then by how we defined 
        $\subintp{\Si}{\MM}{r}$ and since interpretation 
        of terms are equal
        \[
            \MM \model{\Si} \phi(a) \iff \subintp{\Si}{\MM}{t}(a) 
            \in \subintp{\Si}{\MM}{r}
            \iff \subintp{\Si(*)}{\NN}{t}(a) \in 
            \subintp{\Si(*)}{\NN}{r}
            \iff \NN \model{\Si(*)} \phi(a) 
        \]
        \item If $\phi$ is $\NOT \psi$ then using the induction hypothesis
            \[
                \MM \model{\Si} \phi(a) \iff \MM \nodel{\Si} \psi(a)
                \iff \NN \nodel{\Si(*)} \psi(a)
                \iff \NN \model{\Si(*)} \phi(a) 
            \]
        \item If $\phi$ is $\psi \OR \chi$ then using the induction hypothesis
            \begin{align*}
                \MM \model{\Si} \phi(a) \iff \MM \model{\Si} \psi(a) 
                \text{ or } \MM \model{\Si} \chi(a)
                \iff \NN \model{\Si(*)} \psi(a) \text{ or } 
                \NN \model{\Si(*)} \chi(a)
                \iff \NN \model{\Si(*)} \phi(a) 
            \end{align*}
        \item If $\phi$ is $\forall v, \psi$ then
        \begin{align*}
            \MM \model{\Si} \phi(a) &\iff \forall b \in {\MM}, 
            \MM \model{\Si} \psi(a,b)\\
                &\iff \forall b \in {\MM}, \NN \model{\Si(*)} \psi(a,b)
                & \text{by the induction hypothesis}\\
                &\iff \forall b \in {\NN}, \NN \model{\Si(*)} \psi(a,b)
                & \text{by the induction hypothesis}\\
                & \iff \NN \model{\Si(*)} \phi(a) 
        \end{align*}
    \end{itemize}
    Hence $\MM \model{\Si} \phi(a) \iff \NN \model{\Si(*)} \phi(a)$.

    Suppose $T \subs T(*)$ are respectively $\Si$ and $\Si(*)$-theories and 
    $\NN \model{\Si(*)} T(*)$.
    If $\phi \in T \subs T(*)$ then by the previous part, 
    $\NN \model{\Si(*)} \phi$ implies $\MM \model{\Si} \phi$.
    Hence $\MM \model{\Si} T$.
\end{proof}

\begin{lem}[Moving models and theories up signatures] 
    \link{move_up_mod}
    Suppose $\Si \leq \Si(*)$.
    \begin{enumerate}
        \item Suppose $\MM$ is a $\Si$-model of $\Si$-theory $T$.
            Then if there exists $\MM(*)$ a $\Si(*)$-structure 
            whose carrier set is the same as $\MM$
            and whose interpretation agrees with $\intp{\Si}{\MM}$ 
            on constants, functions and relations of $\Si$,
            then $\MM(*)$ is a $\Si(*)$-model of $T$.
        \item Suppose $T$ is a $\Si$-theory and $\phi$ is a 
            $\Si$-sentence such that $T \model{\Si} \phi$.
            Then $T \model{\Si(*)} \phi$.
    \end{enumerate}
    Again, if we have constructed such a $\MM(*)$ from $\MM$
    we tend to just refer to it as $\MM$ and let subscripts involving 
    $\Si$ and $\Si(*)$ describe which one we mean.
\end{lem}
\begin{proof}~
    \begin{enumerate}
        \item Suppose $\MM \model{\Si} T$.
        Let $\phi \in T$.
        To show that 
        $\MM(*) \model{\Si(*)} \phi$ we
        first we prove a useful claim:
        if $t \in \term{\Si}$ with no variables then 
        $\subintp{\Si(*)}{\MM(*)}{t} = \subintp{\Si}{\MM}{t}$.
        Case on what $t$ is:
        \begin{itemize}
            \item If $t$ is a constant symbol $c$ in $\const{\Si}$, 
                then since $\intp{\Si(*)}{\MM(*)} = 
                    \intp{\Si}{\MM}$ on $\Si$,
                \[\subintp{\Si(*)}{\MM(*)}{t} = 
                    \subintp{\Si(*)}{\MM(*)}{c} =
                    \subintp{\Si}{\MM}{c} =
                    \subintp{\Si}{\MM}{t}\]
            \item If $t$ is a variable then it has one variable,
            thus false.
            \item If $t$ is $f(s)$ then since 
            $\intp{\Si(*)}{\MM(*)} = 
            \intp{\Si}{\MM}$ on $\Si$,
            \[
                \subintp{\Si(*)}{\MM(*)}{t} =
                \subintp{\Si(*)}{\MM(*)}{f(s)} =
                \subintp{\Si(*)}{\MM(*)}{f}(\subintp{\Si(*)}{\MM(*)}{s}) =
                \subintp{\Si}{\MM}{f}(\subintp{\Si}{\MM}{s}) =
                \subintp{\Si}{\MM}{f(s)} =
                \subintp{\Si}{\MM}{t}
            \]
        \end{itemize}
        Case on what $\phi$ is ($\phi$ has no variables):
        \begin{itemize}
            \item {If $\phi$ is $\top$ then it is satisfied.}
            \item If $\phi$ is $t = s$, 
                then by the claim above,
                \[
                    \subintp{\Si(*)}{\MM(*)}{t} = 
                    \subintp{\Si}{\MM}{t} = 
                    \subintp{\Si}{\MM}{s} = 
                    \subintp{\Si(*)}{\MM(*)}{s}
                \]
            \item If $\phi$ is $r(t)$,
                then by the claim above
                and the fact that relations are interpreted the same way,
                \[
                    \subintp{\Si(*)}{\MM(*)}{t} = 
                    \subintp{\Si}{\MM}{t} \in
                    \subintp{\Si}{\MM}{r} = 
                    \subintp{\Si(*)}{\MM(*)}{r}
                \]
            \item If $\phi$ is $\NOT \psi$ 
                then using the induction hypothesis on $\psi$,
                \[
                    \MM \model{\Si} \phi \; \iff \;
                    \MM \nodel{\Si} \psi \; \iff \;
                    \MM(*) \nodel{\Si(*)} \psi \; \iff \;
                    \MM(*) \model{\Si(*)} \phi
                \]
            \item If $\phi$ is $\psi \OR \chi$
                then using the induction hypothesis on $\psi$ and $\chi$,
                \[
                    \MM \model{\Si} \phi \iff 
                    \MM \model{\Si} \psi \text{ or } \MM \model{\Si} \chi 
                    \iff \MM(*) \model{\Si(*)} \psi \text{ or } 
                    \MM(*) \model{\Si(*)} \chi 
                    \iff \MM(*) \model{\Si(*)} \phi
                \]
            \item If $\phi$ is $\forall v, \psi(v)$ and 
                $a \in {\MM(*)} = {\MM}$
                then using the induction hypothesis on $\psi$,
                $\MM \model{\Si} \psi(a) \implies \MM(*) \model{\Si(*)} \psi(a)$.
                Hence $\MM(*) \model{\Si(*)} \phi$.
        \end{itemize}
        Thus $\MM(*)$ is a $\Si(*)$-model of $T$.

        \item Suppose $T \model{\Si} \phi$.
        If $\MM(*) \model{\Si(*)} T$ then by 
        \linkto{move_down_mod}{moving $\MM(*)$ down to $\Si$},
        we have a corresponding $\MM \model{\Si} T$ 
        whose carrier set is the same as $\MM(*)$.
        Hence $\MM \model{\Si} \phi$.
        Naturally the models agree on interpretation
        of constants, functions and relations of $\Si$ and so 
        $\MM(*) \model{\Si(*)} \phi$ by the previous part.
    \end{enumerate}
\end{proof}

The following lemma is the bulk of the proof of the compactness theorem.
\begin{lem}[Henkin construction]
    \link{henkin}
    Let $\Si$ be a signature.
    Let $0 < \ka$ be a cardinal such that $|\const{\Si}| \leq \ka$.
    If a $\Si$-theory $T$ 
    \begin{itemize}
        \item has the witness property
        \item is $\Si$-maximal
        \item is finitely consistent
    \end{itemize}
    then it has a non-empty $\Si$-model $\MM$ such that $|\MM| \leq \ka$.
\end{lem}
\begin{proof}
    Without loss of generality $\Si$ is non-empty, 
    since we can add a constant symbol to $\Si$, 
    \linkto{move_up_mod}{see $T$ as a theory 
    in that signature}, and then 
    \linkto{move_down_mod}{take the model we make of $T$ back down} to 
    being a $\Si$-model of $T$.
    \textbf{The $\Si$-structure}: 
    Consider quotienting $\const{\Si}$ by the equivalence relation
    $c \sim d$ if and only if $T \modelsi c = d$.
    Let $\pi : \const{\Si} \to \const{\Si} / \sim $.
    This defines a non-empty $\Si$-structure $\MM$ in the following way:
    \begin{enumerate}
        \item We let the carrier set be the image of the quotient.
        We let the constant symbols be interpreted as their equivalence classes:
        $\intp{\const{\Si}}{\MM} = \pi$.
        We now have the desired cardinality for $\MM$:\footnote{From 
            this point onwards when it is obvious what we mean 
            we write $\MM$ for $\carrier{\MM}$.} 
        $|{\MM}| \leq |\const{\Si}| \leq \ka$
        and $\const{\Si}$ is non-empty so $\MM$ is non-empty.
        \item To interpret functions we must use the witness property.
        Given $f \in \func{\Si}$ and $\pi(c) \in \MM^{n_f}$
        ($\pi$ is surjective),
        we obtain by the witness property some $d \in \const{\Si}$
        such that \[T \modelsi (\exists v, f(c) = v) \to (f(c) = d)\]
        We define $\mmintp{f}$ to map $\pi(c) \mapsto \pi(d)$.
        
        To show that $\mmintp{f}$ is well-defined,
        let $c_0,c_1 \in (\const{\Si})^{n_f}$ 
        such that $\pi(c_0) = \pi(c_1)$ and
        suppose $\pi(d_0),\pi(d_1)$ are their images.
        It suffices that $\pi(d_0) = \pi(d_1)$, i.e. $T \modelsi d_0 = d_1$.
        Indeed, let $\Si$-structure $\NN$ be a model of $T$.
        Then
        \[
            \NN \modelsi (\exists v, f(c_0) = v) 
            \implies 
            \NN \modelsi f(c_0) = d_0
        \]
        Similarly $f(c_1) = d_1$. Hence 
        \[\modintp{\NN}{d_0} = \modintp{\NN}{f}(\modintp{\NN}{c_0})
        = \modintp{\NN}{f}(\modintp{\NN}{c_1}) = \modintp{\NN}{d_1}\]
        and $N \modelsi d_0 = d_1$.
        Hence $T \modelsi d_0 = d_1$.
        \item Let $r \in \rel{\Si}$. 
        We define $\mmintp{r} := 
        \set{\pi(c) \st c \in (\const{\Si})^{m_r} \AND T \modelsi r(c)}$
    \end{enumerate}
    Hence $\MM$ is a $\Si$-structure.
    We want to show that $\MM$ is a $\Si$-model of $T$.

    \textbf{Terms}: to show that $\MM$ satisfies formulas of $T$ 
    we first need that interpretation of terms is working correctly. 
    Claim: if $t \in \term{\Si}$ with variables indexed by $S$,
    $d \in \const{\Si}$ and
    $c$ is in $(\const{\Si})^S$
    then $T \modelsi t(c) = d$ 
    if and only if $\mmintp{t}(\mmintp{c}) = \mmintp{d}$.
    i.e. $\MM \model{\Si} t(c) = d$.
    Case on what $t$ is:
    \begin{itemize}
        \item If $t$ is a constant symbol,
        $T \modelsi t(c) = d$ if and only if 
        $T \modelsi t = d$ 
        if and only if $\pi(t) = \pi(d)$
        if and only if $\mmintp{t} = \mmintp{t}(\mmintp{c}) = \mmintp{d}$.
        \item Suppose $t \in \var{\Si}$,
        then it suffices to show that $T \modelsi c = d$ 
        if and only if $\mmintp{c} = \mmintp{d}$, 
        which we have already done above.
        \item With the induction hypothesis, 
        suppose $t = f(s)$, 
        where $f \in \func{\Si}$ and $s \in (\term{\Si})^{n_f}$.
        \begin{forward} 
            If we can find 
            $e = (e_1, \dots e_{n_f}) \in (\const{\Si})^{n_f}$ 
            such that each 
            $T \modelsi s_i (c) = e_i$
            and $\mmintp{f(e)} = \mmintp{d}$,
            then we have 
            $\mmintp{s_i}(\mmintp{c})
            = \mmintp{e_i}$ and so
            \[
                \mmintp{t}(\mmintp{c}) = \mmintp{(f(s))}(\mmintp{c})
                = \mmintp{f}(\mmintp{s}(\mmintp{c}))
                = \mmintp{f}(\mmintp{e})
                = \mmintp{f(e)} = \mmintp{d}
            \]
            Indeed, using the witness property $n_f$
            times we can construct $e$.
            Suppose we have by induction
            $e_1,\dots, e_{i-1} \in \const{\Si}$
            such that for each $j<i$, 
            they satisfy
            \[
                T \modelsi \exists x_{j+1}, \dots, \exists x_{n_f}, 
                f(e_1, \dots, e_j, x_{j+1},\dots, x_{n_f}) = d 
                \AND e_j = s_j (c)
            \]
            For each $i$ let $\phi_i$ be the formula
            \[
                \exists x_{i+1}, \dots, \exists x_{n_f}, 
                f(e_1, \dots, e_{i-1},v,x_{i+1},\dots, x_{n_f}) = d 
                \AND v = s_i (c)
            \]
            with a single variable $v$.
            Then by the witness property,
            there exists an $e_i \in \const{\Si}$ such that
            $T \modelsi \exists v, \phi_i(v) \to \phi_i(e_i)$.
            To complete the induction we show that $T \modelsi \phi_i(e_i)$.
            Then we will be done since each $\phi_i(e_i)$ 
            will give us $T \modelsi s_i (c) = e_i$
            and the last $\phi_{n_f}$ gives $\mmintp{f(e)} = \mmintp{d}$.

            To this end, let $\NN$ be a model of $T$.
            Then $\NN \modelsi \exists v, \phi_i(v) \to \phi_i(e_i)$.
            By assumption $T \modelsi t(c) = d$ so 
            $\nnintp{t}(\nnintp{c}) = \nnintp{d}$,
            hence 
            \[
                \modintp{\NN}{f}(
                    \modintp{\NN}{s_1}(c),\dots,\modintp{\NN}{s_{n_f}}(c)) = 
                \modintp{\NN}{d}
            \]
            Taking $v$ to be $s_i(c)$ and $x_{i+k}$ to be $s_{i+k}(c)$
            we have that 
            $\NN \modelsi \exists v, \phi_i(v)$.
            Hence $\NN \modelsi \phi_i(e_i)$.
            Thus we have $T \modelsi \phi_i(e_i)$.
        \end{forward}

        \begin{backward} 
            Note that each $\mmintp{(s_i(c))} = \mmintp{e_i}$ 
            for some $e_i \in \const{\Si}$ 
            since $\pi$ is surjective.
            By the induction hypothesis for each $i$ we have 
            $T \model{\Si} s_i(c) = e_i$.
            Hence 
            \[
                \mmintp{f(e)}
                = \mmintp{f}(\mmintp{e}) = \mmintp{f}(\mmintp{(s(c))}) 
                = \mmintp{f}(\mmintp{s}(\mmintp{c})) 
                = \mmintp{t}(\mmintp{c}) = \mmintp{d}
            \]
            Hence $\pi(f(e)) = \pi(d)$ and 
            $T \modelsi f(e) = d$.
            It follows that 
            $T \modelsi t(c) = d$.
        \end{backward}
    \end{itemize}
    Thus $T \modelsi t(c) = d \iff \mmintp{t}(\mmintp{c}) = \mmintp{d}$.

    \textbf{Formulas}:
    now we can show that $\MM \modelsi T$. 
    Since for all
    $\phi \in T$ we have $T \modelsi \phi$,
    it suffices to show that for all $\Si$-sentences
    $\phi, T \modelsi \phi $ 
    implies $\MM \modelsi \phi$.
    We prove a stronger statement
    which will be needed for the induction:
    for all $\Si$-formulas $\phi$ with variables indexed by $S$ and 
    $c \in (\const{\Si})^S$,
    \[
        T \modelsi \phi(c) \iff \MM \modelsi \phi(\mmintp{c})
    \]
    We case on what $\phi$ is:
    \begin{itemize}
        \item Case $\phi$ is $\top$:
        all $\Si$-structures satisfy $\top$.
        \item Case $\phi$ is $t = s$: 
            \begin{forward} 
                Apply the witness property to 
                $t(c) = v$,
                obtaining $d \in \const{\Si}$ such that
                $T \modelsi (\exists v, t(c) = v) \to t(c) = d$.
                Since clearly $T \modelsi \exists v, t(c) = v$
                (take any model 
                and it has an interpretation of $t(c)$) 
                we have $T \modelsi t(c) = d$.
                Also by assumption 
                $T \modelsi t(c) = s(c)$
                so it follows that $T \modelsi s(c) = d$.
                Using the claim from before for terms, 
                we obtain
                $\mmintp{t}(\mmintp{c}) = \mmintp{d} = 
                \mmintp{s}(\mmintp{c})$.
                Hence $\MM \modelsi t(\mmintp{c}) = s(\mmintp{c})$.
            \end{forward}

            \begin{backward}
                If $\MM \modelsi t(\mmintp{c}) = s(\mmintp{c})$ 
                then since $\pi$ is surjective,
                there exists $d \in \const{\Si}$ such that 
                \[\mmintp{t}(\mmintp{c}) = \mmintp{s}(\mmintp{c})= \mmintp{d}\]
                Using the claim for terms
                we obtain $T \model{\Si} t(c) = d$ and 
                $T \model{\Si} s(c) = d$.
                It follows that $T \model{\Si} t(c) = s(c)$.
            \end{backward}

        \item Case $\phi$ is $r(t)$:
            \begin{forward}
                Suppose $T \model{\Si} r(t(c))$.
                By induction, 
                apply the witness property 
                $m_r$ times to the formulas
                \[\exists x_{i + 1}, \dots, 
                \exists x_{m_r}, r(
                    \dots, e_{i-1}, v , x_{i+1},\dots
                    ) \AND v = t_i(c)\]
                each time obtaining $e_i \in \const{\Si}$ satisfying the formula.
                The result is $T \modelsi r(e)$ and each
                $T \model{\Si} t_i(c) = e_i$.
                Using the claim for terms and 
                how we interpreted relations in $\MM$ this implies
                $\mmintp{t}(\mmintp{c}) = \mmintp{e} \in \mmintp{r}$,
                and hence $\MM \modelsi r(t(c))$.
            \end{forward}

            \begin{backward}
                Suppose $\MM \modelsi r(t(c))$.
                Since $\pi$ is surjective, 
                there exists $e \in \const{\Si}$ 
                such that $\mmintp{e} = \mmintp{t}(\mmintp{c}) \in \mmintp{r}$.
                Using the claim for terms again
                we obtain $T \model{\Si} t(c) = e$ and 
                using how $\MM$ interprets relations,
                $T \model{\Si} r(e)$.
                It follows that $T \model{\Si} r(t(c))$.
            \end{backward}
        
        \item Case $\phi$ is $\NOT \chi$:
            Using \linkto{max_prop}{the maximal property of $T$} 
            for the first $\iff$
            and the induction hypothesis for the second $\iff$ we have
            \[T \model{\Si} \NOT \chi(c) \; \iff \;  
            T \nodel{\Si} \chi(c) \; \iff \; \MM \nodel{\Si} \chi(c) \; \iff \;
            \MM \model{\Si} \NOT \chi(c)\]

        \item Case $\phi$ is $\chi_0 \OR \chi_1$
            \begin{align*}
                \MM \model{\Si} \chi_0(\mmintp{c}) \OR \chi_1(\mmintp{c}) 
                    &\;\iff\; \MM \model{\Si} \chi_0(\mmintp{c}) 
                    \text{ or } \MM \model{\Si} \chi_1(\mmintp{c}) \\
                    &\;\iff\; T \model{\Si} \chi_0(c) \text{ or } 
                    T \model{\Si} \chi_1(c) 
                    \text{\quad by the induction hypothesis}
            \end{align*}
            Hence it suffices to show that 
            \[T \model{\Si} \chi_0(c) \text{ or } T \model{\Si} \chi_1(c) 
            \;\iff\; T \model{\Si} \chi_0(c) \OR \chi_1(c)\]
            \begin{forward}
                Suppose $T \model{\Si} \chi_0(c)$ or 
                $T \model{\Si} \chi_1(c)$.
                For $\NN$ a $\Si$-model of $T$,
                \[
                    \NN \model{\Si} \chi_0(\nnintp{c}) \text{ or } 
                    \NN \model{\Si} \chi_1(\nnintp{c}) 
                    \implies \NN \model{\Si} \chi_0(\nnintp{c}) 
                    \OR \chi_1(\nnintp{c})
                \]
                Thus $T \model{\Si} \chi_0(c) \OR \chi_1(c)$.
            \end{forward}

            \begin{backward}
                Suppose $T \model{\Si} \chi_0(c) \OR \chi_1(c)$.
                For $\NN$ a $\Si$-model of $T$,
                \[
                    \NN \model{\Si} \chi_0(\nnintp{c}) 
                    \OR \chi_1(\nnintp{c}) 
                    \implies
                    \NN \model{\Si} \chi_0(\nnintp{c}) \text{ or } 
                    \NN \model{\Si} \chi_1(\nnintp{c}) 
                \]
                Thus $T \model{\Si} \chi_0(c)$ or $T \model{\Si} \chi_1(c)$.
            \end{backward}

        \item Case $\phi$ is $\forall v, \chi(v)$
            \begin{forward}
                Let $d \in {\MM}$, 
                then since $\pi$ surjective $\exists e \in \const{\Si}$
                such that $\pi(e) = d$.
                Since $T \model{\Si} \forall v, \chi(c,v)$ 
                it follows that $T \model{\Si} \chi(c,e)$
                and by induction $\MM \model{\Si} \chi(\mmintp{c},d)$.
                Hence $\MM \model{\Si} \forall v, \chi(\mmintp{c},v)$.
            \end{forward}

            \begin{backward}
                We show the contrapositive.
                If $T \nodel{\Si} \forall v, \chi(c) (v)$, 
                then by \linkto{max_prop}{the maximal property of $T$},
                $T \model{\Si} \exists v, \NOT \chi(c,v)$.
                Applying the witness property to $\NOT \chi(c,v)$,
                there exists $e \in \const{\Si}$ such that 
                \[
                    T \model{\Si} (\exists v, \NOT \chi(c) (v)) \to 
                    (\NOT \chi(c) (v))
                    \quad 
                    \implies
                    \quad T \model{\Si} \NOT \chi(c) (v)
                \]
                Thus $T \nodel{\Si} \chi(c) (v)$ by 
                \linkto{max_prop}{the maximal property of $T$}
                and $\MM \nodel{\Si} \chi(\mmintp{c}) (v)$ 
                by the induction hypothesis.
                Hence $\MM \nodel{\Si} \forall v, \chi(\mmintp{c}) (v)$.
            \end{backward}
    \end{itemize}
    Thus $T \modelsi \phi \iff \MM \modelsi \phi$ and we are done.
\end{proof}

\begin{prop}[Giving Theories the Witness Property]
    \link{make_wit}
    Suppose $\Si(0)$-theory 
    $T(0)$ is finitely consistent.
    Then there exists a signature 
    $\Si(*)$ and $\Si(*)$-theory $T(*)$ such that 
    \begin{enumerate}
        \item $\const{\Si(0)} \subs \const{\Si(*)}$
        and they share the same function and relation symbols.
        \item $|\const{\Si(*)}| = |\const{\Si(0)}| + \aleph_0$
        \item $T(0) \subs T(*)$
        \item $T(*)$ is finitely consistent
        \item Any $\Si(*)$-theory $T'$ such that $T(*) \subs T'$ 
        has the witness property
    \end{enumerate}
\end{prop}
\begin{proof}
    Again without loss of generality $\Si(0)$ is non-empty.
    We want to define $\Si(i),T(i)$, 
    for each $i \in \N$.    
    By induction, 
    we assume we have $\Si(i)$ non-empty a signature and 
    $T(i)$ a $\Si$-theory such that 
    \begin{enumerate}
        \item $\const{\Si(0)} \subs \const{\Si(i)}$
        and they share the same function and relation symbols.
        \item $|\const{\Si(i)}| = |\const{\Si(0)}| + \aleph_0$
        \item $T(0) \subs T(i)$
        \item $T(i)$ is finitely consistent
    \end{enumerate}
    Let 
    \[
        W(i) := \set{\phi \in \form{\Si(i)} 
        \st \phi \text{ has exactly one free variable}}
    \]
    We construct $\Si(i+1)$ 
    by adding constant symbols $c_\phi$ for each $\phi \in W(i)$ and 
    keeping the same function and relation symbols of $\Si(i)$:
    \[
        \const{\Si(i+1)} :=
        \const{\Si(i)} \sqcup \set{c_\phi
        \st \phi \in W(i)}
    \]
    We create a witness formula $w(\phi)$ for each formula $\phi \in W$:
    \begin{align*}
        w : W(i) &\to \form{\Si(i+1)}\\
        \phi &\mapsto ((\exists v, \phi(v)) \to \phi(c_\phi))
    \end{align*}
    Then let \[T(i+1) := T(i) \cup w(W(i))\]
    Certainly
    $T(i+1)$ is a $\Si(i+1)$-theory such that $T(0) \subs T(i+1)$,
    $\const{\Si(0)} \subs \const{\Si(i+1)}$ 
    where the function and relation symbols are unchanged.
    Since $W(i)$ is countibly infinite, 
    $|\const{\Si(i+1)}| = |\const{\Si(i)}| + \aleph_0 = 
    |\const{\Si(0)}| + \aleph_0 $.
    We need to check that $T(i+1)$ is finitely consistent.
    Take a finite subset of $T(i+1)$.
    It is a union of two finite sets 
    $\De_T \subs T(i)$ and $\De_w \subs w(W(i))$.
    Since $T(i)$ is finitely consistent there exists a $\Si(i)$-model
    $\MM(i)$ of $\De_T$.
    Let $\MM(i+1)$ be defined to have carrier set ${\MM(i)}$.
    There exists some $b \in \MM(i) = \MM(i+1)$ since $\Si(i)$ is non-empty.
    To define interpretation for $\MM(i+1)$, 
    let $c \in \const{\Si(i+1)}$:
    \[
        \subintp{\Si(i+1)}{\MM(i+1)}{c} := 
        \begin{cases}
            \subintp{\Si(i)}{\MM(i)}{c} \quad &\text{when }  
            c \in \const{\Si(i)} \\
            a \quad &\text{when } c = c_\phi \text{ and } 
            \exists a \in {\MM(i)}, \MM(i) \model{\Si(i)} \phi(a)\\
            b \quad &\text{when } c = c_\phi \text{ and } 
            \forall a \in {\MM(i)}, \MM(i) \nodel{\Si(i)} \phi(a)
        \end{cases}
        \]
    Then $\MM(i+1)$ is a well defined $\Si(i+1)$-structure.
    We check is is a $\Si(i+1)$-model of $\De_T \cup \De_w$.
    Since 
    $\intp{\Si(i+1)}{\MM(i+1)}$ agrees with 
    $\intp{\Si(i)}{\MM(i)}$ for constants, 
    functions and relations from $\MM(i)$ - a $\Si(i)$-model
    of $\De_T$ -
    \linkto{move_up_mod}{it is a $\Si(*)$-model of $\De_T$}.
    If $\psi \in \De_w$ then it is 
    $\exists v, \phi(v) \to \phi(c_\phi)$ for some 
    $\phi \in W(i)$.
    Supposing that $\MM(i+1) \model{\Si(i+1)} \exists v, \phi(v)$
    it suffices to show $\MM(i+1) \model{\Si(i+1)} \phi(c_\phi)$.
    Then there exists $a \in \MM(i + 1) = \MM(i)$ such that 
    $\MM(i+1) \model{\Si(i+1)} \phi(a)$.
    \linkto{move_down_mod}{Hence} $\MM(i) \model{\Si(i)} \phi(a)$
    and so $c_\phi$ is interpreted as $a$ in $\MM(i+1)$.
    Hence $\MM(i + 1) \model{\Si(i)} \phi(c_\phi)$.
    Thus the induction is complete.

    Let $\Si(*)$ 
    be the signature such that its function and relations 
    are the same as $\Si(0)$
    and $\const{\Si(*)} = \bigcup_{i \in \N} \const{\Si(i)}$.
    Then 
    \[
        |\const{\Si(*)}| = |\bigcup_{i \in \N} \const{\Si(i)}| = 
        \aleph_0 \times (\aleph_0 + \const{\Si(0)}) = 
        \aleph_0 + \const{\Si(0)}
    \]

    Let $T(*) = \bigcup_{i \in \N} T(i)$.
    Any finite subset of $T(*)$ is a subset of some $T(i)$, 
    hence has a non-empty $\Si(i)$-model $\MM$.
    Checking the relevant conditions for
    \linkto{move_up_mod}{moving models up signatures},
    we have
    $\MM(*)$ a $\Si(*)$-model of the finite subset
    (by interpreting the new constant symbols as the 
    element of the non-empty carrier set.).
    Hence $T(*)$ is finitely consistent.
    
    If $T'$ is a $\Si(*)$-theory such that $T(*) \subs T'$,
    and $\phi$ is a $\Si(*)$-formula of exactly one variable.
    There exists an $i \in \N$ such that $\phi \in \form{\Si(i)}$.
    Since $c_\phi \in \Si(i+1)$ satisfies 
    $T(i) \model{\Si(i+1)} \brkt{\exists v, \phi(v)} \to \phi(c_\phi)$,
    by \linkto{move_up_mod}{moving the logical consequence up to $\Si(*)$,}
    we have $T(i) \model{\Si(*)} \brkt{\exists v, \phi(v)} \to \phi(c_\phi)$.
    If $\NN$ is a $\Si(*)$-model of $T'$
    then it is a $\Si(*)$-model of $T(i)$,
    then $\NN \model{\Si(*)} \brkt{\exists v, \phi(v)} \to \phi(c_\phi)$.
    Hence $T' \model{\Si(*)} \brkt{\exists v, \phi(v)} \to \phi(c_\phi)$,
    satisfying the witness property.
\end{proof}

\begin{lem}[Adding Formulas to Consistent Theories]
    \link{adding_formulas}
    If $T$ is a finitely consistent $\Si$-theory 
    and $\phi$ is a $\Si$-sentence then at least one of
    $T \cup \set{\phi}$ or $T \cup \set{\NOT \phi}$ is finitely consistent.
\end{lem}
\begin{proof}
    We show that for any finite $\De \subs T \cup \set{\phi}$ and 
    for any finite $\De_\NOT \subs T \cup \set{\NOT \phi}$,
    one of $\De$ or $\De_\NOT$ is consistent. 
    The finite subset
    \[(\De \setminus \set{\phi}) \cup 
    (\De_\NOT \setminus \set{\NOT \phi}) \subs T\]
    is consistent by finite consistency of $T$.
    Let $\MM$ be the model of 
    $(\De \setminus \set{\phi}) \cup 
    (\De_\NOT \setminus \set{\NOT \phi})$.
    Case on whether $\MM \model{\Si} \phi$ or not.
    In the first case $\MM \model{\Si} \De$
    and in the second $\MM \model{\Si} \De_\NOT$.
    Hence $T \cup \set{\phi}$ or $T \cup \set{\NOT \phi}$ 
    is finitely consistent.
\end{proof}

\begin{ex}
    Find a signature $\Si$, a consistent $\Si$-theory $T$
    and $\Si$-sentence $\phi$ such that  
    $T \cup \set{\phi}$ and 
    $T \cup \set{\NOT \phi}$ are both consistent.
\end{ex}
    
\begin{prop}[Extending a finitely consistent theory to a maximal theory (Zorn)]
    \link{make_max}
    Given a finitely consistent $\Si$-theory $T(0)$
    there exists a $\Si$-theory $T(*)$ such that 
    \begin{enumerate}
        \item $T(0) \subs T(*)$
        \item $T(*)$ is finitely consistent.
        \item $T(*)$ is $\Si$-maximal.
    \end{enumerate}
\end{prop}
\begin{proof}
    We use Zorn's Lemma.
    Let 
    \[Z := \set{T \in \theory{\Si} \st T 
    \text{ finitely consistent and } T(0) \subs T}\]
    be ordered by inclusion.
    Let $T(0) \subs T(1) \subs \cdots $ be a chain.
    Then $\bigcup_{i \in \N} T(i)$ is a $\Si$-theory
    such that any finite subset is a subset of some $T(i)$,
    hence is consistent by finite consistency of $T(i)$.
    Zorn's lemma implies there exists a $T(*) \in Z$
    that is maximal (in the order theory sense).
    Since $T(*)$ is in $Z$, 
    we have that it is a finitely consistent $\Si$-theory containing $T(0)$.
    
    To show that it is $\Si$-maximal we take a $\Si$-sentence $\phi$.
    By \linkto{adding_formulas}{the previous result,} 
    $T(*) \cup \set{\phi}$ or 
    $T(*) \cup \set{\NOT \phi}$ is finitely consistent.
    Hence 
    $T(*) \cup \set{\phi} = T(*)$ or 
    $T(*) \cup \set{\NOT \phi} = T(*)$ by (order theoretic) maximality,
    so $\phi \in T(*)$ or $\NOT \phi \in T(*)$.
\end{proof}

\begin{nttn}[Cardinalities of signatures and structures]
    Given a signature $\Si$, 
    we write
    $|\Si| := |\const{\Si}| + |\func{\Si}| + |\rel{\Si}|$
    and call this the cardinality of the signature $\Si$.
\end{nttn}

\begin{prop}[The compactness theorem]
    \link{compactness}
    If $T$ is a $\Si$-theory, 
    then the following are equivalent:
    \begin{enumerate}
        \item $T$ is finitely consistent.
        \item $T$ is consistent
        \item For any infinite cardinal $\ka$ 
        such that $|\Si| \leq \ka$, 
        there exists a non-empty $\Si$-model of 
        $T$ with cardinality $\leq \ka$.
    \end{enumerate}
\end{prop}
\begin{proof}
    3. implies 2. and 2. implies 1. are both obvious. 
    A proof of 1. implies 3. follows.

    Suppose an
    $\Si(0)$-theory $T(0)$ is finitely consistent.
    Let $\ka$ be an infinite cardinal such that 
    $|\Si(0)| \leq \ka$. Then
    $|\const{\Si(0)}| \leq |\Si(0)| \leq \ka$.
    We wish to find a $\Si$-model of $T$ with cardinality $\leq \ka$.
    We have shown that \linkto{make_wit}{there exists a signature 
    $\Si(1)$ and $\Si(1)$-theory $T(1)$} such that 
    \begin{enumerate}
        \item $\const{\Si(0)} \subs \const{\Si(1)}$
        and they share the same function and relation symbols.
        \item $|\const{\Si(1)}| = |\const{\Si(0)}| + \aleph_0$
        \item $T(0) \subs T(1)$
        \item $T(1)$ is finitely consistent.
        \item Any $\Si(1)$-theory $T$ such that $T(1) \subs T$ 
        has the witness property.
    \end{enumerate}
    $T(1)$ is finitely consistent so
    \linkto{make_max}{there exists a $\Si(1)$-theory $T(2)$} such that 
    \begin{enumerate}[resume]
        \item $T(1) \subs T(2)$
        \item $T(2)$ is finitely consistent.
        \item $T(2)$ is $\Si(1)$-maximal.
    \end{enumerate}
    Furthermore, $T(2)$ has the witness property due to point 5.
    Since $T(2)$ has the witness property, is $\Si(1)$-maximal
    and finitely consistent,
    $T(2)$ has a non-empty $\Si(1)$-model $\MM$ such that $|\MM| \leq \ka$
    \linkto{henkin}{by Henkin Construction}.
    $\MM \model{\Si(1)} T(0)$ since $T(0) \subs T(1) \subs T(2)$.
    We can \linkto{move_down_mod}{move $\MM$ down to $\Si(0)$,}
    obtaining $\MM \model{\Si(0)} T(0)$.
\end{proof}
\subsection{The Category of Structures}
\begin{dfn}[$\Si$-morphism, $\Si$-embedding, $\Si$-isomorphism]
    \link{partial_morph_dfn}
    Given $\Si$ a signature, 
    $\MM, \NN$ both $\Si$-structures, $A \subs \carrier{\MM}$
    and $\io : A \to \carrier{\NN}$,
    we call $\io$ a partial $\Si$-morphism from $\MM$ to $\NN$ when 
    \begin{itemize}
        \item For all $c \in C$ (such that $\mmintp{c} \in A$), 
        \[\io(\mmintp{c}) = \modintp{\NN}{c}\]
        \item For all $f \in F$ and all $a \in M^{n_f}$
        (such that $\mmintp{f}(a) \in A$), 
        \[\io \circ \mmintp{f}(a) = \modintp{\NN}{f} \circ \io(a)\]
        \item For all $r \in R$, for all $a \in M^{m_r} \cap A^{m_r}$,
        \[a \in \mmintp{r} \implies \io (a) \in \modintp{\NN}{r}\]
    \end{itemize}
    If in addition for relations we have
    \[a \in \mmintp{r} \limplies \io (a) \in \modintp{\NN}{r} \quad 
    \text{and} \quad \io \text { is injective,} \]
    then $\io$ is called a partial $\Si$-embedding 
    (the word extension is often used interchangably with embedding). 

    In the case that $A = \carrier{\MM}$
    we write $\io : \MM \to \NN$ and call $\io$ a $\Si$-morphism.
\end{dfn}
The notion of morphisms here will be the same as 
morphisms in the algebraic setting.
For example in the signature of monoids (groups), 
preserving interpretation of constant symbols 
says the identity is sent to the identity 
and preserving interpretation of function symbols
says the multiplication is preserved.

\begin{dfn}[Elementary Embedding]
    A partial $\Si$-embedding $\io : A \to \NN$ (for $A \subs \MM$)
    is elementary if for any $\Si$-formula $\phi$
    with variables indexed by $S$
    and $a \in A^S$,
    \[
        \MM \model{\Si} \phi(a) \quad \iff \quad \NN \model{\Si} \phi(\io(a))
    \]
\end{dfn}

The following is exactly what we expect - that terms are well behaved 
with respect to morphisms.
\begin{lem}[$\Si$-morphisms commute with interpretation of terms]
    \link{morph_comm_term_intp}
    Given a $\Si$-morphism $\io : \MM \to \NN$, 
    we have that for any $\Si$-term $t$ with variables indexed by $S$ and 
    $a \in \MM^S$,
    \[\io(\mmintp{t}(a)) = \nnintp{t}(\io(a))\]
\end{lem}
\begin{proof}
    We case on what $t$ is:
    \begin{itemize}
        \item If $t = c \in \const{\Si}$ then 
            \[
                \io(\mmintp{t}(a)) = \io(\mmintp{c}) = 
                \nnintp{c} = \nnintp{t}(\io(a))
            \]
        \item If $t = v \in \var{\Si}$ then 
            \[
                \io(\mmintp{t}(a)) =
                \io(\mmintp{v}(a)) = \io(a) = 
                \nnintp{v}(\io(a)) = \nnintp{t}(\io(a))
            \]
        \item If $t = f(s)$ then
            \[
                \io(\mmintp{t}(a)) = \io \circ \mmintp{f}(\mmintp{s}(a)) = 
                \nnintp{f} \circ \io(\mmintp{s}(a)) = 
                \nnintp{f} \circ \nnintp{s}(\io(a)) =
                \nnintp{t}(\io(a))
            \]
            Using the induciton hypothesis in the penultimate step.
    \end{itemize}
\end{proof}

It is worth knowing that the set of $\Si$-structures form a category:
\begin{dfn}[The category of $\Si$-structures]
    \link{category_of_structures}
    Given a signature $\Si$, 
    we have $\Mod{\Si}$ - a category where objects are $\Si$-structures 
    and morphisms are $\Si$-morphisms.
    
    Clearly for any $\MM$, 
    the identity exists and is a $\Si$-morphism.
    We show that composition of morphisms are morphisms.
    Furthermore, composition of embeddings are embeddings
    and composition of elementary embeddings are elementary.
    Thus we could also define morphisms between objects to be embeddings
    or elementary embeddings and obtain a subcategory.

    Hence we inherit a notion of isomorphism of $\Si$-structures
    from category theory.
\end{dfn}
\begin{proof}
    Let $\io_1 : \MM_0 \to \MM_1$ and 
    $\io_2 : \MM_1 \to \MM_2$ be $\Si$-morphisms.
    We show that the composition is a $\Si$-morphism:
    \begin{itemize}
        \item If $c \in \const{\Si}$ then 
            \[
                \io_1 \circ \io_0 (\modintp{\MM_0}{c}) = 
                \io_1 (\modintp{\MM_1}{c}) = \modintp{\MM_2}{c}
            \]
        \item If $f \in \func{\Si}$ and 
            $a \in {\MM_0}^{n_f}$ then
            \[
                \io_1 \circ \io_0 \circ \modintp{\MM_0}{f}(a) = 
                \io_1 \circ \modintp{\MM_1}{f} \circ \io_0 (a) = 
                \modintp{\MM_2}{f} \circ \io_1 \circ \io_0 (a)
            \]
        \item If $r \in \rel{\Si}$ and 
            $a \in {\MM_0}^{m_f}$ then
            \[
                a \in \modintp{\MM_0}{r} \implies \io_1 (a) \in \modintp{\MM_1}{r} 
                \implies \io_2 \circ \io_1 (a) \in \modintp{\MM_2}{r}
            \]
    \end{itemize}
    To show that embeddings compose to be embeddings we note that 
    the composition of injective functions is injective 
    and if $r \in \rel{\Si}$ and 
    $a \in {\MM_0}^{m_f}$ then
    \[
        a \in \modintp{\MM_0}{r} \iff \io_1 (a) \in \modintp{\MM_1}{r} 
        \iff \io_2 \circ \io_1 (a) \in \modintp{\MM_2}{r}
    \]
    To show that composition of elementary embeddings are elementary,
    let $\phi \in \form{\Si}$ and $a$ in ${\MM_0}$ be chosen suitably.
    Then
    \[
        \MM_0 \modelsi \phi(a) \iff \MM_1 \modelsi \phi(\io_1 (a))
        \iff \MM_2 \modelsi \phi(\io_2 \circ \io_1 (a))
    \]
\end{proof}

\begin{eg}
    Given the category of structures from the 
    \linkto{dfn_rings}{signature of rings}, 
    we can take the subcategory whose objects are models 
    of the theory of rings (namely rings), 
    hence producing the category of rings.
    Similarly taking the subcategory whose objects are models of the 
    theory of fields (namely fields) produces the category of fields.
\end{eg}

\begin{prop}[Embeddings Preserve Satisfaction of Quantifier Free Formulas]
    \link{emb_preserve_sat_of_quan_free}
    Given $\io : \MM \to \NN$ a $\Si$-embedding and
    $\phi$ a $\Si$-formula with variables indexed by $S$,
    and $a \in \MM^S$,
    \begin{enumerate}
        \item If $\phi$ is $\top$ then it is satisfied by both.
        \item If $\phi$ is $t = s$ then 
            $\MM \model{\Si} \phi(a)  \iff \NN \model{\Si} \phi(\io(a))$.
        \item If $\phi$ is $r(s)$ then 
            $\MM \model{\Si} \phi(a)  \iff \NN \model{\Si} \phi(\io(a))$.
        \item If $\phi$ is $\NOT \chi$ and 
            $\MM \model{\Si} \chi(a)  \iff \NN \model{\Si} \chi(\io(a))$
            then \[\MM \model{\Si} \phi(a)  \iff \NN \model{\Si} \phi(\io(a))\]
        \item If $\phi$ is $\chi_0 \OR \chi_1$ and 
            $\MM \model{\Si} \chi_i (a)  \iff \NN \model{\Si} \chi_i (\io(a))$
            then \[\MM \model{\Si} \phi(a)  \iff \NN \model{\Si} \phi(\io(a))\]
    \end{enumerate}
    Thus from the above we can immediately conclude by induction
    that if $\phi$ is a quantifier free $\Si$-formula, 
    \[\MM \model{\Si} \phi(a)  \iff \NN \model{\Si} \phi(\io(a))\]
    Note that our original result is stronger than this since we didn't
    assume the formula to be quantifier free.
\end{prop}
\begin{proof}~
    \begin{enumerate}
        \item Trivial.
        \item If $\phi$ is $t = s$ then
            \begin{align*}
                \MM \model{\Si} \phi(a) 
                & \iff \mmintp{t}(a) = \mmintp{s}(a)\\
                & \iff \io(\mmintp{t}(a)) = \io(\mmintp{s}(a)) 
                & \text{by injectivity} \\
                & \iff \nnintp{t}(\io(a)) = \nnintp{s}(\io(a)) 
                & \text{\linkto{morph_comm_term_intp}{
                    morphisms commute with}} \\
                & \iff \NN \model{\Si} \phi(\io(a))
                & \text{\linkto{morph_comm_term_intp}{interpretation of terms}}
            \end{align*}
        \item If $\phi$ is $r(s)$ then
            \begin{align*}
                \MM \model{\Si} \phi(a) 
                & \iff a \in \mmintp{r}& \\
                & \iff \io(a) \in \nnintp{r}
                & \text{embeddings property}\\
                & \iff \NN \model{\Si} \phi(\io(a))
            \end{align*}
        \item If $\phi$ is $\NOT \chi$ and 
            $\MM \model{\Si} \chi(a)  \iff \NN \model{\Si} \chi(\io(a))$ then
            \[
                \MM \model{\Si} \phi(a) 
                \iff \MM \nodel{\Si} \chi(a) 
                \iff \NN \nodel{\Si} \chi(\io(a))
                \iff \NN \model{\Si} \phi(\io(a))
            \]
        \item If $\phi$ is $\chi_0 \OR \chi_1$ and
            $\MM \model{\Si} \chi_i (a)  \iff \NN \model{\Si} \chi_i (\io(a))$
            \[
                \MM \model{\Si} \phi(a)
                \iff \MM \model{\Si} \chi_0(a) 
                \text{ or } \MM \model{\Si} \chi_1(a)
                \iff \NN \model{\Si} \chi_0(\io(a)) 
                \text{ or } \NN \model{\Si} \chi_1(\io(a))
                \iff \NN \model{\Si} \phi(\io(a))
            \]
    \end{enumerate}
\end{proof}

\begin{dfn}[Universal Formula, Universal Sentence]
    \link{universal_formula_def}
    A $\Si$-formula is universal if it can be 
    built inductively by the following two constructors:
    \begin{itemize}
        \item[$\vert$] If $\phi$ is a quantifier free $\Si$-formula 
        then it is a universal $\Si$-formula.
        \item[$\vert$] If $\phi$ is a universal $\Si$-formula then
        $\forall v, \phi(v)$ is a universal $\Si$-formula.
    \end{itemize}
    In other words universal $\Si$-formulas are formulas that start with 
    a bunch of `for alls' followed by a quantifier free formula.
\end{dfn}

\begin{prop}[Embeddings preserve satisfaction of universal formulas downwards]
    \link{emb_preserve_sat_of_forall_down}
    Given $\io : \MM \to \NN$ a $\Si$-embedding and
    $\phi$ a universal $\Si$-formula with 
    variables indexed by $S$ ($\chi$ is quantifier free).
    For any $a \in \MM^S$
    \[
        \NN \modelsi \phi(\io(a)) \quad \implies \quad
        \MM \model{\Si} \phi(a)
    \]
    By taking the contrapositive we can show that embeddings
    preserve satisfaction of `existential' $\Si$-formulas upwards.
\end{prop}
\begin{proof}
    We induct on $\phi$:
    \begin{itemize}
        \item If $\phi$ is a quantifier free then since 
        \linkto{emb_preserve_sat_of_quan_free}{embeddings preserve 
        satisfaction of quantifier free formulas}, 
        $\NN \modelsi \phi(\io(a)) \implies
        \MM \model{\Si} \phi(a)$.
        \item If $\phi$ is $\forall v_i, \psi$ with $S$ indexing the variables
        of $\psi$. 
        Let $T := S \setminus \set{i}$
        Assuming the inductive hypothesis: 
        for any $a \in {\MM}^T$ and 
        $b \in {\MM}$,
        \[\NN \modelsi \psi(\io(a),\io(b)) \quad \implies \quad
        \MM \model{\Si} \psi(a,b)\]
        Then for any $a \in {\MM}^T$
        \begin{align*}
            &\quad \NN \modelsi \phi(\io(a))\\
            &\implies \forall b \in {\MM}, 
            \NN \modelsi \psi(\io(a),\io(b)) \\
            &\implies \forall b \in {\MM}, 
            \MM \modelsi \psi(\io(a),\io(b)) & \text{by the induction}\\
            &\implies \MM \model{\Si} \phi(a)
        \end{align*}
    \end{itemize}
\end{proof}

\begin{prop}[Isomorphisms are Elementary]
    \link{iso_imp_elem_equiv}
    If two $\Si$-structures $\MM$ and $\NN$ 
    are $\Si$-isomorphic then the isomorphism is elementary.
\end{prop}
\begin{proof}
    Let $\io : \MM \to \NN$ be a $\Si$-isomorphism.
    We case on what $\phi$ is:
    \begin{itemize}
        \item If $\phi$ is quantifier free, 
            then each case is follows from applying
            \linkto{emb_preserve_sat_of_quan_free}{
                embeddings preserve satisfaction of quantifier free formulas.}
        \item If $\phi$ is $\forall v, \chi(v)$ then
            \begin{forward}
                Let $b \in \const{\NN}$ then
                $\io^{-1}(b) \in \const{\MM}$ is well defined by surjectivity.
                Hence $\MM \modelsi \chi(\io^{-1}(b),a)$ and so
                $\NN \modelsi \chi(b,\iota(a))$ by the induction hypothesis.
                Hence $\NN \model{\Si} \phi(\io(a))$.
            \end{forward}
            \begin{backward}
                The same.
            \end{backward}
    \end{itemize}
\end{proof}

\subsection{Vaught's Completeness Test}
Read ahead to the statement of \linkto{vaught_test}{Vaught's Completeness Test}.

\begin{dfn}[Finitely modelled, infinitely modelled]
    A $\Si$-theory $T$ 
    is finitely modelled when
    there exists a $\Si$-model of $T$ 
    with finite carrier set.
    It is infinitely modelled when
    there exists a $\Si$-model of $T$ 
    with infinite carrier set.
\end{dfn}
Finitely modelled is \emph{not} the same as finitely consistent.

\begin{prop}[Infinitely modelled theories have arbitrary large models]
    \link{inf_mod_theory_has_inf_mod}
    Given $\Si$ a signature, 
    $T$ a $\Si$-theory that is infinitely modelled,
    and a cardinal $\ka$ such that $|\const{\Si}| + \aleph_0 \leq \ka$, 
    there exists $\MM$ a $\Si$-model of $T$ such that 
    $\ka = |{\MM}|$.
\end{prop}
\begin{proof}
    Enrich only the signature's constant symbols to create $\Si(*)$ 
    a signature such that 
    $\const{\Si(*)} = \const{\Si} \cup \set{c_\al \st \al \in \ka}$.
    Let $T(*) = T \cup \set{c_\al \neq c_\be \st \al,\be \in \ka \AND \al \ne \be}$
    be a $\Si(*)$-theory.
    
    Using \linkto{compactness}{the compactness theorem}, 
    it suffices to show that $T(*)$ is finitely consistent.
    Take a finite subset of $T(*)$. 
    This is the union of a finite subset $\De_T \subs T$, 
    and a finite subset of 
    $\De_{\ka} \subs 
    \set{c_\al \neq c_\be \st \al,\be \in \ka \AND \al \ne \be}$.
    Let $\MM$ be the $\Si$-model of $T$ with infinite cardinality.
    We want to make 
    $\MM$ a ${\Si(*)}$-model of $\De_T \cup \De_\ka$ 
    by interpreting the new symbols of $\set{c_\al \st \al \in \ka}$
    in a sensible way.
    
    Since $\De_\ka$ is finite, 
    we can find a finite subset $I \subset \ka$ 
    that indexes the constant symbols appearing in $\De_\ka$. 
    Since $\MM$ is infinite and $I$ is finite,
    we can find distinct elements of $\MM$
    to interpret the elements of
    $\set{c_\al \st \al \in I}$. 
    Interpret the rest of the new constant symbols however,
    for example let them all be sent to the same element,
    then $\MM \model{\Si*} \De_T \cup \De_\ka$.
    Thus $T(*)$ is finitely consistent hence consistent.
    
    Using 
    \linkto{compactness}{the third equivalence of $T(*)$ being consistent},
    there exists
    $\MM$ a $\Si(*)$-model of $T(*)$ with $|{\MM}|\leq \ka$.
    If $|{\MM}| < \ka$ 
    then there would be $c_\al, c_\be$ that are interpreted as equal,
    hence $\MM \model{\Si(*)} c_\al = c_\be$ and $\MM \nodel{\Si(*)} c_\al = c_\be$, 
    a contradiction.
    Thus $|{\MM}| = \ka$.
    \linkto{move_down_mod}{Move $\MM$ down a signature}
    to make it a $\Si$-model of $T$.
    This doesn't change the cardinality of $\MM$,
    so we have a $\Si$-model of $T$ with cardinality $\ka$.
\end{proof}

\begin{dfn}[Elementary equivalence]
    Let $\MM$, $\NN$ be $\Si$-structures.
    They are elementarily equivalent if for any $\Si$-sentence $\phi$,
    $\MM \model{\Si} \phi$ if and only if $\NN \model{\Si} \phi$.
    We write $\MM \equiv_\Si \NN$.
\end{dfn}

\begin{lem}[Not a consequence is consistent]
    \link{not_consequence}
    Let $T$ be a $\Si$-theory
    and $\phi$ is a $\Si$-sentence
    then $T \nodel{\Si} \phi$
    if and only if $T \cup \set{ \NOT \phi}$ is consistent.
    Furthermore, $T \nodel{\Si} \NOT \phi$
    if and only if $T \cup \set{\phi}$ is consistent.
\end{lem}
\begin{proof}
    For the first statement:
    \begin{forward}
        Unfolding $T \nodel{\Si} \phi$,
        we have that there exists a $\Si$-model $\MM$ of $T$ 
        such that $\MM \nodel{\Si} \phi$.
        Hence $\MM \modelsi \NOT \phi$ and we are done.
    \end{forward}
    The backward proof is straightward.

    For the second statement, 
    apply the first to $\NOT \phi$ and obtain 
    $T \nodel{\Si} \NOT \phi$
    if and only if $T \cup \set{\NOT \NOT \phi}$ is consistent.
    Note that for any $\Si$-structure $\MM$, 
    $\MM \model{\Si} \NOT \NOT \phi$ 
    if and only if $\MM \model{\Si} \phi$.
    This completes the proof.
\end{proof}

\begin{prop}[Vaught's Completeness Test]
    \link{vaught_test}
    Suppose that $\Si$-theory $T$ is consistent, 
    not finitely modelled, and $\ka$-categorical 
    for some cardinal satisfying 
    $|\const{\Si}| + \aleph_0 \leq \ka$.
    Then $T$ is maximal (i.e. complete).
\end{prop}
\begin{proof}
    Suppose not: If $T$ is not maximal then there exists
    $\Si$-formula $\phi$ such that $T \nodel{\Si} \phi$ and 
    $T \nodel{\Si} \NOT \phi$.
    These imply $T \cup \set{\NOT \phi}$
    and $T \cup \set{\phi}$ are both
    \linkto{not_consequence}{consistent}.
    Let $\MM_\NOT$ and $\MM$ be models of  
    $T \cup \set{\NOT \phi}$ and $T \cup \set{\phi}$
    respectively.
    Then each are models of $T$ so they are infinite
    and so $T \cup \set{\NOT \phi}$ and $T \cup \set{\phi}$
    are infinitely modelled.

    Since we have $\ka$ such that $|\const{\Si}| + \aleph_0 \leq \ka$, 
    \linkto{inf_mod_theory_has_inf_mod}{
        there exists $\NN_\NOT, \NN$} respectively $\Si$-models of 
        $T \cup \set{\NOT \phi}$ and $T \cup \set{\phi}$
        such that 
        $\ka = |{\NN_\NOT}| = |{\NN}|$.
    Since $T$ is $\ka$-categorical
    $\NN$ and $\NN_\NOT$ are isomorphic
    \linkto{iso_imp_elem_equiv}{
        by an elementary $\Si$-embedding.}
    As $\phi$ has no free variables this implies that
    $\NN \model{\Si} \phi$ and $\NN \model{\Si} \NOT \phi$, 
    a contradiction.
\end{proof}

\subsection{Elementary embeddings and diagrams of models}
\begin{prop}[Tarski-Vaught Elementary Embedding Test]
    \link{tarski_vaught}
    Let $\io : \MM \to \NN$ be a $\Si$-embedding, 
    then the following are equivalent:
    \begin{enumerate}
        \item $\io$ is elementary 
        \item For any 
            $\phi \in \form{\Si}$ with free variables indexed by $S$,
            any $i \in S$ 
            and any $a \in ({\MM})^{S \setminus \set{i}}$,
            \[\forall b \in {\MM}, \NN \model{\Si} \phi(\io(a),\io(b)) 
            \quad \implies \quad 
            \forall c \in {\NN}, \NN \model{\Si} \phi(\io(a),c),\]
            which we call the Tarski-Vaught condition.
        \item For any 
            $\phi \in \form{\Si}$ with free variables indexed by $S$,
            any $i \in S$ 
            and any $a \in ({\MM})^{S \setminus \set{i}}$,
            \[\exists c \in {\NN}, \NN \model{\Si} \phi(\io(a),c)
            \quad \implies \quad
            \exists b \in {\MM}, \NN \model{\Si} \phi(\io(a),\io(b))\]
            This is essentially the contrapositive of the previous statement,
            and is included because it is more commonly 
            version of the statement.
    \end{enumerate}
\end{prop}
\begin{proof}
    We only show the first two statements are equivalent and
    leave the third as an exercise.
    \begin{forward}
        First show that $\MM \model{\Si} \forall v, \phi(a,v)$.
        Let $b \in {\MM}$, 
        then by assumption $\NN \model{\Si} \phi(\io(a),\io(b))$,
        which is implies $\MM \model{\Si} \phi(a,b)$
        as $\io$ is an elementary embedding.
        Thus we indeed have $\MM \model{\Si} \forall v, \phi(a,v)$
        which in turn implies $\NN \model{\Si} \forall v, \phi(\io(a),v)$
        and we are done.
    \end{forward}

    \begin{backward}
        We case on what $\phi$ is, 
        though most of the work was already done before.
        \begin{itemize}
            \item If $\phi$ is quantifier free, 
            then each case follows from applying
            \linkto{emb_preserve_sat_of_quan_free}{
                embeddings preserve satisfaction of quantifier free formulas.}

            \item The backwards implication follows from applying 
                \linkto{emb_preserve_sat_of_forall_down}{
                embeddings preserve satisfaction of universal formulas 
                downwards.}

                For the forwards implication 
                we use the Tarski-Vaught condition
                (so far $\io$ just needed to be a $\Si$-embedding)
\begin{align*}
    \MM \model{\Si} \forall v, \psi (a,v) &
        \implies \forall b \in {\MM}, \MM \model{\Si} \psi(a,b)\\
        &\implies \forall b \in {\MM}, 
            \NN \model{\Si} \psi(\io(a),\io(b))
            & \text{ by the induction hypothesis}\\
        &\implies \forall c \in {\NN}, \NN \model{\Si} \psi(\io(a),c)
            & \text{ by the Tarski-Vaught condition}\\
        &\implies \NN \model{\Si} \phi
\end{align*}
        \end{itemize}
    \end{backward}
\end{proof}

Again we have a technical detail which is not really worth spending too much
time one. 
It is a sensible justification for arguments based on creating new signatures.
\begin{lem}[Moving Morphisms Down Signatures]
    \link{move_down_morph}
    Suppose $\Si \leq \Si(*)$.
    If $\io : \MM \to \NN$ is a $\Si(*)$-morphism then 
    \begin{enumerate}
        \item $\io$ can be made into a $\Si$-morphism.
        \item If $\io$ is an embedding then it remains an embedding.
        \item If $\io$ is an elementary embedding then it remains elementary. 
    \end{enumerate}
\end{lem}
\begin{proof}
    \begin{enumerate}
        \item \linkto{move_down_mod}{
                Move $\MM$ and $\NN$ down to being $\Si$ structures
                (by picking $T(*) = T = \nothing$).}
            We show that the same set morphism 
            $\io : {\MM} \to {\NN}$
            is a $\Si$-morphism.
            \begin{itemize}
                \item If $c \in \const{\Si}$ then since moving structures down
                    signatures preserves interpretation on the lower signature,
                    and since $\io$ is a $\Si(*)$ embedding,
                    \[
                        \io(\subintp{\Si}{\MM}{c}) 
                        = \io(\subintp{\Si(*)}{\MM}{c})
                        = \subintp{\Si(*)}{\NN}{c}
                        = \subintp{\Si}{\NN}{c}
                    \]
                \item If $f \in \func{\Si}$ and $a \in ({\MM})^{n_f}$ 
                    then similarly
                    \[
                        \io \circ \subintp{\Si}{\MM}{f}(a)
                        = \io \circ \subintp{\Si(*)}{\MM}{f}(a)
                        = \subintp{\Si(*)}{\NN}{f}(\io(a))
                        = \subintp{\Si}{\NN}{f}(\io(a))
                    \]
                \item If $r \in \rel{\Si}$ and $a \in ({\MM})^{m_r}$
                    then 
                    \[
                        a \in \subintp{\Si}{\MM}{r}(a)
                        = \subintp{\Si(*)}{\MM}{r}
                        \implies \io(a)  \in \subintp{\Si(*)}{\NN}{r}
                        = \subintp{\Si}{\NN}{r}
                    \]
            \end{itemize}
        \item If we also have that it is an embedding in $\Si(*)$, 
            then injectivity is preserved as it is a property of set morphisms. 
            Given $r \in \rel{\Si}$ and $a \in ({\MM})^{m_r}$,
            \[
                \io(a)  \in \subintp{\Si}{\NN}{r}
                = \subintp{\Si(*)}{\NN}{r}
                \implies 
                a \in \subintp{\Si(*)}{\MM}{r}(a)
                = \subintp{\Si}{\MM}{r}
            \]
        \item If we also have that $\io$ is elementary in $\Si(*)$ 
            then we use the \linkto{tarski_vaught}{Tarski-Vaught Test}:
            let $\phi \in \form{\Si}$ have free variables indexed by $S$,
            let $i \in S$ 
            and let $a \in ({\MM})^{S \setminus \set{i}}$.
            Then due to the construction in
            \linkto{move_down_mod}{moving $\MM$ and $\NN$ down a signature} 
            we have that for any $b \in {\NN}$, 
            \[\NN \model{\Si} \phi (\io(a),\io(b)) 
            \iff \NN \model{\Si(*)} \phi (\io(a),\io(b))\]
            and similarly for $\MM$.
            Hence 
            \begin{align*}
                &\forall b \in {\NN}, 
                \NN \model{\Si} \phi(\io(a),\io(b)) \\
                \implies &\forall b \in {\NN}, 
                \NN \model{\Si(*)} \phi(\io(a),\io(b)) \\
                \implies &\forall c \in {\MM}, 
                \MM \model{\Si(*)} \phi(a,c) 
                & \text{$\io$ is elementary in $\Si(*)$}\\
                \implies &\forall c \in {\MM}, \MM \model{\Si} \phi(a,c)
            \end{align*}
            Hence $\io$ is elementary in $\Si$.
    \end{enumerate}
\end{proof}

\begin{nttn}
    Let $A$ be a set and $\Si$ be a signature,
    enriching only the constant symbols of $\Si$ we can create a signature 
    $\Si(A)$ such that 
    \[
        \const{\Si(A)} := 
        \const{\Si} \cup \set{c_a \st a \in A}
    \]
\end{nttn}

\begin{dfn}[Diagram and the Elementary Diagram of a Structure]
    Let $\MM$ be a $\Si$-structure,
    we move $\MM$ up to the signature $\Si(\MM)$
    by interpreting each new constant symbol $c_a$ as $a$.
    ($\MM$ satisfies the conditions of our lemma for
    \linkto{move_up_sig}{moving models up signatures} 
    by choosing $T = \nothing$).
    Thus we may treat $\MM$ as a $\Si(\MM)$ structure.
    We define the atomic diagram of $\MM$ over $\Si$:
    \begin{itemize}
        \item[$\vert$] If $\phi$ is an atomic
        $\Si(\MM)$-sentence such that $\MM \model{\Si(\MM)} \phi$,
        then $\phi \in \atdiag{\Si}{\MM}$.
        \item[$\vert$] If $\phi \in \atdiag{\Si}{\MM}$ then 
        $\NOT \phi \in \atdiag{\Si}{\MM}$.
    \end{itemize}
    We define the elementary diagram of $\MM$ over $\Si$ as 
    \[
        \eldiag{\Si}{\MM} := 
        \set{\phi \in \Si(\MM)\text{-sentences } \st \MM \model{\Si(\MM)} \phi}
    \]

    The elementary diagram of $\MM$ is a maximal $\Si(\MM)$-theory 
    with $\MM$ as a model of it.
    It is not the same as the set of all $\Si$-sentences satisfied by $\MM$,
    known as the theory of $\MM$ in $\Si$.
\end{dfn}

\begin{prop}[Models of the elementary diagram
        are elementary extensions]
    \link{elem_ext_equiv_eldiag_model}
    Given $\MM$ a $\Si$-structure and 
    $\NN$ a $\Si(\MM)$-structure such that
    $\NN \model{\Si(\MM)} \atdiag{\Si}{\MM}$,
    we can make $\NN$ into a $\Si$-structure and find
    a $\Si$-embedding from $\MM$ to $\NN$.
    Furthermore if
    $\NN \model{\Si(\MM)} \eldiag{\Si}{\MM}$ then 
    the embedding is elementary.

    Conversely, given an elementary $\Si$-embedding from 
    $\MM$ into a $\Si$-structure $\NN$, 
    we can move $\NN$ up to being a $\Si(\MM)$ structure such that 
    $\NN \model{\Si(\MM)} \eldiag{\Si}{\MM}$.
\end{prop}
\begin{proof}
    \begin{forward}
        Suppose $\NN \model{\Si(\MM)} \atdiag{\Si}{\MM}$.
        Firstly we work in $\Si(\MM)$ to define the embedding:
        \linkto{move_up_sig}{move $\MM$ up a signature}
        by taking the same interpretation as used in the 
        definition of $\Si(\MM)$: 
        \[\intp{\Si(\MM)}{\MM} : c_a \mapsto a\]
        and preserving the same interpretation for symbols of $\Si$.
        This makes $\intp{\const{\Si(\MM)}}{\MM}$ surjective.
        Thus we write elements of ${\MM}$ as 
        $\subintp{\Si(\MM)}{\MM}{c}$, 
        for some $c \in \const{\Si(\MM)}$.
        
        Next we define the $\Si(\MM)$-morphism 
        $\io : \MM \to \NN$ such that 
        $\io : \subintp{\Si(\MM)}{\MM}{c} \to \subintp{\Si(\MM)}{\NN}{c}$.
        To check that $\io$ is well defined, 
        take $c,d \in \const{\Si(\MM)}$ such that 
        $\subintp{\Si(\MM)}{\MM}{c} = \subintp{\Si(\MM)}{\MM}{d}$.
        \begin{align*}
            &\implies \MM \model{\Si(\MM)} c = d \\
            &\implies c = d \in \atdiag{\Si}{\MM} \\
            &\implies \NN \model{\Si(\MM)} c = d \\
            &\implies \subintp{\Si(\MM)}{\NN}{c} = \subintp{\Si(\MM)}{\NN}{d}
        \end{align*}
        Thus $\io$ is well defined.
        In fact doing `not' gives us injectivity in the same way:
        Take $c,d \in \const{\Si(\MM)}$ such that 
        $\subintp{\Si(\MM)}{\MM}{c} \ne \subintp{\Si(\MM)}{\MM}{d}$.
        \begin{align*}
            &\implies \MM \model{\Si(\MM)} c \ne d \\
            &\implies c \ne d \in \atdiag{\Si}{\MM} \\
            &\implies \NN \model{\Si(\MM)} c \ne d \\
            &\implies \subintp{\Si(\MM)}{\NN}{c} \ne \subintp{\Si(\MM)}{\NN}{d}
        \end{align*}
        Thus $\io$ is injective.
        To check that $\io$ is a $\Si(\MM)$-morphism, 
        we check interpretation of functions and relations.
        Let $f \in \func{\Si(\MM)} = \func{\Si}$ and 
        $c \in (\const{\Si(\MM)})^{n_f}$.
        $\intp{\const{\Si(\MM)}}{\MM}$ is surjective thus we can find 
        $d \in \const{\Si\brkt{\MM}}$ such that $\MM \model{\Si(\MM)} f(c) = d$.
        Hence $f(c) = d \in \atdiag{\Si}{\MM}$.
        Hence $\NN \model{\Si(\MM)} f(c) = d$.
        \begin{align*}
            \io \circ \subintp{\Si(\MM)}{\MM}{f}(\subintp{\Si(\MM)}{\MM}{c}) 
            &= \io(\subintp{\Si(\MM)}{\MM}{d}) \\
            &= \subintp{\Si(\MM)}{\NN}{d}\\
            &= \subintp{\Si(\MM)}{\NN}{f}(\subintp{\Si(\MM)}{\NN}{c})\\
            &= \subintp{\Si(\MM)}{\NN}{f} \circ \io(\subintp{\Si(\MM)}{\MM}{c})
        \end{align*}
        Let $r \in \rel{\Si(\MM)} = \rel{\Si}$ and 
        $c \in (\const{\Si(\MM)})^{m_r}$.
        \begin{align*}
            \subintp{\Si(\MM)}{\MM}{c}\in \subintp{\Si(\MM)}{\MM}{r} 
            &\implies \MM \model{\Si(\MM)} r(c)\\
            &\implies r(c) \in \atdiag{\Si}{\MM}\\
            &\implies \NN \model{\Si(\MM)} r(c) \\
            &\implies \io(\subintp{\Si(\MM)}{\MM}{c}) = 
            \subintp{\Si(\MM)}{\NN}{c} \in \subintp{\Si(\NN)}{\NN}{r} 
        \end{align*}
        To show that $\io$ is an embedding it remains to show 
        the backward implication for relations.
        Let $r \in \rel{\Si(\MM)} = \rel{\Si}$ and 
        $c \in (\const{\Si(\MM)})^{m_r}$.
        \begin{align*}
            \subintp{\Si(\MM)}{\MM}{c} \notin \subintp{\Si(\MM)}{\MM}{r} 
            &\implies \MM \nodel{\Si(\MM)} r(c)\\
            &\implies \NOT r(c) \in \atdiag{\Si}{\MM}\\
            &\implies \NN \nodel{\Si(\MM)} r(c) \\
            &\implies \io(\subintp{\Si(\MM)}{\MM}{c}) = 
            \subintp{\Si(\MM)}{\NN}{c} \notin \subintp{\Si(\NN)}{\NN}{r} 
        \end{align*}
        Assume furthermore that $\NN \model{\Si(\MM)} \eldiag{\Si}{\MM}$.
        We show that the embedding is elementary.
        Let $\phi$ be a $\Si(\MM)$-formula
        with variables indexed by $S$
        and $a \in ({\MM})^S$.
        Let $c \in (\const{\Si})^S$ be such that 
        $\subintp{\Si(\MM)}{\MM}{c} = a$.
        \begin{align*}
            \MM \model{\Si(\MM)} \phi(a) 
                &\implies \phi(c) \in \eldiag{\Si}{\MM}\\
                &\implies \NN \model{\Si(\MM)} \phi(c)\\
                &\implies \NN \model{\Si(\MM)} \phi(\io(a))
        \end{align*}
        Similarly,
        \begin{align*}
            \MM \nodel{\Si(\MM)} \phi(a) &\implies \NOT \phi(c) 
                \in \eldiag{\Si}{\MM}\\
            &\implies \NN \model{\Si(\MM)} \NOT \phi(c)\\
            &\implies \NN \nodel{\Si(\MM)} \phi(\io(a))
        \end{align*}
        Hence $\io$ is an elementary embedding.
        \linkto{move_down_morph}{Moving $\io : \MM \to \NN$ 
        down} to being a $\Si$-morphism of $\Si$-structures
        completes the proof.
    \end{forward}

    \begin{backward}
        Sketch: Suppose $\io : \MM \to \NN$ is an elementary embedding.
        Make $\MM$ and $\NN$ into $\Si(\MM)$-structures by 
        $\intp{\Si(\MM)}{\MM}: c_a \to a$ and 
        $\intp{\Si(\MM)}{\NN}: c_a \to \io(a)$,
        where $a \in {\MM}$.
        Show that $\io$ is still an elementary embedding
        when moved up to $\Si(\MM)$.
        Then for any $\phi \in \eldiag{\Si}{\MM}$,
        $\MM \model{\Si(\MM)} \phi$ and so by the embedding being elementary
        $\NN \model{\Si(\MM)} \phi$.
        Hence $\NN \model{\Si(\MM)} \eldiag{\Si}{\MM}$.
    \end{backward}
\end{proof}

\subsection{Universal axiomatization}
\begin{dfn}[Axiomatization, universal theory, universal axiomatization]
    A $\Si$-theory $A$ is an axiomatization of a 
    $\Si$-theory $T$ if for all $\Si$-structures $\MM$,
    \[\MM \model{\Si} T \iff \MM \model{\Si} A\]

    If $A$ is a set of universal $\Si$-sentences 
    is called a universal $\Si$-theory.
    We are interested in universal axiomatizations of theories.
\end{dfn}

\begin{lem}[Lemma on constants]
    \link{lemma_on_const}
    Suppose $\const{\Si} \subs \const{\Si(*)}$, 
    $T \in \theory{\Si}$, $\phi \in \form{\Si}$
    with variables indexed by $n \in \N$.
    Suppose there exists a list of constant symbols not from $\Si$, 
    i.e. $c \in (\const{\Si(*)}\setminus \const{\Si})^n$ 
    such that $T \model{\Si(*)} \phi(c)$.
    Then \[
        T \model{\Si} \forall v, \phi(v)
    \]
\end{lem}
\begin{proof}
    If $n = 0$ then the result is clear as there are no quantifiers.
    Suppose $n \ne 0$.
    We prove the contrapositive.
    Suppose $T \nodel{\Si} \forall v, \phi(v)$
    then there exists $\MM$ a $\Si$-model of $T$
    such that $\MM \nodel{\Si} \forall v, \phi(v)$.
    Thus there exists $a \in \MM^n$ 
    such that $\MM \nodel{\Si} \phi(a)$.

    We \linkto{move_up_mod}{move $\MM$ up a signature} 
    by extending the interpretation 
    to the new constant symbols:
    if $d \in \const{\Si(*)} \setminus \const{\Si}$ then
    \[
        \subintp{\Si(\MM)}{\MM}{d} := 
        \begin{cases}
            a_i &, \text{ if for some } 0 \leq i < n \text{ we have }d = c_i,\\
            a_0 &, \text{ otherwise} 
        \end{cases}
    \]
    Then $\MM$ is a $\Si(*)$-model of $T$ such that 
    $\MM \nodel{\Si(*)} \phi(a)$, 
    which by construction is equivalent to 
    $\MM \nodel{\Si(*)} \phi(c)$.
\end{proof}

\begin{nttn}{Universal consequences of $T$}
    Let $T$ be a $\Si$-theory, then 
    \[
        T_\forall := 
        \set{\phi \text{ universal $\Si$-sentences} \st T \modelsi \phi}
    \]
    is called the set of universal consequences of $T$.
\end{nttn}

\begin{prop}[Universal axiomatizations make substructures models]
    \link{universal_axiomatizations_make_subs_mods}
    $T$ a $\Si$-theory has a universal axiomatization if and only if
    for any $\Si$-model $\NN$ of $T$ and any $\Si$-embedding 
    from some $\Si$-structure $\MM \to \NN$,
    $\MM$ is a $\Si$-model of $T$.
\end{prop}
\begin{proof}
    \begin{forward}
        Suppose $A$ is a universal axiomatization of $T$,
        $\NN$ is a $\Si$-model of $T$ and $\MM \to \NN$ is a 
        $\Si$-embedding.
        Let $\phi \in T$. Then
        $\NN \model{\Si} T$ implies $\NN \model{\Si} A$
        by definition of $A$.
        $\NN \model{\Si} A$ implies $\MM \model{\Si} A$
        since \linkto{emb_preserve_sat_of_forall_down}{
            embeddings preserve the satisfaction of quantifier free formulas
            downwards}.
        Finally $\MM \model{\Si} A$ implies $\MM \model{\Si} T$
        by definition of $A$.
    \end{forward}

    \begin{backward}
        We show that $T_\forall$ is a universal axiomatization of $T$.
        Let $\MM \model{\Si} T$ and let $\phi \in T_\forall$.
        Then by definition of $T_\forall$, $T \model{\Si} \phi$.
        Hence $\MM \model{\Si} \phi$ and any $\Si$-model of $T$ is
        a $\Si$-model of $T_\forall$.

        Suppose $\MM \modelsi T_\forall$.
        We first show that $T \cup \atdiag{\Si}{\MM}$ is consistent.
        By the \linkto{compactness}{compactness theorem} 
        it suffices to show that for any subset 
        $\De$ of $\atdiag{\Si}{\MM}$,
        $T \cup \De$ is consistent.
        Write $\De  = \set{\psi_1, \dots, \psi_n}$.
        Let $\psi = \bigwedge_{1 \leq i \leq n} \psi_i$.
        We can find $S$ that indexes the constant symbols in 
        $\const{\Si(\MM)} \setminus \const{\Si}$ that appear in $\psi$ 
        (in the same way as we made indexing sets of the variables).
        Then we can create $\phi \in \form{\Si}$ 
        with variables indexed by $S$ such that 
        $\phi(c) = \psi$, 
        where $c$ is a list of constant symbols in 
        $\const{\Si(\MM)} \setminus \const{\Si}$ indexed by $S$.
        Since $\De \subs \atdiag{\Si}{\MM}$ we have
        $\forall i, \MM \model{\Si} \psi_i$.
        Hence $\MM \modelsi \phi(c)$.
        Then $\MM \modelsi \exists v,  \phi(v)$ and so
        $\MM \nodel{\Si} \forall v,  \NOT \phi(v)$.

        Since each $\psi_i$ is from the the atomic diagram of $\MM$ 
        they are all quantifier free.
        Thus $\phi$ is a quantifier free $\Si$-formula and 
        $\forall v,  \NOT \phi(v)$ is universal.
        Hence $T \nodel{\Si} \forall v, \NOT \phi(v)$ 
        by the definition of $T_\forall$.
        By \linkto{lemma_on_const}{the lemma on constants}
        this implies that $T \nodel{\Si(\MM)} \NOT \phi(c)$.
        Hence there exists a $\Si(\MM)$-model of $T \cup \phi(c)$.
        Then it follows that this is also a $\Si(\MM)$-model of $T \cup \De$.
        Thus $T \cup \De$ is consistent so
        $T \cup \atdiag{\Si}{\MM}$ is consistent.

        Thus there exists $\NN$ a $\Si$-model of 
        $T \cup \atdiag{\Si}{\MM}$.
        This is a model of $\atdiag{\Si}{\MM}$ thus by 
        \linkto{elem_ext_equiv_eldiag_model}{there 
        is a $\Si(\MM)$-embedding $\MM \to \NN$}.
        We \linkto{move_down_morph}{make this a $\Si$-embedding}, 
        hence using the theorem's hypothesis $\MM$ is a $\Si$-model of $T$.
    \end{backward}
\end{proof}

The following result has doesn't come up at all until much later,
but is included here as another demonstration of the lemma on constants in use.
It appears as an exercise in the second chapter of Marker's book \cite{marker}.

\begin{cor}[Amalgamation]
    \link{amalgamation}
    Let $\AA$, $\MM$ and $\NN$ be $\Si$-structures,
    and suppose we have \linkto{partial_morph_dfn}{partial} 
    elementary $\Si$-embeddings
    $\io_\MM : A \to \MM$ and 
    $\io_\NN : A \to \NN$, for $\nothing \ne A \subs \AA$.
    Then there exists a common elementary extension $\PP$ of $\MM$ and $\NN$
    such that the following commutes:
    \begin{cd}
        \MM \ar[r] &\PP \\
        A \ar[u, "\io_\MM"] \ar[r, "\io_\NN"] &\NN \ar[u]
    \end{cd}
    $\PP$ is the `amalgamation' of $\MM$ and $\NN$.
\end{cor}
\begin{proof}
    We show first that the theory $\eldiag{\Si}{\MM} \cup \eldiag{\Si}{\NN}$
    is consistent as a $\Si(\MM,\NN)$-theory, where $\const{\Si(\MM,\NN)}$ is 
    defined to be 
    \[
        \set{c_a \st a \in A} \cup 
        \set{c_a \st a \in \MM \setminus \io_\MM(A)}
        \cup \set{c_a \st a \in \NN \setminus \io_\NN(A)}
    \]
    where terms and formulas from $\Si(A), \Si(\MM), \Si(\NN)$ 
    are interpreted in the natural way: 
    the constants $c_{\io_\MM(a)} \mapsto c_a$ (similarly for $\NN$).
    For the rest of the proof we identify 
    $\const{\Si(\MM)}$ with
    $\const{\Si(A)} \cup \set{c_a \st a \in \MM \setminus \io_\MM(A)}$
    (similarly with $\NN$). 

    By the \linkto{compactness}{compactness theorem} it suffices to show that 
    for any finite subset $\De \subs \eldiag{\Si}{\NN}$,
    $\eldiag{\Si}{\MM} \cup \De$ is consistent.
    Let $\phi$ be the $\Si(\MM)$-formula and $a \in \NN^\star$ be such 
    that \footnote{Take out all the finitely many constants appearing from 
        $\NN \setminus \io_{\NN}(A)$ in 
        $\De$ and make them into a tuple $a$, 
        replacing them with free variables.
        What remains is a finite set of $\Si(A)$-formulas, 
        which are naturally also $\Si(\MM)$-formulas. 
        We take the `and' of all of them to be $\phi$.}
    \[\phi(a) = \bigand{\psi \in \De}{} \psi\]
    $\phi(a)$ is naturally a $\Si(\MM,\NN)$-sentence such that 
    $\NN \model{\Si(\MM,\NN)} \phi(a)$.
    
    Suppose for a contradiction $\eldiag{\Si}{\MM} \cup \De$ is inconsistent.
    Then any $\Si(\MM,\NN)$-model of 
    $\eldiag{\Si}{\MM}$ is not a model of $\De$,
    which implies it does not satisfy $\phi(a)$.
    Hence
    \[\eldiag{\Si}{\MM} \model{\Si(\MM,\NN)} \NOT \phi(a)\]
    By the \linkto{lemma_on_const}{lemma on constants} applied to 
    $\Si(\MM) \leq \Si(\MM,\NN)$, $\eldiag{\Si}{\MM}$ and 
    $a \in \const{\Si(\MM,\NN)} \setminus \const{\Si(\MM)}$ we have
    \[\eldiag{\Si}{\MM} \model{\Si(\MM)} \forall v, \NOT \phi(v)\]
    Noting that $\MM$ is a $\Si$-model of its elementary diagram, 
    and \linkto{move_down_mod}{moving $\MM$} down a signature we have that 
    \[
        \MM \model{\Si(\MM)} \forall v, \NOT \phi(v) \implies 
        \MM \model{\Si(A)} \forall v, \NOT \phi(v) \implies 
    \]
    Since $A \to \MM$ and $A \to \NN$ are partial elementary $\Si$-embeddings
    (and thus naturally $\Si(A)$-embeddings) 
    we have that 
    $\AA \model{\Si(A)} \forall v, \NOT \phi(v)$ and so 
    $\NN \model{\Si(A)} \forall v, \NOT \phi(v)$.
    \linkto{move_up_mod}{Move this up} 
    to $\Si(\MM,\NN)$ we have a contradiction, by choosing 
    $v$ to be $a$:
    $\NN \model{\Si(\MM,\NN)} \NOT \phi(a)$, 
    but we remarked before that $\NN \model{\Si(\MM,\NN)} \phi(a)$.

    Hence $\eldiag{\Si}{\MM} \cup \eldiag{\Si}{\NN}$
    is consistent as a $\Si(\MM,\NN)$-theory. 
    Let $\PP$ be a $\Si(\MM,\NN)$-model of this
    (and naturally a $\Si(\MM)$ or a $\Si(\NN)$ structure).
    Then \linkto{elem_ext_equiv_eldiag_model}{there exist elementary 
    $\Si(\MM)$ and $\Si(\NN)$-extensions}
    $\la_\MM : \MM \to \PP$ and $\la_\NN : \NN \to \PP$ such that 
    $\la_\MM(\subintp{\Si(\MM)}{\MM}{c_m}) = \subintp{\Si(\MM)}{\PP}{c_m}$ 
    for each constant symbol $c_m$ for $m \in \MM \setminus \io_\MM(A)$ 
    and 
    $\la_\MM(\subintp{\Si(\MM)}{\MM}{c_a}) = \subintp{\Si(\MM)}{\PP}{c_a}$ 
    for each constant symbol $c_a$ for $a \in A$
    (similarly with $\NN$).

    Naturally, we can move everything down to $\Si(A)$.
    Thus for any $a \in A$
    \[
        \la_\MM \circ \io_\MM(a) = \la_\MM (\subintp{\Si(A)}{\MM}{c_a})
        = \la_\MM (\subintp{\Si(\MM)}{\MM}{c_a}) = 
        \subintp{\Si(\MM)}{\PP}{c_a} = \subintp{\Si(A)}{\PP}{c_a}
    \]
    By symmetry we have 
    $\la_\MM \circ \io_\MM(a) = \subintp{\Si(A)}{\PP}{c_a} =
    \la_\NN \circ \io_\NN(a)$.
\end{proof}

\subsection{The L\"{o}wenheim-Skolem Theorems}
The results in this subsection aren't used until much later. %? when?
It is worth skipping for now, 
but the material can be covered with the foundations made so far.

\begin{prop}[Upward L\"{o}wenheim-Skolem Theorem]
    \link{upwards_lowenheim_skolem}
    If $\MM$ is an infinite $\Si$-structure 
    and $\ka$ a cardinal such that 
    $|{\MM}|+ |\const{\Si}| \leq \ka$,
    there exists a $\Si$-structure $\NN$ with cardinality 
    $\ka$ as well as an elementary 
    $\Si$-embedding from $\MM$ to $\NN$.
\end{prop}
\begin{proof}
    \linkto{move_up_sig}{$\MM$ is naturally a $\Si(\MM)$-structure},
    and $\MM \model{\Si(\MM)} \eldiag{\Si}{\MM}$.
    Thus $\eldiag{\Si}{\MM}$ is a 
    $\Si(\MM)$-theory with an infinite model $\MM$,
    \linkto{inf_mod_theory_has_inf_mod}{
        hence it has a $\Si(\MM)$-model $\NN$ of cardinality $\ka$}.
    Making $\NN$ a $\Si$-structure, 
    \linkto{elem_ext_equiv_eldiag_model}{
        we obtain a $\Si$-embedding from $\MM$ to $\NN$}.
\end{proof}

\begin{dfn}[Skolem Functions]
    We say that a $\Si$-theory $T$ 
    \emph{has built in Skolem functions} when for any $\Si$-formula $\phi$
    that is not a sentence, 
    with free variables indexed by $S$,
    there exists $f \in \func{\Si}$ such that $n_f = k$ and 
    \[T \model{\Si} \bigforall{i \in S}{} w_i, 
        \brkt{\exists v, \phi(v,w) \to \phi(f(w),w)},\]
    Note that $w$ can be length $0$, 
    in which case $f$ has arity $0$ 
    and so would be interpreted as a constant map.
    We would have 
    \[T \model{\Si}  
        \exists v, \phi(v) \to \phi(f)\]
\end{dfn}

\begin{prop}[Skolemization]
    \link{skolemization}
    Let $T(0)$ be a $\Si(0)$-theory, 
    then there exists $T$ a $\Si$ theory such that
    \begin{enumerate}
        \item $|\func{\Si}| = |\func{\Si(0)}| + \aleph_0$
        \item $\func{\Si(0)} \subs \func{\Si}$, 
            and they share the same constant and relation symbols
        \item $T(0) \subs T$
        \item $T$ has built in Skolem functions
        \item All models of $T(0)$ can be moved up to being models of $T$ 
            with interpretations agreeing on $\Si$.
    \end{enumerate} 
    We call $T$ the Skolemization of $T(0)$.
\end{prop}
\begin{proof}
    Similarly to the \linkto{make_wit}{Witness Property proof}, 
    we define $\Si(i), T(i)$ for each $i \in \N$.
    Suppose by induction that we have $T(i) \in \theory{\Si}$, 
    such that 
    \begin{enumerate}
        \item $|\func{\Si(i)}| = |\func{\Si(0)}| + \aleph_0$
        \item $\func{\Si(0)} \subs \func{\Si(i)}$
            and they share the same constant and relation symbols
        \item $T(0) \subs T(i)$
        \item All models of $T(0)$ can be moved up to being models of $T(i)$
            with interpretations agreeing on $\Si$
    \end{enumerate} 
    Then define $\Si(i+1)$ such that only the function symbols are enriched:
    \[
        \func{\Si(i+1)} := \func{\Si(i)} \cup 
        \set{f_{\phi} \st \phi \in \form{\Si(i)} 
        \text{ and $\phi$ is not a sentence}}
    \]
    extending the arity $n_\star$ to by having 
    $n_{f_\phi} = |S| - 1$, where $S$ is indexes the free variables of $\phi$.
    There are countably infinite $\Si(i)$-formulas, 
    thus $|\func{\Si(i)}| = |\func{\Si(0)}| + \aleph_0$.

    Define $\Psi : \form{\Si(i)} \to \form{\Si(i+1)}$ mapping 
    \[
        \phi \mapsto \forall w, (\exists v, \phi(v,w)) \to (\phi(f_\phi (w),w)),
    \] 
    where $w$ is a list of variables of the suitable length. 
    We then define 
    \[
        T(i+1) := T(i) \cup \Psi(\form{\Si(i)})
    \]
    which is a $\Si(i+1)$-theory because the image of $\Psi$ has only 
    $\Si$-sentences.
    Note that $T(0) \subs T(i) \subs T(i+1)$.

    Let $\MM(0)$ be a $\Si(0)$-model of $T(0)$, 
    then we have $\MM(i)$ a $\Si(i)$-model of $T(i)$ 
    with the same carrier set and same
    interpretation on $\Si(0)$ as $\MM(0)$.
    Let $\MM(i+1)$ have the same carrier set as $\MM(0)$.
    To extend interpretation to $\Si(i+1)$,
    we first deal with the case where ${\MM(i)}$ is empty by simply
    interpreting all new function symbols as the empty function.
    Otherwise we have a $c \in {\MM(0)}$. 
    For $f_\phi \in \Psi(\form{\Si(i)})$ define
    \begin{align*}
        \modintp{\MM(i+1)}{f_\phi} : 
        {\MM(i+1)}^{n_{f_\phi}} &\to {\MM(i+1)}\\
        a &\mapsto \begin{cases}
            b &, \text{ if } \exists b \in {\MM}, 
            \MM(i) \model{\Si(i)} \phi(b,a)\\
            c &, \text{ otherwise}
        \end{cases}\\
    \end{align*}
    Then by construction, 
    $\MM(i+1) \model{\Si(i+1)} \Psi(\form{\Si(i)})$.
    By checking the conditions on \linkto{move_up_sig}{
        moving $\MM(i)$ up to $\MM(i+1)$},
    we can also conclude
    $\MM(i+1) \model{\Si(i+1)} T(i)$.
    Hence $\MM(i+1) \model{\Si(i+1)} T(i+1)$.
    
    Let $\Si(*)$ 
    be the signature such that its constants and relations agree with $\Si(0)$
    and $\func{\Si(*)} = \bigcup_{i \in \N} \func{\Si(i)}$.
    Then 
    \[
        |\func{\Si(*)}| = |\bigcup_{i \in \N} \func{\Si(i)}| = 
        \aleph_0 \times (\aleph_0 + \func{\Si(0)}) = 
        \aleph_0 + \func{\Si(0)}
    \]

    Let $T(*) = \bigcup_{i \in \N} T(i)$.
    We show that $T(*)$ has built in Skolem functions.
    Let $\phi$ be a $\Si(*)$-formula that is not a $\Si$-sentence.
    Then $\phi \in \form{\Si(i)}$ for some $i \in \N$. 
    Thus $\Psi(\phi) \in T(i+1) \subs T(*)$,
    hence 
    \[T(*) \model{\Si(*)} \forall w, (\exists v, \phi(v,w)) 
    \to (\phi(f_\phi (w),w))\]
    Thus $T(*)$ has built in Skolem functions.
    
    If $\MM \model{\Si} T$ then let ${\MM(*)} = {\MM}$ 
    and define the interpretation such that for all $i \in \N$, 
    and $f \in T(i)$, 
    $\subintp{\Si(*)}{\MM(*)}{f} = \subintp{\Si(i)}{\MM(i)}{f}$.
    Since all interpretations agree upon intersection this is well-defined.
    To show that $\MM(*)$ is a $\Si(*)$-model of $T(*)$,
    let $\phi$ be in $T(*)$; 
    there is some $i \in \N$ such that $\phi \in T(i)$.
    Using our lifted $\MM(i)$ from before we have 
    $\MM(i) \model{\Si(i)} \phi$.
    By checking the conditions on \linkto{move_up_sig}{
        moving $\MM(i)$ up to $\MM(*)$},
    we can also conclude
    $\MM(*) \model{\Si(*)} \phi$
    (by taking the $\Si(*)$ theory $\set{\phi}$).
    Hence
    $\MM(*) \model{\Si(*)} T(*)$.
\end{proof}

\begin{dfn}[Theory of a Structure]
    We define the theory of a $\Si$-structure $\MM$ to be
    \[\Theory_{\MM} := \set{\phi \in \form{\Si} \st 
    \phi \text{ is a $\Si$-sentence and } \MM \model{\Si} \phi}\]
\end{dfn}

\begin{prop}[Downward L\"{o}wenheim-Skolem Theorem]
    Let $\NN$ be an infinite $\Si(0)$-structure and $M(0) \subs {\NN}$.
    Then there exists a $\Si(0)$-structure $\MM$ such that 
    \begin{itemize}
        \item $M(0) \subs {\MM} \subs {\NN}$
        \item $|{\MM}| \leq |M(0)| + |\func{\Si(0)}| + \aleph_0$
        \item The inclusion $\subs : \MM \to \NN$ is an elementary embedding.
    \end{itemize}
\end{prop}
\begin{proof}
    We first take the \linkto{skolemization}{Skolemization} 
    of $\Theory_{\NN}$ and call the new signature and theories 
    $\Si$ and $T$.
    Since $\NN \model{\Si(0)} \Theory_{\NN}$, 
    we can move it up to being a $\Si$-structure so that 
    $\NN \model{\Si} T$.

    We want to create the carrier set of $\MM$,
    it has to be big enough so that interpreted functions are closed on $\MM$.
    Given $M(i)$ such that $|M(i)| \leq |M(0)| + |\func{\Si}| + \aleph_0$, 
    we inductively define $M(i+1)$:
    \[
        M(i+1) := M(i) \cup \set{\subintp{\Si}{\NN}{f}(a) 
        \st f \in \func{\Si} \AND a \in {M(i)}^{n_f}}
    \]
    Then \begin{align*}
        |M(i+1)| &\leq |M(i)| + |\func{\Si}| \times |M(i)^{n_f}|\\
        &\leq |M(i)| + |\func{\Si}| \times (|M(i)| + \aleph_0)\\
        &\leq |M(0)| + |\func{\Si}| + \aleph_0 + |\func{\Si}| 
        \times (|M(0)| + |\func{\Si}| + \aleph_0)\\
        &\leq |M(0)| + |\func{\Si}| + \aleph_0
    \end{align*}
    Then ${\MM} := \bigcup_{i} M(i)$ and 
    $|{\MM}| \leq |M(i)| \times \aleph_0 =
    |M(0)| + |\func{\Si}| + \aleph_0 
    \leq |M(0)| + |\func{\Si(0)}| + \aleph_0$.

    We first interpret function symbols, 
    which will give us a way to interpret constant symbols. 
    For $f \in \func{\Si}$ and $a \in ({\MM})^{n_f}$,
    define $\subintp{\Si}{\MM}{f}(a) = \subintp{\Si}{\NN}{f}(a)$.
    This is well-defined as there exists 
    $i \in \N$ such that $a \in (M(i))^{n_f}$,
    \[\subintp{\Si}{\MM}{f}(a) \in M(i+1) \subs {\MM}\]
    Then to interpret constant symbols,
    we consider for each $c \in \const{\Si}$ the formula $v = c$.
    Since $T$ has built in Skolem functions and $\NN \model{\Si} T$, 
    there exists $f$ with arity $n_f = 0$ such
    $\NN \model{\Si} (\exists v, v = c) \to f = c$.
    Since $\NN \model{\Si} \exists v, v = c$, 
    we have $\subintp{\Si}{\NN}{f} = \subintp{\Si}{\NN}{c}$.
    Since $\subintp{\Si}{\NN}{f} = \subintp{\Si}{\MM}{f} : 
        ({\MM})^0 \to {\MM}$ 
    we can define $\subintp{\Si}{\MM}{c} = \subintp{\Si}{\NN}{c} = 
        \subintp{\Si}{\NN}{f} = \subintp{\Si}{\MM}{f} \in {\MM}$.
    
    Lastly define the interpretation of relations as 
    $\subintp{\Si}{\MM}{r} = ({\MM})^{m_r} \cap \subintp{\Si}{\NN}{r}$.

    By construction the inclusion $\subs$ is a $\Si$-embedding. 
    We check that it is elementary using the third equivalent condition in the
    \link{tarski_vaught}{Tarski Vaught Test}:
    let $\phi \in \form{\Si}$ with free variables indexed by $S$,
    $i \in S$ and
    $a \in ({\MM})^{S \setminus \set{i}}$.
    Suppose $\exists c \in {\NN}, \NN \model{\Si} \phi(a,c)$.
    $T$ has built in Skolem functions,
    and $\NN \model{\Si} T$. 
    Hence there exists $f \in \func{\Si}$ such that
    \[
        \NN \model{\Si} (\exists v, \phi(a,v)) \to \phi(a,f(a)) 
    \]
    We can deduce $\NN \model{\Si} \phi(a,f(a))$.
    Noting that $\subintp{\Si}{\MM}{f}(a) = \subintp{\Si}{\NN}{f}(a)$ 
    completes the Tarski Vaught Test. 
    Hence $\subs$ is an elementary $\Si$-embedding.
    
    We \linkto{move_down_morph}{move $\subs : \MM \to \NN$ down a signature} 
    since by 
    \linkto{skolemization}{Skolemization} we have 
    $\Si(0) \leq \Si$.
    Then $\subs : \MM \to \NN$ is an elementary $\Si$-embedding.
\end{proof}

%\section{Types}
This section mainly follows material from Tent and Ziegler's book \cite{tent}.
\subsection{Types on theories}
\begin{dfn}[$F(\Si,n)$ and formulas consistent with a theory]
    \link{dfn_types_on_theories}
    Let ${v_1,\dots v_n}$ be variables,
    $T$ be a $\Si$-theory.
    Let $F(\Si,n)$ be the set of $\Si$-formulas with at most 
    $v_1,\dots v_n$ as their free variables.
    For any $c \in \const{\Si}^n$, $p \subs F(\Si,n)$
    we write 
    \[p(c) = \set{\phi(c) \st \phi \in p}\]
    and if $\MM$ is a $\Si$-structure with $a \in {\MM}^n$ we write 
    \[\MM \model{\Si} p(a)\]
    to mean for every $\phi \in p$, $\MM \model{\Si} \phi(a)$.

    We say $p$ is a maximal (or complete) if for any $\phi \in F(\Si,n)$,
    $\phi \in p$ or $\NOT \phi \in p$.
    This generalises the notion of maximality (completeness) of a theory.
\end{dfn}

\begin{dfn}[Consistency for types and compactness for types]
    \link{equiv_def_of_consistent_with_theory}
    \link{compactness_for_types}
    Let $T$ be a $\Si$-theory and $p$ be a subset of $F(\Si,n)$.
    Let $c_1,\dots,c_n$ be new constant symbols and let $\Si(c)$ be 
    the signature with these added constant symbols.
    The following are equivalent:
    \begin{enumerate}
        \item $T \cup p(c)$ is a consistent $\Si(c)$-theory.
            (Where $p(c)$ is the formulas of $p$ with the variables substituted 
            by $c_1,\dots,c_n$.)
        \item (Consistent with $T$) There exists 
            $\MM \model{\Si} T$ and $a \in {\MM}^n$
            such that $\MM \modelsi p(a)$. 
        \item (Finitely consistent with $T$) 
            For any finite subset $\De \subs p$, there exists 
            $\MM \model{\Si} T$ and $a \in {\MM}^n$
            such that $\MM \modelsi p(a)$. 
    \end{enumerate}
    If any of the above is true then we say $p$ is consistent with $T$ and say 
    $p$ is an $n$-type on $T$ since there are up to $n$ variables in the 
    formulas of $p$.
    The second and third definitions being equivalent is the generalisation 
    of compactness for $n$-types.
\end{dfn}
\begin{proof}
    ($1. \iff 2.$) 
    \begin{forward}
        Suppose we have a $\Si(c)$-structure 
        $\MM \model{\Si(c)} T \cup p(c)$.
        Then by taking the images of the interpretation of each $c_i$ in $\MM$
        we obtain $a = \modintp{\MM}{c} \in {\MM}^n$ such that 
        $\MM \model{\Si(c)} p(a)$.
        \linkto{move_down_mod}{Moving this down to $\Si$} preserves 
        satisfaction of $p(a)$  as elements of $p(a)$ are $\Si$-formulas
        with values in $\MM$ (and $T$ for the same reason):
        \[\MM \model{\Si} T \cup p(a)\]
        and we have what we want.
    \end{forward}

    \begin{backward}
        Suppose we have $\MM \model{\Si} T$ and $a \in {\MM}^n$
        such that $\MM \modelsi p(a)$.
        We can make $\MM$ a $\Si(c)$-structure such that 
        everything from $\Si$ is interpreted in the same way 
        and each constant symbol $c_i$ is interpreted as $a_i$.
        Thus \linkto{move_up_mod}{$\MM \model{\Si(c)} T$} and for any 
        $\phi(c) \in p(c)$,
        \[\MM \model{\Si} \phi(a) \implies \MM \model{\Si(c)} \phi(a)
        \implies \MM \model{\Si(c)} \phi(c)\]
        as $c$ is interpreted as $a$.
        Hence $\MM \model{\Si(c)} T \cup p(c)$ 
        and $T \cup p(c)$ is consistent in $\Si(c)$.
    \end{backward}

    ($2. \iff 3.$) 
    \begin{align*}
        &p \text{ consistent with } T \\
        &\iff T \cup p(c) \text{ consistent in } \Si(c) 
        \quad \text{by (1. $\iff$ 2.)}\\
        &\iff \text{for any finite } \De(c) \subs p(c), 
        T \cup \De(c) \text{ consistent in } \Si(c) 
        \quad \text{by \linkto{compactness}{compactness}}\\
        &\iff \text{for any finite } \De \subs p, 
        T \cup \De(c) \text{ consistent in } \Si(c) \\
        &\iff \text{for any finite } \De \subs p, 
        \De \text{ consistent with } T
        \quad \text{by (1. $\iff$ 2.)}
    \end{align*}
\end{proof}

\begin{dfn}[Stone space of a theory]
    Let $T$ be a $\Si$-theory.
    Let the stone space of $T$, 
    $S_n(T)$ be the set of all maximal $n$-types on $T$.
    (The signature of the $n$-types of the on $T$ is implicit, 
    given by the signature of $T$.)
    We give a topology on $S_n(T)$ by specifying an open basis;
    $U \subs S_n(T)$ is an element of the basis when there exists 
    $\phi \in F(\Si,n)$ such that 
    \[U = [\phi]_T := \set{p \in S_n(T) \st \phi \in p}\]
\end{dfn}

\begin{prop}[Extending to maximal $n$-types (Zorn)]
    \link{extend_to_maximal_type_zorn}
    Any $n$-type can be extended to a maximal $n$-type.
\end{prop}
\begin{proof}
    Let $T$ be a theory and $p$ be a $n$-type.
    Order by inclusion the set
    \[ 
        Z = \set{q \in S_n(T) \st q \text{ is an } n\text{-type and }
        p \subs q}
    \]
    This is non-empty as it contains $p$.
    Let $p_0 \subs p_1 \subs \dots$ be a chain in $Z$.
    Then $m = \bigcup_{i \in \N} p_i$ is finitely consistent with $T$
    (by taking large enough $i$)
    \linkto{compactness_for_types}{hence} consistent with $T$. 
    By Zorn we have the existence of a maximal element $q$ in $Z$.
    To show that $q$ is a maximal $n$-type let $\phi \in F(\Si,n)$. 
    As $q$ is consistent with $T$ there exists a $\Si$-structure 
    $\MM \model{\Si} T$ and $a \in {\MM}^n$ such that 
    $\MM \model{\Si} q(a)$. 
    In the case that $\MM \model{\Si} \phi(a)$
    we have $q \cup \set{\phi}$ is consistent with $T$ and so by 
    maximality $\phi \in q$.
    In the other case $q \cup \set{\NOT \phi}$ is consistent and so
    $\NOT \phi \in q$.
\end{proof}

\begin{dfn}[Equivalence modulo a theory]
    \link{dfn_modulo_theory}
    We say two $\Si$-formulas $\phi$ and $\psi$ with 
    free variables indexed by $S$ are equivalent modulo a 
    $\Si$-theory $T$ if 
    \[T \model{\Si} \forall v, \brkt{\phi \iff \psi}\]
    where $v = (v_i)_{i \in S}$.
\end{dfn}

\begin{prop}[Basic facts about the basis elements]
    \link{basic_facts_basis_elems}
    Let $T$ be a $\Si$-theory, $\phi, \psi \in F(\Si,n)$.
    \begin{itemize}
        \item $(\NOT \phi) \in p$ if and only if $p \notin [\psi]_T$.
        \item $[\phi]_T = [\psi]_T$ if and only if $\phi$ and $\psi$ are 
            equivalent modulo $T$.

        The basis elements are closed under Boolean operations
        \item $[\NOT \phi]_T = S_n(T) \setminus [\phi]_T$
        \item $[\phi \OR \psi]_T = [\phi]_T \cup [\psi]_T$
        \item $[\phi \AND \psi]_T = [\phi]_T \cap [\psi]_T$
        \item $[\top]_T = S_n(T)$ and $[\bot]_T = \nothing$
    \end{itemize}
\end{prop}
\begin{proof}
    We will just prove a couple of these.
    \begin{itemize}
        \item Suppose $(\NOT \phi) \in p$.
            Then if $p \in [\phi]_T$ then since $p$ is consistent with $T$
            there exists a model $\MM$ and $a$ from 
            $\MM$ such that
            $\MM \modelsi \phi(a)$ and $\MM \nodel{\Si} \phi(a)$, 
            a contradiction.
            For the other direction, 
            $p \notin [\psi]_T$ and so $ \psi \notin p$ and
            by maximality $\NOT \phi \in p$.
        \item \begin{forward}
            Suppose for a contradiction
            $T \nodel{\Si} \forall v, \brkt{\phi \iff \psi}$.
            then \linkto{not_a_consequence}{there 
            exists $\MM \model{\Si} T$} and $a \in {\MM}^n$ such that 
            $\MM \modelsi \phi \AND \NOT \psi$ or 
            $\MM \modelsi (\NOT \phi) \AND \psi$.
            In the first case we have that $\set{\phi, \NOT \psi}$ 
            is consistent with $T$ and so can be 
            \linkto{extend_to_maximal_type_zorn}{extended
            to a maximal $n$-type $p$}.
            Thus $p \in [\phi]_T = [\psi]_T$ and $p \notin [\psi]_T$, 
            a contradiction.
            \end{forward}
            \begin{backward}
                Suppose $T \model{\Si} \forall v, \brkt{\phi \iff \psi}$.
                Let $p \in [\phi]_T$.
                It suffices to show that $p \in [\psi]_T$.
                Since $p$ is consistent with $T$ there exists a $\Si$-structure
                $\MM \model{\Si} T$ and $a \in {\MM}^n$ such that 
                $\MM \model{\Si} p(a)$.
                By assumption $\MM \model{\Si} \brkt{\phi \iff \psi}(a)$
                and $p \in [\phi]_T$ so 
                $\MM \model{\Si} \psi(a)$.
                Suppose $p \notin \psi$, then $\NOT \psi \in p$ hence 
                we have a contradiction.
            \end{backward}
        \item $p \in [\phi \OR \psi]_T$ if and only if 
            $(\phi \OR \psi) \in p$. 
            Suppose $\phi \notin p$ then by maximality 
            $(\NOT \phi) \in p$ and so $p \in [\psi]_T$. 
            In the other case $p \in [\phi]_T$.
            For the other direction $p \in [\phi]_T \cup [\psi]_T$
            implies $\phi \in p$ or $\psi \in p$.
            In the first case we have $\MM \model{\Si} T$ such that 
            $\MM \model{\Si} \phi$.
            Then $\MM \model{\Si} \phi \OR \psi$ and so $(\phi \OR \psi) \in p$.
    \end{itemize}
\end{proof}

\begin{prop}[Properties of the Stone space]
    \link{properties_of_stone_space}
    Let $T$ be a theory.
    \begin{itemize}
        \item Elements of the basis of $S_n(T)$ are clopen.
        \item $S_n(T)$ is Hausdorff.
        \item $S_n(T)$ is compact.
    \end{itemize}
\end{prop}
\begin{proof}~
\begin{itemize}
    \item By maximality of each $p$ the complement of $U$ is also 
        in the open basis:
        \[  
            \set{p \in S_n(T) \st \phi \notin p}
            = \set{p \in S_n(T) \st (\NOT \phi) \in p}
        \]
        Hence each element of the basis is clopen.
    \item Let $p,q \in S_n(T)$ and suppose $p \ne q$. 
        By maximality and the fact that $\form{\Si}$ is non-empty we can assume
        without loss of generality that there is $\phi \in p \setminus q$.
        Again by maximality $(\NOT \phi) \in q$, 
        and so $p \in [\phi], q \in [\NOT \phi]_T$.
        These opens are disjoint:
        if $r \in [\phi]_T \cap [\NOT \phi]_T$ then as 
        $r$ is consistent with $T$, there exists 
        $\MM \model{\Si} T$ such that 
        $\MM \model{\Si} \phi$ and $\MM \model{\Si}(\NOT \phi)$
        a contradiction. 
    \item Naturally we will use the
        \linkto{compactness_for_types}{compactness theorem for types}.
        Let $C$ be a collection of closed sets with finite intersection
        property.
        Then each closed set can be written as an intersection of
        basis elements (a finite union of closed sets is still a basis element
        since \linkto{basic_facts_basis_elems}{$[\phi]\cup[\psi]_T = 
            [\phi \OR \psi]_T$}):
        \[C = \set{\bigcap_{\phi \in \al}[\phi]_T \st \al \in I}\]
        Let 
        \[\Ga = \set{\phi \in \al \st \al \in I} \quad 
        \text{ and } \quad [\Ga]_T = \set{[\phi]_T \st \phi \Ga}\]
        Then the intersection any finite subset of $[\Ga]_T$ is non-empty as it
        contains a finite intersection of elements in $C$.
        Thus for any finite subset $\De \subs \Ga$ 
        there exists $p \in S_n(T)$ such that $\De \subs p$,
        and as $p$ is consistent with $T$ so is $\De$.
        Hence $\Ga$ is finitely consistent with $T$ and so
        \linkto{compactness_for_types}{$\Ga$ is consistent with $T$}.
        \linkto{extend_to_maximal_type_zorn}{Extending $\Ga$ to a 
        maximal $n$-type} $q$ gives us $\phi \in q$ for every $\phi \in \Ga$.
        Hence for all $\al \in I$ and for all $\phi \in \al$, $p \in [\phi]_T$
        and the intersection of $C$ is non-empty.
\end{itemize}\end{proof}

The Stone space is meant to have a geometric interpretation as 
\linkto{prime_spec_zariski_top}{$\spec(\MM[x_1,\dots,x_n])$}%?%?
when $\MM$ is an algebraically closed field.

\subsection{Types on structures}
\begin{dfn}[Realisation]
    Let $\MM$ be a $\Si$-structure and $A \subs {\MM}$.
    Let $p$ be a subset of $F(\Si(A),n)$
    (we will often be considering the $n$-types on $\eldiag{\Si}{\MM}$,
    a special case of this where $A = {\MM}$).
    Let $\NN$ be a $\Si(A)$-structure.
    \begin{itemize}
        \item $p$ is realised in $\NN$ by $a \in {\NN}^n$ over $A$ if
            \[\NN \model{\Si(A)} p(a)\]
            We also just say $p$ is realised in $\NN$.
            If $p$ is not realised in $\NN$
            then we say $\NN$ omits $p$.
        \item $p$ is finitely realised in $\NN$ 
            over $A$ if for each finite subset $\De \subs p$ 
            there exists $a \in {\NN}^n$
            such that $\De$ is realised in $\MM$ by $a$.
    \end{itemize}
\end{dfn}

\begin{lem}[Finite realisation and embeddings]
    \link{finite_realisation_and_embeddings}
    Let $\MM$ be a $\Si$-structure, 
    $A$ a subset of ${\MM}$ and $p$ a subset of $F(\Si(A),n)$.
    Then the following are equivalent 
    \begin{itemize}
        \item $p$ is consistent with $\eldiag{\Si}{\MM}$ 
            (i.e. it is an $n$-type over $\eldiag{\Si}{\MM}$).
        \item There exists an elementary embedding $\MM \to \NN$ 
            such that $p$ is realised in $\NN$.
        \item There exists an elementary embedding $\MM \to \NN$ 
            such that $p$ is finitely realised in $\NN$.
        \item $p$ is finitely realised in $\MM$.
    \end{itemize}
    The elementary embeddings can be seen as both $\Si$-embeddings or 
    $\Si(A)$-embeddings for any subset $A \subs \MM$.
\end{lem}
\begin{proof}
    $(1. \implies 2.)$ There exists $\NN$ and a
    $b \in {\NN}^n$ such that 
    $\NN \model{\Si(\MM)} \eldiag{\Si}{\MM}$ and 
    $\NN \model{\Si(\MM)} p(b)$.
    Then since \linkto{elem_ext_equiv_eldiag_model}{models of the 
    elementary diagram correspond to elementary extensions}, 
    there exists an elementary $\Si(\MM)$-embedding $\MM \to \NN$ 
    and $b \in {\NN}^n$ such that $p$ 
    is realised by $b$ in $\NN$.
    (This can be moved down to being a $\Si(A)$-embedding for any 
    subset $A \subs \MM$.)

    $(2. \implies 3.)$ Let $\De \subs p$ be finite.
    Then for the same embedding into $\NN$ we can see that $\De$ 
    is realised by $b$ in $\NN$.

    $(3. \implies 4.)$ Let $\De \subs p$ be finite.
    Then by assumption there exists an elementary $\Si$-embedding 
    $\io : \MM \to \NN$ and $b \in {\NN}^n$ 
    such that  
    $\NN \model{\Si} \De(b)$. 
    Hence 
    \[\NN \modelsi \exists v, \bigand{\psi \in \De}{} \psi(v)\]
    where $v$ represents the free variables in $\De$.
    By the embedding being elementary we have 
    \[\MM \modelsi \exists v, \bigand{\psi \in \De}{} \psi(v)\]
    Hence there exists $a \in \MM^\star$ which will realise $\De$.

    $(4. \implies 1.)$
        By \linkto{compactness_for_types}{compactness for types},
        it suffices to show that $p$ 
        is finitely consistent with the elementary diagram.
        Let $\De \subs p$ be finite.
        Then by assumption there is $a \in {\MM}^n$ such that 
        $\MM \model{\Si} \De(a)$ and so $\MM \model{\Si(\MM)} \De(a)$.
        Clearly $\MM$ is a model of its elementary diagram.
\end{proof}

\begin{dfn}[Type of an element]
    Let $\MM$ be a $\Si$-structure containing
    $A \subs {\MM}$ and $a \in {\MM}^n$
    Then 
    \[\subintp{A,n}{\MM}{\tp}(a) := 
    \set{\phi \in F(\Si(A),n) \st \MM \model{\Si(A)} \phi(a)}\]
    is the type of $a$ in $\MM$ over $A$.
    One can verify that this is a maximal $n$-type on $T$
    if $\MM$ is a model of $T$.
\end{dfn}

\begin{prop}[Elements of the Stone space are types of elements]
    \link{elems_of_stone_space_are_types_of_elements}
    Let $\MM$ be a $\Si$-structure,
    $A$ a subset of ${\MM}$ and $p$ a subset of $F(\Si(A),n)$.
    Let $a \in {\MM}^n$.
    Then \begin{itemize}
        \item $p$ is a maximal $n$-type on $\eldiag{\Si}{\MM}$ 
            that is realised by $a \in {\MM}^n$ 
            if and only if $p = \subintp{A,n}{\MM}{\tp}(a)$.
            Hence elements of the Stone space of $\eldiag{\Si}{\MM}$
            are types of elements.
        \item If $\MM \subs \NN$ is an elementary embedding then
            \[\subintp{A,n}{\MM}{\tp}(a) = \subintp{A,n}{\NN}{\tp}(a)\]
    \end{itemize}
\end{prop}
\begin{proof}~\begin{itemize}
    \item \begin{forward}
        As $p$ is realised by $a$, 
        $p \subs \subintp{A,n}{\MM}{\tp}(a)$.
        By maximality of $p$ any formula in $\subintp{A,n}{\MM}{\tp}(a)$ 
        is either in $p$
        or its negation is in $p$.
        If its negation is in 
        $p \subs \subintp{A,n}{\MM}{\tp}(a)$ 
        then $\MM \model{\Si(A)} \phi(a)$ and
        $\MM \nodel{\Si(A)} \phi(a)$, a contradiction.
    \end{forward}
    \begin{backward}
        If $p = \subintp{A,n}{\MM}{\tp}(a)$ 
        then clearly $p$ is realised by $a$ and so it is consistent with 
        $\eldiag{\Si}{\MM}$ thus it is an $n$-type over $\eldiag{\Si}{\MM}$.
        For any $\phi \in F(\Si(A),n)$,
        $\MM \model{\Si(A)} \phi(a)$ or $\MM \nodel{\Si(A)} \phi(a)$.
        Hence $\phi$ or $\NOT \phi$ is in $p$ and so it is maximal.
    \end{backward}
    \item \[\phi \in \subintp{A,n}{\MM}{\tp}(a) \iff \MM \model{\Si(A)} \phi(a)
        \iff \NN \model{\Si(A)} \phi(a) \iff 
        \phi \in \subintp{A,n}{\NN}{\tp}(a)\]
\end{itemize}
\end{proof}
%\section{Quantifier elimination and model completeness}
Written whilst following section on algebraically closed fields.
\subsection{Quantifier elimination}

\begin{dfn}[Equivalence modulo a theory]
    We say two $\Si$-formulas $\phi$ and $\psi$ with 
    free variables indexed by $S$ are equivalent modulo a 
    $\Si$-theory $T$ if 
    \[T \model{\Si} \forall v, \brkt{\phi \iff \psi}\]
    where $v = (v_i)_{i \in S}$.
\end{dfn}

\begin{dfn}[Quantifier elimination]
    Let $T$ be a $\Si$-theory and $\phi$ a $\Si$-formula.
    We say the quantifiers of $\phi$ can be eliminated if there exists a
    quantifier free $\Si$-formula $\psi$ that is equivalent to $\phi$ 
    modulo $T$.
    We say $\phi$ is reduced to $\psi$.

    We say $T$ has quantifier elimination if the quantifiers of any 
    $\Si$-formula can be eliminated.
\end{dfn}

\begin{lem}[Deduction]
    \link{deduction}
    Let $T$ be a $\Si$-theory, $\De$ a finite $\Si$-theory and 
    $\psi$ a $\Si$-sentence.
    Then $T \cup \De \model{\Si} \psi$ if and only if 
    \[T \model{\Si} \brkt{\bigand{\phi \in \De}{} \phi} \to \psi\]
\end{lem}
\begin{proof}
    We first case on if $\De$ is empty or not.
    If it is empty then $T \cup \De \model{\Si} \psi$ if and only if
    $T \model{\Si} \psi$ if and only if $T \model{\Si} \top \to \psi$
    if and only if 
    \[T \model{\Si} \brkt{\bigand{\phi \in \De}{} \phi} \to \psi\]
    \begin{forward}
        Suppose $\MM \model{\Si} T$ then we need to show 
        $\MM \model{\Si} \brkt{\bigand{\phi \in \De}{} \phi} \to \psi$. 
        Indeed, suppose $\MM \model{\Si} \brkt{\bigand{\phi \in \De}{} \phi}$ 
        then by induction $\MM \model{\Si} T \cup \De$ and so by assumption
        that $T \cup \De \model{\Si} \psi$ we have 
        $\MM \model{\Si} \psi$. 
        Hence $\MM \model{\Si} \brkt{\bigand{\phi \in \De}{} \phi} \to \psi$.
    \end{forward}

    \begin{backward}
        Suppose $\MM \model{\Si} T \cup \De$ then 
        $\MM \model{\Si} T$ thus by assumption that 
        $T \model{\Si} \brkt{\bigand{\phi \in \De}{} \phi} \to \psi$
        we have $\MM \model{\Si} \brkt{\bigand{\phi \in \De}{} \phi} \to \psi$.
        By induction $\MM \model{\Si} \brkt{\bigand{\phi \in \De}{} \phi}$ 
        thus we have $\MM \model{\Si} \psi$.
    \end{backward}
\end{proof}

\begin{lem}[Proofs are finite]
    \link{proofs_are_finite}
    Suppose $T$ is a $\Si$-theory and $\phi$ a $\Si$-sentence such that 
    $T \model{\Si} \phi$. 
    Then there exists a finite subset $\De$ of $T$ such that 
    $\De \model{\Si} \phi$.
\end{lem}
\begin{proof}
    We show the contrapositive.
    Suppose for all finite subsets $\De$ of $T$, $\De \nodel{\Si} \phi$,
    then \linkto{not_consequence}{$\De \cup \set{\phi}$ is consistent} and
    by \linkto{compactness}{compactness} $T \cup \set{\phi}$ is consistent.
    Hence \linkto{not_consequence}{$T \nodel{\Si} \phi$}.
\end{proof}

\begin{prop}[Eliminating quantifiers of a formula]
    \link{elim_quant_of_form}
    Let $\Si$ be a signature such that $\const{\Si} \ne \nothing$.
    Suppose $T$ is a $\Si$-theory and $\phi$ is a $\Si$-formula
    with free-variables $v = (v_1,\dots,v_n)$.
    Then the quantifiers of $\phi$ 
    can be eliminated if and only if the following holds:
    for any two $\Si$-models $\MM,\NN $ of $T$ and any 
    $\Si$-structure $\AA$ that with $\Si$-embeddings into both $\MM$ and $\NN$ 
    ($\io_\MM, \io_\NN$),
    if $a \in \AA^n$ then 
    \[\MM \model{\Si} \phi(\io_\MM (a)) 
    \iff \NN \model{\Si} \phi(\io_\NN (a))\]
\end{prop}
\begin{proof}
    \begin{forward}
        Let $a \in \AA^n$.
        By assumption there exists $\psi \in \form{\Si}$ such that 
        $T \model{\Si} \forall v, \brkt{\phi(v) \IFF \psi(v)}$
        Then $\MM \model{\Si} \phi(\io_\MM(a))$ if and only if 
        $\MM \model{\Si} \psi(\io_\MM(a))$ if and only if 
        $\AA \model{\Si} \psi(a)$, 
        since \linkto{emb_preserve_sat_of_quan_free}{
            embeddings preserve the satisfaction of quantifier free formulas}.
        Similarly, this is if and only if $\NN \model{\Si} \phi(\io_\NN(a))$.
    \end{forward}

    \begin{backward}
        Let 
        \[
            \Ga := \set{\psi \text{ quantifier free } \form{\Si} \st
            T \model{\Si} \forall v, (\phi \to \psi)}
        \]
        and let $\Si(*)$ be such that 
        $\const{\Si(*)} = \const{\Si} \cup \set{d_1,\dots,d_n}$
        for some new constant symbols $d_i$ 
        (indexed according to the free-variables of $\phi$).
        We claim that 
        $T \cup \set{\psi(d) \st \psi \in \Ga} \model{\Si(*)} \phi(d)$.
        We first look at how this would complete the proof.
        If it is true then as 
        \linkto{proofs_are_finite}{proofs are finite}
        we have a finite subsets $\De \subs \Ga$ such that
        $T \cup \set{\psi(d) \st \psi \in \De} \model{\Si(*)} \phi(d)$.
        By \linkto{deduction}{deduction} we have
        \[T \model{\Si(*)} \brkt{\bigand{\psi \in \De}{} \psi(d)} \to \phi(d)\]
        and by the \linkto{lemma_on_const}{lemma on constants}
        \[
            T \model{\Si} \forall v, 
            \brkt{\bigand{\psi \in \De}{} \psi(v)} \to \phi(v) 
        \]
        where $\brkt{\bigand{\psi \in \De}{} \psi(v)}$ is quantifier free.
        By the definition of $\De$ we have the other implication as well:
        \[
            T \model{\Si} \forall v, 
            \brkt{\bigand{\psi \in \De}{} \psi(v)} \IFF \phi(v) 
        \]
        hence the result.

        Suppose for a contradiction 
        $T \cup \set{\psi(d) \st \psi \in \Ga} \nodel{\Si(*)} \phi(d)$.
        Then there exists a model
        $\MM$ of $T \cup \set{\psi(d) \st \psi \in \Ga}$ such that 
        $\MM \nodel{\Si} \phi(d)$.

        Suppose for a second contradiction that the $\Si(*)(\MM)$-theory
        $T \cup \atdiag{\Si(*)}{\MM} \cup \set{\phi(d)}$ is inconsistent.
        Then by \linkto{compactness}{compactness} 
        some subset
        $T \cup \De \cup \set{\phi(d)}$ is inconsistent, 
        where $\De \subs \atdiag{\Si(*)}{\MM}$ is finite.
        \linkto{not_a_consequence}{This implies}
            $T \cup \De \model{\Si(*)(\MM)} \NOT \phi(d)$.
        Hence by \linkto{deduction}{deduction} we have 
        \[T \model{\Si(*)(\MM)} 
            \brkt{\bigand{\psi(d) \in \De}{} \psi(d)} \to \NOT \phi(d)\]
        By the \linkto{lemma_on_const}{lemma on constants
        applied to $\const{\Si} \subs \const{\Si(*)(\MM)}$}
        \[T \model{\Si} 
            \forall v, 
            \sqbrkt{\brkt{\bigand{\psi(d) \in \De}{}\psi(v)} 
            \to \NOT \phi(v)}\]
        Taking the contrapositive, 
        \[T \model{\Si} 
            \forall v, 
            \sqbrkt{\phi(v) 
            \to \brkt{\bigor{\psi(d) \in \De}{}\NOT \psi(v)}}\]
        Hence $\bigor{\psi(d) \in \De}{}\NOT \psi(v) \in \Ga$
        and so $\MM \model{\Si(*)} \bigor{\psi(d) \in \De}{}\NOT \psi(v)$
        by definition of $\MM$.
        However each $\De \subs \atdiag{\Si(*)}{\MM}$ and so
        $\MM \model{\Si(*)} \bigand{\psi(d) \in \De}{}\psi(v)$, 
        a contradiction.
        Thus there exists a model
        \[\NN \model{\Si(*)(\MM)} T \cup 
            \atdiag{\Si(*)}{\MM} \cup \set{\phi(d)}\]
        Since $\NN \model{\Si(*)(\MM)} \atdiag{\Si(*)}{\MM}$ there exists a 
        $\Si(*)(\MM)$ morphism $\io : \MM \to \NN$.
        \linkto{move_down_morph}{Move this morphism down to $\Si$}, then
        by assumption with $\AA := \MM$, for any sentence $\chi$
        \[\MM \model{\Si} \chi
            \iff \NN \model{\Si} \chi\]
        Since $\NN \model{\Si(*)(\MM)} \phi(d)$ by the 
        \linkto{lemma_on_const}{lemma on constants} 
        $\NN \model{\Si} \forall v, \phi(v)$
        and so $\MM \model{\Si} \forall v, \phi(v)$.
        Which is a contradiction because
        $\MM \model{\Si(*)} \NOT \phi(d)$ and so 
        by the \linkto{lemma_on_const}{lemma on constants} 
        $\MM \model{\Si} \forall v, \NOT \phi(v)$.
        (We have
        $\MM \model{\Si} \phi(\modintp{\MM}{c}, \dots \modintp{\MM}{c})$
        and $\MM \nodel{\Si} \phi(\modintp{\MM}{c}, \dots \modintp{\MM}{c})$.)
    \end{backward}
\end{proof}

\begin{lem}[Sufficient condition for quantifier elimination]
    \link{condition_for_q_e}
    Let $T$ be a $\Si$-theory and suppose for any quantifier free $\Si$-formula
    $\psi$ with at least one free variable $w$, the quantifier of 
    $\forall w, \psi (w)$ can be eliminated.
    Then $T$ has quantifier elimination.
\end{lem}
\begin{proof}
    Induct on what $\phi$ is.
    \begin{itemize}
        \item If $\phi$ is $\top$, an equality or a relation then it is already 
            quantifier free.
        \item If $\phi$ is $\NOT \chi$ and there exists a quantifier free 
            $\Si$-formula $\psi$ such that 
            $T \model{\Si} \forall v, \chi \IFF \psi$.
            Then $T \model{\Si} \forall v, \NOT \chi \IFF \NOT \psi$.
            Hence $\phi$ can be reduced to $\NOT \psi$ which is quantifier free.
        \item If $\phi$ is $\chi_0 \OR \chi_1$ and there exist respective 
            reductions of these $\psi_0$ and $\psi_1$ then
            $\phi$ reduces to $\psi_0 \OR \psi_1$ which is quantifier free.
        \item If $\phi$ is $\forall w, \chi(w)$ and there exists quantifier free
            $\psi$ such that 
            \[T \model{\Si} \forall w, \bigforall{v \in S}{} v, 
                \brkt{\chi \IFF \psi}\]
            where $S$ indexes the rest of the free variables in 
            $\chi$ and $\psi$.
            Then we can show that 
            \[
                T \model{\Si} \bigforall{v \in S}{}
                \brkt{\phi \IFF (\forall w, \psi)}
            \]
            By assumption there exists $\om$ a quantifier free $\Si$-formula
            such that 
            \[
                T \model{\Si} \bigforall{v \in S}{}
                \brkt{\om \IFF (\forall w, \psi)}
            \]
            Hence $\phi$ can be reduced to $\om$.
    \end{itemize}
\end{proof}

\begin{cor}[Improvement: Sufficient condition for quantifier elimination]
    \link{improved_condition_for_q_e}
    If $T$ be a $\Si$-theory if
    for any quanfier free $\Si$-formula $\phi$ with at least one free variable
    $w$ (index the rest by $S$),
    for any $\MM, \NN$ $\Si$-models of $T$, for any $\Si$-structure $\AA$ 
    that embeds into $\MM$ and $\NN$ (via $\io_\MM,\io_\NN$) and 
    any $a \in ({\AA})^S$,
    \[\MM \model{\Si} \forall w, \phi(\io_{\MM}(a)) \implies 
    \NN \model{\Si} \forall w, \phi(\io_\NN (a))\]
    then $T$ has quantifier elimination.

    Equivalently we can use the statement 
    \[\MM \model{\Si} \exists w, \phi(\io_{\MM}(a)) \implies 
    \NN \model{\Si} \exists w, \phi(\io_\NN (a))\]
    by negating $\phi$.
\end{cor}
\begin{proof}
    To show that $T$ has quantifier elimination
    \linkto{condition_for_q_e}{it suffices to show that} 
    for any quanfier free $\Si$-formula $\phi$ with at least one free variable
    $w$ (index the rest by $S$), 
    the quantifiers of $\forall w, \phi$ can be eliminated.
    This is true \linkto{elim_quant_of_form}{if and only if} for any 
    $\MM, \NN$ $\Si$-models of $T$, 
    for any $\Si$-structure $\AA$ 
    that injects into $\MM$ and $\NN$ (via $\io_\MM,\io_\NN$) and 
    any $a \in (\AA)^S$,
    \[\MM \model{\Si} \forall w, \phi(\io_{\MM}(a)) \implies 
    \NN \model{\Si} \forall w, \phi(\io_\NN (a))\]
    By symmetry of $\MM$ and $\NN$ we only require one implication.
    Hence the proposition.
\end{proof}
\begin{rmk}
    For quantifier elimination it also suffices to show that for any 
    $\Si$-models $\MM$ of $T$, for any $\Si$-structure $\AA$ 
    that embeds into $\MM$ (via $\io_\MM$) and 
    any $a \in ({\AA})^S$,
    \[\MM \model{\Si} \forall w, \phi(\io(a)) \implies 
    \AA \model{\Si} \forall w, \phi(a)\]
    since \linkto{emb_preserve_sat_of_forall_down}{
        embeddings preserve satisfaction of universal formulas downwards}.
    Equivalently we can use
    \[\AA \model{\Si} \exists w, \phi(a) \implies 
    \MM \model{\Si} \exists w, \phi(\io(a))\]
\end{rmk}

\subsection{Back and Forth}
`Back and forth' is a technique used to determine 
elementary equivalence of models,
quantifier elimination of theories
and completeness of theories.
This section draws together work from Poizat \cite{poizat}, 
OLP \cite{openlogicproject},
and Pillay \cite{pillay}.
It is motivated by the \linkto{infinite_infinite_classes}{example at the end}, 
which should be looked at first.

\begin{dfn}[Substructure generated by a subset]
    Let $\MM$ be a $\Si$-structure.
    Let $A \subs \MM$.
    Then the following are equal:
    \begin{itemize}
        \item The set $\<A\>$ defined inductively: $A\subs \<A\>$;
        if $c \in \const{\Si}$ then 
        $\mmintp{c} \in \<A\>$; if $f \in \func{\Si}$ and 
        $\al \in \<A\>^{n_f}$ then $\mmintp{f}(\al) \in \<A\>$.
        \item $\bigcap \set{\NN \text{ substructure of } \MM \st A \subs \NN}$
    \end{itemize}
    and define a substructure of $\MM$.
    We say it is the `substructure of $\MM$ generated by $A$'.
    
    We say a substructure is finitely generated if there exists a finite set 
    $A$ such that it is equal to $\<A\>$.
\end{dfn}
\begin{proof}
    We note show that $\<A\>$ is a substructure of $\MM$ containing $A$:
    It contains the interpretations of constant symbols from $\MM$.
    By definition $\modintp{\<A\>}{f} := \mmintp{f}$ is well defined.
    Each relation $r$ is naturally interpreted as the intersection of relations
    on $\MM$ intersected with $\<A\>^{m_r}$.
    Hence $\bigcap \NN \subs \<A\>$.

    For the other direction note that if $a \in \<A\>$ then it is in $A$,
    $\mmintp{c}$
    or $\mmintp{f}(\al)$ for some $\al \in \<A\>^{n_f}$.
    If it is in $A$ then we are done.
    Any substructure of $\MM$ contains the $\mmintp{c}$ for each constant symbol
    hence the first case is fine.
    Any substructure of $\MM$ is closed under $\mmintp{f}$ and by induction
    $\al \in \NN^{n_f}$ for any substructure $\NN$.
    Hence $f(\al) \in \NN$ for any substructure.
    Thus $\<A\> \subs \bigcap \NN$ and we are done.
\end{proof}

\begin{prop}[Image of generators are generators of the image]
    \link{im_of_gen_are_gen_of_im_substructures}
    The image of a substructure generated by a subset is a substructure 
    generated by the image of a set.
    In particular,
    a finitely generated substructure has finitely generated image under a 
    $\Si$-morphism given by the image of the generators.
\end{prop}
\begin{proof}
    Let $\io : \<A\> \to \NN$ be a $\Si$-morphism.
    We show that $\<\io(A)\> = \io(\<A\>)$.
    If $b \in \<\io(A)\>$ then $b = \nnintp{c}$ or $b = \nnintp{f}(\io(\al))$ 
    for $al \in \<A\>$.
    Hence $b = \nnintp{c} = \io(\mmintp{c}) \in \io(\<A\>)$
    or 
    \[b = \nnintp{f}(\io(\al)) = \io(\mmintp{f}(\al)) \in \io(\<A\>)\]
    Thus $\<\io(A)\> \subs \io(\<A\>)$.
    The other direction is similar.
\end{proof}

\begin{dfn}[Partial isomorphisms]
    Let $\MM$ and $\NN$ be $\Si$-structures.
    A partial isomorphism from $\MM$ to $\NN$ is a $\Si$-isomorphism $p$
    with finitely generated domain in of $\MM$ 
    and codomain in $\NN$.
\end{dfn}

\begin{prop}[Equivalent definition of partial isomorphism]
    \link{equiv_def_partial_iso}
    Let $\MM$ and $\NN$ be $\Si$-structures. 
    Let $a \in \MM^n$ and $b \in \NN^n$.
    The following are equivalent:
    \begin{itemize}
        \item There exists a partial isomorphism $p : \<a\> \to \<b\>$ 
            such that $p(a) = b$.
        \item $\subintp{\nothing}{\MM}{\qftp}(a) = 
            \subintp{\nothing}{\NN}{\qftp}(b)$
    \end{itemize}
\end{prop}
\begin{proof}
    \begin{forward}
        We induct on terms to show that $\mmintp{t}(a) = \modintp{\<a\>}{t}(a)$
        for each term $t$:
        \begin{itemize}
            \item If $t$ is a constant symbol or a variable 
                then by definition of the 
                the substructure interpretation they are equal.
            \item If $t$ is $f(s)$ and we have the inductive hypothesis 
                $\mmintp{s}(a) = \modintp{\<a\>}{s}(a)$
                then by definition of the substructure interpretation
                \[
                    \mmintp{t}(a) = \mmintp{f}(\mmintp{s}(a))
                    = \mmintp{f}(\modintp{\<a\>}{s}(a))
                    = \modintp{\<a\>}{f}(\modintp{\<a\>}{s}(a))
                    = \modintp{\<a\>}{t}(a)
                \]
        \end{itemize}
    
    Let $\phi$ be a quantifier free $\Si$-formula with up to $n$ variables.
    We show by induction on $\phi$ that 
    \[
        \MM \model{\Si} \phi(a)
        \iff \<a\> \model{\Si} \phi(a)
    \]
    \begin{itemize}
        \item If $\phi$ is $\top$ it is trivial.
        \item If $\phi$ is $t = s$ then it is clear that
            \[
                \mmintp{t}(a) = \mmintp{s}(a)
                \iff \modintp{\<a\>}{t}(a) = \modintp{\<a\>}{s}(a)
            \]
            by what we showed for terms.
        \item If $\phi$ is $r(t)$ then 
            \[  
                (a_{i_1},\dots, a_{i_m}) \in \mmintp{r}
                \iff (a_{i_1},\dots, a_{i_m}) \in 
                \mmintp{r} \cap \<a\> = \modintp{\<a\>}{r}
            \]
        \item If $\phi$ is $\NOT \psi$ or $\psi \OR \chi$ then it is 
            clear by induction.
    \end{itemize}
    As $p$ is an $\Si$-isomorphism, for any quantifier free $\Si$-formula
    with up to $n$ variables,
    \[\MM \model{\Si} \phi(a) \iff \<a\> \model{\Si} \phi(a)
    \iff \<b\> \model{\Si} \phi(b)
    \NN \model{\Si} \phi(b)\]
    \end{forward}

    \begin{backward}
        Suppose $\subintp{\nothing}{\MM}{\qftp}(a) = 
        \subintp{\nothing}{\NN}{\qftp}(b)$.
        We define $p : \<a\> \to \NN$ by the following:
        if $\al \in \<a\>$ then one can write $\al$ as a term $t$ evaluated
        at $a$: $\al = \mmintp{t}(a)$; $p$ maps $a$ to $\nnintp{t}(b)$.
        To show that $p$ is well-defined, note that if two terms $t$ and $s$
        are such that $\mmintp{t}(a) = \mmintp{s}(a)$ then 
        $t = s$ is a formula in $\subintp{\nothing}{\MM}{\qftp}(a) = 
        \subintp{\nothing}{\NN}{\qftp}(b)$ and so 
        $\nnintp{t}(b) = \nnintp{s}(b)$.
        it is injective because if two terms $t$ and $s$
        are such that $\nnintp{t}(b) = \nnintp{s}(b)$ then 
        $t = s$ is a formula in $\subintp{\nothing}{\NN}{\qftp}(b) = 
        \subintp{\nothing}{\MM}{\qftp}(a)$ and so 
        $\mmintp{t}(a) = \mmintp{s}(a)$.

        By definition $p$ commutes with the interpretation of constant symbols,
        function symbols, and relations.
        Furthermore, for each $i$, $p(a_i) = b_i$ by taking the term to be a 
        variable and evaluating at $a_i$. 
        \linkto{im_of_gen_are_gen_of_im_substructures}{The 
            image of $p$ is $\<b\>$ as the image of $a$ is $b$}.
        Hence it is a partial isomorphism $\<a\> \to \<b\>$ 
        such that $p(a) = b$.
    \end{backward}
\end{proof}

\begin{prop}[Basic facts about partial isomorphisms]
    \link{basic_facts_partial_isomorphisms}
    Let $\MM$ and $\NN$ be $\Si$-structures.
    \begin{itemize}
        \item The inverse of a partial isomorphism is a partial isomorphism.
        \item The restriction of a partial isomorphism is a partial isomorphism.
        \item The composition of partial isomorphisms is a partial isomorphism.
    \end{itemize}
\end{prop}

\begin{dfn}[Partially isomorphic structures]
    Let $\MM$ and $\NN$ be $\Si$-structures.
    A partial isomorphism from $\MM$ to $\NN$ is said to have 
    the back and forth property if 
    \begin{itemize}
        \item (Forth) For each $a \in \MM$
            there exists a partial isomorphism $q$ such that 
            $q$ extends $p$ and $a \in \dom{p}$.
        \item (Back) For each $p \in I$ 
            there exists a partial isomorphism $q$ such that 
            $q$ extends $p$ and $b \in \codom{q}$.
    \end{itemize}

    We say $\MM$ and $\NN$ are back and forth equivalent 
    when all partial isomorphisms from $\MM$ to $\NN$ 
    have the back and forth property.
\end{dfn}

\begin{prop}[Equivalent definition of back and forth property]
    \link{equiv_def_back_and_forth}
    Let $\MM$ and $\NN$ be $\Si$-structures.
    Let $p : \<a\> \to \<b\>$ for $a \in \MM^n$ and $b \in \NN^n$
    be a partial isomorphism such that $p(a) = b$.
    It has the back and forth property if and only if the two conditions hold
    \begin{itemize}
        \item (Forth) For any $\al \in \MM$, 
            there exists $\be \in \NN$ such that 
            $\subintp{\nothing}{\MM}{\qftp}(a,\al) = 
            \subintp{\nothing}{\NN}{\qftp}(b,\be)$
        \item (Back) For any $\be \in \NN$, 
        there exists $\al \in \MM$ such that 
        $\subintp{\nothing}{\MM}{\qftp}(a,\al) = 
        \subintp{\nothing}{\NN}{\qftp}(b,\be)$
    \end{itemize}
\end{prop}
\begin{proof}
    \begin{forward}
        Suppose $p$ has the back and forth property.
        We only show `forth' as the `back' case is similar.
        Let $\al \in \MM$.
        By `forth' there exists $q$ 
        a partial isomorphism extending $p$ such that 
        $\al \in \dom(q)$.
        By \linkto{basic_facts_partial_isomorphisms}{restriction}
        and the fact that \linkto{image_of_generators}{the image of 
        generators generates the image},
        there exists $\be \in \NN$ such that 
        \[\res{q}{\<a,\al\> \to \<b,\be\>}\]
        is a local isomorphism.
        Using the \linkto{equiv_def_of_partial_iso}{the equivalent definition}
        we obtain ${\qftp}(a,\al) = {\qftp}(b,\be)$.
    \end{forward}

    \begin{backward}
        We show that $p$ has the `forth' property.
        Let $\al \in \MM$.
        By assumption there exists $\be \in \MM$ such that 
        \[\subintp{\nothing}{\MM}{\qftp}(a,\al) = 
        \subintp{\nothing}{\NN}{\qftp}(b,\be)\]
        Thus \linkto{equiv_def_of_partial_iso}{there exists 
        $q : \<a,\al\> \to \<b,\be\>$} such that $q(a) = b$ and $q(\al) = \be$.
        Hence $p$ is extended by $q$ with $\al$ in its domain.
    \end{backward}
\end{proof}

\begin{prop}[Quantifier elimination for types]
    \link{quant_elim_for_types}
    Let $T$ be a $\Si$-theory.
    $T$ has quantifier elimination if and only if for any $n \in \N$,
    any two $\Si$-models of $T$ and any $a \in \MM^n, b \in \NN^n$,
    if \[\subintp{\nothing}{\MM}{\qftp}(a) = 
    \subintp{\nothing}{\NN}{\qftp}(b)\]
    then \[\subintp{\nothing}{\MM}{\tp}(a) = 
    \subintp{\nothing}{\NN}{\tp}(b)\]
\end{prop}
\begin{proof}
    \begin{forward}
        Let $\phi \in {\tp}(a)$.
        By quantifier elimination there exists quantifier free $\psi$
        such that they are equivalent modulo $T$.
        Then $\MM \model{\Si} \psi(a)$ and 
        $\psi \in {\qftp}(a) = {\qftp}(b)$.
        Thus $\NN \model{\Si} \psi(b)$ and by equivalence modulo $T$
        $\NN \model{\Si} \phi(b)$.
        Hence $\phi \in {\tp}(b)$.
        The other inclusion is similar.
    \end{forward}

    \begin{backward}
        Let $n \in \N$. Define a map $f : S_n(T) \to S_n^{\qf}(T)$ 
        that takes a maximal $n$-type $p$ to $p \cap QFF(\Si,n)$.
        It is well-defined as the image is indeed a maximal $n$-type.
        It is a surjection as any quantifier free maximal 
        $n$ type is an $n$-type 
        and therefore 
        \linkto{extend_to_maximal_type_zorn}{can be extended to a maximal 
        $n$-type.}
        To show injectivity we note that
        \linkto{elems_of_stone_space_are_types_of_elements}{any two elements 
        of $S_n(T)$ can be written as types of elements}
        $\subintp{\nothing}{\MM}{\tp}(a)$ and  
        $\subintp{\nothing}{\NN}{\tp}(b)$.
        If their images are equal then 
        \[\subintp{\nothing}{\MM}{\qftp}(a) = 
        \subintp{\nothing}{\NN}{\qftp}(b)\]
        thus by assumption they are equal.

        To show that $f$ is continuous we show that elements of 
        the clopen basis have clopen preimage.
        Let $[\phi]_T^{\qf}$ be in the clopen basis of $S_n^{\qf}(T)$.
        Then $p \in [\phi]_T$ if and only if $\phi \in p$ if and only if 
        $\phi \in f(p)$ if and only if $f(p) \in [\phi]_T^{\qf}$.
        Hence the preimage is $[\phi]_T$ which is clopen.

        A continuous bijection between Hausdorff compact spaces is a 
        homeomorphism. 
        Hence for any $\phi \in F(\Si,n)$ the image of the clopen set generated
        by $\phi$ is clopen: there exists $\psi \in QFF(\Si,n)$
        such that $[\phi]_T = f^{-1}[\psi]_T^{\qf} = [\psi]_T$.
        $[\phi]_T = [\psi]_T$ \linkto{basic_facts_basis_elems}{if and only if} 
        they are equivalent modulo $T$.
        Thus we can eliminate quantifiers for any 
        $\phi \in F(\Si,n)$ for any $n$.
        Thus $T$ has quantifier elimination.
    \end{backward}
\end{proof}

\begin{lem}[Back and forth equivalence implies quantifier elimination for types]
    \link{back_and_forth_gives_quantifier_elimination_lem}
    Let $\MM$ and $\NN$ be $\Si$-structures.
    If $\MM$ and $\NN$ are back and forth equivalent and
    $a \in \MM^n$ and $b \in \NN^n$ are such that
    \[\subintp{\nothing}{\MM}{\qftp}(a) = 
    \subintp{\nothing}{\NN}{\qftp}(b)\]
    then \[\subintp{\nothing}{\MM}{\tp}(a) = 
    \subintp{\nothing}{\NN}{\tp}(b)\]
\end{lem}
\begin{proof}
    Let $\phi \in F(\Si,n)$.
    If $\phi$ is quantifier free then 
    $\MM \modelsi \phi(a) \iff \NN \modelsi \phi(b)$.
    By induction on formulas it suffices to show that if 
    $\phi$ is the formula $\forall v, \psi$ and 
    for any $\al \in \MM$ there exists $\be \in \NN$ such that 
    $\MM \modelsi \psi(a,\al) \iff \NN \modelsi \psi(b,\be)$, 
    then we have 
    $\MM \modelsi \forall v, \psi(a) \iff \NN \modelsi \forall v, \psi(b)$.

    By the \linkto{equiv_def_partial_iso}{
        equivalent definition of partial isomorphisms,}
    there exists $p : \<a\> \to \<b\>$ 
    a partial isomorphism in $p$ such that $p(a) = b$.
    Suppose $\MM \modelsi \forall v, \psi(a)$ and let $\be \in \NN$, 
    then $\MM \modelsi \forall v, \psi(a,\al)$.
    By `back' in 
    \linkto{equiv_def_back_and_forth}{
        the equivalent definition of the back and forth property}
    there exists $\al \in \MM$ such that 
        $\subintp{\nothing}{\MM}{\qftp}(a,\al) = 
        \subintp{\nothing}{\NN}{\qftp}(b,\be)$
    Hence $\NN \modelsi \forall v, \psi(a,\al)$.
    The other direction is similar.
\end{proof}

\begin{cor}[Back and forth implies elementary equivalence]
    \link{back_and_forth_implies_elem_equiv}
    Let $\MM$ and $\NN$ be $\Si$-structures.
    If $\MM$ and $\NN$ are back and forth equivalent
    then they are elementarily equivalent.
\end{cor}
\begin{proof}
    Let $\phi$ be a quantifier free $\Si$-formula with $0$ variables,
    i.e. a quantifier free sentence.
    As the empty set is a partial isomorphism.
    Thus by the \linkto{equiv_def_partial_iso}{equivalent 
        definition of a partial isomorphism,}
    \[\subintp{\nothing,0}{\MM}{\qftp}(\nothing) = 
    \subintp{\nothing,0}{\NN}{\qftp}(\nothing)\]
    By the fact that 
    \linkto{back_and_forth_gives_quantifier_elimination_lem}{back 
        and forth equivalence implies quantifier elimination for types},
    \[\subintp{\nothing,0}{\MM}{\tp}(\nothing) = 
    \subintp{\nothing,0}{\NN}{\tp}(\nothing)\]
    
    Thus for any $\Si$-sentence $\phi$, $\MM \modelsi \phi$ if and only if 
    $\phi \in \subintp{\nothing,0}{\MM}{\tp}(\nothing) = 
    \subintp{\nothing,0}{\NN}{\tp}(\nothing)$
    if and only if $\NN \modelsi \phi$.
\end{proof}

\begin{dfn}[$\om$-saturation]
    \link{om_saturation_dfn}
    Let $\MM$ be a $\Si$-structure. 
    $\MM$ is $\om$-saturated if for every finite subset $A \subs \MM$, 
    every $n \in \N$ and every $p \in S_n(\Theory_\MM(A))$,
    $p$ is realised in $\MM$.

    See the general version $\ka$-saturated \linkto{ka_saturation_dfn}{here}.
\end{dfn}

\begin{prop}[$\infty$-equivalence]
    \link{infty_equivalence_01}
    Let $\MM$ and $\NN$ be $\om$-saturated $\Si$-structures.
    If $a \in \MM^n$ and $b \in \NN^n$ satisfy
    \[\subintp{\nothing,n}{\MM}{\tp}(a) = 
    \subintp{\nothing,n}{\NN}{\tp}(b)\]
    then 
    \begin{itemize}
        \item (Forth) For any $\al \in \MM$ there exists $\be \in \NN$ such that
        \[\subintp{\nothing,n+1}{\MM}{\tp}(a,\al) = 
        \subintp{\nothing,n+1}{\NN}{\tp}(b,\be)\]
        \item (Back) For any $\be \in \NN$ there exists $\al \in \MM$ such that
        \[\subintp{\nothing,n+1}{\MM}{\tp}(a,\al) = 
        \subintp{\nothing,n+1}{\NN}{\tp}(b,\be)\]
    \end{itemize}
    If this property holds for any pair $a,b$ related by a partial isomorphism
    we say $\MM$ and $\NN$ are $\infty$-equivalent.
\end{prop}
\begin{proof}
    Let $\al \in \MM$ and consider 
    \[p(a,v) := \subintp{a,1}{\MM}{\tp}(\al) \in S_1(\Theory_\MM(a))\]
    Any formula in $p(a,v)$ can be written as a 
    $\Si$-formula $\phi(w,v)$ with variables $w$
    replaced with elements of $a$ 
    ($v$ represents a single variable to be replaced by $\al$). 
    Let 
    \[p(w,v) := \set{\phi(w,v) \st \phi(a,v) \in p(a,v)}\]
    We claim that 
    \[p(b,v) := \set{\phi(b,v) \st \phi \in p(w,v)} \in S_1(\Theory_\NN(b))\]
    To this end, we note that it is indeed 
    a maximal subset of $F(\Si(b),1)$ since for any $\phi(b) \in F(\Si(b),1)$
    \[\phi(a) \in p(a,v) \text{ or } \NOT \phi(a) \in p(a) \implies 
    \phi(b) \in p(b,v) \text{ or } \NOT \phi(b) \in p(b)\]
    We just need to show that it is consistent with $\Theory_\NN(b)$.

    By \linkto{compactness_for_types}{compactness for types} 
    and noting that $\NN$ is a $\Si(b)$-model of $\Theory_\NN(b)$,
    it suffices to show
    that for any finite subset $\De(w,v) \subs p(w,v)$ 
    there exists $\be \in \NN^m$ such that 
    $\NN \model{\Si(b)} \De(b,\be)$.
    \begin{align*}
        &\MM \model{\Si(a)} \bigand{\phi \in \De}{} \phi(a,\al)\\
        \implies &\MM \model{\Si} \exists v, \bigand{\phi \in \De}{} \phi(a,v)\\
        \implies &\brkt{\exists v, \bigand{\phi \in \De}{} \phi(a,v)} \in 
        \subintp{\nothing}{\MM}{\tp}(a) = 
        \subintp{\nothing}{\NN}{\tp}(b)\\
        \implies &\NN \modelsi \exists v, \bigand{\phi \in \De}{} \phi(b,v)\\
        \implies &\exists \be \in \NN, 
        \NN \modelsi \bigand{\phi \in \De}{} \phi(b,\be)\\
        \implies &\exists \be \in \NN, \NN \model{\Si(b)} \De(b,\be)
    \end{align*}
    Thus $p(b,v)\in S_1(\Theory_\NN(b))$ and since $\NN$ is $\om$-saturated
    $p(b,v)$ is realised in $\NN$ by some $\be$.
    Thus by maximality, $p(b,v) = \subintp{b,1}{\NN}{\tp}(\be)$.

    Finally, for $\phi(v,w) \in F(\Si,n+1)$
    \begin{align*}
        &\phi(v,w) \in \subintp{\nothing}{\MM}{\tp}(a,\al)
        \iff &\MM \model{\Si} \phi(a,\al) \iff \MM \model{\Si(a)} \phi(a,\al)\\
        \iff &\phi(a,v) \in \subintp{a,1}{\MM}{\tp}(\al) = p(a,v)\\
        \iff &\phi(b,v) \in p(b,v) = \subintp{b,1}{\MM}{\tp}(\be)\\
        \iff &\NN \model{\Si(b)} \phi(b,\be) \iff \NN \modelsi \phi(b,\be)\\
        \iff &\phi(w,v) \in \subintp{\nothing}{\NN}{\tp}(b,\be)
    \end{align*}
\end{proof}

\begin{prop}[Back and forth method for showing quantifier elimination]
    \link{om_sat_models_and_quantifier_elimination}
    Let $T$ be a $\Si$-theory.
    If $T$ has quantifier elimination then
    for any two $\om$-saturated $\Si$-models of $T$
    are back and forth equivalent.

    If any two $\Si$-models of $T$
    are back and forth equivalent then $T$ has quantifier 
    elimination.
    \footnote{We could also phrase this as $T$ has 
    quantifier elimination if and only if
    any two $\om$-saturated $\Si$-models of $T$
    are back and forth equivalent, but the saturation requirement becomes 
    redundant in one direction.}
\end{prop}
\begin{proof}
    \begin{forward}
        Let $p$ be a partial isomorphism from $\MM$ to $\NN$.
        By the \linkto{equiv_def_partial_iso}{
            equivalent definition of partial isomorphisms}
        there exists $a \in \MM^n$ and $b \in \NN^n$ such that 
        $p(a) = b$ and 
        \[\subintp{\nothing}{\MM}{\qftp}(a) = 
        \subintp{\nothing}{\NN}{\qftp}(b)\]
        By \linkto{quant_elim_for_types}{
            quantifier elimination for types}
        \[\subintp{\nothing}{\MM}{\tp}(a) = 
        \subintp{\nothing}{\NN}{\tp}(b)\]
        The models are $\om$-saturated, hence  
        \linkto{infty_equivalence_01}{by $\infty$-equivalence}
        for any $\al \in \MM$ there exists $\be \in \NN$ such that 
        \[\subintp{\nothing}{\MM}{\tp}(a,\al) = 
        \subintp{\nothing}{\NN}{\tp}(b,\be)\]
        Taking only the quantifier free elements,
        we obtain 
        \[\subintp{\nothing}{\MM}{\qftp}(a,\al) = 
        \subintp{\nothing}{\NN}{\qftp}(b,\be)\]
        and by the
        \linkto{equiv_def_back_and_forth}{
            equivalent definition of the back and forth property}
        we have that $p$ has the back and forth property.
    \end{forward}

    \begin{backward}
        Let $n \in \N$, $\MM$ and $\NN$ be models of $T$,
        $a \in \MM^n$ and $b \in \NN^n$.
        By \linkto{quant_elim_for_types}{
            quantifier elimination for types} it suffices to show that 
        if \[\subintp{\nothing}{\MM}{\qftp}(a) = 
        \subintp{\nothing}{\NN}{\qftp}(b)\]
        then \[\subintp{\nothing}{\MM}{\tp}(a) = 
        \subintp{\nothing}{\NN}{\tp}(b)\]

        This is satisfied as
        \linkto{back_and_forth_gives_quantifier_elimination_lem}{any 
        two models of $T$ are back and forth equivalent}.
    \end{backward}
\end{proof}

\begin{cor}[Back and forth condition for completeness]
    \link{back_and_forth_implies_completeness}
    Let $T$ be a $\Si$-theory.
    If any two models of $T$ are back and forth equivalent then 
    $T$ is complete.
\end{cor}
\begin{proof}
    If any two models are back and forth equivalent 
    \linkto{back_and_forth_implies_elem_equiv}{then any two non-empty models 
        are elementarily equivalent} (the non-empty is redundant information).
    Hence \linkto{equiv_def_completeness_0}{$T$ is complete}.
\end{proof}

We end this section with a nice example of all of this in action.
\begin{eg}[Infinite infinite equivalence classes]
    \link{infinite_infinite_classes}
    \[\Si_E := (\nothing,\nothing,n_f,\set{E},m_r)\]
    where $m_E = 2$ and $n_f$ is the empty function, 
    defines the signature of binary relations.
    We write for variables $x$ and $y$, 
    we write $x \sim y$ as notation for $E(x,y)$
    The theory of equivalence relations $\ER$ 
    is set set containing the following formulas:
    \begin{align*}
        &\text{Reflexivity - } \forall x, x \sim x\\
        &\text{Symmetry - } \forall x \forall y, x \sim y \to y \sim x\\
        &\text{Transitivity - } 
        \forall x \forall y \forall z, (x \sim y \AND y \sim z) \to x \sim z
    \end{align*}
    For $n \in \N_{>1}$ define 
    \begin{align*}
        &\phi_n := \bigexists{i = 1}{n} x_i, \bigand{i < j}{} x_i \nsim x_j\\
        &\psi_n := \forall x, \bigexists{i = 1}{n} x_i, 
            \bigand{i = 1}{n} \brkt{x \sim x_i} \AND 
            \bigand{i < j}{} \brkt{x_i \ne x_j}
    \end{align*}
    Show that the theory $T = \ER \cup {\phi_n, \psi_n}_{1 < i}$ has 
    quantifier elimination and is complete.
    (You may wonder if it is indeed a theory
    and what nasty induction must be done to 
    show that its formulas can be constructed.)
\end{eg}
\begin{proof}
    We first define the projection into the quotient:
    if $\MM \model{\Si_E} T$ and $a \in {\MM}$ then 
    \[\pi_\MM(a) := \set{b \in {\MM} \st \MM \model{\Si_E} a \sim b}\]
    If $A \subs \MM$ we write $\pi_\MM(A)$ to be the image
    \[\set{\pi_\MM(a) \st \exists a \in A}\]
    Note that the quotient is $\pi_\MM(\MM)$.

    Let $\MM, \NN$ be $\Si_E$-models of $T$
    and let $p$ be a partial isomorphism from $\MM$ to $\NN$.
    By \linkto{om_sat_models_and_quantifier_elimination}{the back and forth
        condition for quantifier elimination} and 
        \linkto{back_and_forth_implies_completeness}{the 
        back and forth condition for completeness}
    it suffices to show that $p$ has the back and forth property.
    
    We only show `forth'.
    Let $\al \in \MM$.
    Suppose $\pi_\MM(\al) \cap \dom p$ is empty.
    We can show that $\pi_\NN(\NN)$ is infinite whilst 
    $\pi_\NN(\codom p)$ is finite, 
    hence there exists $\be \in \NN$ such that 
    $\pi(\be) \in \pi_\NN(\NN) \setminus \pi_\NN(\codom p)$ is non-empty.
    Then define $q : \dom p \cup \set{\al} \to \codom p \cup \set{\be}$
    to agree with $p$ on its domain and send $\al$ to $\be$.
    Note that the domain and codomain of $q$ are substructures
    as the language only contains a relation symbol 
    (thus all subsets are substructures).
    We show that $q$ is an isomorphism.
    It is clearly bijective, and to be an embedding it just needs to preserve
    interpretation of the relation.
    Let $a,b \in \dom q$, if 
    both are in $\dom p$ then as $p$ is a partial isomorphism
    \[a \modintp{\MM}{\sim} b \iff p(a) \modintp{\NN}{\sim} p(b) \iff 
    q(a) \nnintp{\sim} q(b)\]
    Otherwise WLOG $a = \al$.
    If $b = \al$ then it is clear.
    If $b \in \dom p$ then by assumption $b \notin \pi_\MM(\al) = \pi(a)$
    hence $\NOT a \mmintp{\sim} b$.
    By construction 
    \[q(a) = q(\al) = \be \implies 
    \pi_\NN(q(a)) \notin \pi_\NN(\codom p) \quad \text{ and } \quad
    q(b) = p(b) \in \codom p\]
    hence $\NOT q(a) \nnintp{\sim} q(b)$.
    Thus $q$ is a local isomorphism extending $p$.

    Suppose $\pi_\MM(\al) \cap \dom p$ is non-empty,
    i.e. there exists $a \in \dom p$ such that $\al \mmintp{\sim} a$
    We can show that 
    $\pi_\NN(p(a))$ is infinite and $\codom p$ is finite
    hence there exists $\be \in \pi_\NN(p(a)) \setminus \codom p$.
    Then define $q : \dom p \cup \set{\al} \to \codom p \cup \set{\be}$
    to agree with $p$ on its domain and send $\al$ to $\be$.
    Again $p$ is clearly a bijection on substructures, 
    and we show that the relation is preserved.
    Let $b,c \in \dom q$. 
    If $b,c \in \dom p$ then it is clear as $p$ is an isomorphism,
    it is also clear if $b,c = \al$.
    Otherwise WLOG $c = \al$ and $b \in \dom p$.
    Then $c = \al \mmintp{\sim} a$ and
    by construction of $\be$ 
    \[q(c) = q(\al) = \be \nnintp{\sim} p(a)\]
    Noting $a \mmintp{\sim} b$ if and only if $p(a) \nnintp{\sim} p(b)$
    as $p$ is a partial isomorphism
    thus $c \mmintp{\sim} a \mmintp{\sim} b$ 
    if and only if $q(c) \nnintp{\sim} p(a) \nnintp{\sim} p(b) = q(b)$.
    Hence $q$ is a local isomorphism extending $p$.
    Thus $p$ has the `forth' property (and similarly the `back' property).
\end{proof} 

\begin{prop}[Countable back and forth equivalent structures are isomorphic]
    %? Where to put this?
    Let $\MM$ and $\NN$ be countably infinite $\Si$-structures.
    If $\MM$ and $\NN$ are back and forth equivalent then 
    $\MM$ and $\NN$ are isomorphic. 
\end{prop}
\begin{proof}
    Write $\MM = \set{a_i}_{i \in \N}$ and $\NN = \set{b_i}_{i \in \N}$.
    Inductively define partial isomorphisms $p_n$ for $n \in \N$:
    \begin{itemize}
        \item Take $p_0$ to be the empty function.
        \item If $n + 1$ is odd then ensure $a_{n/2}$ 
            is in the domain:
            by the `forth' property of $p$ there exists $p_{n+1}$ 
            extending $p_n$ such that $a_{n/2} \in \dom(p_{n+1})$.
        \item If $n + 1$ is even then ensure $b_{(n+1)/2}$ is in the codomain:
            by the `back' property of $p$ there exists $p_{n+1}$ 
            extending $p_n$ such that $b_{(n-1)/2} \in \codom(p_{n+1})$.
    \end{itemize}
    We claim that $p$, the union of the partial isomorphisms 
    $p_n$ for each $n \in \N$, is an isomorphism.
    Note that it is well-defined and has image $\NN$ as the $p_i$ are nested and
    for any $a_i \in \MM$ and $b_i \in \NN$, 
    $a_i \in \dom(p_{2i+1})$ and $b_i \in \dom(p_{2i+2})$.
    It is injective: if $a_i, a_j \in \MM$ and $p(a_i) = p(a_j)$ then 
    $p_{2i+2}(a_i) = p_{2i+2}(a_j)$ and so $a_i = a_j$ as $p_{2i+2}$ is a 
    partial isomorphism.
    One can show that it is an $\Si$-embedding.
\end{proof}

\subsection{Model completeness}%where to put this?
\begin{dfn}[Model Completeness]
    We say a $\Si$-theory $T$ is model complete when given two $\Si$-models
    of $T$ and a $\Si$-embedding $\io : \MM \to \NN$,
    the embedding is elementary.
\end{dfn}
\begin{rmk}
    \link{quantifier_elimination_implies_model_completeness}
    Any $\Si$-theory $T$ with quantifier elimination is model complete.
    If $\phi$ is a $\Si$-formula and $a \in ({\MM})^S$.
    Then given two $\Si$-models
    of $T$ and a $\Si$-embedding $\io : \MM \to \NN$ we can take
    $\psi$ a quantifier free formula such that 
    $T \modelsi \forall v, \phi \IFF \psi$.
    Since \linkto{emb_preserve_sat_of_quan_free}{embeddings preserve
    satisfaction of quantifier free formulas}
    \[
        \MM \modelsi \phi(a) \iff \MM \modelsi \psi(a)
        \iff \NN \modelsi \psi(\io (a)) \iff \NN \modelsi \phi(\io(a))
    \]
    Thus the extension is elementary.
\end{rmk}

%\section{Morley Rank}
\subsection{Saturation}

\begin{prop}[Elementarily equivalent structures have a common extension 
        \cite{poizat}]
    \link{elem_equiv_struc_has_common_ext}
    Let $\NN_i$ be $\Si$-structures, 
    for each $i \in I$ a non-empty set.
    Suppose that for each $i, j \in I$, 
    $\NN_i$ is elementarily equivalent to $\NN_j$.
    Then there exists $\MM$ a $\Si$-structure and elementary $\Si$-embeddings 
    $\io_i : \NN_i \to \MM$ for each $i \in I$.
\end{prop}
\begin{proof}
    We show that 
    $\bigcup_{i \in I} \eldiag{\Si}{\NN_i}$ is consistent
    as a $\Si(*):=\Si(\bigcup_{i \in I} ({\NN_i}))$ theory.
    This would give us 
    \linkto{elem_ext_equiv_eldiag_model}{elementary $\Si(*)$-embeddings}
    from each $\NN_i$ into some non-empty model $\MM$,
    and we would then be done (by moving this embedding down to $\Si$).

    By \linkto{compactness}{compactness} it suffices to show that 
    each finite subset of $\bigcup_{i \in I} \eldiag{\Si}{\NN_i}$ is consistent.
    The finite subset can be written in the form
    \[\bigcup_{i \in S} \De_i\]
    where $S \subs I$ is a finite subset 
    and each $\De_i \subs \eldiag{\Si}{\NN_i}$ is finite.
    Let $0$ denote the element of $I$ (non-empty). 
    We show that 
    \[\NN_0 \model{\Si(*)} \bigcup_{i \in S} \De_i\]
    By definition of the elementary diagram,
    each $\NN_i$ is a $\Si(\NN_i)$-model of $\De_i$. 
    We can write $\De_i$ as $\Ga_i(a_i)$ 
    where $\Ga_i$ is a set of $\Si$-formulas
    with at most $n$ variables and $a_i \in {\NN_i}^n$.
    Then using elementary equivalence
    \begin{align*}
        &\NN_i \model{\Si(\NN_i)} \bigand{\phi \in \De_i}{} \phi
        \implies \NN_i \model{\Si} \bigand{\phi \in \Ga_i}{} \phi(a_i)\\
        \implies &\NN_i \model{\Si} \bigexists{j = 1}{n} v_j, 
        \bigand{\phi \in \Ga_i}{} \phi(v)
        \implies \NN_0 \model{\Si} \bigexists{j = 1}{n} v_j, 
        \bigand{\phi \in \Ga_i}{} \phi(v)\\
        \implies &\exists b_i \in {\NN_0}^n, \NN_0 \model{\Si} 
        \bigand{\phi \in \Ga_i}{} \phi(b_i)
    \end{align*}
    We proceed to use this fact to define interpretation on $\NN_0$ as 
    a $\Si(*)$ structure.

    If $\NN_0$ were empty then any sentence is satisfied by it, 
    hence by elementary equivalence every $\NN_i$ satisfies all sentences.
    Hence all the structures are empty and we can find a common extension,
    namely the empty set.
    Otherwise let $c \in {\NN_0}$.
    We only need to update interpretation of constant symbols from 
    ${\NN_i}$ for $i \ne 0$.
    Let $d \in {\NN_i}$.
    If $d$ appears in the tuple $a_i$ from above, 
    we interpret it as the corresponding term in the tuple $b_i$ such that 
    $\modintp{\NN_0}{a_i} = b_i$.
    Otherwise intperpret $d$ as $c$.
    Clearly for each $i \in S$ and each $\phi \in \Ga_i$
    \[\NN_0 \model{\Si} \phi(b_i) \implies 
    \NN_0 \model{\Si(*)} \phi(b_i) \implies 
    \NN_0 \model{\Si(*)} \phi(a_i)\]
    and so 
    \[\NN_0 \model{\Si(*)} \bigcup_{i \in S} \Ga_i(a_i)
    \implies \NN_0 \model{\Si(*)} \bigcup_{i \in S} \De_i\]
\end{proof}

\begin{dfn}[More general version of the theory of a structure]
    Let $\MM$ be a $\Si$-structure and let $A \subs {\MM}$.
    Move $\MM$ up to $\Si(A)$ in the obvious way. 
    The theory of $\MM$ over $A$ is defined by
    \[\Theory_\MM(A):= 
    \set{\phi \in \Si(A)\text{-sentences} \st \MM \model{\Si(A)} \phi}\]
    Note that if $\MM$ is non-empty then 
    $\Theory_\MM(A)$ is a consistent and complete $\Si(A)$-theory
    as it is modelled by $\MM$ 
    and any formula is either satisfied by $\MM$ or not.
    Note also that the theory of $\MM$ over 
    ${\MM}$ is the elementary diagram.
\end{dfn}

\begin{lem}[Types over $\Theory_\MM(A)$ are realised in extensions]
    \link{types_of_models_realised_in_extensions}
    Let $\MM$ be a $\Si$-structure, 
    $A \subs \MM$ and $p \in S_n(\Theory_\MM(A))$.
    Then there exists an elementary $\Si(A)$-embedding $\MM \to \NN$ such that 
    $p$ is realised in $\NN$.
\end{lem}
\begin{proof}
    By definition of the Stone space, 
    $p$ is consistent with $\Theory_\MM(A)$ and so there exists $\RR$
    a $\Si(A)$-model of $\Theory_\MM(A)$ that realises $p$.
    Since $\Theory_\MM(A)$ is complete, 
    and both $\MM$ and $\RR$ are $\Si(A)$-models of it,
    $\MM$ and $\RR$ are elementarily equivalent.
    \linkto{elem_equiv_struc_has_common_ext}{Hence they have a common
        elementary extension $\NN$.}
    Since $\RR$ realises $p$ and the extension $\RR \to \NN$ is elementary 
    $\NN$ realises $p$.
    Hence we have an elementary $\Si(A)$-extension of $\MM$ that realises $p$.
\end{proof}

\begin{lem}[Types are preserved downwards in elementary embeddings]
    \link{technical_lemma_types_preserved_downwards}
    Suppose $\io : \MM \to \NN$ is an elementary $\Si$-embedding
    and $A \subs \MM$ is a finite subset.
    Consider $p \in F(\Si(\io(A)),n)$.
    For convinience we write this as $q(A)$, 
    where $q \in F(\Si, n + m)$,
    $m$ is the cardinality of $A$.
    Then 
    \[q(\io(A)) \in S_n(\Theory_{\NN}(\io(A)))
    \implies q(A) \in S_n(\Theory_{\MM}(A))\]
\end{lem}
\begin{proof}
    Suppose $q(\io(A)) \in S_n(\Theory_{\NN}(\io(A)))$.
    Then there exists $\RR$ a $\Si(\io(A))$-model of $\Theory_\NN(\io(A))$ 
    that realises $q(\io(A))$. 
    It suffices to show that $\RR \model{\Si(A)} \Theory_\MM(A)$ and 
    realises $q(A)$.
    Take the interpretation of a constant symbol $a \in A$ as the 
    $\modintp{\RR}{\io(a)}$, 
    the interpretation in $\Si(\io(A))$.
    Then for any $\phi \in \form{\Si}$ (with $m$ free variables) 
    and $a \in \RR^m$
    \[\RR \model{\Si(A)} \phi(A)(a) \iff 
    \RR \model{\Si(\io(A))} \phi(\io(A))(a)\]

    Let $\phi(A) \in \Theory_\MM(A)$ such that $\phi \in \form{\Si}$.
    Then $\MM \model{\Si(A)} \phi(A)$ hence $\MM \model{\Si} \phi(A)$
    and as the embedding is elementary $\NN \model{\Si} \phi(\io(A))$
    and $\NN \model{\Si(\io(A))} \phi(\io(A))$.
    Hence $\phi(\io(A)) \in \Theory_\NN(A)$ and 
    $\RR \model{\Si(\io(A))} \phi(\io(A))$ and 
    $\RR \model{\Si(A)} \phi(A)$.
    Thus $\RR \model{\Si(A)} \Theory_\MM(A)$
    
    There exists $a \in \RR^n$ such that $\RR \model{\Si(\io(A))} q(\io(A))(a)$.
    If $\phi \in q$ then $\RR \model{\Si(\io(A))} \phi(\io(A))(a)$
    and $\RR \model{\Si(A)} q(A)(a)$.
\end{proof}

\begin{dfn}[Embedding Chain, Elementary Chain \cite{marker}]
    Given $(I,\leq)$ a non-empty linear order and a functor $M: I \to \Mod{\Si}$
    that sends each $\al \in I$ to a $\Si$-structure $\MM(\al)$
    and each $\al \leq \be$ in $I$ to a 
    $\Si$-embedding $\lift{\al}{\be} : \MM(\al) \to \MM(\be)$, called a lift.
    Then we call the functor $M$ an embedding chain of $\Si$-structures.

    Furthermore if $M$ only results in elementary $\Si$-embeddings
    then $M$ is an elementary chain.
\end{dfn}

\begin{prop}[$\om$-saturated elementary extensions \cite{poizat}]
    \link{om_sat_elem_ext_of_models}
    Every $\Si$-structure $\MM$ has an 
    \linkto{om_saturation_dfn}{$\om$-saturated} elementary extension.
\end{prop}
\begin{proof}
    Let $\MM$ be a $\Si$-structure.
    We will inductively create an elementary chain of $\Si$-structures
    $\MM = \MM(0) \to \MM(1) \to \cdots$ such that for each $\al \in \N$,
    every finite subset $A \subs \MM(\al)$, 
    every $n \in \N$ and every $p(A) \in S_n(\Theory_{\MM(\al)}(A))$,
    $p(\lift{\al}{\al + 1} A)$ is realised in $\MM(\al + 1)$.
    \linkto{direct_limit_of_chains}{Taking the direct limit} 
    gives us a $\Si$-structure $\NN$ and elementary $\Si$-embeddings 
    $\io_\al : \MM(\al) \to \NN$ for each $\al \in \N$ that commute with 
    the lifts from the chain.
    In particular we will have that $\NN$ is an elementary $\Si$-extension
    of $\MM$.
    For any finite subset $A \subs \NN$, 
    any $n \in \N$ and any $p(A) \in S_n(\Theory_{\NN}(A))$,
    since $A$ is finite there exists $\al \in \N$ such that 
    ${\io_\al}^{-1}(A) \subs \MM(\al)$ bijects with $A$.
    \linkto{technical_lemma_types_preserved_downwards}{As types are preserved
        downwards in elementary embeddings} 
    $p(A) \in S_n(\Theory_{\NN}(A))$ implies 
    $p(\io_\al^{-1} A) \in S_n(\Theory_{\MM(\al)}(\io_\al^{-1} A))$.
    By construction we see that
    $p$ is realised in $\MM(\al + 1)$
    thus $p$ is realised in $\NN$ as $\io_{\al + 1}$ is elementary.
    This $\NN$ is $\om$-saturated.

    Furthermore, 
    if $A \subs \MM$, $n \in \N$ and $p(A) \in S_n(\Theory_{\MM})(A)$,
    then $p(\lift{0}{1} A)$ is realised in $\MM(1)$.
    Hence $p(\io_0 A)$ is realised in $\NN$ as $\io_{1}$ is elementary.

    It suffices to build $\MM(1)$ from $\MM$.
    Let $A \subs \MM$, $n \in \N$ and $p_{A,n} \in S_n(\Theory_\MM (A))$.
    \linkto{types_of_models_realised_in_extensions}{$p_{A,n}$ 
        is realised in some $\Si(A)$-extension $\MM(p_{A,n})$.}
    As $\MM(p_{A,n})$ extends $\MM$, 
    it is naturally a $\Si(\MM)$-model of $\eldiag{\Si}{\MM}$.
    We create such a $\Si(\MM)$-structure for each $A$, $n$ and $p_{A_n}$.
    Consider the set 
    \[\set{\MM(p_{A,n}) \st A \subs \MM, 
    n \in \N, p_{A,n} \in S_n(\Theory_\MM (A))}\]
    Any two structures in the set are elementarily equivalent since 
    they would both model $\eldiag{\Si}{\MM} = \Theory_\MM(\MM)$ 
    which is a complete $\Si(\MM)$-theory.
    Thus \linkto{elem_equiv_struc_has_common_ext}{there 
        exists a common $\Si(\MM)$-extension $\MM(1)$}
    of all the structures in the set. 
    In particular there exists 
    an elementary $\Si(\MM)$-embedding $\lift{0}{1} : \MM \to \MM(1)$
    (for the existance of one intermediate structure take 
    $A = \nothing, n = 0$ and \linkto{make_max}{extend 
    $\Theory_\MM(\nothing)$ to a maximal $\Si$-theory}).

    \begin{cd}
        &&&\MM(1)&&&\\
        \MM(\dots)\ar[urrr, shorten >= 10pt]&\MM(\dots)
        \ar[urr, shorten >= 5pt]&\MM(\dots)\ar[ur]
        &\MM(p_{A,n})\ar[u]
        &\MM(\dots)\ar[ul]&\MM(\dots)
        \ar[ull, shorten >= 5pt]&\MM(\dots)
        \ar[ulll, shorten >= 10pt]\\
        &&&\MM \ar[ulll]\ar[ull]\ar[ul]\ar[u]\ar[ur]\ar[urr]\ar[urrr]&&&
    \end{cd}

    To show that $\MM(1)$ has the desired property let
    $A \subs \MM$, 
    $n \in \N$ and $p \in S_n(\Theory_{\MM}(A))$.
    By design there exists $\MM(p)$ that realises $p$.
    Since $\MM(1)$ is an elementary $\Si(\MM)$-extension of $\MM(p)$
    (we can turn this into a $\Si(A)$-embedding for things to make sense)
    we have that 
    $p$ is realised in $\MM(1)$.
\end{proof}

\begin{lem}[Disjunctive normal form]
    \link{disjunctive_normal_form_0}
    Let $\phi$ be a quantifier free $\Si$-formula with variables indexed
    by $S$.
    Then there exist atomic $\Si$-formulas
    $f_{ij}$ such that
    for any $\Si$-structure $\MM$
    \[ 
        \MM \model{\Si} \bigforall{s \in S}{}v_s, \phi(v) 
        \IFF \bigor{i \in I}{} 
        \bigand{j \in J_{i}}{} f_{ij}(v)
    \]
    An immediate improvement to this is that there exist $\Si$-formulas
    $f_{ij}, g_{ij}$ of the form $s = t$ or $r(t)$ 
    (equality of terms or a relation) such that
    for any $\Si$-structure $\MM$
    \[ 
        \MM \model{\Si} \bigforall{s \in S}{}v_s, \phi(v) 
        \IFF \bigor{i \in I}{} 
        \brkt{\bigand{j \in J_{i0}}{} f_{ij}(v)\AND 
        \bigand{j \in J_{i1}}{} \NOT g_{ij}(v)}
    \]
\end{lem}
\begin{proof}
    We induct on $\phi$:
    \begin{itemize}
        \item If $\phi$ is $\top$ 
            then we take a single `or' $I = \set{0}$ and the empty `and' for 
            $J_{0}$.
        \item If $\phi$ is $t = s$ then
            let $I = J_{0} = \set{0}$ and take $f_{00}$ to be $t = s$.
        \item If $\phi$ is $r(t)$ then 
            let $I = J_{0} = \set{0}$ and take $f_{00}$ to be $r(t)$.
        \item If $\phi$ is $\NOT \psi$ and there exist the relevant things for 
            $\psi$ that satisfy
            \[ 
                \MM \model{\Si} \bigforall{s \in S}{}v_s, \psi(v) 
                \IFF \bigor{i \in I}{} 
                \bigand{j \in J_{i}}{} f_{ij}(v)
            \]
            for all $\Si$-structures $\MM$ and for all $a \in {\MM}^S$.
            We can negate and rearrange this statement to give us what we 
            want for $\phi$.
            The gist of the induction goes as follows:
            \begin{align*}
                &\NOT \bigor{i \in I}{} 
                \bigand{j \in J_{i}}{} f_{ij}(v) \IFF 
                \bigand{i \in I}{} 
                \bigor{j \in J_{i}}{} \NOT f_{ij}(v)\\
                \IFF &\brkt{\bigor{j \in J_0}{} \NOT f_{0j}} \AND
                \brkt{\bigand{0 \ne i \in I}{} 
                \bigor{j \in J_{i}}{} \NOT f_{ij}(v)} \\
                 \IFF 
                &\bigor{k_0 \in J_0}{} 
                \sqbrkt{\NOT f_{0k_0} \AND \bigand{0 \ne i \in I}{} 
                \bigor{j \in J_{i}}{} \NOT f_{ij}(v)}
                \text{ (induction on $J_0$)}\\
                \IFF & \bigor{k_0 \in J_0}{} 
                \sqbrkt{\NOT f_{0k_0} \AND \bigor{k_1 \in J_1}{} 
                \brkt{\NOT f_{1k_1} \AND \bigand{i \ne 0, 1}{} 
                \bigor{j \in J_{i}}{} \NOT f_{ij}(v)}} \\
                \IFF & \bigor{k_0 \in J_0}{} 
                \bigor{k_1 \in J_1}{} 
                \brkt{\NOT f_{0k_0} \AND \NOT f_{1k_1} 
                \AND \bigand{i \ne 0, 1}{} 
                \bigor{j \in J_{i}}{} \NOT f_{ij}(v)} \\
                \IFF & \dots \text{ induction on $I$} 
                \IFF \bigor{k_0 \in J_0}{} \dots \bigor{}{}
                \sqbrkt{\bigand{}{}g_{ij}}
            \end{align*}
        \item If $\phi$ is $\chi_0 \OR \chi_1$ 
            and there exist the relevant things 
            for $\chi_0$ and $\chi_1$, then we can simply take the or of
            the two formulas found and obtain what we want.
    \end{itemize}
\end{proof}

\begin{lem}[Type of an element from the model is isolated]
    \link{type_of_an_element_from_the_model}
    Let $T$ be a complete $\Si$-theory
    and $\MM$ be a non-empty $\Si$-model of $T$. 
    \linkto{elems_of_stone_space_are_types_of_elements}{Any 
        type over $\MM$ is of the form}\footnote{Here we have $\nothing$ in 
            $\subintp{\nothing,1}{\NN}{\tp}(a)$ as we are working in the theory 
            $\eldiag{\Si}{\MM}$, 
            which is in the language $\Si_E(\MM)$,
            thus we don't need to add symbols from $\MM$.}
        $\subintp{\nothing,1}{\NN}{\tp}(a)$, 
    where $\NN$ is an
    \linkto{om_sat_elem_ext_of_models}{$\om$-saturated elementary extension} 
        of $\MM$ and $a \in \NN$.
    Then $a$ is in the image of $\MM$ if and only if 
    $\tp(a)$ is an \linkto{dfn_isolated_point}{isolated point} 
    in $S_1(\MM)$.
    Thus $\MM$ bijects with the isolated points of $S_1(\MM)$
    by taking $c \mapsto \set{\tp(io(c))}$ and the 
    \linkto{dfn_derived_set}{derived set} is
    \[  
        S_1(\MM)' = S_1(\MM) \setminus \bigcup_{c \in \MM} [x = c]
        = \bigcap_{c \in \MM} [x \ne c] 
    \]
\end{lem}
\begin{proof}
    \begin{forward}
        Suppose $a$ is in the image of $\MM$.
        Then we have $c \in \MM$ such that $a = \nnintp{c}$.
        Hence $x = c$ is an element of $\tp(a)$ and so $\tp(a) \in [x = c]$.
        If $q \in [x = c]$ then as $q$ is realised by some $b \in \NN$, 
        $b = \nnintp{c} = a$, 
        which implies $q = \tp(a)$.
        Thus the open set $[x = c]$ is a singleton and the point is isolated.
        Thus $c \mapsto \tp(\io(c))$ has image a subset of the isolated points.
        It is injective since for $c, d \in \MM$, 
        if $\tp(\io(c)) = \tp(\io(d))$ then 
        $x = c \in \tp(\io(c)) = \tp(\io(d))$
        and so $\io(d) = \io(c)$ which implies $c = d$ as $\io$ is injective.
    \end{forward}
    \begin{backward}
        On the other hand, suppose there is an isolated point $p$ in $S_1(\MM)$,
        then $\set{p}$ is open. 
        This open set is equal to $\bigcup_{\phi \in I} [\phi]$
        as a finite intersection of clopen sets is clopen,
        hence $\psi \in I$ such that $p \in [\psi]$, 
        whence $[\psi] = \set{p}$.
        As $\psi \in p$ (which is consistent with $T$) we have that 
        $\exists x, \psi(x)$ is realised in some model of $T$ 
        (any elementary extension of $\MM$ is a model of $T$). 
        Since $T$ is complete
        this implies $T \model{\Si_E} \exists x, \psi(x)$
        In particular $\MM$ is a model of $T$ and so there exists 
        $a \in \MM$ such that $\MM \model{\Si_E} \psi(a)$.
        Thus $\NN \model{\Si_E} \psi(\io(a))$ so $\psi \in \tp(\io(a))$
        and $\tp(\io(a)) \in [\psi]$.
        Since there is only one element in this set $p = \tp(\io(a))$.
        Thus $c \mapsto \set{\tp(\io(c))}$ is surjective.
    \end{backward}
\end{proof}

\begin{eg}[Infinite infinite equivalence classes revisited]
    \link{infinite_infinite_classes1}
    Consider again the theory $T$ of 
    \linkto{infinite_infinite_classes}{infinite infinite equivalence classes},
    and $\MM$ a non-empty $\Si_E$-model of $T$. 
    We wish to classify all the types over $\MM$, i.e. the elements of 
    $S_1(\MM) := S_1(\eldiag{\Si}{\MM})$.
    We know that
    \linkto{elems_of_stone_space_are_types_of_elements}{
        any type over $\MM$ is of the form}
        $\subintp{\nothing,1}{\NN}{\tp}(a)$, 
    where $\NN$ is the 
    \linkto{om_sat_elem_ext_of_models}{$\om$-saturated elementary extension} 
        of $\MM$ and $a \in \NN$.
    We show that there are three cases:
    \begin{enumerate}
        \item The element $a$ is in the image of $\MM$ if and only if 
            $\tp(a)$ is an \linkto{dfn_isolated_point}{isolated point} 
            in $S_1(\MM)$.
            Thus $\MM$ bijects with the isolated points of $S_1(\MM)$
            by taking $c \mapsto \set{\tp(io(c))}$.
        \item The element $a$ is not in the image but it equivalent under the 
            relation to something
            in the image if and only if $\tp(a)$ is isolated in $S_1(\MM)'$,
            the \linkto{dfn_derived_set}{derived set} of the Stone space.
        \item Otherwise it is an isolated point 
            in $S_1(\MM)''$ the second derived set. 
            There is exactly one such type.
    \end{enumerate}
    \[  
        S_1(\MM) = \set{\tp(a) \st a \in \MM} \sqcup 
        \set{\tp(a) \st a \notin \MM \AND \exists c \in \MM, a \sim c} \sqcup
        \set{\tp(a) \st \forall c \in \MM, a \nsim c}
    \]
    Hence $\abs{\MM} \leq \abs{\MM} + \abs{\MM} + 1 \leq \abs{S_1(\MM)}$.
    By $(1.)$, $\MM$ bijects with a subset of $S_1(\MM)$ and so 
    $\abs{\MM} = \abs{S_1(\MM)}$.
\end{eg}
\begin{proof}
    (1.) This is covered in \linkto{type_of_an_element_from_the_model}{the 
        lemma}.
    
    (2.)
    Suppose $a$ is not in the image of $\MM$ but is equivalent to 
    an element of $c \in \MM$.
    Then 
    \[\tp(a) \in \bigcap_{d \in \MM} [x \ne d] \cap [x \sim c]\]
    Notice that we cannot make the intersection a formula because 
    this would then be quantifying over $\NN$ instead of $\MM$,
    also this intersection is infinite as $\MM$ is infinite.
    To show that this is the only element of this set, 
    suppose $\tp(b)$ also satisfies 
    \[(\forall d \in \MM, x \ne d \in \tp(b)) \text{ and } x \sim c \in \tp(b)\]
    then let $\phi \in QFF(\Si(\MM),1)$.
    We case on $\phi$ to show that $\phi \in \tp(a) \iff \phi \in \tp(b)$.
    By \linkto{infinite_infinite_classes}{quantifier elimination of $T$} and the 
    \linkto{quant_elim_for_types}{types version of quantifier elimination}
    it suffices to show that $\qftp(a) = \qftp(b)$.
    (for each step we only show $\phi \in \qftp(a) \implies \phi \in \qftp(b)$
    the other direction is the same)
    \begin{itemize}
        \item If $\phi = \top$ then it is trivial.
        \item If $\phi$ is $r = s$ then as the only $\Si_E(\MM)$-terms 
            that exist are constants $d_i \in \MM$ or variables $x_i$, 
            we case on $r$ and $s$.
            If both are constants $d_0, d_1$ then since $\phi$ is in $\qftp(a)$,
            $\nnintp{d_0} = \nnintp{d_1}$ and so $\phi$ is in $\qftp(b)$.
            If one is a constant $d$ and the other a variable $x$ then 
            $\nnintp{d} = \nnintp{x}(a) = a$ which is a contradiction as 
            $x \ne d \in \qftp(a)$.
            If both are variables then as $\phi$ has at most one variable it 
            is $x = x$ which is clearly in $\tp(b)$.
        \item If $\phi$ is $r \sim s$ then we again case on what the terms are.
            If they are constants $d_0,d_1$ 
            we have $\nnintp{d_0} \nnintp{\sim} \nnintp{d_1}$ 
            and so $\phi$ is in $\tp(b)$.
            If one is a constant $d$ and the other a variable $x$ then by 
            the fact that $x \ne d \in \qftp(a) \cap \qftp(b)$ 
            \[\nnintp{d} \nnintp{\sim} \nnintp{x}(a) = a 
            \nnintp{\sim} c \nnintp{\sim} b\]
            Hence $d \sim x \in \qftp(b)$.
            If both are variables then as above it is $x \sim x$ which is in 
            $\tp(b)$.
        \item If $\phi$ is $\NEG \psi$ then $\NN \model{\Si_E(\MM)} \phi(a)$
            implies $\NN \nodel{\Si_E(\MM)} \psi(a)$ which by the induction 
            hypothesis tells us $\NN \nodel{\Si_E(\MM)} \psi(b)$, 
            whence $\phi(b) \in \qftp(b)$.
        \item If $\phi$ is $\psi \OR \chi$ then 
            $\NN \model{\Si_E(\MM)} \psi(a)$ or 
                $\NN \model{\Si_E(\MM)} \chi(a)$
            and by the induction hypothesis 
            $\NN \model{\Si_E(\MM)} \psi(b)$ or 
            $\NN \model{\Si_E(\MM)} \chi(b)$
            hence $\phi \in \qftp(b)$.
    \end{itemize}
    Thus (though $a$ is not unique) $\tp(a)$ is the unique type satisfying 
    this characterisation.
    Hence the intersection of the derived set and 
    $[x \sim c]$ is $\set{\tp(a)}$,
    as any $p$ in the derived set is (as we proved above) \emph{not} due 
    to some element from $\MM$.
    Thus these are isolated in the derived set.

    We prove a lemma:
    if $\psi$ is a formula of one variable
    then $[\psi] \cap S_1(\MM)' = [x \sim d] \cap S_1(\MM)'$ for some 
    $d \in \MM$ or 
    $S_1(\MM)' \subs [\psi]$.
    As $T$ has \linkto{infinite_infinite_classes}{quantifier elimination} 
    we can assume $\psi$ is quantifier free.
    The \linkto{disjunctive_normal_form_0}{disjunctive normal form} of $\psi$
    gives us 
    \[[\psi] = \bigcup_{i \in I} \bigcap_{j \in J_{i}} [f_{ij}(v)]\]
    There exists an $i$ such that $\bigcap_{j \in J_{i}} [f_{ij}(v)]$ 
    contains $p$ hence 
    \[  
        [\psi] \cap S_1(\MM)' =  \bigcap_{j \in J_{i}} [f_{ij}(v)] 
            \cap \bigcap_{c \in \MM} [x \ne c]
    \]
    If $f_{ij}$ is of the form $r = s$ then if $r$ and $s$ are both constant
    symbols or both variable symbols it is true and we can remove it,
    otherwise it is $x = c$ for some $c \in \MM$, which implies that $p$
    contains $x = c$ and $x \ne c$, a contradiction.
    If $f_{ij}$ is of the form $r \ne s$ then if $r$ and $s$ are both constant
    symbols or both variable symbols it is either true and we can remove it,
    or false and we have a contradiction;
    otherwise it is $x \ne c$ for some $c \in \MM$, 
    which is already given in the second intersection and we can remove it.
    If $f_{ij}$ is $r \nsim s$ then $r \ne s$ is also in $p$ so by the previous
    point we can remove it.
    Thus the only remaining case is when the $f_{ij}$ are $r \sim s$,
    again we can remove the cases where $r$ and $s$ are both constants or 
    variables and we are left with $x \sim c$ for some $c \in \MM$.
    If there is more than one of these we can remove them as they either 
    contradict one another (via transitivity) or they are redundant information
    and can be removed.
    Hence we are left with either $[\psi] \cap S_1(\MM)' = [x \sim d] 
    \cap S_1(\MM)'$ for some $d \in \MM$ or 
    $[\psi] \cap S_1(\MM)' = S_1(\MM)'$.
    In the second case $S_1(\MM)' \subs [\psi]$.

    Using the lemma we show that for any isolated point 
    $\tp(a)$ in the derived set $S_1(\MM)'$,
    there exists $d \in \MM$ such that $a \sim d$.
    If it is isolated
    \[\set{\tp(a)} = \bigcup_{\phi \in I} [\phi] \cap S_1(\MM)'\] 
    which implies 
    there is some $\psi$ such that 
    \[  
        \set{\tp(a)} = [\psi] \cap S_1(\MM)'
    \]
    By the lemma, either $\set{\tp(a)} = [x \sim d] 
    \cap \bigcap_{c \in \MM} [x \ne c]$ for some $d \in \MM$ or 
    $\set{\tp(a)} = \bigcap_{c \in \MM} [x \ne c]$.
    In the second case we have a contradiction as for two elements $d, e$in 
    distinct equivalence classes of $\MM$
    $\bigcap_{d \in \MM} [x \ne d] \cap [x \sim c]$ and 
    $\bigcap_{e \in \MM} [x \ne e] \cap [x \sim c]$ contain distinct types
    and are both subsets of $\bigcap_{c \in \MM} [x \ne c]$, 
    which is a singleton set.
    Thus there exists $d \in \MM$ such that $x \sim d \in \tp(a)$.

    (3.) There is the final case where $a$ is 
    not equivalent to anything from $\MM$ (therefore not in the image). 
    By the above two cases 
    we have that $\tp(a)$ is not isolated in $S_1(\MM)$ or $S_1(\MM)'$.
    We show that there is exactly one such type, 
    which implies $S_1(\MM)''$ consists of one isolated point.
    $\MM$ has infinitely many equivalence classes 
    and so the set $\Ga := \set{ x \nsim c \st c \in \MM}$ is finitely 
    realised in $\MM$, 
    \linkto{finite_realisation_and_embeddings}{hence consistent}
    with the elementary diagram of $\MM$, 
    which implies it can be 
    \linkto{extend_to_maximal_type_zorn}{extended to a maximal type}.
    Hence there exists a type $\tp(a)$ such that 
    $a$ is not equivalent to any thing from $\MM$.
    There is only one:
    Suppose $\tp(a)$ and $\tp(b)$ both satisfy the above.
    Let $\phi \in \tp(a)$.
    By the lemma $[\psi] \cap S_1(\MM)' = [x \sim d] \cap S_1(\MM)'$ 
    for some $d \in \MM$
    or $S_1(\MM)' \subs [\psi]$.
    The first case is false since it implies 
    $\tp(a) \in [\psi] \cap S_1(\MM)' \subs [x \sim d]$ 
    but $x \nsim d$ is in $\tp(a)$ by assumption.
    Thus $\tp(b) \in S_1(\MM)' \subs [\psi]$ and so $\psi \in \tp(b)$.
    The other direction is the same and so $\tp(a) = \tp(b)$ and 
    this point is unique.
\end{proof}

\subsection{Morley Rank}
This subsection follows Marker's \cite{marker} material again.

\begin{dfn}[Definable]
    \link{definable_set}
    Let $\MM$ be a $\Si$-structure and $A \subs \MM$.
    We say $X \subs \MM^0$ is $\Si(A)$-definable if $X$ is non-empty.
    In the non-degenerate case:
    $X \subs {\MM}^n$ is $\Si(A)$-definable over $\MM$ if there exists a 
    $\Si(A)$-formula in $n$ free variables such that 
    \[X = \set{a \in {\MM}^n \st \MM \model{\Si} \phi(a)}\]

    For a $\Si(A)$-formula $\phi$ with $n$ free variables we use $\phi(\MM)$ 
    to denote $\set{a \in \MM^n \st \MM \model{\Si(\MM)} \phi(a)}$,
    the set defined by $\phi$.
    
    Note that if $\phi$ is a sentence then 
    $\phi(\MM) := \set{\nothing}$ when $\phi$ is 
    satisfied by $\MM$ and it is empty otherwise.
    For now we only concern outselves with the case where $A = \MM$.
\end{dfn}

\begin{dfn}[Morley rank with respect to a structure (not necessarily saturated)]
    Let $\MM$ be a $\Si$-structure, 
    let $\al$ be an ordinal.
    By transfinite induction of $\al$
    we define what it means for any $\Si(\MM)$-formula $\phi$ to satisfy 
    $\al \leq \MR{\MM}{\phi}$:
    \begin{itemize}
        \item If $\phi(\MM)$ is non-empty then $0$ is in $R$.
        \item If $\al$ is a non-zero limit ordinal and for each $\be < \al$
            and $\phi$, 
            $\be \leq \MR{\MM}{\phi}$ then $\al \leq \MR{\MM}{\phi}$.
        \item Suppose $\al + 1$ is a successor ordinal. 
            If there exists for each 
            $n \in \N$ a $\Si(\MM)$-formula $\psi_n$ satisfying 
            $\al \leq \MR{\MM}{\psi_n}$ such that the $\psi_n(\MM)$
            are pairwise disjoint subsets of $\phi(\MM)$, 
            then $\al \leq \MR{\MM}{\phi}$.
            \footnote{The relations $\al \leq \MR{\MM}{\psi_n}$ 
            is already defined by induction.
            Each $\psi_n$ has the same number of variables as $\phi$
            since we require $\psi_n(\MM) \subs \phi(\MM)$.}
    \end{itemize}

    We then define $\MR{\MM}{\phi} \in \set{-\infty,\infty} \cup \ord$, 
    the Morley rank of $\phi$:
    \begin{itemize}
        \item If for each ordinal $\al$, 
            $\al \nleq \MR{\MM}{\phi}$ then we take 
            $\MR{\MM}{\phi} := -\infty$.
        \item If for all ordinals $\al$, 
            $\al \leq \MR{\MM}{\phi}$ then we take $\MR{\MM}{\phi} := \infty$.
        \item If there exists an ordinal $\al$ 
            such that $\al \leq \MR{\MM}{\phi}$ but 
            $\al + 1 \nleq \MR{\MM}{\phi}$, 
            then $\MR{\MM}{\phi} := \al$.
    \end{itemize}
\end{dfn}
\begin{proof}
    To show that the second part of the definition is well-defined
    we need a lemma:
    By induction on $\al$ we show that if $\be \leq \al$,
    then for any $\Si(\MM)$-formula $\phi$,
    $\al \leq \MR{\MM}{\phi}$ implies $\be \leq \MR{\MM}{\phi}$.
    \begin{itemize}
        \item If $\al = 0$ then it is vaccuously true.
        \item Suppose $\al$ is a non-zero limit ordinal.
            If $\al \leq \MR{\MM}{\phi}$, $\be < \al$ and 
            $\phi \in \form{\Si(\MM)}$ then by definition of 
            $\al \leq \MR{\MM}{\phi}$ for non-zero limit ordinals
            we have $\be \leq \MR{\MM}{\phi}$.
        \item Suppose $\al$ satisfies the condition. 
            We show that $\al + 1$ satisfies the condition as well.
            Let $\phi$ be a $\Si(\MM)$-formula.
            Suppose $\al + 1 \leq \MR{\MM}{\phi}$.
            If $\be \leq \al + 1$ then either $\be = \al$ or $\be < \al$.
            It suffices to show the first case - $\al \leq \MR{\MM}{\phi}$ - 
            as by the induction hypothesis this implies that for any $\be < \al$
            we also have $\be \leq \MR{\MM}{\phi}$, 
            which covers the second case.
            By definition we have $\Si(\MM)$-formulas $\psi_n$ for each natural
            $n$ with $\psi_n(\MM)$ disjoint subset of $\phi(\MM)$ and 
            $\al \leq \MR{\MM}{\psi_n}$.
            This is not quite what we want so we 
            show $\al \leq \MR{\MM}{\phi}$ by induction on $\al$ once again.
            \begin{itemize}
                \item If $\al = 0$ then since $\psi_0(\MM) \subs \phi(\MM)$
                    and $\psi_0(\MM) \ne \nothing$ by the fact that 
                    $0 \leq \MR{\MM}{\psi_n}$, 
                    we have that $\phi(\MM) \ne \nothing$ and so 
                    $0 \leq \MR{\MM}{\phi}$.
                \item If $\al$ is a non-zero ordinal and all $\be < \al$
                    satisfy $\be \leq \MR{\MM}{\phi}$ then clearly 
                    $\al \leq \MR{\MM}{\phi}$.
                \item If $\al = \be + 1$ 
                    then by the original induction hypothesis, 
                    we have for each $n \in \N$, 
                    \[\al \leq \MR{\MM}{\psi_n} \implies 
                    \be \leq \MR{\MM}{\psi_n}\]
                    Thus $\al \leq \MR{\MM}{\phi}$.
            \end{itemize}
    \end{itemize}

    To show then that Morley rank is well-defined: 
    if the first and second cases don't hold,
    we can find the minimal element $\al \in \ord$ such that 
    $\al \nleq \MR{\MM}{\phi}$
    as \linkto{basic_facts_ordinals}{$\ord$ is well-ordered}.
    This is not a limit ordinal, since if $\be$ less than it satisfy
    $\be \leq \MR{\MM}{\phi}$ and it it a limit ordinal then 
    $\al \leq \MR{\MM}{\phi}$, a contradiction.
    Thus it must be a successor and so we can find its predecessor.
    This is unique: suppose for a contradiction that $\al < \be$ satisfy 
    \[\al \leq \MR{\MM}{\phi}, \al + 1 \nleq \MR{\MM}{\phi}, 
    \be \leq \MR{\MM}{\phi}, \be + 1 \nleq \MR{\MM}{\phi}\] 
    then \linkto{less_than_and_succ_of_ord}{$\al + 1 < \be$ or $\al + 1 = \be$},
    so using the lemma above for the first case,
    we have in either case $\al + 1 \leq \MR{\MM}{\phi}$,
    which is a contradiction.
    Thus $\al = \be$ and this ordinal is unique.
\end{proof}
\begin{rmk}
    We tacitely order $\set{-\infty,\infty} \cup \ord$ by the usual ordering on 
    $\ord$ together with $-\infty \leq$ everything $\leq \infty$.
    We can then compare the Morley rank of two formulas by $\leq$.
\end{rmk}

\begin{lem}[Morley rank for elementary extensions between saturated structures]
    \link{Morley_rank_elem_ext_lem}
    If $\MM$ and $\NN$ are both $\om$ saturated $\Si$-structures and 
    $\MM \to \NN$ is an elementary $\Si$-extension, 
    then for any $\Si(\MM)$-formula $\phi$,
    \[\MR{\MM}{\phi} = \MR{\NN}{\phi}\]
    (where in the second case $\phi$ is considered to be a $\Si(\NN)$-formula).
\end{lem}
\begin{proof}
    Again it suffices to show by induction that for each $\al \in \ord$, 
    given $\MM$ and $\NN$ and $\phi$ as above, we have
    \[\al \leq \MR{\MM}{\phi} \quad \iff \quad \al \leq \MR{\NN}{\phi}\]

    The $0$ case: \begin{align*}
        & 0 \leq \MR{\MM}{\phi}\\
        \iff & \MM \model{\Si(\MM)} \exists v, \phi(v)\\
        \iff & \NN \model{\Si(\MM)} \exists v, \phi(v)\\
        \iff & 0 \leq \MR{\NN}{\phi}
    \end{align*} 

    The non-zero limit ordinal case is standard.

    Successor:
    With the induction hypothesis for $\al$, 
    suppose $\al + 1 \leq \MR{\MM}{\phi}$
    then there exist $\psi_n$ for each $n \in \N$ such that 
    $\al \leq \MR{\MM}{\psi_n}$, 
    $\psi_n(\MM) \subs \phi(\MM)$ and these sets are pairwise disjoint.
    By the induction hypothesis we have for each $n$ that 
    $\al \leq \MR{\NN}{\psi_n}$. 
    Since the embedding is elementary
    \begin{align*}
        &\psi_n(\MM) \subs \phi(\MM)\\
        \implies & \MM \model{\Si} \forall v, \psi_n \to \phi\\
        \implies & \NN \model{\Si} \forall v, \psi_n \to \phi\\
        \implies & \psi_n(\NN) \subs \phi(\NN)
    \end{align*}
    Similarly 
    \begin{align*}
        &\psi_n(\MM) \cap \psi_{l}(\MM) = \nothing\\
        \implies & \MM \model{\Si} \forall v, \NOT (\psi_n \AND \phi_{n+1})\\
        \implies & \NN \model{\Si} \forall v, \NOT (\psi_n \AND \phi_{n+1})\\
        \implies & \psi_n(\NN) \cap \psi_{l}(\NN) = \nothing
    \end{align*}
    Hence $\al + 1 \leq \MR{\NN}{\phi}$.

    For the other direction assume the induction hypothesis and suppose 
    $\al + 1 \leq \MR{\NN}{\phi}$.
    First replace $\phi$ with a $\Si$-formula $\phi(v,a)$ at some 
    $a \in \MM^{i}$.
    For each $n \in \N$
    there exist $\Si$-formulas $\psi_n$ and $c_n \in \N^{i_n}$
    such that $\al \leq \MR{\NN}{\psi_n(v,c_n)}$ and $\psi_n(\NN, c_n)$
    are disjoint subset of $\phi(\NN,a)$.
    Since the extension is elementary $\mmintp{\tp}(a) = \nnintp{\tp}(a)$
    and by \linkto{infty_equivalence_01}{$\infty$-equivalence} for each $n$
    there exist $d_0 \in \MM^{i_0},\dots,d_n \in \MM^{i_n}$ such that 
    \[
        \nnintp{\tp}(a,c_0,\dots,c_n) = \mmintp{\tp}(a,d_0,\dots,d_n)
        = \nnintp{\tp}(a, d_0,\dots,d_n)
    \]
    Similarly to before, for each $n \in \N$
    \begin{align*}
        &\psi_n(\NN,c_n) \subs \phi(\NN,a)\\
        \iff & \NN \model{\Si} \forall v, \psi_n(v,c_n) \to \phi(v,a)\\
        \iff & \forall v, \psi_n(v,c_n) \to \phi(v,a) \in 
        \nnintp{\tp}(a,c_n) = \mmintp{\tp}(a,d_n)\\
        \iff & \MM \model{\Si} \forall v, \psi_n(v,d_n) \to \phi(v,a)\\
        \iff & \psi_n(\MM,d_n) \subs \phi(\MM,a)
    \end{align*}
    They are disjoint:
    \begin{align*}
        &\psi_n(\NN,c_n) \cap \psi_l(\NN,c_l) = \nothing\\
        \iff & \NN \nodel{\Si} \exists v, \psi_n(v,c_n) \AND \psi_l(v,c_l)\\
        \iff & \exists v, \psi_n(v,w_n) \AND \psi_l(v,w_l) \notin 
        \tp(a,c_n,c_l) = 
        \tp(b,d_n,d_l)\\
        \iff & \MM \nodel{\Si} \exists v, \psi_n(v,d_n) \AND \psi_l(v,d_l)\\
        \iff & \psi_n(\MM,d_n) \cap \psi_l(\MM,d_l) = \nothing
    \end{align*}
    Thus the induction is complete.
\end{proof}

\begin{cor}[Saturated elementary extensions give equal Morley rank]
    \link{sat_elem_ext_give_eq_Morley}
    Let $\AA,\MM,\NN$ be $\Si$-structures such that $\MM$ and $\NN$ are 
    $\om$-saturated elementary $\Si$-extensions of $\AA$. 
    Then for any $\Si(\AA)$-formula $\phi$, 
    \[\MR{\MM}{\phi} = \MR{\NN}{\phi}\]
\end{cor}
\begin{proof}
    Let $\LL$ be the \linkto{amalgamation}{amalgamation} of $\MM$ and $\NN$,
    and let $\Om$ be an 
    \linkto{om_sat_elem_ext_of_models}{$\om$-saturated elementary extension} 
    of $\LL$.
    The \linkto{category_of_structures}{composition} 
    of elementary extensions is elementary and so 
    $\MM \to \Om$ and $\NN \to \Om$ is elementary.
    Hence by the \linkto{Morley_rank_elem_ext_lem}{previous lemma}
    we have that 
    \[\MR{\MM}{\phi} = \MR{\Om}{\phi} = \MR{\NN}{\phi}\]
\end{proof}

\begin{prop}[Subset implies inequality of Morley rank]%?
    \link{implication_subset_inequality}
    Let $\phi$ and $\psi$ be two $\Si(\MM)$-formulas
    and $\NN$ be an elementary extension of $\MM$.
    If $\psi(\NN) \subs \phi(\NN)$ then
    \[\MR{\NN}{\psi} \leq \MR{\NN}{\phi}\]

    Hence if $X$ is defined by two formulas then the Morley 
    rank of the two formulas in some $\NN$ is the same.
    \[
        \phi(\NN) = X = \psi(\NN) 
        \quad \implies \quad 
        \MR{\NN}{\phi} = \MR{\NN}{\psi}
    \]
\end{prop}
\begin{proof}
    %Note that $\psi(\NN) \subs \phi(\NN)$ (see important footnote\footnote{
    %    $\psi(\NN) \subs \phi(\NN)$
    %    is not true when the sets of free variables differ.
    %    For example if $\psi$ is a sentence then it defines a point but if 
    %    $\phi$ were $x = x$ then it would define all of $\NN$.)
    %    }).
    Induct on $\al \in \set{-\infty} \cup \ord$ to show that 
    \[\al \leq \MR{\NN}{\psi} \implies \al \leq \MR{\NN}{\phi}\]

    The $-\infty$ case is clear.
    The $0$ case follows from noting that if 
    $0 \leq \MR{\NN}{\psi}$ then $\nothing \ne \psi(\NN) \subs \phi(\NN)$.
    The non-zero limit case is the same as usual.
    The successor case follows from noting that the
    $\psi_i$ given for each $i \in \N$ define subsets of $\psi(\NN)$ and 
    therefore subsets of $\phi(\NN)$.

    Now that we have that it is true for all ordinals the $\infty$ case also
    follows.

    The `hence' uses anti-symmetry of the ordering on 
    $\set{-\infty,\infty} \cup \ord$.
\end{proof}

\begin{dfn}[Morley rank in saturated extensions]
    Let $\MM$ be a $\Si$-structure and $\phi$ a $\Si(\MM)$-formula.
    Then $\MR{}{\phi} := \MR{\NN}{\phi}$ for $\NN$ any $\om$-saturated extension 
    of $\MM$.
    This is unique with respect to the choice of $\NN$ since 
    \linkto{sat_elem_ext_give_eq_Morley}{saturated elementary extensions 
    give equal Morley rank}.
    Suppose $X = \phi(\MM)$ is the subset of $\MM^n$ defined by $\phi$.
    Then we write $\MR{}{X} = \MR{}{\phi}$ to be the Morley rank of the
    $\Si(\MM)$-definable set.
    It is unique with respect to the choice of $\phi$ since
    \linkto{implication_subset_inequality}{equal 
        $\Si(\MM)$-definable sets implies equal Morley rank}.
\end{dfn}
Note that for an $\om$-saturated structure $\MM$ and $\phi$ a $\Si(\MM)$-formula
\[\MR{}{\phi} = \MR{\MM}{\phi}\]
since $\MM$ is an $\om$-saturated extension of itself.

\begin{prop}[Basic facts about Morley rank of a definable set]
    \link{basic_facts_morley_rank_of_dfnbl_set}
    Let $X$ and $Y$ be $\Si(\MM)$-definable sets in $\MM^n$ a $\Si$-structure.
    \begin{enumerate}
        \item If $X \subs Y$ then $\MR{}{X} \leq \MR{}{Y}$.
        \item $\MR{}{X} = 0$ if and only if $X$ is finite and non-empty.
        \item $\MR{}{X \cup Y} = \max \set{\MR{}{X}, \MR{}{Y}}$.
    \end{enumerate}
\end{prop}
\begin{proof}
    $(1.)$ Follows from the \linkto{implication_subset_inequality}{
        previous lemma}.

    $(2.)$ First note that $0 \leq \MR{}{X}$ if and only if the 
    formula defining $X$ is satisfied if and only if $X$ is non-empty.
    It remains to show that $1 \nleq \MR{}{X}$ if and only if 
    $X$ is finite.
    \begin{forward}
        If $X$ were infinite, we can let $\phi$ be the formula defining $X$ and 
        define formulas $\phi \AND x = a$ for each $a \in X$,
        hence having infinitely many satisfiable formulas corresponding to
        disjoint subsets of $X$.
        Hence $1 \leq \MR{}{X}$, a contradiction.
    \end{forward}

    \begin{backward}
        If $1 \leq \MR{}{X}$ then we would have infinitely many disjoint
        subsets of $X$, which implies $X$ is infinite, a contradiction.
    \end{backward}

    $(3.)$ It suffices to show that for all $X$ and $Y$ 
    $\Si(\MM)$-definable subsets of $\MM$
    if $\MR{}{X} \leq \MR{}{Y}$ then $\MR{}{Y} = \MR{}{X \cup Y}$.
    By the first part we have that $\MR{}{Y} \leq \MR{}{X \cup Y}$,
    thus we need only to show the other inequality by induction on 
    $\al \in \set{-\infty} \cup \ord$:
    \[\al \leq \MR{}{X \cup Y} \implies \al \leq \MR{}{Y}\]
    For $\al = - \infty$ then it is clear.
    For $\al = 0$ suppose $0 \leq \MR{}{X \cup Y}$. 
    Then $X \cup Y$ is non empty.
    Suppose for a contradiction $Y$ is empty then $X$ must be non-empty
    and we have that $\MR{}{Y} < \MR{}{X}$, a contradiction.
    Hence $0 \leq \MR{}{Y}$.

    The non-zero limit ordinal case is trivial as usual.
    Suppose it is true for $\al$ and suppose $\al + 1 \leq \MR{}{X \cup Y}$.
    Then there exist disjoint subsets $S_i \subs X \cup Y$ such that 
    $\al \leq \MR{}{S_i}$.
    Consider the subsets $S_i \cap X \subs X$ and $S_i \cap Y \subs Y$.
    If infinitely many $S_i \cap Y$ satisfy $\al \leq \MR{}{S_i \cap Y}$
    then we are done.
    On the other hand if only finitely many do so then by induction
    \[\al \leq \MR{}{S_i} = \max \set{\MR{}{S_i \cap X}, \MR{}{S_i \cap Y}}\]
    Hence infinitely many satisfy $\al \leq \MR{}{S_i \cap X}$.
    Thus $\al + 1 \leq \MR{}{X} \leq \MR{}{Y}$ and the induction is complete.
    
    As usual the $\infty$ case follows from the result given by the induction.
\end{proof}
Naturally parts $2.$ and $3.$ also have analogues in the formula version of 
Morley rank.

\subsection{Morley degree}
\begin{nttn}
    Let $A$ be a set.
    Then $A^{<\om}$ is the set 
    \[A^{<\om} := \bigsqcup_{n \in \N} A^n\]
    To be explicit we mean a disjoint union of the sets of maps 
    (a.k.a $n$-tuples) $n \to A$.

    For $a = (a_1,\dots,a_n) \in A^n \subs A^{<\om}$ 
    and $b \in A$ we write $a;b$ to mean
    $a$ with $b$ appended to it: 
    \[(a_1,\dots,a_n,b) \in A^{n+1} \subs A^{<\om}\]
\end{nttn}

The following lemma has been adapted to be suitable without pre-requisites.
For the general version see Proof Wiki \cite{wiki1}.
\begin{lem}[K\"{o}nig's Tree]
    \link{konigs_tree}
    Partially order $2^{<\om}$ by $t \leq s$ 
    if and only if $s$ restricts to the map
    $t$ on the domain domain of $t$.
    Let $T \subs 2^{<\om}$ be such that 
    \begin{itemize}
        \item For any $s \in T$ and $t \in 2^{<\om}$ if $t \leq s$ 
            then $t \in T$. 
            We call this property $T$ being a `binary tree'.
        \item $T$ is countably infinite.
    \end{itemize}
    Then there exists an `infinite branch' $b \in 2^\N$ 
    (i.e. $b : \N \to 2$) such that 
    for each $n \in \N$ the restriction $\res{b}{n}$ is in $T$.
\end{lem}
\begin{proof}
    By induction we construct for each $n \in \N$, $t_n \in 2^n$ such that 
    \begin{itemize}
        \item $t_n \in T$
        \item for each $i < n$, $t_{i + 1} \in D(t_i)$
        \item for each $i \leq n$, $D(t_i)$ is infinite
    \end{itemize}
    We take $t_0$ to be the empty map. 
    $T$ is non-empty (it is countably infinite), 
    the empty map is in $2^\nothing$, 
    and anything is equal to the empty map upon restriction to $\nothing$.
    Thus applying the binary tree property we have $t_0 \in T$.
    For the same reason $D(t_0) = T$ is infinite.

    Assuming by induction that we have defined $t_0, \dots, t_n$
    we assume for a contradiction that both $D(t_n;0)$ and $D(t_n;1)$ are
    finite.
    Then clearly 
    \[D(t_n) \subs \set{t_n;0\,,\, t_n;1} \cup D(t_n;0) \cup D(t_n;1)\]
    and so $D(t_n)$ is finite, a contradiction.
    Hence we pick one that has infinitely many descendents to be $t_{n+1}$.
    Taking any descendent and restricting its domain to $2^{n+1}$ we see that 
    $t_{n+1} \in T$; restricting $t_{n+1}$ to $2^{n}$ we obtain $t_{n}$.
    Hence the induction is complete.

    We take $b = \bigcup_{n \in \N} t_n$ to be the infinite branch.
    By construction each finite restriction lies in $T$.
\end{proof}

\begin{dfn}[$\al$-minimality]
    \link{al_minimality_dfn}
    Let $\M$ be an $\om$-saturated $\Si$-structure and $\al$ an ordinal.
    We say a $\Si(\M)$-formula $\phi$ is $\al$-minimal if 
    \[\MR{}{\phi} = \al \text{ and there does not exist } 
    \psi \in \form{\Si(\M)} \text{ such that } 
    \MR{}{\phi \AND \psi} = \MR{}{\phi \AND \NOT \psi} = \al\]
\end{dfn}

The following lemma gives us the existence of Morley degree,
a way to specify how many disjoint subsets of a definable set we can produce.
This also gives us a way to split up any formula into $\al$-minimal formulas
which are easier to work with.
\begin{lem}[Morley degree decomposition \cite{marker}]
    \link{morley_degree_lem}
    Let $\M$ be an $\om$-saturated $\Si$-structure.
    Let $\phi$ be a $\Si(\M)$-formula.
    Suppose that $\MR{}{\phi} = \al \in \ord$.
    Then there exist $\al$-minimal formulas $\psi_1,\dots,\psi_d$ 
    defining disjoint subsets of $\phi(\M)$
    each with $\MR{}{X_i} = \al$.
    Futhermore if $X_1,\dots,X_n$ are 
    pairwise disjoint subsets of $\phi(\M)$ 
    each with $\MR{}{X_i} = \al$ then $n \le d$.
\end{lem}
\begin{proof}
    We build a binary tree $T \subs 2^{<\om}$ and apply 
    \linkto{konigs_tree}{K\"{o}nig's tree lemma}.
    Inductively we define for each $n \in \N$ a finite `binary tree' 
    $T_n \subs 2^{\leq n}$ and $\Phi_n : T_n \to \form{\Si(\M)}$
    such that \begin{itemize}
        \item for each $i < n$, $T_{i+1} \cap 2^{\leq i} = T_i$
        \item for each $i < n$, $\res{\Phi_{i+1}}{2^{\leq i}} = \Phi_i$
        \item for each $t \in T_n$, $\MR{}{\Phi_n(t)} = \al$
    \end{itemize}
    We take $T_0 = \set{\nothing}$ and have $\Phi_n : \nothing \mapsto \phi$.

    By induction, supposing that we have defined $T_i$ and $\Phi_i$ for 
    each $i \leq n$, we define $T_{n+1}$ as
    \begin{align*}
        T_{n+1} = T_n &\cup 
        \set{t;0 \st t \in T_n \text{ is } \al-\text{minimal}}\\
        &\cup \set{t;1 \st t \in T_n \text{ is } \al-\text{minimal}}
    \end{align*}
    Then we find (using the axiom of choice) for each $t;0 \in T_{n+1}$
    a $\Si(\M)$-formula $\chi_{n+1}$ such that 
    \[
        \MR{}{\Phi_n(t) \AND \chi_{n+1}} = 
        \MR{}{\Phi_n(t) \AND \NOT \chi_{n+1}} = \al
    \]
    We define $\Phi_{n+1}$ such that it agrees upon restriction with $\Phi_n$
    and for each $t \in T_n \cap 2^{n}$,
    \[t;0 \mapsto \Phi_{n+1}(t) \AND \chi_{n+1}
    \text{ and } t;1 \mapsto \Phi_{n+1}(t) \AND \NOT \chi_{n+1}\]
    By assumption
    \[\MR{}{\Phi_{n+1}(t;0)} = \MR{}{\Phi_{n}(t) \AND \chi_{n+1}} = \al\]
    Hence $\Phi_{n+1}$ satisfies the conditions we required.
    It then makes sense to define 
    \[  
        T := \bigcup_{n \in \N} T_n \quad \text{ and } \quad
        \Phi := \bigcup_{n \in \N} \Phi_n : T \to \form{\Si(\M)} 
    \]
    $T$ is a binary tree:
    if $s \in T \cap 2^{n}$ and we have $t < s \in T$ then $0 < n$
    and so it must be that $s$ is created due to the induction step.
    Thus we have $\res{s}{s^{n-1}} \in T$ and by induction 
    $t = \res{s}{s^m} \in T$, where $m$ is power of $2$ which $t$ lies in.

    Suppose for a contradiction $T$ is infinite.
    Then 
    \[\abs{T} \leq \abs{2^{<\om}} = \abs{\bigsqcup_{n \in \N} 2^n} 
    \linkto{infinite_union_of_finite}{= \aleph_0}\]
    Hence $T$ is countably infinite and we can apply 
    \linkto{konigs_tree}{K\"{o}nig's tree lemma},
    giving us an `infinite brach' $b : \N \to 2$ such that for each $n \in \N$
    we have $\res{b}{n} \in T_n \subs T$.
    From this we obtain for each $n \in \N$ a formula $\Phi_n(\res{b}{n})$
    and define a $\Si(\M)$-formula
    \[\psi_n := \Phi_n(\res{b}{n}) \AND \NOT \Phi_{n+1}(\res{b}{n+1})\]
    These will define infinitely many disjoint subsets of $\phi(\M)$
    with Morley rank $\al$.
    Without loss of generality $\res{b}{n+1} = \res{b}{n};0$, 
    hence for each $n$:
    \begin{align*}
        \psi_n(\M) &
        = \brkt{\Phi_n(\res{b}{n}) \AND \NOT \Phi_{n+1}(\res{b}{n+1})}(\M)\\
        &= \brkt{\Phi_n(\res{b}{n}) \AND 
        (\NOT \Phi_{n}(\res{b}{n}) \OR \NOT \chi_{n+1})}(\M)\\
        &= \brkt{\Phi_n(\res{b}{n}) \AND \NOT \chi_{n+1}}(\M)
    \end{align*}
    \linkto{basic_facts_morley_rank_of_dfnbl_set}{Computing} 
    the Morley rank is then straightforward:
    \[\MR{}{\psi_n} = \MR{}{\Phi_n(\res{b}{n}) \AND \NOT \chi_{n+1}} = \al\]
    which follows from the definition of $\chi_{n+1}$.
    Furthermore to see that any two distinct $\psi_n$ define disjoint 
    subsets of $\phi$, let $n < m$, then 
    \[\Phi_{n}(\res{b}{n})(\M) \subs \Phi_{n+1}(\res{b}{n+1})(\M) \subs 
    \dots \Phi_{m}(\res{b}{m})(\M)\]
    Hence 
    $\NOT \Phi_{m}(\res{b}{m})(\M) \subs \NOT \Phi_n(\res{b}{n})(\M)$ and
    \[
        \psi_m(\M) \subs (\NOT \Phi_{m+1}(\res{b}{m+1}))(\M)
        \subs (\NOT \Phi_{n}(\res{b}{n}))(\M)
    \]
    the last of which does not intersect with $\psi_n(\M)$.
    Hence they are disjoint.
    They are subsets of $\phi(\M)$ since they are subset of 
    $\Phi_n(\res{b}{n})$ which by induction are subset of $\phi(\M)$.
    Hence $\phi$ has Morley rank $\al + 1$ which is a contradiction.
    Thus $T$ is finite.

    We consider the set of terminal nodes 
    $N = \set{t \in T \st \forall s \in T, t \nless s}$.
    Since $T$ is finite $N$ is also finite;
    we take $d$ to be the cardinality of $N$.
    Consider $\Phi(N)$, the image of $N$.
    There are three points to note about this: 
    \begin{itemize}
        \item For each $t \in N$, $\Phi(t)$ is $\al$-minimal.
        \item The members of $\Phi(N)$ define pairwise disjoint subsets of 
            $\phi(\M)$.
        \item Thus $\phi(\M)$ is equal to a disjoint union
            $\brkt{\bigor{t \in N}{} \Phi(t)}(\M) = 
            \bigsqcup_{t \in \N} \Phi(t)(\M)$
    \end{itemize}
    We complete the proof using these points first, 
    then come back and prove them.

    Let $\Si(\M)$-formulas 
    $\om_1,\dots,\om_n$ have Morley rank $\al$ and 
    define pairwise disjoint subsets of $\phi(\M)$.
    We need to show that $n \leq d$.
    Suppose for a contradiction $d < n$.
    Suppose for another contradiction 
    that for all $i \leq n$ there exists $t \in N$ such that 
    $\al \leq \MR{}{\Phi(t) \AND \om_i}$ 
    (hence $\al = \MR{}{\Phi(t) \AND \om_i}$).
    Then there exist $1 \leq i < j \leq n$ such that 
    \[\MR{}{\om_i \AND \Phi(t)} = \MR{}{\om_j \AND \Phi(t)} = \al\]
    Then 
    \[\al = \MR{}{\om_j \AND \Phi(t)} 
    \leq \MR{}{(\NOT \om_i) \AND \Phi(t)} \leq \MR{}{\Phi(t)} = \al\]
    This implies $\Phi(t) \notin N$, a contradiction.
    Then there must be some $i \leq n$ such that for all $t \in N$
    $\MR{}{\Phi(t) \AND \om_i} < \al$.
    However 
    \[
        \om_i(\M) = (\om_i \AND \phi)(\M) = 
        \brkt{\om_i \AND \bigor{t \in N}{} \Phi(t)}(\M)
        = \brkt{\bigor{t \in N}{} \om_i \AND \Phi(t)}(\M)
    \]
    Hence $\MR{}{\om_i} \linkto{basic_facts_morley_rank_of_dfnbl_set}{=} 
    \max_{t \in \N} \MR{}{\om_i \AND \Phi(t)} < \al$, a contradiction.
    Hence $n \leq d$.

    We show the three facts from before.
    The first is clear from the definition of $\N$.
    Let $s$ and $t$ be distinct elements of $N$. 
    Let $n \in \N$ be the maximal natural such that $\res{s}{n} = \res{t}{n}$.
    If $s \in 2^n$ then $s < t$ which is a contradiction.
    Thus $s, t \notin 2^n$.
    As $n$ is maximal, $\res{s}{n+1} \ne \res{t}{n+1}$ and so 
    $\Phi(\res{s}{n+1})(\M)$ is disjoint with $\Phi(\res{t}{n+1})(\M)$.
    $\Phi(s)(\M) \subs \Phi(\res{s}{n+1})(\M)$ and similarly with $t$ hence 
    $\Phi(s)(\M)$ and $\Phi(t)(\M)$ are disjoint.

    Clearly $\brkt{\bigor{t \in N}{} \Phi(t)}(\M) \subs \phi(\M)$.
    For the other direction note that for any $t \in T$ such that 
    $\exists s \in T, t < s$, 
    \[\Phi(t)(\M) = \Phi(t;0)(\M) \sqcup \Phi(t;1)(\M)\]
    Hence each element of $T$ (in $2^n$) 
    is either in $N$ or defines a set that is 
    a disjoint union due two elements in $2^{n+1} \cap T$.%
    As $T$ is finite we can find the maximal $n \in \N$ such that 
    $\exists t \in 2^n \cap T, a \in \Phi(t)(\M)$.
    If $t \in N$ then we are done.
    Suppose $t \notin N$ then $a \in \Phi(t;0)(\M) \sqcup \Phi(t;1)(\M)$
    and so $\exists s \in 2^{n+1} \cap T, a \in \Phi(s)(\M)$,
    contradicting maximality of $t$.
    Hence $a \in \bigsqcup_{t \in N} \Phi(t)(\M)$ and 
    \[\phi(\M) = \brkt{\bigor{t \in N}{} \Phi(t)}(\M) = 
    \bigsqcup_{t \in \N} \Phi(t)(\M)\]
\end{proof}

\begin{dfn}[Morley degree]
    Let $\phi$ be a $\Si(\MM)$-formula for a $\Si$-structure $\MM$.
    We natually interpret $\phi$ as a $\Si(\M)$-formula
    where $\M$ is an \linkto{om_sat_elem_ext_of_models}{
        $\om$-saturated elementary $\Si$-extension} of $\MM$.
    We define the Morley degree of $\phi$, which we write as $\mdeg (\phi)$:
    If $\MR{}{\phi} \in \ord$ then
    $\MR{}{\phi}$ is the minimal $d \in \N$ such that 
    if $X_1,\dots,X_n$ are 
    pairwise disjoint subsets of $\phi(\M)$ 
    each with $\MR{}{X_i} = \MR{}{\phi}$ then $n \le d$.
    The existence of such a $d$ comes from the \link{morley_degree_lem}{lemma}.
    If $\MR{}{\phi} = - \infty$ then $\mdeg(\phi) := -\infty$.
    If $\MR{}{\phi} = \infty$ then $\mdeg(\phi) := \infty$.

    We also write $\mdeg(\phi(\MM)):= \mdeg{(\phi)}$ for the Morley degree of 
    a $\Si(\MM)$-definable subset of $\MM$.
\end{dfn}
Note that Morley degree for formulas with ordinal Morley rank will always 
be natural numbers greater than $0$, 
since the formula itself defines a (trivially pairwise disjoint) subset.

\subsection{Morley rank and degree for types}
\begin{dfn}[Morley rank and degree for types]
    \link{morley_rank_for_types_dfn}
    Let $\MM$ be a $\Si$-structure with $A \subs \MM$.
    Let $p \in S_n(\Theory_\MM(A))$.
    Then we define the Morley rank of $p$ to be 
    \[\MR{}{p} := \min \set{\MR{}{\phi} \st \phi \in p 
    \text{ with exactly $n$ free variables}}\]
    which is well-defined since ordinals are 
    \linkto{basic_facts_ordinals}{well-ordered}.

    Note that for any $p$ (by choice) there exists $\phi \in p$ with exactly $n$ 
    free variables such that $\MR{}{\phi} = \MR{}{p}$,
    we call $\phi$ the rank representative.

    We also define Morley degree of $p$:
    If $\MR{}{p} = -\infty$ then $\mdeg(p) = -\infty$
    and if $\MR{}{p} = \infty$ then $\mdeg(p) = \infty$.
    Otherwise we take 
    \[
        \mdeg(p) := 
        \min \set{\mdeg(\phi) \st \phi \in p 
        \text{ and } \MR{}{\phi} = \MR{}{p} \text{ and } 
        \phi \text{ has exactly $n$ free variables}}
    \]
\end{dfn}

\begin{lem}[$\al$-minimal rank representative of a type]
    \link{smallest_rank_rep}
    Suppose $\M$ is an $\om$-saturated strongly minimal $\Si$-structure,
    $A \subs \M$ and $p \in S_n(\Theory_\M(A))$.
    Let $\al = \MR{}{p}$.
    Then there exists an $\al$-minimal rank representative $\phi$ of $p$.
\end{lem}
\begin{proof}
    Suppose no such $\phi$ exists, then starting with any rank representative 
    $\phi_0$ and using maximality of $p$
    there exists $\psi_0 \in p$ such that 
    \[\MR{}{\phi_0 \AND \psi_0} = \MR{}{\phi_0 \AND \NOT \psi_0} = \al\]
    We define $\phi_1$ as $\phi_0 \AND \psi_0$ 
    which by assumption is not $\al$-minimal and proceed by induction 
    to create $\phi_i$ and $\psi_i$.
    Now $\phi_i \AND \NOT \psi_i$ are infinitely 
    many formulas with Morley rank $\al$ defining disjoint subsets of $\phi_0$,
    so $\al + 1 \leq \MR{}{\phi_0}$, a contradiction.
\end{proof}

The following lemma tells us we can also go the other way.
Given a formula we can find a type with equal Morley rank 
containing the formula.
This uses the Morley degree decomposition of a formula.
\begin{lem}[Formulas are represented by types \cite{tent}]
    \link{formulas_rep_by_types}
    Let $A \subs \M$, for $\M$ an $\om$-saturated $\Si$-structure.
    If $\phi$ is a $\Si(A)$-formula with $\MR{}{\phi} \in \ord$, 
    then there exists a type $p \in S_n(\Theory_\M(A))$ such that 
    $\MR{}{\phi} = \MR{}{p}$ and $\phi \in p$.

    Hence 
    \[\MR{}{\phi} = \max \set{\MR{}{q} \st \phi \in q \in S_n(\Theory_\M(A))}\]
\end{lem}
\begin{proof}
    We first show that this holds when $\phi$ is $\al$-minimal,
    where $\al := \MR{}{\phi}$.
    We take 
    \[
        p := \set{\psi \in \linkto{dfn_types_on_theories}{F(\Si(A),n)} 
        \st \MR{}{\phi \AND \NOT \psi}} < \al
    \]
    as our $n$-type.
    To show that it is finitely consistent with $\Theory_\M(A)$
    let $\De$ be a finite subset of $p$,
    it suffices to show that the conjunction 
    $\Psi := \bigand{\psi \in \De}{} \psi$
    has Morley rank at least $\al$.
    \[
        \MR{}{\phi \AND \NOT \Psi} = 
        \MR{}{\bigor{\psi \in \De}{} \phi \AND \NOT \psi} 
        \linkto{basic_facts_morley_rank_of_dfnbl_set}{=} 
        \max_{\psi \in \De} (\MR{}{\phi \AND \NOT \psi} ) < \al
    \]
    Hence the conjunction $\Phi$ is in $p$.
    Supposing $\MR{}{\Psi} < \al$ implies
    $\MR{}{\phi \AND \Psi} < \al$ and 
    \[
        \MR{}{\phi} \linkto{basic_facts_morley_rank_of_dfnbl_set}{=} 
        \max (\MR{}{\phi \AND \Psi}, \MR{}{\phi \AND \NOT \Psi})
        < \al
    \]
    which is a contradiction.
    Hence $\al \leq \MR{}{\Psi}$ and so $p$ is finitely consistent.
    To show that it is maximal, suppose $\psi \notin p$. Then 
    \[\al \leq \MR{}{\phi \AND \NOT \psi} \leq \MR{}{\phi} = \al\]
    and by $\al$-minimality of $\phi$ we cannot have 
    $\al \leq \MR{}{\phi \AND \psi}$.
    Hence $\NOT \psi \in p$.

    Clearly $\phi \in p$, hence $\MR{}{p} \leq \MR{}{\phi}$.
    Furthermore if $\psi \in p$ then $\al \nleq \MR{}{\phi \AND \NOT \psi}$
    hence 
    \[
        \MR{}{\phi} \linkto{basic_facts_morley_rank_of_dfnbl_set}{=} 
        \max (\MR{}{\phi \AND \psi}, \MR{}{\phi \AND \NOT \psi})
        = \MR{}{\phi \AND \psi} \leq \MR{}{\psi}
    \]
    hence $\MR{}{p} = \MR{}{\psi}$.

    Now we can do the case without the $\al$-minimal hypothesis.
    Write $\al \in \ord$ for the Morley rank and 
    $d \in \N$ the Morley degree of $\phi$.
    \linkto{morley_degree_lem}{There exist $\al$-minimal formulas}
    $\psi_1,\dots,\psi_d$ that partition $\phi(\M)$,
    each with $\MR{}{\psi_i} = \al$.
    By the first part we have for $\psi_1$ a maximal type 
    $p \in S_n(\Theory_\M(A))$ such that $\psi_1 \in p$ and 
    $\MR{}{p} = \MR{}{\psi_1} = \al$.
    In fact $\phi \in p$ as well since $\psi_1(\M) \subs \phi(\M)$
    so we have found the type that we required.

    The `hence' follows immediately.
\end{proof}

%? Where to put?
\subsection{Constructable sets}
\begin{dfn}[Constructable]
    Let $\MM$ be a $\Si$-structure.
    The set of constructable subsets of $\MM^n$ are defined by:
    \begin{itemize}
        \item[$\vert$] If $\phi$ is an atomic $\Si(\MM)$-formula with up to 
            $n$ free variables then $\phi(\MM) \subs K^n$ is constructable.
        \item[$\vert$] If $X \subs \MM^n$ is constructable 
            then $\MM^n \setminus X$ is constructable. 
        \item[$\vert$] If $X,Y \subs \MM^n$ are constructable then $X \cup Y$
            is constructable. 
    \end{itemize}
    Thus these are `finite boolean combinations' of sets 
    $\Si(\MM)$-definable by atomic 
    formulas, i.e. sets $\Si(\MM)$-definable by a quantifier free formula.
\end{dfn}

\begin{prop}[Constructable is definable]
    \link{definable_is_constructable_01}
    Let $T$ be a $\Si$-theory with quantifier elimination and 
    $\MM$ be a $\Si$-model of $T$.
    Then subsets of $\MM^n$ are constructable if and only if they are 
    $\Si(\MM)$-definable by a quantifier free formula if and only if they 
    are $\Si(\MM)$-definable.
\end{prop}
\begin{proof}
    We only show the first equivalence as the second is trivial.
    \begin{forward}
        Suppose $X \subs \MM^n$ is constructable.
        Then we induct on $X$:
        \begin{itemize}
            \item If $X$ is defined by an atomic formula then 
                there is nothing to show.
            \item If $X$ is $\MM^n \setminus Y$ and $Y$ is constructable and 
                by induction $Y$ is $\Si(\MM)$-definable 
                then we take the negation of its defining formula.
            \item If $X$ is $Y \cup Z$, both constructable and 
                by induction $Y,Z$ are both $\Si(\MM)$-definable 
                then we take the `or' of their defining formulas.
        \end{itemize}
    \end{forward}
    \begin{backward}
        Suppose $X$ is $\Si(\MM)$-definable by the quantifier free formula 
        $\phi(v,b)$ with $b \in \MM^m$ for some $m$.
        We induct on what $\phi$ is:
        \begin{itemize}
            \item If $\phi$ is atomic then $X$ is constructable.
            \item If $\phi$ is $\NOT \psi$ and $\psi(\MM,b)$
                is constructable then $X = \MM^n \setminus \psi(\MM,b)$
                hence it is constructable.
            \item If $\phi$ is $\psi \OR \chi$ then $X$ is the union of 
                the sets $\psi(\MM,b)$ and $\chi(\MM,b)$ which are 
                constructable by induction. 
                The union of constructable is constructable.
        \end{itemize}
    \end{backward}
\end{proof}

%\begin{eg}[Infinite infinite equivalence classes re-revisited]
%    \link{infinite_infinite_classes2}
%    Consider again again the theory $T$ of 
%    \linkto{infinite_infinite_classes}{infinite infinite equivalence classes},
%    and $\MM$ a non-empty $\Si_E$-model of $T$. 
%    As an exercise we try to compute the Morley rank of some 
%    $\Si_E(\MM)$-formulas.
%
%    Say we are interested in computing the Morley rank of $x=x$.
%    We shown that $2 \leq \MR{\MM}{x = x}$ by first taking 
%    $\psi_i$ to be $x \sim a_i$ such that the $a_i$ are mutually dissimilar
%    (for each $i \ne k$, $a_i \nsim a_k$).
%    These $\psi_i$ should have Morley rank $1$ or greater: take 
%    $\psi_{ij}$ to be $x = b_{ij}$ where the $b_{ij}$ are all similar to $a$
%    but mutually distinct ($b_{ij} \sim a_i$ and $b_{ij} \ne b_{ik}$).
%    These formulas define singleton sets, hence they have Morley rank $0$.
%    Hence the $1 \leq \MR{\MM}{\psi_i}$ and $2 \leq \MR{\MM}{x = x}$.
%
%    I conjecture that $3 \nleq \MR{\MM}{x = x}$.
%    This requires some knowledge of the disjoint 
%    non-empty $\Si(\MM)$-definable subsets of $\MM$.
%    By quantifier elimination%?
%    $\Si(\MM)$-definable 
%    is constructable so we first consider sets defined by atomic
%    formulas (the `generators').%?%?
%
%    If the atomic formula is true regardless of what we evaluate at:
%    $x = x, c = c, x \sim x, c \sim c, c \sim d$ 
%    ($c$ and $d$ represent constants that are interpreted distinct but similar 
%    in $\MM$,
%    $c$ and $e$ represent constants that are dissimilar.)
%    then the set they define is $\MM$.
%    If the atomic formula is false regardless of what we pick:
%    $c = d, c \sim e$
%    then they define the empty set.
%    The remaining cases are $x = c$ which corresponds to a singleton set
%    and $x \sim c$ which gives an infinite equivalence class.
%
%    Now suppose $3 \leq \MR{\MM}{x = x}$.
%    We then have for each $i \in \N$, $\psi_i$ such that 
%    $\psi_i(\MM)$ are disjoint subsets of $\MM$
%    and $2 \leq \MR{\MM}{\psi_i}$.
%    Each $\psi_i(\MM)$ then has disjoint subsets $\chi_{ij}(\MM)$ such that 
%    $1 \leq \MR{\MM}{\chi_{ij}}$.
%    By disjunctive normal form%?
%    we have that each $\chi_{ij}(\MM)$ is a finite union of intersections of 
%    atomically defined sets and their negations.
%    We can ignore all instances of `true' and `false' atomic formulas 
%    since 
%    \[X \cap \nothing = \nothing, X \cup \nothing = X, X \cap \MM = X,
%    X \cup \MM = \MM\]
%    and $\chi_{ij}$ must be non-empty and a proper subset of $\MM$.
%    Hence it is a finite union of intersections of 
%    singleton sets, equivalence classes and their complements. 
%    \[\chi_{ij}(\MM) = \bigcup_{\al \in I} \bigcap_{\be \in J_i} \om_{\al\be}\]
%    Since $1 \leq \MR{\MM}{\chi_{ij}}$, $\chi_{ij}(\MM)$ 
%    \linkto{basic_facts_morley_rank_of_dfnbl_set}{must be infinite},
%    thus there exists $\al \in I$ such that 
%    $\bigcap_{\be \in J_i} \om_{\al\be}$ is infinite,
%    ruling out the presence of singleton sets (but not their complements).
%    What does this all mean??%?
%\end{eg}
%
%%?Where to put?
%\begin{lem}[Morley rank for formulas at elements of the same type]
%    Let $\MM$ be a $\om$-saturated $\Si$-structure.
%    Let $\phi(v,w)$ be a $\Si$-formula.
%    Let $a, b \in \MM^m$ satisfy 
%    \[\subintp{\nothing}{\MM}{\tp}(a) = \subintp{\nothing}{\MM}{\tp}(b)\]
%    then $\MR{\MM}{\phi(v,a)} = \MR{\MM}{\phi(v,b)}$, 
%    where $\phi(v,a),\phi(v,b)$ 
%    are seen as $\Si(\MM)$-formulas with $n$ variables.
%\end{lem}
%\begin{proof}
%    It suffices to show by induction on $\al$ that for all $\al \in \ord$,
%    for any $\Si$-formula $\phi(v,w)$, if $a, b \in \MM^m$ satisfy 
%    \[\subintp{\nothing}{\MM}{\tp}(a) = \subintp{\nothing}{\MM}{\tp}(b)\]
%    then 
%    $\al \leq \MR{\MM}{\phi(v,a)} \iff \al \leq \MR{\MM}{\phi(v,b)}$.
%
%    If $\al = 0$ then since 
%    $\subintp{\nothing}{\MM}{\tp}(a) = \subintp{\nothing}{\MM}{\tp}(b)$
%    we have 
%    \[\MM \model{\Si} \exists v, \phi(v,a) 
%        \iff \MM \model{\Si} \exists v, \phi(v,b)\]
%    Hence 
%    \begin{align*}
%        &0 \leq \MR{\MM}{\phi(v,a)} \quad \iff \quad 
%        \phi(v,a)(\MM) \ne \nothing\\
%        \iff \quad & \MM \model{\Si} \exists v, \phi(v,a) \quad \iff \quad
%        \MM \model{\Si} \exists v, \phi(v,b) \\
%        \iff \quad & \phi(v,b)(\MM) \ne \nothing \quad \iff 
%        \quad 0 \leq \MR{\MM}{\phi(v,a)}
%    \end{align*}
%    
%    If $\al$ is a non-zero limit ordinal and all $\be$ less than $\al$ satisfy
%    the condition then 
%    \begin{align*}
%        &\al \leq \MR{\MM}{\phi(v,a)}\\
%         \iff 
%        &\forall \be < \al, \be \leq \MR{\MM}{\phi(v,a)}\\
%        \iff & \forall \be < \al, \be \leq \MR{\MM}{\phi(v,b)}\\
%        \iff & \al \leq \MR{\MM}{\phi(v,b)}
%    \end{align*}
%
%    For the succesor ordinal case by symmetry it suffices to show that if 
%    $\al$ satisfies the condition and $\al + 1 \leq \MR{\MM}{\phi(v,a)}$
%    then $\al + 1 \leq \MR{\MM}{\phi(v,b)}$.
%    As $\al + 1 \leq \MR{\MM}{\phi(v,a)}$, 
%    we have for each $n \in \N$ a $\Si(\MM)$-formula $\psi_n(v,c_n)$ 
%    (where $\psi_n \in \form{\Si}$ and $c_n \in \MM^{i_n}$) satisfying 
%    $\al \leq \MR{\MM}{\psi_n(v,c_n)}$ such that the $\psi_n(\MM,c_n)$
%    are pairwise disjoint subsets of $\phi(\MM)$.
%    For $n \in \N$, since $\MM$ is $\om$-sturated we can apply 
%    the `forth' property of 
%    \linkto{infty_equivalence_01}{$\infty$-equivalence} for 
%    $\MM$ with itself $i_0 + \dots + i_n$ times, 
%    using the fact that the types of $a$ and $b$ are equal in the first step:
%    \[\exists d_0 \in \MM^{i_n},\dots,
%    \exists d_n \in \MM^{i_n}, 
%    \subintp{\nothing}{\MM}{\tp}(a,c_0,\dots,c_n) = 
%    \subintp{\nothing}{\MM}{\tp}(b,d_0,\dots,d_n)\]
%    In particular we have for each $n$
%    \[
%        \subintp{\nothing}{\MM}{\tp}(c_n) = 
%        \subintp{\nothing}{\MM}{\tp}(d_n)
%    \]
%    and by the induction hypothesis 
%    \[\al \leq \MR{\MM}{\psi_n(v,c_n)} \text{ if and only if }
%    \al \leq \MR{\MM}{\psi_n(v,d_n)}\]
%    It remains to show that the $\psi_n(\MM,d_n)$ 
%    are disjoint subsets of $\phi(\MM,b)$.
%    Indeed, for each $n \in \N$
%    \begin{align*}
%        &\psi_n(\MM,c_n) \subs \phi(\MM,a)\\
%        \iff & \MM \model{\Si} \forall v, \psi_n(v,c_n) \to \phi(v,a)\\
%        \iff & \forall v, \psi_n(v,c_n) \to \phi(v,a) \in 
%        \subintp{\nothing}{\MM}{\tp}(a,c_n) = 
%        \subintp{\nothing}{\MM}{\tp}(b,d_n)\\
%        \iff & \MM \model{\Si} \forall v, \psi_n(v,d_n) \to \phi(v,b)\\
%        \iff & \psi_n(\MM,d_n) \subs \phi(\MM,b)
%    \end{align*}
%    To show that they are disjoint:
%    \begin{align*}
%        &\psi_n(\MM,c_n) \cap \psi_l(\MM,c_l) = \nothing\\
%        \iff & \MM \nodel{\Si} \exists v, \psi_n(v,c_n) \AND \psi_l(v,c_l)\\
%        \iff & \exists v, \psi_n(v,w_n) \AND \psi_l(v,w_l) \notin 
%        \tp(a,c_n,c_l) = \tp(b,d_n,d_l)\\
%        \iff & \MM \nodel{\Si} \exists v, \psi_n(v,d_n) \AND \psi_l(v,d_l)\\
%        \iff &\psi_n(\MM,d_n) \cap \psi_l(\MM,d_l) = \nothing
%    \end{align*}
%    Thus the induction is complete.
%\end{proof}
%
%
%\chapter{Appendix}
%\section{Direct limits}
\begin{prop}[Direct Limit of Chains \cite{marker}]
    \link{direct_limit_of_chains}
    If $M: I \to \Mod{\Si}$ is an embedding chain of 
    $\Si$-structures then there exists
    $\NN \in \struc{\Si}$ such that for all 
    $\al \in I$ there is a $\Si$-embedding $\io_\al : \MM(\al) \to \NN$
    and for any $\al \leq \be$ in $I$, the diagram
    \begin{center}
        \begin{tikzcd}
            \MM(\al) \ar[r, "\lift{\al}{\be}"] \ar[rd, "\io_\al", swap] 
            &\MM(\be) \ar[d, "\io_\be"]\\
            &\NN
        \end{tikzcd}
    \end{center}
    commutes. 
    Furthermore if $M$ were elementary then each $\io_\al$ 
    is elementary by the same construction.
\end{prop}
\begin{proof}
    We consider the disjoint union $\bigsqcup_{\al \in I} \MM(\al)$.
    For $a, b \in \bigsqcup_{\al \in I} \MM(\al)$ we want to define 
    $a \sim b$.
    There exist $\al, \be \in I$ such that $a \in \MM(\al)$ and 
    $b \in \MM(\be)$. 
    Since $I$ is a linear order, $\al \leq \be$ or vice versa.
    Then we say $a \sim b$ if $\lift{\al}{\be}(a) = b$ or vice versa.
    To show that the relation is transitive we use that fact that 
    $M$ is a functor so any 
    $\al \leq \be \leq \ga$ in $I$, 
    $\lift{\be}{\ga} \circ \lift{\al}{\be} = \lift{\al}{\ga}$.

    We define the carrier set as the quotient:
    \[\carrier{\NN} := \bigsqcup_{\al \in I} \MM(\al) / \sim\]
    Then for each $\al \in I$ there is an induced map of sets 
    $\io_\al : {\MM(\al)} \to {\NN}$ sending 
    $a \mapsto [a]$:
    \begin{center}
        \begin{tikzcd}
            {\MM(\al)} \ar[r, "\subs"] 
            \ar[rd, "\io_\al", swap] &\bigsqcup \ar[d]\\
            & \bigsqcup / \sim
        \end{tikzcd}
    \end{center}
    We immeditately have that it commutes with lifts:
    \begin{center}
        \begin{tikzcd}
            {\MM(\al)} \ar[r, "\lift{\al}{\be}"] 
            \ar[rd, "\io_\al", swap] 
            &{\MM(\be)} \ar[d, "\io_\be"]\\
            &{\NN}
        \end{tikzcd}
    \end{center}
    let $\al \leq \be$ in $I$ and let $a \in {\MM(\al)}$,
    then \[
        \io_\be \circ \lift{\al}{\be} (a) = 
        [\lift{\al}{\be} (a)] = [a] = \io_\al (a)
    \]
    since $\lift{\al}{\be} \sim a$. 
    Furthermore, for any $\al \in I$, 
    $\io_\al$ is injective:
    let $a, b \in {\MM(\al)}$ such that $\io_\al (a) = \io_\al (b)$,
    then by definition of $\io_\al$ we have $a \sim b$. 
    Note that $\lift{\al}{\al}$ is the identity, hence 
    \[
        a \sim b \quad \implies \quad a = \, \lift{\al}{\al} (b) = b
        \]

    We define interpretation for $\NN$ such that it commutes with 
    $\io_\al$ for each $\al$.
    We note that $I$ is non-empty and take $\al \in I$.
    \begin{itemize}
        \item[$\vert$] For $c \in \const{\Si}$, 
            $\modintp{\NN}{c} := \io_\al(\modintp{\MM(\al)}{c})$
        \item[$\vert$] For $f \in \func{\Si}$ define 
            $\modintp{\NN}{f} : {\NN}^{n_f} \to {\NN}$ 
            such that for $a \in {\NN}^{n_f}$ there exists a 
            $\be \in I$ (the maximum element of a finite totally ordered set) and
            $b \in {\MM(\be)}^{n_f}$ such that $a = \io_\be (b)$.
            Then have 
            $\modintp{\NN}{f} : a \mapsto \io_be (\modintp{\MM(\be)}{f}(b))$.
            To check that $\modintp{\MM}{f}$ is well defined, 
            let $\ga \in I$ and $c \in \MM(\ga)$ 
            be such that $a = \io_\ga (c)$.
            WLOG $\be \leq \ga$. 
            First note that 
            $\io_\ga (c) = a = \io_\be (b) = \io_\ga \circ \lift{\be}{\ga} (b)$
            since we showed that the $\io_\star$ commute with lifts.
            Since $\io_\ga$ is injective, $c = \lift{\be}{\ga} (c)$.
            Hence 
            \begin{align*}
                & \io_\be \circ \modintp{\MM(\be)}{f}(b) 
                =& \io_\ga \circ \lift{\be}{\ga} \circ \modintp{\MM(\be)}{f}(b)
                =& \io_\ga \circ \modintp{\MM(\ga)}{f} \circ \lift{\be}{\ga} (b)
                =& \io_\ga \circ \modintp{\MM(\ga)}{f}(c)
            \end{align*}
            Hence $\modintp{\NN}{f}$ is well defined.
        \item[$\vert$] For $r \in \rel{\Si}$, define 
            \[
                \modintp{\MM}{r} = \bigcup_{\be \in I} 
                \io_\be (\modintp{\MM(\be)}{r})
            \]
    \end{itemize}
    By the way we define interpretation it is clear that for any $\al$,
    $\io_\al$ is a $\Si$-morphism.
    We already have that it is injective.
    To show that it is an embedding, 
    take $a \in {\MM(\al)}^{m_r}$ such that 
    \[
        io_\al (a) \in \modintp{\MM}{r} = \bigcup_{\be \in I} 
        \io_\be (\modintp{\MM(\be)}{r})
    \]
    There exists a $\be$ and a $b \in \modintp{\MM(\be)}{r}$ such that
    $\io_\al (a) = \io_\be (b)$.
    Case on $\al \leq \be$ or $\be \leq \al$:
    If $\al \leq \be$ then 
    \[\io_\be \lift{\al}{\be} (a) = \io_\al (a) = \io_\be(b) 
    \quad \implies \quad \lift{\al}{\be} (a) = b\]
    by injectivity of $\io_\be$.
    Hence $\lift{\al}{\be} (a) \in \modintp{\MM(\be)}{r}$ 
    so $a \in \modintp{\MM(\al)}{r}$ since 
    $\lift{\al}{\be}$ is a $\Si$-embedding.
    If $\be \leq \al$ then we obtain $\lift{\be}{\al} (b) = a$ in the same way.
    Therefore $b \in \modintp{\MM(\be)}{r}$ implies 
    $a = \lift{\be}{\al}(b) \in \modintp{\MM(\al)}{r}$
    as $\lift{\be}{\al}$ is a $\Si$-morphism.

    Lastly we show that if the chain $M$ were elementary, 
    then for all $\al \in I$, $\io_\al$ is elementary.
    We prove the equivalent statement:
    for any $\phi \in \form{\Si}$ with free variables indexed by $S$, 
    given $\al \in I$ and $a \in {\MM(\al)}^S$,
    \[\MM(\al) \model{\Si} \phi(a) \iff \NN \model{\Si} \phi(\io_\al (a))\]
    by induction on $\phi$.
    By the two lemmas \linkto{emb_preserve_sat_of_quan_free}{
        `embeddings preserve satisfaction of quantifier free formulas'} and 
    \linkto{emb_preserve_sat_of_forall_down}{
        `embeddings preserve satisfaction of universal formulas downwards'}
    we only need to show that under the assumption of the inductive hypothesis,
    \[\MM(\al) \model{\Si} \forall v, \psi(a,v) \
    \implies \NN \model{\Si} \forall v, \psi(\io_\al (a),v)\]
    Suppose $\MM(\al) \model{\Si} \forall v, \psi(a,v)$
    and let $c \in {\NN}$.
    There exist $\be \in I$ and $b \in {\MM(\be)}$ such that 
    $\io_\be(b) = c$.
    Case on $\al \leq \be$ or $\be \leq \al$.
    If $\al \leq \be$ then the lift $\lift{\al}{\be}$ is elementary by assumption
    and so 
    \begin{align*}
            & \MM(\be) \model{\Si} \forall v, \psi(\lift{\al}{\be}(a),v)\\
        \implies & \MM(\be) \model{\Si} \psi(\lift{\al}{\be}(a),b)\\
        \implies & \NN \model{\Si} \psi(\io_\be \circ \lift{\al}{\be}(a),\io_\be (b))
                & \text{induction hypothesis}\\
        \implies &  \NN \model{\Si} \psi(\io_\al (a),c)
    \end{align*}
    Thus $\io_\al$ is indeed elementary.
\end{proof}
%\section{Boolean Algebras, Ultrafilters and the Stone Space}
\subsection{Boolean Algebras}
There is a very detailed wikipedia page \cite{wiki0} on Boolean algebras,
which can be used as references for elementary proofs.

\begin{dfn}[Partially ordered set]
    \link{partial_ordering}
    The signature of partially ordred sets $\Si_\PO$ consists of 
    $(\nothing,\nothing ,n_f,\set{\leq},m_f)$,
    where $n_\leq = 2$.
    The theory of partially ordered sets $\PO$ consists of 
    \begin{itemize}
        \item[$\vert$] Reflexivity: 
            $\forall x, 
            x \leq x$ (this is just notation for $\leq(x,x)$)
        \item[$\vert$] Antisymmetry: 
            $\forall x \forall y, 
            (x \leq y \AND y \leq x) \to x = y$
        \item[$\vert$] Transitivity:
            $\forall x \forall y \forall z, 
            (x \leq y \AND y \leq z) \to x \leq z$
    \end{itemize}
\end{dfn}

\begin{dfn}[Boolean algebra]
    \link{dfn_boolean_algebra}
    The signature of Boolean algebras $\Si_\BLN$ consists of 
    $(\set{1,0},\set{\leq,\sqcap,\sqcup,-},n_f,\nothing,m_f)$,
    where $n_\leq = 2$, $n_\sqcap = n_\sqcup = 2$ and $n_- = 1$.
    The theory of Boolean algebras $\BLN$ consists of the theory of 
    partially ordered sets\footnote{Often 
        the ordering is defined afterwards as 
        $a \leq b$ if and only if $a = a \sqcap b$.} 
    $\PO$ together with the formulas
    \begin{itemize}
        \item[$\vert$] Assosiativity of conjunction: 
            $\forall x \forall y \forall z, 
            (x \sqcap y) \sqcap z = x \sqcap (y \sqcap z)$
        \item[$\vert$] Assosiativity of disjunction: 
            $\forall x \forall y \forall z, 
            (x \sqcup y) \sqcup z = x \sqcup (y \sqcup z)$
        \item[$\vert$] Identity for conjunction:
            $\forall x, x \sqcap 1 = x$ 
        \item[$\vert$] Identity for disjunction:
            $\forall x, x \sqcup 0 = x$ 
        \item[$\vert$] Commutativity of conjunction: 
            $\forall x \forall y, x \sqcap y = y \sqcap x$
        \item[$\vert$] Commutativity of disjunction: 
            $\forall x \forall y, x \sqcup y = y \sqcup x$
        \item[$\vert$] Distributivity of conjunction:
            $\forall x \forall y \forall z, 
            x \sqcap (y \sqcup z) = (x \sqcap y) \sqcup (x \sqcap z)$
        \item[$\vert$] Distributivity of disjunction:
            $\forall x \forall y \forall z, 
            x \sqcup (y \sqcap z) = (x \sqcup y) \sqcap (x \sqcup z)$
        \item[$\vert$] Negation on conjunction: 
            $\forall x, x \sqcap - x = 0$ 
        \item[$\vert$] Negation on disjunction: 
            $\forall x, x \sqcup - x = 1$ 
        \item[$\vert$] Order on conjunction:
            $\forall x \forall y, (x \sqcap y) \leq x$
        \item[$\vert$] Maximal property of conjunction:
            $\forall x \forall y \forall z, 
            (x \leq y) \AND (x \leq z) \to (x \leq y \sqcap z)$ 
        \item[$\vert$] Order on disjunction:
            $\forall x \forall y, x \leq (x \sqcup y)$ 
        \item[$\vert$] Minimal property of disjunction:
            $\forall x \forall y \forall z, 
            (x \leq z) \AND (y \leq z) \to (x \sqcup y \leq z)$ 
        \end{itemize}
    Often `absorption' is also included, 
    but it can be deduced from the other axioms.
    I have not used the usual logical symbols due to obvious clashes with 
    our notation, 
    and we will be using this in the context of sets anyway.
    If $B$ is a $\Si_\BLN$-model of $\BLN$ we call it a Boolean algebra.
\end{dfn}

\begin{dfn}[Filters and ultrafilters on Boolean algebras]
    \link{dfn_filter_on_bool_alg}
    Let $P$ be a partial order.
    A subset $\FF$ of $P$ is an filter on $P$ if
    \begin{itemize}
        \item $\FF$ is non-empty.
        \item Closure under `superset': if $a \in \FF$ then any $b \in P$ 
            such that $a \leq b$ is also a member of $\FF$. %?Analogy between filters and prime ideals?
        \item Closure under `finite intersection':
            for any two members $a,b \in P$ there is a common smaller 
            $c \in P$: $c \leq a$ and $c \leq b$. For 
            Boolean algebras this is equivalent to the conjunction
            $a \sqcap b$ being in $\FF$.
    \end{itemize}
    We say a filter on $P$ is proper if it is not equal to $P$
    - for Boolean algebras this is equivalent to not containing $0$.
    A proper filter $\FF$ on $P$ is an ultrafilter (maximal filter) when
    for any filter $\GG$ on $P$ such that $\FF \subs \GG$ we have $\GG = \FF$ 
    or $\GG = P$.
\end{dfn}

\begin{lem}[Facts about Boolean algebras]
    \link{bare_boolean_facts}
    Let $B$ be a Boolean algebra, let $\FF$ be an ultrafilter
    on $B$, let $a,b \in B$ and let $f : B \to C$ be a morphism.
    \begin{itemize}
        \item $a \sqcap a = a$ and $a \sqcup a = a$.
        \item $a \sqcup 1 = 1$ and $a \sqcap 0 = 0$.
        \item If $a \sqcap b = 0$ and $a \sqcup b = 1$ then $a = - b$
            (negations are unique.)
        \item $- (a \sqcup b) = (- a) \sqcap (- b)$ and its dual. 
            (De Morgan)
        \item ($a \in \FF$ or $b \in \FF$) if and only if $a \sqcup b \in \FF$.
        \item Morphisms are order preserving.
        \item Morphisms commute with negation.
        \item $a \sqcap b = 1$ if and only if $a = 1$ and $b = 1$.
            Similarly for $\sqcup$ with $0$.
        \item If $a \sqcap b = 0$ then $b \leq - a$.
    \end{itemize}
\end{lem}
%\begin{proof}~
%    \begin{itemize}
%        \item We prove only $a \sqcap a = a$ as the other has the same proof.
%            \[a \sqcap a = (a \sqcap a) \sqcup 0 = (a \sqcap a) \sqcup (a \sqcap - a)
%            = a \sqcap (a \sqcup - a) = a \sqcap 1 = a\]
%        \item Again, we prove only $a \sqcup 1 = 1$. Using the previous part,
%            \[a \sqcup 1 = a \sqcup (a \sqcup - a) = (a \sqcup a) \sqcup - a
%            = a \sqcup - a = 1\]
%        \item \begin{align*}
%            a &= a \sqcup 0 = a \sqcup (b \sqcap - b) = 
%            (a \sqcup b) \sqcap (a \sqcup - b) = 1 \sqcap (a \sqcup - b)\\
%            &= 0 \sqcup (a \sqcup - b) = (b \sqcap - b) \sqcup (a \sqcup - b)
%            = (b \sqcup a) \sqcup - b = 0 \sqcup - b \\
%            &= - b
%        \end{align*} 
%        \item By the previous part it suffices to show that 
%            \[\brkt{(- a) \sqcap (- b)} \sqcap (a \sqcup b) = 0 \text{ and }
%            \brkt{(- a) \sqcup (- b) } \sqcup (a \sqcup b) = 1\]
%            These are clear.
%        \item \begin{forward}
%            In the case that $a \in \FF$ we have $a \leq a \sqcup b \in \FF$ 
%            as $\FF$ is under superset. 
%            The other case is the same.
%        \end{forward}
%        \begin{backward}
%            Suppose $a \notin \FF$ and $b \notin \FF$ then
%            \linkto{negation_is_in_ultrafilter}{$- a \in \FF$ and 
%                $- b \in \FF$} hence $- a \sqcap - b \in \FF$
%                by closure under intersection. 
%                By the previous part this is equal to $- (a \sqcup b)$,
%                which implies $(a \sqcup b) \notin \FF$ as $\FF$ 
%                is an \linkto{negation_is_in_ultrafilter}{ultrafilter}.
%        \end{backward}
%        \item Suppose $b \leq a$. 
%            We show that $f(a) \leq f(b)$.
%            By the minimal property of disjunction,
%            \[b \leq a \AND a \leq a \implies b \sqcup a \leq a\]
%            Clearly $a \leq b \sqcup a$ and so $a = b \sqcup a$
%            Hence $f(b) \leq f(b) \sqcup f(a) = f(b \sqcup a) = f(a)$.
%        \item Suppose $f(a) = b$.
%            Then $f(a) \AND f(- a) = f(a \AND - a) = f(0) = 0$
%            and $f(a) \OR f(- a) = f(a \OR - a) = f(1) = 1$.
%            As negations are unique (shown above) 
%            this gives us that $f(- a) = - f(a)$.
%        \item Suppose $a \sqcap b = 1$ then $1 \leq a \sqcap b \leq a$ hence $1 = a$
%            and similarly $1 = b$.
%        \item\begin{align*}
%                &a \sqcup - a = 1\\
%                \implies & b \sqcap (a \sqcup - a) = b \sqcap 1 = b\\
%                \implies & (b \sqcap a) \sqcup (b \sqcap - a) = b\\
%                \implies & 0 \sqcup (b \sqcap - a) = b\\
%                \implies & b = b \sqcap - a \leq - a
%            \end{align*}
%    \end{itemize}
%\end{proof}

\begin{prop}[Equivalent definition of ultrafilter for Boolean algebras]
    \link{negation_is_in_ultrafilter}
    Let $B$ be a boolean algebra. 
    Let $\FF$ be a proper filter on $B$.
    The following are equivalent:
    \begin{enumerate}
        \item $\FF$ is an ultrafilter over $B$.
        \item If $a \sqcup b \in \FF$ then $a \in \FF$ or $b \in \FF$.
            `$\FF$ is prime'.
        \item For any $a$ of $B$, 
            $a \in \FF$ or $(- a) \in \FF$.
    \end{enumerate}
    For $2.$ the or is in fact `exclusive or' since if both $a$ 
    and its negation were in $\FF$ then $\FF$ would not be proper.
\end{prop}
\begin{proof}
    $(1. \implies 2.)$
        Suppose $a_0 \sqcup a_1 \in \FF$.
        \[\GG_{a_0} = \set{b \in B \st \exists c \in \FF, c \sqcap a_0 \leq b}\]
        Then clearly $\GG_{a_0}$ is a filter containing $\FF$ and $a_0$.
        Similarly have $\GG_{a_1}$.
        Since $\FF$ is an ultrafilter, 
        we can case on whether $\GG_{a_i}$ is $\FF$ or $B$.
        If either is $\FF$ then $a_i \in \FF$ and we are done.
        If $\GG_{a_0} = \GG_{a_1} = B$ then both contain $0$ and so 
        there exist $c_i \in \FF$ such that 
        $c_i \sqcap a_i = 0$.
        \linkto{bare_boolean_facts}{Hence $c_i \leq - a_i$,}
        and so $(- a_i) \in \FF$.
        Thus $- (a_0 \sqcup a_1) = - a_0 \sqcap - a_1 \in \FF$,
        and so $0 \in \FF$,
        a contradiction.

    $(2. \implies 3.)$ 
        Let $a \in \FF$.
        Then we have that $a \sqcup - a = 1$, 
        which is by definition a member of $\FF$.
        By assumption this implies that $a \in \FF$ or $- a \in \FF$.
    
    $(3. \implies 1.)$
        Let $\GG$ be a proper filter such that $\FF \subs \GG$.
        It suffices to show that $\GG = \FF$.
        Then let $a \in \GG$.
        Suppose $a \notin \FF$.
        Then $- a \in \FF$ and so $- a \in \GG$.
        Thus $0 = (a \sqcap - a) \in \GG$ 
        thus $\GG$ is not proper, a contradiction.
        Hence $\GG = \FF$.
\end{proof}

\begin{prop}[Extending filters to ultrafilters on Boolean algebras]
    \link{zorn_ultrafilter}
    Any Boolean algebra 
    can be extended to an ultrafilter.
\end{prop}
\begin{proof}
    Let $\FF$ be a proper filter on $B$ a Boolean algebra. 
    The set of proper filters on $B$ that contain $\FF$ is a non-empty set 
    ordered by inclusion.
    Any chain of proper filters is a proper filter 
    so by Zorn we have a 
    maximal element.
\end{proof}

\begin{dfn}
    The category of Boolean algebras consists of Boolean algebras as the objects
    and for any two Boolean algebras $B, C$ morphisms $f:B \to C$ such that
    for any $a,b \in B$
    \begin{align*}
        f(0) = 0\\
        f(1) = 1\\
        f(a \sqcup b) = f(a) \sqcup f(b)\\
        f(a \sqcap b) = f(a) \sqcap f(b)\\
    \end{align*}
\end{dfn}

\begin{dfn}[Stone space]
    A topological space is $0$-dimensional if it has a clopen basis.

    The category of Stone spaces has
    $0$-dimensional, 
    compact and Hausdorff topological spaces as objects and 
    continuous maps as morphisms.
\end{dfn}

\begin{prop}[Contravariant functor from Boolean algebras to Stone spaces
        \cite{tent}]
    \link{bool_alg_to_stone_functor}
    Given $B$ a Boolean algebra, 
    the following topological spaces are isomorphic:
    \begin{itemize}
        \item $S(B)$, 
            the set of ultrafilters on $B$ with a clopen basis given by 
            collections of ultrafilters containing some element of $B$.
        \item $\mor{B}{2}{}$, 
            the set of Boolean algebra morphisms from $B \to 2$ 
            with the subspace topology from $2^B$, where $2$ is given the 
            discrete topology and $2^B = \prod_{b \in B} 2$ 
            inherits a product topology.
    \end{itemize}
    This is a Stone space and there is a 
    contravariant functor from the category of 
    Boolean algebras to the category of Stone spaces taking each 
    $B$ to $S(B)$
    and each Boolean algebra morphism
    $f: A \to B$ to a continuous map of Stone spaces:
    \[S(f) := f^{-1}(\star) : S(B) \to S(A)\]

    Note:
    In the second definition the functor would take $f$ to the precomposition 
    map $\star \circ f : \mor{B}{2}{} \to \mor{A}{2}{}$.
\end{prop}
\begin{proof}
    We first show that the spaces are isomorphic.
    In fact their isomorphism is the restriction of a larger isomorphism between 
    $2^B$ and the power set of $B$.
    We take the bijection
    \[2^B \to \PP(B) := f \mapsto f^{-1}(1)\]
    inducing a topology on the power set of $B$.
    We must show that $S(B)$ is the image of $\mor{B}{2}{}$
    under this isomorphism and that the subspace topology on $S(B)$
    from $\PP(B)$ is the topology defined by the basis given.

    \textbf{$S(B)$ is the image of 
    $\mor{B}{2}{}$}: Let $f : B \to 2$ be a Boolean algebra morphism. 
    We must show that $\FF := f^{-1}(1)$ is an ultrafilter.
    Since $f(1) = 1$, $1 \in \FF$. 
    If $a,b \in \FF$ then $f(a) = f(b) = 1$ so 
    $f(a \sqcap b) = f(a) \sqcap f(b) = 1$ thus 
    $\FF$ is closed under intersection.
    If $a \leq b$ and $a \in \FF$ then as $f$ is 
    \linkto{bare_boolean_facts}{order preserving} $f(a) \leq f(b)$.
    It is a proper filter as $f(0) = 0 \ne 1$.
    To show it is an ultrafilter we use the
    \linkto{negation_is_in_ultrafilter}{equivalent definition}:
    let $a \in \PP(B)$. 
    If $f(a) = 1$ then we are done,
    otherwise
    \[f(- a) = - f(a) = - 0 = 1 \implies - a \in \FF\]
    and we are done.
    To show that this is a surjection we use the inverse:
    \[\FF \mapsto \brkt{a \to \begin{cases}
        1, a \in \FF\\
        0, a \notin \FF
    \end{cases}}\] 
    To show that this is a morphism we note that $\FF$ 
    is proper and contains $1$ thus $f(0) = 0$ and $f(1) = 1$. 
    Also 
    \begin{align*}
        & f(a \sqcap b) = 1 \iff a \sqcap b \in \FF\\ 
        &\iff a \in \FF \text{ and } b \in \FF 
        \quad \text{ by closure under finite intersection and superset}\\
        &\iff f(a) = 1 \text{ and } f(b) = 1 \\
        &\iff f(a) \sqcap f(b) = 1 
        \quad \text{ \linkto{bare_boolean_facts}{ as proven before}}
    \end{align*}
    Hence $f(a \sqcap b) = f(a) \sqcap f(b)$.
    Lastly since 
    \[- f(a) = 1 \iff f(a) = 0 \iff a \notin \FF \iff - a \in \FF 
    \iff f(- a) = 1\]
    thus by \linkto{bare_boolean_facts}{De Morgan} and 
    \linkto{bare_boolean_facts}{uniqueness of negations} we have
    \[f(a \sqcup b) = f(-(- a \sqcap - b)) = 
    - \sqbrkt{- f(a) \sqcap - f(b)} = 
    \brkt{- - f(a)} \sqcup - - f(b) = f(a) \sqcup f(b)\]
    Thus the inverse map gives back a Boolean algebra morphism and 
    so the image of $\mor{B}{2}{}$ is $S(B)$ under the isomorphism.

    \textbf{The topologies agree}:
    It suffices that any open set in the basis of each can be written 
    as a open set in the other.
    For each $b$ we let $\pi_b$ be 
    the continuous projection/evaluation maps from $2^B \to 2$
    mapping $f \mapsto f(b)$.
    \begin{forward}
        Let $[b]$ be an element of the 
        basis for $S(B)$ under the Stone topology.
        Then this is the image of the open subset 
        $\pi_b^{-1}\set{1}$ under the isomorphism:
        \[
            \pi_b^{-1}\set{1} = \set{f \in 2^B \st f(b) = 1} 
            \IFF \set{f^{-1}(1) \st f(b) = 1}
            = \set{\FF \text{ ultrafilter} \st b \in \FF}
        \]
    \end{forward}
    \begin{backward}
        Conversely, 
        any open in $\mor{B}{2}{}$ is of the form 
        $\pi_b^{-1}X$ where $b \in B$ and $X \subs 2$.
        When $X$ is empty or $2$ then $\pi_b^{-1}X$ is empty or the whole space
        so it is open in $S(B)$.
        Otherwise if $X = \set{1}$ we have the case above again.
        The case $X = \set{0}$ is also covered by taking the clopen complement.
    \end{backward}
    Thus the topologies are the same under this isomorphism.

    We now show that the equivalent topologies are Stone spaces.
    $S(B)$ is $0$-dimensional:
    an element of the open basis is given by
    \[[b] := \set{\FF \subs B \st b \in \FF}\]
    for some $b \in B$.
    The complement of each open in the basis
    $S(B) \setminus [b] = [-b]$ is again in the open basis.
    $S(B)$ is Hausdorff: if $\FF, \GG \in S(B)$ are not equal then 
    without loss of generality there exists $b \in \FF$ such that 
    $b \notin \GG$. 
    Then the points $\FF$ and $\GG$ 
    are separated $\FF \in [b]$ and $\GG \in [-b]$.
    %? Analogy between filters and prime ideals?

    $\mor{B}{2}{}$ is compact: 
    note that $2 = \set{0,1}$ with the discrete topology is compact and
    so $2^B = \prod_{a \in B} 2$ with the product topology is also
    compact by Tychonoff.
    Closed in compact is compact, so it remains to show that $\mor{B}{2}{}$
    is closed in $2^B$:
    \[
        \mor{B}{2}{} = \set{f \st f(0) = 0} \cap \set{f \st f(1) = 1}
        \cap \set{f \st f \text{ commutes with $\sqcap$}} \cap 
        \set{f \st f \text{ commutes with $\sqcup$}}
    \]
    It suffices that these four sets are closed.
    The first is $\pi_0^{-1}(0)$ and the second $\pi_1^{-1}(1)$,
    which are both closed (and open).
    For the third 
    \[
        \set{f \st f \text{ commutes with $\sqcap$}}=
        \bigcap_{a,b \in B} \set{f \in 2^B \st f(a \sqcap b) = f(a) \sqcap f(b)}
    \]
    So it suffices that each set in the intersection is closed:
    \begin{align*}
        2^B \setminus \set{f \st f(a \sqcap b) = f(a) \sqcap f(b)}\\
        = &\set{f \st f(a \sqcap b) = 0 \text{ and } f(a) \sqcap f(b) = 1}
        \cup \set{f \st f(a \sqcap b) = 1 \text{ and } f(a) \sqcap f(b) = 0}\\
        = &\brkt{\pi_{a \sqcap b}^{-1}\set{0} \cap \pi_a^{-1}\set{1}
        \cap \pi_b^{-1}\set{1}} \cup 
        \brkt{\pi_{a \sqcap b}^{-1}
        \cap (\pi_a^{-1}\set{0}) \cup \pi_b^{-1}\set{0}}
    \end{align*}
    Each projection map's preimage is open so this whole complement is open.
    Hence $\set{f \st f \text{ commutes with $\sqcap$}}$ is closed.
    Similarly for $\set{f \st f \text{ commutes with $\sqcup$}}$ so we 
    have shown that $S(B) \iso \mor{B}{2}{}$ is compact.

    To show that $S(\star)$ 
    is a contravariant functor we need to check that the morphism map
    \[S(f) := f^-1(\star) : S(B) \to S(A)\]
    is a well-defined, respects the identity and composition. 
    We show that $S(f)$ is continuous: 
    it suffices that preimages of clopen elements are clopen.
    Let $[b] \subs S(A)$ be clopen. 
    \begin{align*}
        & S(f)^{-1}[b]\\
        =& \set{\FF \in S(B) \st f^{-1}(\FF) \in [b]}\\
        =& \set{\FF \in S(B) \st f(b) \in \FF}\\
        =& [f(b)]
    \end{align*}
    which is clopen.
\end{proof}

\begin{prop}[Stone Duality]%?Incomplete
    There is an equivalence between the category of Stone
    spaces and the category of Boolean algebras.
    Given by the functor $\BB_\star$ 
    sending any Stone space $X$ to the set of 
    its clopen subsets (this is a basis of $X$ as it is $0$-dimensional):
    \begin{cd}
        \text{ Boolean algebras }  \ar[r, shift left, "S(\star)"] 
        & \text{ Stone spaces } \ar[l, shift left, "\BB_\star"]
    \end{cd}
    and its inverse \linkto{bool_alg_to_stone_functor}{$S(\star)$}.
\end{prop}
\begin{proof}
    Let $X$ be a $0$-dimensional compact Hausdorff topological space.
    There is an obvious Boolean algebra to take on 
    \[\BB_X := \set{a \subs X \st a \text{ is clopen}}\]
    which is interpreting $0$ to be $\nothing$, $1$ to be $X$, 
    $\leq$ as $\subs$,
    conjunction as intersection, disjunction as union and negation to be 
    taking the complement in $X$.
    One can check that this is a Boolean algebra.

    We make this a contravariant functor by taking any continuous map 
    $f : X \to Y$
    to an induced map $f^{\diamond}:= f^{-1}(\star):\BB_Y \to \BB_X$.
    One can check that this is a well-defined functor.

    It remains to show the equivalence of categories by giving 
    natural transformations $S(\BB_{\star}) \to \id{\star}$ in
    the category of topological spaces
    and $\id{\star} \to \BB_{S(\star)}$ in the category of Stone spaces.
    This is omitted.
\end{proof}

\subsection{Isolated points of the Stone space}
\begin{dfn}[Isolated point]
    \link{dfn_isolated_point}
    Let $X$ be a topological space and $x \in X$. 
    We say $x$ is isolated if $\set{x}$ is open.
\end{dfn}

\begin{dfn}[Derived set]
    \link{dfn_derived_set}
    Let $X$ be a topological space. 
    The derived set of $X$ is defined as
    \[X' : = X \setminus \set{x \in X \st x \text{ isolated}}\]
\end{dfn}

\begin{ex}[Equivalent definition of derived set]
    Let $U$ be a subspace of $X$ a topological space.
    $x \in U$ is not isolated (in the subspace topology) if and only if 
    for any open set $O_x$ containing $x$, 
    $O_x \cap U \setminus \set{x}$ is non-empty. 
    (it is a limit point of $U$.)
\end{ex}

\begin{dfn}[Atom]
    Let $B$ be a Boolean algebra. 
    We say $a \in B$ is an atom if it is non-zero and for any $b \in B$
    if $b \leq a$ then $b = a$ or $b = 0$.
\end{dfn}

\begin{dfn}[Principle filter]
    Let $B$ be a Boolean algebra and $a \in B$.
    Suppose $a$ is non-zero.
    Then principle filter of $a$ is defined as 
    \[\upa{a} := \set{b \in B \st a \leq b}\]
    One should check that this is a proper filter.
\end{dfn}

\begin{prop}
    Let $a \in B$ a Boolean algebra.
    Then $\upa{a}$ is an ultrafilter if and only if $a$ is atomic. 
\end{prop}
\begin{proof}
    \begin{forward}
        Suppose $b \leq a$.
        As $\upa{a}$ is \linkto{negation_is_in_ultrafilter}{ultrafilter} 
        either $b \in \upa{a}$ or $- b \in \upa{a}$.
        In the first case $a \leq b$ hence $a = b$.
        If $- b \in \upa{a}$ then $b \leq a \leq - b$ and so 
        \[1 = b \sqcup - b \leq - b \sqcup - b = - b\]
        and so $1 = - b$ and $b = 0$.
    \end{forward}

    \begin{backward}
        Suppose $a$ is an atom. 
        We show that for any $b \in B$, 
        it is in $\upa{a}$ or its negation is in $\upa{a}$
        Since $1 \in \upa{a}$, 
        $b \sqcup - b \in \upa{a}$ and so $a \leq b \sqcup - b$.
        Thus
        \[a = a \sqcap (b \sqcup - b) = (a \sqcap b) \sqcup (a \sqcap - b)\]
        Since $a$ is non-zero, either $a \sqcap b$ or $a \sqcap - b$ is non-zero.
        If $a \sqcap b \ne 0$ then together with the fact that $a \sqcap b \leq a$
        we conclude that $a \sqcap b = a$ as $a$ is atomic.
        Hence $a \leq b$ and $b \in \upa{a}$.
        Similarly the other case results in $- b \in \upa{a}$.
    \end{backward}
\end{proof}

\begin{prop}[Correspondence between atoms and isolated points]
    Let $a \in B$ a Boolean algebra.
    If $a$ is an atom if and only if $[a]$ is a singleton in $S(B)$.
    Hence $\upa{a}$ is an isolated point in $S(B)$.
\end{prop}
\begin{proof}
    \begin{forward}
        If $a$ is an atom then $[a] = \set{\upa{a}}$ is the only ultrafilter 
        containing $a$ is the principle filter of $a$:
        Let $\FF$ be an ultrafilter containing $a$.
        Then for any $b \in \FF$, 
        $a \sqcap b \in \FF$ and non-zero.
        Therefore $a \sqcap b = a$ as $a$ is an atom and $a \leq b$.
        Hence $b \in \upa{a}$.
        By maximality $\FF = \upa{a}$.
        Hence $[a]$ is a singleton.
    \end{forward}

    \begin{backward}
        Suppose $a$ is not atomic. 
        Then there exists $b \leq a$ such that $b \ne 0$ and $b \ne a$.
        \begin{align*}
            & b \sqcup (a \sqcap - b)
            = (a \sqcap b) \sqcup (a \sqcap - b)\\
            = & a \sqcup (b \sqcap - b) = a \sqcup 0 = a
        \end{align*}
        There exist \linkto{zorn_ultrafilter}{ultrafilters} 
        extending the principle filters of $b$ and 
        $a \sqcap - b$.
        These are not equal since $b$ and $- b$ cannot be in the same 
        proper filter.
        Both filters contain $a$. 
        Hence $[a]$ is not a singleton.
    \end{backward}
\end{proof}
%\subsection{Ultraproducts and Łos's Theorem}
This section introduces ultrafilters and ultraproducts and uses Łos's Theorem
to prove the compactness theorem.
Łos's theorem appears as an exercise in Tent and Ziegler's book \cite{tent}.

\begin{dfn}[Filters on sets]
    Let $X$ be a set. 
    The power set of $X$ is a \linkto{dfn_boolean_algebra}{Boolean algebra} 
    with $0$ interpreted as 
    $\nothing$, 
    $1$ interpreted as $X$ and $\leq$ interpreted as $\subs$.
    We say $\FF$ is a filter on $X$ if it is a 
    \linkto{dfn_filter_on_bool_alg}{filter} on the power set.
    The definition translates to:
    \begin{itemize}
        \item $X \in \FF$.
        \item For any two members of $\FF$ their intersection is in $\FF$.
        \item If $a \in \FF$ then any $b$ in the power set of $X$ 
            such that $a \subs b$ is also a member of $\FF$.
    \end{itemize}
    Translating definitions over we have that 
    a filter on $X$ is proper if and only if it does not contain the empty set,
    if and only if the filter is not equal to the power set.
    Furthermore a proper filter $\FF$ is an ultrafilter if and only if
    for any filter $\GG$, 
    if $\FF \subs \GG$ then $\FF = \GG$ 
    or $\GG$ is the power set of $X$.
\end{dfn}

\begin{dfn}[Ultraproduct]
    Let $\FF$ be an ultrafilter on $X$.
    We define a relation on $\prod_{x \in X} x$ by 
    \[
        (a_x)_{x \in X} \sim (b_x)_{x \in X} 
        := \set{x \in X \st a_x = b_x} \in \FF
    \]
    This is an equivalence relation as 
    \begin{itemize}
        \item $(a_x)_{x \in X} \sim (a_x)_{x \in X} \iff 
            \set{x \in X \st a_x = a_x} = X \in \FF$ 
        \item Symmetry is due to symmetry of $=$.
        \item If $\set{x \in X \st a_x = b_x} \in \FF$ and 
            $\set{x \in X \st b_x = c_x} \in \FF$ then 
            $\set{x \in X \st a_x = b_x = c_x}$ is their intersection
            and so is in $\FF$.
            Thus its superset $\set{x \in X \st a_x = c_x}$ is in $\FF$.
    \end{itemize}
    We define the ultraproduct of $X$ over $\FF$:
    \[\prod X / \FF := \prod_{x \in X} x / \sim\]
\end{dfn}

\begin{prop}[Equivalent definition of ultrafilter (translated to the power set)]
    \link{complement_is_in_ultrafilter}
    Let $X$ be a set. 
    Let $\FF$ be a proper filter on $X$.
    $\FF$ is an ultrafilter over $X$ if and only if for every subset 
    $U \subs X$, 
    \emph{either} $U \in \FF$ or $X \setminus U \in \FF$.
\end{prop}
\begin{proof}
    Follows immediately from \linkto{negation_is_in_ultrafilter}{the 
    equivalent definition of an ultrafilter}.
\end{proof}

\begin{prop}[Łos's Theorem]
    \link{los_theorem}
    Let $\f{M} \subs \struc{\Si}$ where $\Si$ is a signature
    such that \emph{each carrier set is non-empty}.
    Let \[X = \set{{\MM} \st \MM \in \f{M}}\]
    Write $\MM = {\MM}$ and $\f{M} = X$ to make things look nicer.
    Suppose $\FF$ is an ultrafilter on $\f{M}$ 
    (i.e. an ultrafilter on the Boolean algebra $P(\f{M})$).
    Then we want to make $\NN := \prod \f{M} / \FF$ into a $\Si$-structure.
    Let $\pi$ be the natural surjection 
    $\prod_{\MM \in \f{M}} \MM \to \prod \f{M} / \FF$.
    If $a = (a_1, \dots, a_n) \in \prod_{\MM \in \f{M}} \MM$ then
    write $a_\MM := ((a_1)_\MM, \dots, (a_n)_\MM)$.
    \begin{itemize}
        \item Constant symbols $c \in \const{\Si}$ are interpreted as 
            \[\modintp{\NN}{c} := \pi(\modintp{\MM}{c})_{\MM \in \f{M}}\]
        \item Any function symbol $f \in \func{\Si}$ is interpreted as the 
            function
            \[\modintp{\NN}{f} : 
            \brkt{\prod \f{M} / \FF}^n \to \prod \f{M} / \FF
            := \pi\brkt{(a_\MM)_{\MM \in \f{M}}} \mapsto 
            \pi(f^\MM (a_\MM))_{\MM \in \f{M}}\]
            where $\pi\brkt{(a_\MM)_{\MM \in \f{M}}} = 
            \brkt{\pi((a_i)_\MM)_{\MM \in \f{M}}}_{i=1}^n$.
        \item Any relation symbols 
            $r \in \rel{\Si}$ is interpreted as the subset such that 
            \[\pi(a) \in \modintp{\NN}{r}
            \iff \set{\MM \in \f{M} \st 
            a_\MM \in \modintp{\MM}{r}} \in \FF\]
            where $a = a_1,\dots, a_m$ and $\pi(a) = \brkt{\pi(a_i)}_{i=1}^m$.
    \end{itemize}
    Then for any $\Si$-formula $\phi$ with free variables indexed by $S$,
    If $a = (a_1, \dots, a_n) \in \prod_{\MM \in \f{M}} \MM$ then 
    \[\NN \model{\Si} \phi(\pi(a)) \iff 
    \set{\MM \in \f{M} \st \MM \model{\Si} \phi(a_\MM)} \in \FF\]
\end{prop}
\begin{proof}
    We show that the interpretation of functions is well defined.
    Let $a, b \in \brkt{\prod_{\MM \in \f{M}}}^{n_f}$.
    Suppose for each $i \in \set{1,\dots,n_f}$, 
    $a_i \sim b_i \in \prod_{\MM \in \f{M}} \MM$.
    Then 
    \begin{align*}
        &\text{for each } i, 
        \set{\MM \in \f{M} \st (a_i)_\MM = (b_i)_\MM} \in \FF \\
        &\implies
        \set{\MM \in \f{M} \st \bigand{i = 1}{n} (a_i)_\MM = (b_i)_\MM} \in \FF
        \text{ by closure under finite adjunction}\\
        &\implies
        \set{\MM \in \f{M} \st a_\MM = b_\MM} \in \FF\\
        &\implies
        \set{\MM \in \f{M} \st 
            \modintp{\MM}{f}(a_\MM) = \modintp{\MM}{f}(b_\MM)} \in \FF
        \text{ by closure under superset}\\
        &\implies \pi(\mmintp{f}(a_\MM)) = \pi(\mmintp{f}(b_\MM))
        \text{ by definition of the quotient}
    \end{align*}

    We use the following claim:
    If $t$ is a term with variables $S$ and 
    $a \in (\prod_{\MM \in \f{M}} \MM)^S$
    then there exists $b \in \prod_{\MM \in \f{M}} \MM$ such that 
    \[\modintp{\NN}{t}\circ \pi(a) = \pi(b) \text{ and } 
    \forall \MM \in \f{M}, \modintp{\MM}{t}(a_\MM) = b_\MM\]
    We prove this by induction on $t$:
    \begin{itemize}
        \item If $t$ is a constant symbol $c$ then pick 
            $b := (\modintp{\MM}{c})_{\MM \in \f{M}}$.
        \item If $t$ is a variable then let 
            $a \in \prod_{\MM \in \f{M}} \MM$ (only one varible), 
            pick $b := a$.
        \item If $t$ is a $f(s)$ then let $a \in (\prod_{\MM \in \f{M}} \MM)^S$
            by the inducition hypothesis there exists 
            $c \in \prod_{\MM \in \f{M}} \MM$ such that 
            \[\modintp{\NN}{s} \circ \pi(a) = \pi(c) \text{ and } 
            \forall \MM \in \f{M}, \modintp{\MM}{s}(a_\MM) = c_\MM\]
            Then we can take 
            $b = \brkt{\modintp{\MM}{f}(c_\MM)}_{\MM \in \f{M}}$. 
            Thus 
            \[\nnintp{t}\circ \pi(a) = \nnintp{f}(\nnintp{s}\circ (\pi(a)))
            = \nnintp{f}(\pi(c)) = \pi(\mmintp{f}(c_\MM))_{\MM \in \f{M}}
            = \pi(b)\]
            and for any $\MM \in \f{M}$,
            \[\mmintp{t}(a_\MM) = \mmintp{f} \circ \mmintp{s} (a_\MM)
            = \mmintp{f}(c_\MM) = b_\MM\]
    \end{itemize}
    We now induct on $\phi$ to show that for any appropriate $a$,
    \[\NN \model{\Si} \phi(\pi(a)) \iff 
    \set{\MM \in \f{M} \st \MM \model{\Si} \phi(a_\MM)} \in \FF\]
    \begin{itemize}
        \item The case where $\phi$ is $\top$ is trivial 
            (noting that anything models $\top$ and $\f{M} \in \FF$).
        \item If $\phi$ is $s = t$ then it suffices to show that 
        \[  
            \modintp{\NN}{s}\circ \pi(a) = \modintp{\NN}{t}\circ \pi(a)
            \iff 
            \set{\MM \in \f{M} \st \MM \model{\Si} \modintp{\MM}{t}(a_\MM) = 
            \modintp{\MM}{s}(a_\MM)} \in \FF
        \]
        \begin{forward}
            If for two terms $s,t$ we have
            $\modintp{\NN}{s}\circ \pi(a) = \modintp{\NN}{t}\circ \pi(a)$
            then by the claim
            there exists $b \in \prod_{\MM \in \f{M}} \MM$ such that
            \[  
                \f{M} = 
                \set{\MM \in \f{M} \st 
                \modintp{\MM}{s}(a_\MM) = b_\MM = \modintp{\MM}{t}(a_\MM)} 
                = 
                \set{\MM \in \f{M} \st 
                    \modintp{\MM}{t}(a_\MM) = \modintp{\MM}{s}(a_\MM)} 
            \]
            which is therefore in the filter $\FF$.
        \end{forward}
        \begin{backward}
            If for two terms $s,t$ we have
            $\modintp{\NN}{s}\circ \pi(a) \ne \modintp{\NN}{t}\circ \pi(a)$
            then by the claim there exist 
            $b \ne c \in \prod_{\MM \in \f{M}} \MM$ such that
            \[  
                \set{\MM \in \f{M} \st 
                    \modintp{\MM}{t}(a_\MM) = \modintp{\MM}{s}(a_\MM)} =
                \set{\MM \in \f{M} \st b_\MM = c_\MM} = 
                \nothing
            \]
            which is not in the filter $\FF$ as it is proper.
        \end{backward}
        \item If $\phi$ is $r(t)$ then by the claim we have 
            $b \in \prod_{\MM \in \f{M}} \MM$ with the desired properties.
            It suffices to show that 
            \[\pi(b) \in \nnintp{r} \iff 
            \set{\MM \in \f{\MM} \st b_\MM \in \mmintp{r}} \in \FF\]
            This follows from our definition of 
            interpretation of relation symbols.
        \item If $\phi$ is $\NOT \psi$ then 
            $\NN \model{\Si} \phi(\pi(a))$ if and only if 
            $\set{\MM \in \f{M} \st \MM \model{\Si} \psi(a_\MM)} \notin \FF$
            by induction.
            This holds if and only if its 
            \linkto{complement_is_in_ultrafilter}{complement is in the filter}:
            $\set{\MM \in \f{M} \st \MM \nodel{\Si} \psi(a_\MM)} \in \FF$
            which is if and only if 
            $\set{\MM \in \f{M} \st \MM \model{\Si} \phi(a_\MM)} \notin \FF$
        \item Without loss of generality we can use $\AND$ instead of $\OR$
            to make things simpler 
            (replacing this comes down to dealing with a couple of $\NOT$
            statements).
            If $\phi$ is $\psi \AND \chi$ then one direction 
            follows filters being closed under intersection:
            \begin{align*}
                &\NN \model{\Si} \phi(\pi(a))\\
                &\iff 
                \set{\MM \in \f{M} \st \MM \model{\Si} \psi(a_\MM)} \in \FF
                \text{ and }
                \set{\MM \in \f{M} \st \MM \model{\Si} \chi(a_\MM)}\in \FF\\
                &\implies 
                \set{\MM \in \f{M} \st \MM \model{\Si} \psi(a_\MM)}
                \cap 
                \set{\MM \in \f{M} \st \MM \model{\Si} \chi(a_\MM)}\in \FF\\
                &\iff
                \set{\MM \in \f{M} \st \MM \model{\Si} 
                \psi(a_\MM) \AND \chi(a_\MM)}\in \FF
            \end{align*}
            To make second implication a double implication we note that 
            each of the two sets 
            \[\set{\MM \in \f{M} \st \MM \model{\Si} \psi(a_\MM)} \in \FF
            \text{ and }
            \set{\MM \in \f{M} \st \MM \model{\Si} \chi(a_\MM)}\in \FF\]
            are supersets of the intersection which is in $\FF$.
        \item Without loss of generality we can use $\exists$ instead of 
            $\forall$ to make things simpler.
            \begin{forward}
                Suppose $\NN \model{\Si} \exists v, \psi(\pi(a),v)$.
                Then there exists $b \in \prod_{\MM \in \f{M}} \MM$ such that 
                $\NN \model{\Si} \psi(\pi(a),\pi(b))$.
                Then by induction 
                \[\set{\MM \in \f{M} \st \MM \model{\Si} \psi(a_\MM,b_\MM)}
                \in \FF\]
                This is a subset of 
                \[  
                    \set{\MM \in \f{M} \st \exists c \in \MM, \MM \model{\Si} 
                    \psi(a_\MM,c)} = 
                    \set{\MM \in \f{M} \st \MM \model{\Si} 
                    \exists v, \psi(a_\MM,v)}
                \]
            \end{forward}
            \begin{backward}
                Suppose $Y:=\set{\MM \in \f{M} \st \MM \model{\Si} 
                \exists v, \psi(a_\MM,v)} \in \FF$.
                Then by the axiom of choice we have for each $\MM$
                \[\begin{cases}
                    b_\MM \in \MM, \MM \model{\Si} &\text{if } \MM \in Y\\
                    b_\MM \in \MM   &\MM \notin Y
                \end{cases}\]
                since each $\MM$ is non-empty.
                By induction we have 
                $\NN \model{\Si} \psi(\pi(a),\pi(b))$
                and so $\NN \model{\Si} \exists v, \psi(\pi(a),v)$
            \end{backward}
    \end{itemize}
\end{proof}

\begin{cor}[The Compactness Theorem]
    A $\Si$-theory is consistent if and only if it is finitely consistent.
\end{cor}
\begin{proof}
    Suppose $T$ is finitely consistent.
    For each finite subset $\De \subs T$ we let $\MM_\De$ be the given 
    non-empty model of $\De$,
    which exists by finite consistency.
    We generate an ultrafilter $\FF$ on 
    $\f{M} := \set{\MM_\De \st \De \in I}$ and use 
    \linkto{los_theorem}{Łos's Theorem} to show that
    $\prod \f{M} / \FF$ is a model of $T$.
    Let
    \[I = \set{\De \subs T \st \De \text{ finite}}
    \quad \text{ and } [\star] : I \to \PP(I) := 
    \De \mapsto \set{\Ga \in I \st \De \subs \Ga}\]
    Writing $[I]$ for the image of I,
    we claim that $\FF := \set{U \in \PP(I) \st \exists V \in [I], V \subs U}$
    forms an ultrafilter on $I$ 
    (i.e. an ultrafilter on the Boolean algebra $\PP(I)$).
    Indeed 
    \begin{itemize}
        \item $\nothing \in I$ thus $I = \set{[\nothing]  \in [I] \subs \FF}$.
        \item Suppose $\nothing \in \FF$ then $\nothing \in [I]$ and so there 
            exists $\De \in I$ such that $[\De] = \nothing$.
            This is a contradiction as $\De \in [\De]$.
        \item If $U, V \in \FF$ then there exist $\De_U, \De_V \in I$ 
            such that $[\De_U] \subs U$ and $[\De_V] \subs V$. 
            \begin{align*}
                [\De_U] \cap [\De_V] &
                = \set{\Ga \in I \st \De_0 \subs \Ga 
                \text{ and } \De_1 \subs \Ga}\\
                &= \set{\Ga \in I \st \De_0 \cup \De_1 \subs \Ga}\\
                &= [\De_0 \cup \De_1] \in [I] \subs \FF
            \end{align*}
        \item Closure under superset is clear.
    \end{itemize}
    We identify each $\MM_\De \in \f{M}$ with $\De \in I$ and generate
    the same filter (which we will still call $\FF$) on $\f{M}$ 
    (this is okay as the power sets are isomorphic as Boolean algebras.)
    By \linkto{los_theorem}{Łos's Theorem} $\prod \f{M} / \FF$ is 
    a well-defined $\Si$-structure such that for any $\Si$-sentence $\phi$
    \[\prod \f{M} / \FF \model{\Si} \phi \iff 
    \set{\MM \in \f{M} \st \MM \model{\Si} \phi} \in \FF\]
    Let $\phi \in T$, 
    then $\set{\De \in I \st \set{\phi} \subs \De} \in \FF$ and so
    \[
        \set{\De \in I \st \set{\phi} \subs \De} \subs 
        \set{\De \in I \st \phi \in \De} \subs
        \set{\De \in I \st \MM \model{\Si} \phi} \in \FF
    \]
    The image of this under the isomorphism is 
    $\set{\MM_\De \in X \st \MM \model{\Si} \phi}$ thus is in $\FF$ and so 
    $\prod \f{M} / \FF \model{\Si} \phi$.
\end{proof}
%\subsection{Stone Duality}
This section of the appendix gives a more algebraic way of 
constructing Stone spaces.

\begin{dfn}[A Boolean algebra on $F(\Si,n)$]
    Let $T$ be a $\Si$-theory.
    We quotient out $F(\Si,n)$ by the equivalence relation
    \[  
        \phi \sim \psi \quad := 
        \quad \phi \text{ and } \psi \text{ equivalent modulo } T
        := T \model{\Si} \forall v, (\phi \IFF \psi)
    \]
    Call the projection into the quotient $\pi$
    and the quotient $F(\Si,n) / T$.
    We make $F(\Si,n) / T$ into a Boolean algebra by 
    interpreting $0$ as $\pi(\bot)$, $1$ as $\pi(\top)$, 
    $\pi(\phi) \cap \pi(\psi)$ as $\pi(\phi \AND \psi)$,
    $\pi(\phi) \cup \pi(\psi)$ as $\pi(\phi \OR \psi)$,
    $\NEG \pi(\phi)$ as $\pi(\NOT \phi)$ and 
    $\pi(\phi) \leq \pi(\psi)$ as 
    \[\set{(\pi(\phi),\pi(\psi)) \st T \model{\Si} \forall v, (\phi \to \psi)}\]
    One can verify that these are well-defined and satisfy the axioms of 
    a Boolean algebra.
    Notice we need $T$ (potentially chosen to be the empty set) 
    to make $\to$ look like $\leq$ and that we had to quotient modulo $T$ 
    to make $\leq$ satisfy antisymmetry. 
    Antisymmetry in this context looks very much like 
    `propositional extensionality'.
    Thus it makes sense to consider the Stone space of this Boolean algebra 
    $S(F(\Si,n) / T)$.
\end{dfn}

\begin{lem}
    \link{technical_lem_max_subs_pullback}
    If $p \subs F(\Si,n)$ is a maximal subset 
    ($\forall \phi \in F(\Si,n), \phi \in p$ or $\NOT \phi \in p$) then 
    $\pi(\phi) \in \pi(p)$ in the quotient implies $\phi \in p$.
\end{lem}
\begin{proof}
    If $\pi(\phi) \in \pi(p)$ then there exists $\psi \in p$ such that 
    $\psi$ is equivalent to $\phi$ modulo $T$.
    By consistency with $T$ there exists a non-empty $\Si$-model $\MM$ of $T$
    and $b \in {\MM}^n$ 
    such that $\MM \model{\Si} p(b)$, 
    in particular $\MM \model{\Si} \psi(b)$.
    Equivalence modulo $T$ then gives us that $\MM \model{\Si} \phi(b)$.
    By maximality of $p$, 
    $\phi$ or $\NOT \phi$ is in $p$ but the latter would lead to 
    $\MM \nodel{\Si} \phi(b)$, 
    a contradiction.
\end{proof}

\begin{prop}[The Stone space is a set of ultrafilters]
    The Stone space of a $\Si$-theory $T$ is homeomorphic 
    to the set of ultrafilters from $S(F(\Si,n) / T)$ 
    that have preimage consistent with $T$ with the subspace topology.
    In other words, if $\pi$ is the projection to the quotient then
    $S_n(T) \iso X$, where
    \[X := 
    \set{\FF \in S(F(\Si,n) / T) \st \pi^{-1}(\FF) \text{ consistent with } T}\]
\end{prop}
\begin{proof}
    Warning: this proof uses $\pi$ to be three different things,
    the quotient $F(\Si,n) \to F(\Si,n) / T$, 
    the image map (of the quotient) $S_n(T) \to X$, 
    and the map of clopen sets (the image map of the image map) 
    $\PP(S_n(T)) \to \PP(X)$.
    The second will be a homeomorphism and the third will be a map between
    subsets of the topologies (in particular the clopen subsets).

    We show that sending $p \in S_n(T)$ to its image under the projection 
    to the quotient $\pi(p)$ is a homeomorphism.
    To show that it is well-defined it suffices to show that for any $p$, a 
    maximal $n$-type over $T$, $\pi(p)$ is an ultrafilter of $F(\Si,n) / T$
    with preimage consistent with $T$.
    Preimage being consistent with $T$ 
    follows from the definition of $n$-types over theories.
    To show that it is a proper filter:
    \begin{itemize}
        \item $\top \in p$ by consistency and maximality. 
            Hence $\pi(\top) \in \pi(p)$.
        \item If $\pi(\bot) \in \pi(p)$ then
            \linkto{technical_lem_max_subs_pullback}{$\bot \in p$} 
            which is a contradiction with consistency.
        \item If $\pi(\phi), \pi(\psi) \in \pi(p)$ then 
            \linkto{technical_lem_max_subs_pullback}{$\phi,\psi \in p$} and so
            $\phi \AND \psi \in p$ thus by definition of the Boolean algebra 
            $F(\Si,n) / T$, 
            \[\pi(\phi) \cap \pi(\psi) = \pi(\phi \AND \psi) \in \pi(p)\]
        \item If $\pi(\phi) \in \pi(p)$ and $\pi(\phi) \leq \pi(\psi)$ then 
            \linkto{technical_lem_max_subs_pullback}{$\phi \in p$} 
            and by definition of $\leq$,
            \[T \model{\Si} \forall v, (\phi \to \psi)\]
            Since $p$ is consistent with $T$ there exists a non-empty
            $\Si$-structure $\MM$ and $b \in {\MM}^n$ such that 
            $\MM \model{\Si} \phi(b) \to \psi(b)$ and $\MM \model{\Si} \phi(b)$.
            Hence $\MM \model{\Si} \psi(b)$ and by maximality of $p$ we have 
            $\psi \in p$ and $\pi(\psi) \in \pi(p)$.
    \end{itemize}
    The image $\pi(p)$ is an ultrafilter by the 
    \linkto{negation_is_in_ultrafilter}{equivalent definition}:
    if $\pi(\phi) \in F(\Si,n) / T$ then either 
    $\phi \in p$ or $\NEG \phi \in p$ by maximality of $p$, 
    hence $\pi(\phi) \in \pi(p)$ or $\pi(\NEG \phi) \in \pi(p)$.
    Thus we have $\pi$ is a map into \[
    \set{\FF \in S(F(\Si,n) / T) \st \pi^{-1}(\FF) \text{ consistent with } T}\]

    Injectivity: if $p,q \in S_n(T)$ and $\pi(p) = \pi(q)$ then 
    if $\phi \in p$, we have $\pi(\phi) \in \pi(p) = \pi(q)$ and by our 
    claim above $\phi \in q$.
    Surjectivity: let $\FF \in S(F(\Si,n) / T)$ 
    have its preimage consistent with $T$.
    Then its preimage is an $n$-type.
    If its preimage is a maximal $n$-type then we have surjectivity.
    Indeed since $\FF$ is an ultrafilter if $\phi \in F(\Si,n)$ then 
    $\pi(\phi) \in \FF$ or $\pi(\NOT \phi) = \NEG \pi(\phi) \in \FF$,
    hence $\phi \in \pi^{-1}(\FF)$ or $\NOT \phi \in \pi^{-1}(\FF)$.

    To show that the map is continuous in both directions,
    it suffices to show that images of clopen sets are clopen and preimages of 
    clopen sets are clopen,
    as each topology is generated by their clopen sets.
    For $\phi \in F(\Si,n)$ since 
    $\pi : S_n(T) \to X$ is a bijection we have that
    \[\pi([\phi]_T) = \set{\FF \in X \st \phi \in \pi^{-1}(\FF)} =
    \set{\FF \in F(\Si,n) / T \st \pi(\phi) \in \FF \text{ and } \FF \in X} = 
    [\pi(\phi)] \cap X\]
    and similarly $\pi^{-1}([\pi(\phi)] \cap X) = [\phi]_T$.
    Hence there is a correspondence between clopen sets.
\end{proof}

\begin{lem}[Topological consistency]
    \link{topological_consistency}
    Let $\FF \in S(F(\Si,n) / T)$ and $T$ be a $\Si$-theory.
    $\pi^{-1}(\FF)$ is consistent with $T$ if and only if 
    \[\pi^{-1}(\FF) \in \bigcap_{\phi \in \pi^{-1}(\FF)} [\phi]_T\]
    if and only if 
    \[\bigcap_{\phi \in \pi^{-1}(\FF)} [\phi]_T \text{ is non empty}\]
\end{lem}
\begin{proof}
    ($1. \implies 2. \implies 3.$)
        Suppose $\pi^{-1}(\FF)$ is consistent with $T$.
        Then $\pi^{-1}(\FF) \in S_n(T)$ thus for any $\phi \in \pi^{-1}(\FF)$, 
        $\pi^{-1}(\FF) \in [\phi]_T$. 
        Hence 
        \[\pi^{-1}(\FF) \in \bigcap_{\phi \in \pi^{-1}(\FF)} [\phi]_T\]
        and it is non-empty.

    ($3. \implies 1.$)
        Suppose 
        \[p \in \bigcap_{\phi \in \pi^{-1}(\FF)} [\phi]_T\]
        then $\forall \phi \in \pi^{-1}(\FF), \phi \in p$.
        As $\FF$ is an ultrafilter, for any $\phi \in p$, 
        \[\phi \notin \pi^{-1}(\FF) \implies \NOT \phi \in \pi^{-1}(\FF) 
        \implies \NOT \phi \in p \text{ a contradiction}\]
        Hence $p = \pi^{-1}(\FF)$.
        Hence $\pi^{-1}(\FF) \in S_n(T)$ and thus is consistent with $T$.
\end{proof}

\begin{prop}[Topological compactness implies compactness for types]
    \link{compactness_for_types_2}
    Let $\FF \in S(F(\Si,n)/T)$ and $T$ be a $\Si$-theory.
    Then $\pi^{-1}(\FF)$ is consistent with $T$ 
    if and only if $\pi^{-1}(\FF)$ if finitely consistent
    with $T$.
\end{prop}
\begin{proof}
    By definition $\pi^{-1}(\FF)$ is finitely consistent with $T$ if and only if 
    any finite subset of $\pi^{-1}(\FF)$ is consistent with $T$.
    Translating this to the topology, this is
    \linkto{topological_consistency}{if and only if} 
    for any finite subset $\De \subs \pi^{-1}(\FF)$,
    \[\bigcap_{\phi \in \De} [\phi]_T \text{ is non empty}\]
    By \linkto{stone_of_theo_compact}{topological compactness of $S_n(T)$} 
    this is if and only if 
    \[\bigcap_{\phi \in \pi^{-1}(\FF)} [\phi]_T \text{ is non empty}\]
    \linkto{topological_consistency}{Translating this back to model theory}
    this is if and only if $\pi^{-1}(\FF)$ is consistent with $T$.
\end{proof}
%\section{Ordinals}
\cite{jech}

\begin{dfn}[Linear ordering, well-ordering, transitive]
    Let $X$ be a set (or class or whatever) 
    \linkto{partial_ordering}{partially ordered}
    by $\leq$. 
    It is a linearly ordered if for each $x, y \in X$, 
    $x \leq y$ or $y \leq x$.
    It is a well-ordered if each non-empty subset $S \subs X$ 
    has a least element,
    i.e. there exists $s \in S$ such that for any $x \in S$, $s \leq x$.

    A set $X$ is transitive if for any member of $X$ is a subset of $X$.
    An example of transitive sets are the naturals:
    \[0 := \nothing, 1 := \set{0}, 2:= \set{0,1}, \dots\]

\end{dfn}

\begin{ex}[Transitive]
    The reason for the use of the word transitive is due 
    to its equivalent definition: 
    $x$ is transitive if for any $y \in x$ and any $z \in y$, $z \in x$.
\end{ex}


\begin{ex}[Well-orderings are linear]
    \link{well_ord_to_lin}
    Take two elements and consider
    the set containing these two elements. 
    This has a minimum.
\end{ex}

\begin{dfn}[Ordinal, succesion, limit ordinal]
    A set in an ordinal if it is transitive and well-ordered by taking 
    $<$ to be $\in$.
    The class of ordinals is denoted $\ord$.

    If $\al \in \ord$ then $\al + 1$ is defined to be $\al \cup \set{\al}$,
    $\al + 1$ is then called a successor ordinal.
    Check that the successor of an ordinal is an ordinal.
    If $\al \in \ord$ is not a successor ordinal then 
    it is called a limit ordinal.

    We endow $\ord$ with an ordering given by 
    $\al \leq \be$ if and only if $\al \subs \be$.
\end{dfn}

\begin{lem}[Transitivity of $<$]
    \link{transitivity_of_less_than}
    Suppose $\leq$ is a partial order.
    The relation $<$ (defined by $\al < \be$ if and only if $\al \leq \be$ and 
    $a \ne b$) is transitive.

    Hence $\al \leq \be < \ga$ implies $\al < \ga$.
\end{lem}
\begin{proof}
    Suppose $a < b$ and $b < c$.
    Then $a \leq b$ and $b \leq c$ and by transitivity of $\leq$ we have $a \leq c$.
    It remains to show that $a \ne c$.
    Suppose $a = c$, 
    then $a \leq b$ and $b \leq a$ and by antisymmetry $a = b$.
    This is false since $a < b$ and so $a \ne b$ by definition.
\end{proof}

\begin{prop}[Basic facts about ordinals]
    \link{basic_facts_ordinals}
    Most importantly, we show that $\ord$ is well-ordered.
    \begin{enumerate}
        \item $0$ is a limit ordinal.
        \item Subsets of ordinals are well-ordered by $\in$.
        \item The intersection of a non-empty subclass of is an ordinal.
        \item Ordinals are closed under membership: if $\al \in \ord$
            and $\be \in \al$ then $\be \in \ord$.
        \item Let $\al \ne \be$ be ordinals. 
            If $\al \subs \be$ then $\al \in \be$.
        \item $\ord$ is well-ordered by $\subs$
        (which is by the fifth part the same thing as $=$ or $\in$).
    \end{enumerate}
\end{prop}
\begin{proof}~
    \begin{enumerate}
        \item $0$ is trivially well-ordered and transitive.
            If $0 = \al \cup \set{\al}$ then $\al \in 0$, 
            which is a contradiction.
        \item Let $\al \in \ord$ and $S \subs \al$.
            By moving elements into $\al$ we see that $S$ 
            inherits the linear ordering given by $\in$.
            Any subset of $S$ is a subset of $\al$,
            hence any subset has a minimal element.
            Thus $S$ is well-ordered
        \item Let $S$ be a non-empty class of ordinals, containing $\al$, say.
            $\bigcap S$ is a subset of $\al$ thus is
            well-ordered by the second part. 
            It is transitive: any element in the intersection satisfies 
            \[\forall \be \in S, x < \be\] 
            hence for every $\be \in S$, as $\be$ is transitive, $x \subs \be$.
            Thus $x \subs \bigcap S$.
        \item Let $\al \in \ord$ and $\be \in \al$.
            As $\al$ is transitive $\be \subs \al$.
            By the second part we have that $\be$ is well-ordered by $<$.

            It remains to show that $\be$ is transitive.
            Let $\ga \in \be$ and suppose for a contradiction that 
            $\ga \nsubseteq \be$.
            Then there exists $\de \in \ga$ such that $\de \notin \be$.
            By \linkto{transitivity_of_less_than}{transitivity of $<$}
            (in this case $\in$), 
            $\de \in \ga \in \be \in \al$ implies $\de \in \al$.
            
            As $\al$ is linearly-ordered we have that $\be \leq \de$ or 
            $\de \leq \be$, equivalently three cases
            $\be \in \de$ or $\be = \de$ or $\de \in \be$.
            The last case is false by assumption. 
            Then first case gives $\de \in \ga \in \be \in \de$,
            hence $\de < \de$ by \linkto{transitivity_of_less_than}{
                transitivity of $<$}.
            In the second case $\de \in \ga \in \be  = \de$,
            hence by transitivity again $\de < \de$.
            In either case $\de \ne \de$, a contradiction.
        \item Let $\ga$ be the minimum element of $\be \setminus \al$,
            using well-ordering of $\be$.
            Then we claim that $\al = \ga$, which implies $\al \in \be$.
            Indeed if $x \in \al$ then $x \in \be$ by assumption.
            By linearity of $\be$, 
            $\ga \leq x$ or $x < \ga$.
            In the first case we have by 
            \linkto{transitivity_of_less_than}{transitivity of $<$} that 
            $\ga < \al$, which is a contradiction as 
            $\ga \in \be \setminus \al$.
            Thus $x \in \ga$ and so $\al \subs \ga$.

            On the other hand suppose $x \in \ga$ and $x \notin \al$.
            Then $x \in \be \setminus \al$ and so by minimality $\ga \leq x$.
            This is a contradiction as $\ga \leq x < \ga$ and by transitivity
            we have $\ga < \ga$.

            We can see $\al$ as the initial segment $\be$ given by $\ga$, 
            i.e. $\set{x \in \be \st x < \ga}$.
        \item Reflexivity, antisymmetry and transitivity are clear.
            It \linkto{well_ord_to_lin}{suffices to show} 
            that it is a well-ordering.
            Let $S$ be a non-empty set of ordinals.
            $\bigcap_{\al \in S} \al$ is an ordinal by the third part.
            We want to show that there exists $\al \in S$ such that 
            $\bigcap S = \al$.
            Suppose not, then by the fifth part we have for any $\al \in S$,
            $\bigcap S \in \al$.
            Hence $\bigcap S \in \bigcap S$, which is a contradiction.
    \end{enumerate}
\end{proof}

\begin{prop}[Transfinite induction]
    \link{transfinite_induction}
    Let $C \subs \ord$ be defined inductively:
    \begin{itemize}
        \item If $\al$ is a limit ordinal and $\forall \be < \al, \be \in C$
        then $\al \in C$. (In particular $0 \in C$.)
        \item If $\al \in C$ then $\al + 1 \in C$.
    \end{itemize}
    Then $C = \ord$. 
    We have the first constructor as the limit case because 
    the smallest ordinal $0$ is a limit ordinal.

    (`Strong') Let $D \subs \ord$ be defined with a single constructor:
    \begin{itemize}
        \item If $\al \in \ord$ and $\forall \be < \al, \be \in C$
        then $\al \in C$.
    \end{itemize}
    Then $D = \ord$.
\end{prop}
\begin{proof}
    Suppose $C \ne \ord$.
    Then as \linkto{basic_facts_ordinals}{$\ord$ is well-ordered},
    there exists $\be \in \ord$ that is the least ordinal such that 
    $\be \notin C$.
    If $\be$ is a successor $\al + 1$ then by minimality of $\be$, $\al \in C$ 
    and applying the first property gives $\be \in C$,
    which is a contradiction. 
    Otherwise, $\be$ is a limit ordinal. 
    Thus by minimality
    $\forall \al < \be, \al \in C$.
    Applying the second property of $C$ gives $\be \in C$.

    Check that the $C \subs D$ which implies $D = \ord$.
\end{proof}

Here is an example of transfinite induction in use:
\begin{lem}[Less than and the successor]
    \link{less_than_and_succ_of_ord}
    If $\al < \be$ are ordinals then $\al + 1 \in \be$ or $\al + 1 = \be$.
\end{lem}
\begin{proof}
    We induct on $\be$.
    Suppose $\be$ is a limit ordinal.
    Then $\al + 1 \notin \be$ implies $\be \leq \al + 1$ by $\ord$ being 
    \linkto{basic_facts_ordinals}{well-ordered}.
    This implies $\al < \al$ which is a contradiction.
    Hence in this case $\al + 1 \in \be$.

    Suppose $\be = \ga + 1$.
    then $\al < \ga$ or $\al = \ga$.
    In the second case $\al + 1 = \ga + 1 = \be$ and we are done.
    In the first case, by the induction hypothesis $\al + 1 \in \ga$ or 
    $\al + 1 = \ga$. 
    In either case $\al + 1 \in \be$.
\end{proof}


To define a function on $\ord$, it then suffices to define a function on $C$, 
using the recursor (in Type theory) of $C$ or of $D$.
For our purposes it means that only need to say what the function does 
to successor and limit ordinals, 
or only what it does to ordinals given the images of all smaller ordinals.

\begin{prop}[Transfinite recursion]
    We denote the class of all sets by $V$.
    Let $G : V \to V$.
    Then there exists a unique map $F : \ord \to V$ such that 
    for each ordinal $\al$, 
    \[F(\al) = G(\res{F}{\al})\]
    $F$ can be thought of as a sequence indexed by $\ord$.\footnote{As 
        we are in set theory it makes 
        sense to consider $\res{F}{\al}$ as a set.}
\end{prop}
\begin{proof}
    We show by \linkto{transfinite_induction}{transfinite induction}
    that for each ordinal $\al$ there exists a unique $F_\al : \al \to V$
    such that for each $\xi < \al$ 
    \[F_\al(\xi) = G(\res{F_\al}{\xi})\]
    \begin{itemize}
        \item If $\al$ is a successor ordinal and there exists a unique 
            $F_\be : \be \to V$ such that $\forall \xi < \be$,
            \[F_\be(\xi) = G(\res{F_\be}{\xi})\]
            Then take $F_\al : \al \to V$ such that it restricts to $F_\be$
            on $\be$ and maps $\be \mapsto G(F_\be)$.
            $F_\al$ is the unique map that satisfies 
            $\forall \xi < \al,F(\xi) = G(\res{F}{\xi})$ since its restriction
            to $\be$ is the unique. 
        \item If $\al$ is a limit ordinal and for any $\be < \al$ there exists a
            unique $F_\be : \be \to V$ satisfying $\forall \xi < \be$
            \[F_\be(\xi) = G(\res{F_\be}{\xi})\]
            Define $F_\al : \al \to V$ as the union of all the $F_\be$.
            $F_\al$ is well-defined since all of the $F_\be$ agree upon restriction
            by uniqueness.
            $F_\al$ satisfies $\forall \xi < \al, F_\al(\xi) = G(\res{F}{\xi})$ by 
            construction.
            $F_\al$ is the unique map satisfying this: if $H$ satisfies the same
            then for any $\be \in \al$, 
            $\res{F_\al}{\be}$ (similarly $\res{H}{\be}$) 
            is the unique map $f$ satisfying 
            \[\forall \xi < \be, f(\xi) = G(\res{f}{\xi})\]
            Thus for any $\be \in \al$, $\res{F_\al}{\be} = \res{H}{\be}$
            \[F_\al(\be) = G(\res{F_\al}{\be}) = G(\res{H}{\be}) = H(\be)\]
            and so $F_\al = H$.
    \end{itemize}
    We define $F$ to map any ordinal $\al$ to $G(F_\al)$.
    By \linkto{transfinite_induction}{strong induction} on 
    $\al$ we show that $F_\al = \res{F}{\al}$:
    suppose for any $\be < \al$, $F_\be = \res{F}{\be}$,
    then for any $\be < \al$
    \[F_\al(\be) = G(\res{F_\al}{\be}) = G(F_\be) = F(\be) = \res{F}{\al}(\be)\]
    and so $F_\al = \res{F}{\al}$ and for any $\al$,
    \[F(\al) = G(F_\al) = G(\res{F}{\al})\]
    Lastly $F$ is unique: if $H$ also satisfies the conditions then suppose
    $F(\be) = H(\be)$ for all $\be < \al$.
    This implies $\res{F}{\al} = \res{H}{\al}$ and
    \[F(\al) = G(\res{F}{\al}) = G(\res{H}{\al}) = H(\al)\]
    thus again by strong induction on $\al$ we have $F = H$.
\end{proof}
%\subsection{Pregeometries}
This subsection walks through the basics of pregeometries,
leading to the theorem on bases having the same cardinality, 
allowing us to define dimension.
It follows the lecture notes by Prof. C. Ward Henson \cite{henson}.

\begin{dfn}[Pregeometry]
    \link{pregeometry_dfn}
    Let $X$ be a set and $\cl$ be a map from the power set of $X$ to itself,
    called the `closure map'.
    $(X,\cl)$ is a pregeometry the following hold
    \begin{enumerate}
        \item $\cl$ is a morphism of ordered sets: 
            for any $U \subs V \subs X$, $\cl(U) \subs \cl(V)$.
        \item Idempotence: for any $U \subs X$, $\cl(\cl(U)) = \cl(U)$.
        \item Finite character: if $U \subs X$ and $a \in \cl(U)$ then 
            there exists a finite subset $F \subs U$ such that $a \in \cl(F)$.
        \item Exchange: if $a \in \cl(U \cup \set{b})$
            then $a \in \cl(U)$ or $b \in \cl(U \cup \set{a})$.
    \end{enumerate}
\end{dfn}

\begin{dfn}[Span, independence, basis]
    Let $(X,\cl)$ be a pregeometry.
    Let $U,V \subs X$.
    \begin{itemize}
        \item $U$ spans $V$ if $U \subs V$ and $\cl(U) = \cl(V)$.
        \item $U$ is independent if for any $a \in U$, 
            $a \notin \cl(U \setminus \set{a})$.
        \item $U$ is a basis of $V$ if $U$ spans $V$ and is independent.
    \end{itemize}
\end{dfn}

\begin{prop}[Independence and span]
    \link{independence_and_span_basic}
    Let $(X,\cl)$ be a pregeometry, $a \in X$ and $U,V \subs X$.
    \begin{enumerate}
        \item If $U$ is independent then any subset of $U$ is independent.
        \item If $U$ is independent and $a \notin \cl(U)$ then then 
            $U \cup \set{a}$ is independent.
        \item $U$ is independent if and only if every finite subset of $U$ is 
            independent.
        \item $U$ spans $V$ if and only if there exists a subset of 
            $U$ that is a basis of $V$. 
        \item $U$ is a basis of $V$ if and only if $U$ 
            is a maximally independent subset of $V$
            if and only if $U$ is a minimally spanning subset of $V$, 
            i.e. for any independent $W \subs V$, $U \not\subset W$;
            for any spanning $W \subs V$, $W \not\subset U$.
        \item Any independent subset $U$ is contained in a basis.
    \end{enumerate}
\end{prop}
\begin{proof}~
    \begin{enumerate}
        \item Let $a \in A \subs U$. 
        Suppose for a contradiction 
        $a \in \cl(A \setminus \set{a})$ then as $\cl$ is a morphism of 
        ordered sets $a \in \cl(U \setminus \set{a})$, 
        which is false as $U$ is independent.
        Hence $A$ is independent.
        \item The forward direction follows from the part $1$. 
        
        \begin{backward}
            Let $a \in U$ and suppose for a contradiction 
            $a \in \cl(U \setminus \set{a})$. 
            As $\cl$ has `finite character' there exists a finite subset 
            $F \subs U \setminus \set{a}$ such that $a \in \cl(F)$.
            This is a contradiction as every subset of $U$ is independent
            so $F$ is independent.
        \end{backward}

        \item Let $x \in U \cup \set{a}$.
        Suppose for a contradiction that 
        $x \in \cl(U \cup \set{a} \setminus \set{x})$.
        We will show that $a \in \cl(U)$, which is false.
        If $x = a$ then $a \in \cl(U \cup \set{a} \setminus \set{a}) = \cl(U)$
        and we are done.
        Otherwise $x \in U$ then by the exchange property
        we have $x \in \cl(U \setminus \set{x})$ or $a \in \cl(U)$.
        The first case is false since $U$ is independent,
        hence we are done.

        \item The backward direction follows from the fact that 
        $\cl$ is a morphism of ordered sets.

        \begin{forward}(Zorn)
            Suppose $U$ spans $V$ and consider 
            \[\set{W \subs U \st W \text{ is independent}}\]
            This set is non-empty since the empty set is independent.
            Let $C$ be a chain in the set. 
            Then $\bigcup C$ is a subset of $U$.
            It is independent because every finite subset of it is independent
            by part $3$.
        \end{forward}
        \item We show that $U$ is basis of $V$ if and only if it is a 
            maximally independent subset of $V$.

            \begin{forward}
                Suppose for a contradiction that
                $U \subset W$ and $W$ is independent.
                Let $a \in W \setminus U$.
                Since $U$ is a basis 
                $a \in \cl(U) \subs \cl(W \setminus \set{a})$, 
                contradicting independence of $W$.
            \end{forward}

            \begin{backward}
                Suppose $U$ is maximally independent.
                Then to show $U$ is also spanning we show that $V \subs \cl(U)$.
                Let $a \in V$. 
                If $a \in U$ then we are done.
                If $a \notin U$ then $U \cup \set{a}$ is dependent by maximality
                and so there exists $b \in U \cup \set{a}$ such that 
                $b \in \cl(U \cup \set{a} \setminus \set{b})$.
                In the case that $b \in U$ by independence of $U$ and the 
                exchange property
                \[b \in \cl(U \setminus {b} \cup \set{a}) \AND 
                b \notin \cl(U \setminus \set{b}) \implies 
                a \in \cl(U)\]
                In the case that $b = a$ then 
                \[a \in \cl(U \cup \set{a} \setminus \set{a}) = \cl(U)\]
            \end{backward}

            Now we show that $U$ is a basis of $V$ if and only if it is a 
            minimally spanning subset of $V$.

            \begin{forward}
                Suppose for a contradiction we have $W \subset U$ such that 
                $W$ spans $V$.
                Then let $a \in U \setminus W$.
                Since $W$ is spanning we have 
                \[a \in \cl(W) \subs \cl(U \setminus \set{a})\]
                contradicting independence of $U$.
            \end{forward}

            \begin{backward}
                Let $U$ be a minimally spanning subset.
                To show it is independent, 
                let $a \in U$ and suppose for a contradiction 
                $a \in \cl(U \setminus \set{a})$.
                Then $U \subs \cl(U \setminus \set{a})$ and so 
                \[V \subs \cl(U) \subs \cl(U \setminus \set{a})\]
                and $U \setminus \set{a}$ spans $V$
                which contradicts the minimality of $U$.
            \end{backward}
        \item (Zorn) Consider 
            $\set{V \subs X \st U \subs V \text{ and } V \text{ independent}}$.
            This is non-empty as it contains $U$.
            The union of any chain in it independent if and only if 
            every finite subset of it is independent if and only if each 
            set in it is independent by part $3$.
    \end{enumerate}
\end{proof}

\begin{lem}[Support]
    \link{support_pregeometry}
    Let $(X,\cl)$ be a pregeometry and let $U$ be a basis of $V \subs X$.
    Then for any $a \in \cl(V)$ there exists a minimal finite subset $F \subs U$
    such that $a \in \cl(F)$, 
    i.e. if $a \in \cl(G)$ for some $G \subs V$ then $F \subs G$.
    We call $F$ the support of $a$ in the basis $U$.
\end{lem}
\begin{proof}
    Since $\cl$ has finite character we have that there exists a finite subset 
    of $U$ containing $a$ in its closure, 
    and by well ordering of the naturals (and choice) we can take one such 
    subset $F$ with minimal cardinality.
    Let $G$ be another subset containing $a$ in its closure.
    Suppose for a contradiction there exists $b \in F \setminus G$.
    Then $a \in \cl(G) \subs \cl(U \setminus \set{b})$.
    Hence 
    \[
        U \setminus \set{b} \cup \set{a} \subs \cl(U \setminus \set{b})
        \implies 
        \cl(U \setminus \set{b} \cup \set{a}) = \cl(U \setminus \set{b})
    \]
    We show that this contradicts the independence of $U$.
    Since $b \in F$ and $F$ is minimal,
    $a \notin \cl(F \setminus \set{b})$ and by exchange 
    \[
        b \in \cl(F \setminus \set{b} \cup \set{a}) 
        \subs \cl(U \setminus \set{b} \cup \set{a}) 
        = \cl(U \setminus \set{b})
    \]
    contradicting the independence of $U$.
\end{proof}

\begin{prop}[Existence of dimension]
    \link{existence_of_dim}
    Let $(X,\cl)$ be a pregeometry.
    Then for any $U$ there exists a basis of $U$.
\end{prop}
\begin{proof}
    We apply Zorn's lemma.
    The set $\set{W \subs U \st W \text{ is independent}}$
    contains the empty set so is non-empty.
    Let $C$ be a chain in there.
    Then $\bigcup C$ is a subset of $U$.
    $\bigcup C$ is independent 
    \linkto{independence_and_span_basic}{if and only if} 
    every finite subset of it is independent, 
    which is true since any part of the chain is independent.
    Hence there exists a maximally independent subset of $U$,
    \linkto{independence_and_span_basic}{i.e. a basis of $U$}.
\end{proof}

\begin{prop}[Uniqueness of dimension]
    \link{uniqueness_of_dim}
    Let $(X,\cl)$ be a pregeometry.
    Then if $U$ and $V$ are bases of $Y \subs X$ then $\abs{U} = \abs{V}$.
\end{prop}
\begin{proof}
    Without loss of generality $\abs{U} \leq \abs{V}$.
    It suffices to show that $\abs{V} \leq \abs{U}$.
    We case on whether $V$ is finite or not.

    If $V$ is infinite then we reconstruct $V$ in the following way:
    for each $a \in U$ there exists a finite subset $F_a$ of $V$ such that 
    $a \in \cl(F_a)$.
    By choice we can construct $(F_a)_{a \in U}$ and take the union 
    $\bigcup_{a \in U} F_a$.
    For any $a \in U$, the union satisfies 
    $a \in \cl(F_a) \subs \cl(\bigcup F_a)$
    thus $U \subs \cl(\bigcup F_a)$ and
    \[Y \subs \cl(U) \subs \cl(\bigcup F_a)\]
    Since $\bigcup F_a$ is a subset of $V$ that spans $Y$, 
    and bases are \linkto{independence_and_span_basic}{minimally spanning sets}, 
    we have $V = \bigcup F_a$.

    Suppose for a contradiction $U$ is finite. 
    Then $V = \bigcup_{a \in U} F_a$ would be a finite union of finite sets 
    which is finite,
    contradicting our assumption about $V$.
    Hence 
    \[
        \abs{V}  = \abs{\bigcup_{a \in U} F_a} 
        \linkto{infinite_union_of_finite}{\leq \abs{U}}
    \]
    Thus we are done.

    If $V$ is finite then we use induction on $n \in \N$ to show that 
    if $V_0$ is a independent set of cardinality $n$ then there exists 
    $W \subs U$ such that $V_0$ and $W$ are disjoint, 
    $V_0 \cup W$ is a basis of $Y$ 
    and $\abs{V_0 \cup W} = \abs{U}$.
    This will imply the existance of such a $W$ for $V$ such that 
    \[\abs{V} \leq \abs{V \cup W} = \abs{U},\]
    completing the proof.

    For the base case we take $W = U$.
    For the induction step we suppose $V_0$ is non-empty (containing $a$) and 
    obtain via the induction hypothesis 
    $W \subs U$ such that $V_0 \setminus \set{a}$ and $W$ are disjoint, 
    $V_0 \setminus \set{a} \cup W$ is a basis of $Y$ 
    and $\abs{V_0 \setminus \set{a} \cup W} = \abs{U}$.
    If $a \in W$ then we can take our new $W$ to be $W \setminus \set{a}$
    and see that $V_0$ and $W \setminus \set{a}$ are disjoint, 
    $V_0 \cup W \setminus \set{a}$ is a basis of $Y$ 
    and $\abs{V_0 \cup W \setminus \set{a}} = \abs{U}$.
    
    Otherwise, take $F$ the 
    \linkto{support_pregeometry}{support of $a$ in the basis 
        $V_0 \setminus \set{a} \cup W$}.
    Supposing for a contradiction that $F \cap W$ were empty,
    we get $F \subs V_0 \setminus \set{a}$, which implies 
    $a \in \cl(F) \subs \cl(V_0 \setminus \set{a})$, 
    contradicting independence of $V_0$.
    Hence there exists $b \in F \cap W$.

    We claim $W \setminus \set{b}$ is our new $W$.
    They are disjoint: since $V_0 \setminus \set{a} \cap W$ is empty 
    and we are in the case where $a \notin W$, we have $V_0 \cap W$ is empty
    and so $V_0 \cap W \setminus \set{b}$ is empty.

    To show that their union is independent:
    we first note that as subsets of independent sets are independent
    \[(V_0 \setminus \set{a}) \cup (W \setminus \set{b})
    = (V_0 \cup W \setminus \set{b}) \setminus \set{a}\]
    is independent.
    \linkto{independence_and_span_basic}{It suffices to show that}
    \[a \notin \cl((V_0 \setminus \set{a}) \cup (W \setminus \set{b}))\]
    Suppose for a contradiction 
    $a \in \cl((V_0 \setminus \set{a}) \cup (W \setminus \set{b}))$.
    Then by minimality of $F$ we have 
    \[
        b \in F \subs (V_0 \setminus \set{a}) \cup (W \setminus \set{b})
        \implies 
        b \in V_0 \setminus \set{a}
        \implies
        V_0 \setminus \set{a} \cap W \ne \nothing
    \]
    The last of which is a contradiction.

    To show that their union is spanning it suffices to show that 
    $b \in \cl(V_0 \cup W \setminus \set{b})$ since this implies 
    \[V_0 \setminus \set{a} \cup W \subs V_0 \cup W 
    \subs \cl(V_0 \cup W \setminus \set{b})\]
    and so $Y \subs \cl(V_0 \setminus \set{a} \cup W)
    \subs \cl(V_0 \cup W \setminus \set{b})$.
    Suppose $b \in V_0$, 
    then $b \in \cl(V_0 \cup W \setminus \set{b})$ and we are done.
    Otherwise 
    \[
        (V_0 \cup W  \setminus \set{b}) \setminus \set{a} =
        (V_0 \setminus \set{a}) \cup (W \setminus \set{b}) = 
        (V_0 \setminus \set{a} \cup W) \setminus \set{b}
    \]
    Since $V_0 \cup W \setminus \set{b}$ is independent
    $a \notin \cl(V_0 \setminus \set{a} \cup W \setminus \set{b})$. 
    Thus by the exchange principle
    \[a \in \cl((V_0 \setminus \set{a}) \cup (W \setminus \set{b}) \cup \set{b}) 
    \implies 
    b \in \cl(V_0 \cup W \setminus \set{b})\]
    and so the union is spanning.

    Finally we note that 
    \[\abs{V_0 \cup W \setminus \set{b}} = \abs{V_0} + \abs{W} - 1 = 
    \abs{V_0 \setminus \set{a} \cup W} = \abs{U}\]
    and the proof is complete.
\end{proof}

\begin{dfn}[Dimension]
    \link{dimension_dfn}
    Let $U$ be a subset of $(X,\cl)$ a pregeometry.
    Then the dimension of $U$ is defined to be the cardinality of any basis of 
    $U$.
    The \linkto{existence_of_dim}{existence} and 
    \linkto{uniqueness_of_dim}{uniqueness} 
    of dimension are given by the previous two theorems.
\end{dfn}
%
\chapter{Model Theory of Fields}
\section{Ax-Grothendieck}
This section studies the theories of fields in the language of rings,
with particular focus on algebraically closed fields.
\subsection{Language of Rings}
We introduce rings and fields and construct the field of fractions
of integral domains to see the models in action.

\begin{dfn}[Signature of rings, theory of rings]
    \link{dfn_rings}
    We define 
    $\Si_\RNG := (\set{0,1},\set{+,-,\cdot},n_\star,\nothing,m_\star)$ 
    to be the signature of rings, 
    where
    $n_+ = n_\cdot = 2$, $n_- = 1$ and $m_\star$ is the empty function.

    Using the obvious abbreviations
    $x + (-y) = x - y, x \cdot y = xy$ and so on,
    we define the theory of rings $\RNG$ as the set containing:
    \begin{itemize}
        \item[$\vert$] Assosiativity of addition: 
            $\forall x \forall y \forall z, (x + y) + z = x + (y + z)$
        \item[$\vert$] Identity for addition:
            $\forall x, x + 0 = x$ 
        \item[$\vert$] Inverse for addition: $\forall x, x - x = 0$ 
        \item[$\vert$] Commutativity of addition: 
            $\forall x \forall y, x + y = y + x$
        \item[$\vert$] Assosiativity of multiplication: 
        $\forall x \forall y \forall z, 
        (x \cdot y) \cdot z = x \cdot (y \cdot z)$
        \item[$\vert$] Identity for multiplication:
        $\forall x, x \cdot 0 = x$ 
        \item[$\vert$] Commutativity of multiplication: 
        $\forall x \forall y, x \cdot y = y \cdot x$
        \item[$\vert$] Distributivity:
        $\forall x \forall y \forall z, x \cdot (y + z) = x\cdot y + x \cdot z$
    \end{itemize}
    Note that we don't have axioms for closure of functions 
    and existence or uniqueness of inverses as
    it is encoded by interpretation of 
    $+, -, \cdot$ being well-defined.
    Note that the theory of rings is universal.
\end{dfn}

\begin{dfn}[Theory of integral domains and fields]
    We define the $\Si_\RNG$-theory of integral domains
    \[\ID := \RNG \cup \set{0 \ne 1,
    \forall x \forall y, xy = 0 \to (x = 0 \OR y = 0)}\]
    and the $\Si_\RNG$-theory of fields
    \[\FLD := \RNG \cup \set{\forall x, x = 0 \OR \exists y, xy = 1}\]
    Note that the theory of integral domains is universal but the 
    theory of fields is not.
\end{dfn}

\begin{prop}[Field of fractions]
    \link{field_of_fractions}
    Suppose $\AA \model{\Si_\RNG} \ID$.
    Then there exists an $\Si_\RNG$-embedding $\io : \AA \to \BB$
    such that $\BB \model{\Si_\RNG} \FLD$.
    We call $\BB$ the field of fractions.
\end{prop}
\begin{proof}
    We construct $X = \set{(x,y) \in {{\AA}}^2 \st y \ne 0}$ and
    and equivalence relation $(x,y) \sim (v, w) \iff xw = yv$. 
    (Use $\AA \model{\Si_\RNG} \ID$ 
    to show that this is an equivalence relation.)
    Let ${\BB} = X / \sim$ with $\pi : X \to \BB$ as the quotient map. 
    Denote $\pi(x,y) := \frac{x}{y}$, 
    interpret $\modintp{\BB}{0} = \frac{\modintp{\AA}{0}}{\modintp{\AA}{1}}$ 
    and $\modintp{\BB}{1} = \frac{\modintp{\AA}{1}}{\modintp{\AA}{1}}$.
    Interpret $+$ and $\cdot$ as standard fraction addition and multiplication
    and use $\AA \model{\Si_\RNG} \ID$ to check that these are well defined.

    Check that $\BB$ is an $\Si_\RNG$ structure and that 
    $\BB \model{\Si_\RNG} \FLD$.
    Define $\io : {\AA} \to {\BB} := a \mapsto \frac{a}{1}$
    and show that this well defined and injective.
    Check that $\io$ is a $\Si_\RNG$-morphism
    and note that since there are no relation symobls in $\Si_\RNG$
    it is also an embedding.
\end{proof}

\begin{prop}[Universal property of field of fractions]
    \link{uni_prop_field_of_fractions}
    Suppose $A \model{\Si_\RNG} \ID$ and $K$ its field of fractions.
    Then if $L \model{\Si_\RNG} \FLD$ and there exists a
    $\Si_\RNG$-embedding $\io_L : A \to L$, 
    then there exists a unique $\Si_\RNG$-embedding $K \to L$
    that commutes with the other embeddings:
    \begin{cd}
        A \ar[r] \ar[dr] & K \ar[d, dashed]\\
        & L
    \end{cd}
\end{prop}
\begin{proof}
    Define the map $\io : K \to L$ sending 
    $\frac{a}{b} \mapsto \frac{\io_L (a)}{\io_L(b)}$.
    Check that this is well-defined and a $\Si_\RNG$-morphism.
    It is injective because $\io_L$ is injective:
    \[\frac{\io_L(a)}{\io_L(b)} = 0 \implies \io_L(a) = 0
    \implies a = 0\]
    Thus it is an embedding.

    It is unique: suppose $\phi : K \to L$ is a $\Si_\RNG$-embedding
    that commutes with the diagram.
    Then for any $a \in K$, 
    $\phi(\frac{a}{1}) = \io_L(a) = \io (\frac{a}{1})$.
    Since both $\phi, \io$ are embeddings
    they commute with taking the inverse for $a \ne 0$:
    $\phi(\frac{1}{a}) = \io (\frac{1}{a})$.
    Since any element of $K$ can be written as $\frac{a}{b}$,
    we have shown that $\phi = \io$.
\end{proof}

\subsection{Algebraically closed fields}
\begin{dfn}[Theory of algebraically closed fields]
    We define the $\Si_\RNG$ theory of algebraically closed fields
    \[
        \ACF := 
        \FLD \cup \set{\bigforall{i = 0}{n - 1} a \exists x, \;
        x^n + \sum_{i = 0}^{n-1} a_i x^i = 0
        \st n \in \N_{>0}, a \in {\var{\Si_\RNG}^{n-1}}}
        \]
    Unlike the theories $\RNG,\ID,\FLD$ 
    this theory is countably infinite.
\end{dfn}

\begin{prop}
    $\ACF$ is not complete.
\end{prop}
\begin{proof}
    Take the $\Si_\RNG$-formula $\forall x, x + x = 0$.
    This is satisfied by the 
    \linkto{algebraic_closure_of_fields}{algebraic closure} 
    of $\F_2$ but not by that of $\F_3$,
    since field embeddings preserve characteristic. 
\end{proof}

\begin{dfn}[Algebraically closed fields of characteristic $p$]
    For $p \in \Z_{>0}$ prime define 
    \[\phi_p := \forall x, \sum_{i = 1}^{p} x = 0\]
    and let $\ACF_p := \ACF \cup \set{\phi_p}$.
    Furthermore, let
    \[\ACF[0] := \ACF \cup \set{\NOT \phi_p \st p \in \Z_{>0} \text{ prime}}\]
\end{dfn}

An important fact about algebraically closed fields of characteristic $p$:
\begin{prop}[Transcendence degree and characteristic
    determine algebraically closed fields of characteristic $p$ 
    up to isomorphism]
    \link{trans_deg_and_char_determine_ACF_p}
    If $K_0,K_1$ are fields with same characteristic and 
    transcendence degree over their minimal subfield ($\zmo{}$ or $\Q$)
    then they are (non-canonically) isomorphic.
\end{prop}
\begin{proof}
    \linkto{appendix_trans_deg_and_char_determine_ACF_p}{See appendix}.
\end{proof}

\begin{nttn}
    If $K \model{\Si_\RNG} ACF[p]$, write $\tdeg(K)$ to mean the 
    transcendence degree over its minimal subfield ($\zmo{}$ or $\Q$).
\end{nttn}

\begin{lem}[Cardinality of algebraically closed fields]
    \link{card_of_alg_closed_fields}
    If $L$ is an algebraically closed field then it has cardinality
    $\aleph_0 + \tdeg(L)$.
\end{lem}
\begin{proof}
    Let $S$ be a transcendence basis and call the minimal subfield $K$.
    Since $L$ is algebraically closed it splits the seperable polynomials
    $x^n - 1$ for each $n$. 
    Hence $L$ is infinite.
    Also $S \subs L$ and so $\aleph_0 + \tdeg(L) \leq \abs{L}$.
    For the other direction note that
    \[M = \bigcup_{f \in I}\set{a \in M \st f = \linkto{min_poly}{\min(a,K)}}\]
    where $I \subs K(S)[x]$ 
    is the set of monic and irreducible polynomials over $K(S)$.
    Thus 
    \begin{align*}
        \abs{M} &\leq \abs{I} \times \aleph_0 
        \leq \abs{K(S)[x]} \times \aleph_0\\
        &\leq \abs{K(S)} \times \aleph_0 
        \leq \abs{K[S]}\times \abs{K[S]} \times \aleph_0\\
        &= \abs{K[S]} \times \aleph_0 
        \leq \abs{\bigcup_{n \in \N} (K\cup S)^n} \times \aleph_0\\
        &= \abs{\bigcup_{n \in \N} K \cup S} \times \aleph_0
        = \abs{K \cup S} \times \aleph_0\\
        &= \abs{S} \times \aleph_0
    \end{align*}
    Noting that $K = \Q$ or $\F_p$ and so is at most countable.
    By Schröder–Bernstein we have $\aleph_0 + \tdeg(L) = \abs{L}$.
\end{proof}

\begin{prop}
    \link{ACF_ka_categorical_and_complete}
    $\ACF_p$ is $\ka$-categorical for uncountable $\ka$, consistent 
    and complete.
    %it is also decidable.?
\end{prop}
\begin{proof}
    Suppose $K,L \model{\Si_\RNG} \ACF_p$ and $\abs{K} = \abs{L} = \ka$.
    Then \linkto{card_of_alg_closed_fields}{
        $\tdeg(K) + \aleph_0 = \abs{K} = \ka$}
    and so $\tdeg(K) = \ka$ (as $\ka$ is uncountable).
    Similarly $\tdeg(L) = \ka$ and so $\tdeg(K) = \tdeg(L)$.
    Thus \linkto{trans_deg_and_char_determine_ACF_p}{
        $K$ and $L$ are isomorphic}.
    
    $\ACF_p$ is consistent due to the
    \linkto{algebraic_closure_of_fields}{existence of the algebraic closures}
    for any characteristic,
    \linkto{card_of_alg_closed_fields}{it is not finitely modelled}
    and is $\aleph_1$-categorical with 
    $\const{\Si_\RNG} + \aleph_0 \leq \aleph_1$,
    hence it is complete by \linkto{vaught_test}{Vaught's test}.
    %It is decidable because
    %it is recursively axiomatized and complete.?
\end{proof}

\subsection{Ax-Grothendieck}

\begin{prop}[Lefschetz principle]
    \link{lefschetz}
    Let $\phi$ be a $\Si_\RNG$-sentence.
    Then the following are equivalent:
    \begin{enumerate}
        \item There exists a $\Si_\RNG$-model of $\ACF_0$ 
        that satisfies $\phi$.
        (If you like $\C \model{\Si_\RNG} \phi$.)
        \item $\ACF_0 \model{\Si_\RNG} \phi$
        \item There exists $n \in \N$ such that for any prime $p$
        greater than $n$,
        $\ACF_p \model{\Si_\RNG} \phi$
        \item There exists $n \in \N$ such that for any prime $p$
        greater than $n$ there exists a non-empty $\Si_\RNG$-model
        of $\ACF_p$ that satisfies $\phi$.
    \end{enumerate}
\end{prop}
\begin{proof}~
    \begin{itemize}
        \item[${}$]$1. \implies 2.$ If $\C \model{\Si_\RNG} \phi$ then since
        $\ACF_0$ is complete 
        $\ACF_0 \model{\Si_\RNG} \phi$ or $\ACF_0 \model{\Si_\RNG} \NOT \phi$.
        In the latter case we obtain a contradiction.
        \item[${}$]$2. \implies 3.$ Suppose $\ACF[0] \model{\Si_\RNG} \phi$
        then since \linkto{proofs_are_finite}{`proofs are finite'}
        there exists a finite subset $\De$ of $\ACF[0]$ such that
        $\De \model{\Si_\RNG} \phi$.
        Let $n$ be maximum of all $q \in \N$ such that 
        $\NOT \phi_q \in \De$.
        By uniqueness of characteristic, 
        if $p$ is prime and greater than $n$ and $q$ is prime such that 
        $\NOT \phi_q \in \De$ then
        $\ACF_p \model{\Si_\RNG} \NOT \phi_q$.
        Thus if $\MM$ is a $\Si_\RNG$-model of $\ACF_p$ then 
        $\MM \model{\Si_\RNG} \De$ and so $\MM \model{\Si_\RNG} \phi$.
        Hence for all primes $p$ greater than $n$,
        $\ACF_p \model{\Si_\RNG} \phi$.
        \item[${}$]$3. \implies 4.$ $\ACF_p$ is consistent thus there exists a 
        non-empty $\Si_\RNG$-model of $\ACF_p$. 
        Our hypothesis implies it satisfies $\phi$.
        \item[${}$]$4. \implies 1.$ Let $n \in \N$ such that for any prime $p$
        greater than $n$ there exists a non-empty $\Si_\RNG$-model of $\ACF_p$
        that satisfies $\phi$.
        Then because $\ACF_p$ is complete $\ACF_p \model{\Si_\RNG} \phi$.
        Suppose for a contradiction $\ACF_0 \nodel{\Si_\RNG} \phi$.
        Then by completeness $\ACF_0 \model{\Si_\RNG} \NOT \phi$.
        Hence by the above we obtain there exists $m$ such that 
        for all $p$ greater than $m$, 
        $\ACF_p \model{\Si_\RNG} \NOT \phi$.
        Then since there are infinitely many primes, 
        take $p$ greater than both $m$ and $n$,
        then $\ACF_p$ is inconsistent, a contradiction.
        Hence $\ACF_0 \model{\Si_\RNG} \phi$
        and in particular $\C \model{\Si_\RNG} \phi$. 
    \end{itemize}
\end{proof}

\begin{lem}[Ax-Grothendieck for algebraic closures of finite fields]
    \link{algebraic_closure_ax_groth}
    If $\Om$ is an algebraic closure of a finite field
    then any injective polynomial map over $\Om$ is surjective.
\end{lem}
\begin{proof}
    \linkto{appendix_algebraic_closure_ax_groth}{See appendix.}
\end{proof}

\begin{lem}[Construction of Ax-Grothendieck formula]
    \link{construction_of_ax_grothendieck_formula}
    There exists a $\Si_\RNG$-sentence $\Phi_{n,d}$ such that 
    for any field $K$, $K \model{\Si} \Phi_{n,d}$ if and only if
    for all $d,n \in \N$ any injective polynomial map $f : K^n \to K^n$
    of degree less than or equal to $d$ is surjective.
\end{lem}
\begin{proof}
    We first need to be able to express polynomials in $n$ varibles 
    of degree less than or equal to $d$ in an elementary way.
    We first note that for any $n,d \in \N$ there exists a finite set $S$
    and powers $r_{s,j}\in \N$ (for each $(s,j) \in S \times \set{1, \dots, n}$).
    such that any polynomial $f \in K[x_1, \dots, x_n]$ can be written as
    \[\sum_{s \in S} \la_s \prod_{j=1}^n x_j^{r_{s,j}}\]
    for some $\la_s \in K$.
    Now we have a way of quantifying over all such polynomials,
    which is by quantifying over all the coefficients.
    We define $\Phi_{n,d}$:
    \begin{align*}
        \Phi_{n,d}:= \bigforall{i = 1}{n} \bigforall{s \in S}{} \la_{s,i},
        &\sqbrkt{\bigforall{j = 1}{n}x_j \bigforall{j = 1}{n}y_j,
        \bigand{i = 1}{n} \brkt{
            \sum_{s \in S}\la_{s,i}\prod_{j = 1}^{n} x_j^{r_{s,j}} = 
            \sum_{s \in S}\la_{s,i}\prod_{j = 1}^{n} y_j^{r_{s,j}}
         }
         \longrightarrow \bigand{i = 1}{n} x_i = y_i}\\
         &\longrightarrow \bigforall{j = 1}{n} x_j, \bigexists{i = 1}{n} z_i,
         \bigand{i=1}{n}\brkt{z_i = \sum_{s \in S} 
         \la_{s,i}\prod_{j = 1}^{n} x_j^{r_{s,j}}
         }
    \end{align*}
    At first it quantifies over all of the coefficients of all the $f_i$.
    The following part says that if the polynomial map is injective then 
    it is surjective.
    Thus $K \model{\Si} \Phi_{n,d}$ if and only if
    for all $d,n \in \N$ any injective polynomial map $f : K^n \to K^n$
    of degree less than or equal to $d$ is surjective.
\end{proof}

\begin{prop}[Ax-Grothendieck]
    If $K$ is an algebraically closed field of characterstic $0$
    then any injective polynomial map over $K$ is surjective.
    In particular injective polynomial maps over $\C$ are surjective.
\end{prop}
\begin{proof}
    We show an equivalent statement:
    for any $n, d \in \N$, any injective polynomial map $f : K^n \to K^n$ 
    of degree less than or equal to $d$ is surjective.
    This is true if and only if $K \model{\Si_\RNG} \Phi_{n,d}$
    \linkto{construction_of_ax_grothendieck_formula}{(by construction of
    the A-G formula)}
    which is true if and only if 
    for all $p$ prime greater than some natural number there exists 
    an algebraically closed field of characteristic $p$ that satisfies 
    $\Phi_{n,d}$, by \linkto{lefschetz}{the Lefschetz principle}.
    Indeed, take the natural $0$ and let $p$ be a prime greater than $0$.
    Take $\Om$ an algebraic closure of $\F_p$, 
    which indeed models $\ACF_p$.
    $\Om \model{\Si_\RNG} \Phi_{n,d}$ if and only if 
    for any $n, d \in \N$, any injective polynomial map $f : \Om^n \to \Om^n$ 
    of degree less than or equal to $d$ is surjective  
    \linkto{construction_of_ax_grothendieck_formula}{
       (by construction of the A-G formula)}.
    The final statement is true due to \linkto{algebraic_closure_ax_groth}{
        A-G for algebraic closures of finite fields.}
\end{proof}


%\subsection{Quantifier elimination in algebraically closed fields
and Nullstellensatz}

\begin{lem}[Terms in the language of rings are polynomials]
    \link{terms_in_RNG_are_polynomials}
    Let $X$ be a subset of a ring $A$.
    Then $\Si_\RNG(X)$-terms are interpreted as 
    polynomials with coefficients from $\<X\>$,
    the smallest subring of $A$ generated by $X$.
\end{lem}
\begin{proof}
    Let $t$ be a $\Si_\RNG(X)$-term.
    \begin{itemize}
        \item If $t$ is a constant $c$ then it's interpretation is
            $\modintp{\MM}{c}$, a constant polynomial. 
            Since $c$ is $0$, $1$ or something in $X$,
            this is in any polynomial ring over $\<X\>$.
        \item If $t$ is a variable $v_i$ then it is interpreted as
            the single variable polynomial $v_s \in \<X\>[v_s]$.
        \item If $t$ is $f(s_1,\dots,s_n)$ and each $s_i$ 
            is interpreted as a polynomial $q_i$,
            take the polynomial ring $\<X\>[v_1,\dots,v_k]$
            containing all the $q_i$ (everything is finite).
            Then $f(q_1,\dots,q_n)$ 
            is still a polynomial in $\<X\>[v_1,\dots,v_k]$
            as $f$ is either $+,-$ or $\cdot$.
    \end{itemize} 
\end{proof}

\begin{lem}[Disjunctive normal form for rings]
    \link{disjunctive_normal_form}
    Let $A$ be a ring or the empty set.
    Let $\phi$ be a quantifier free $\Si_\RNG(A)$-formula with variables indexed
    by $S$.
    Then there exist $\Si_\RNG(A)$-terms 
    (i.e. \linkto{terms_in_RNG_are_polynomials}{polynomials})
    $p_{ij}, q_{ij}$ such that
    for any $\Si_\RNG(A)$-structure $\MM$
    \[ 
        \MM \model{\Si} \bigforall{s \in S}{v_s}, \sqbrkt{\phi(v) 
        \IFF \bigor{i \in I}{} 
        \brkt{\bigand{j \in J_{i0}}{} p_{ij}(v) = 0 \AND 
        \bigand{j \in J_{i1}}{} q_{ij}(v) \ne 0}}
    \]
\end{lem}
\begin{proof}
    Applying the general 
    \linkto{disjunctive_normal_form_0}{disjunctive normal form} we have that
    any formula of the form $s = t$ or $r(t)$ is just a polynomial equation, 
    since there are no relation symbols.
    Moving everything to one side we have that they are of the form 
    $p_{ij}(v) = 0$ and $q_{ij}(v) \ne 0$.
\end{proof}

\begin{rmk}[Substructures of algebraically closed fields]
    Since $\GRP, \RNG, \ID$ are all universal axomatizations of themselves,
    by \linkto{universal_axiomatizations_make_subs_mods}{'universal 
    axiomatizations make substructures models'} we have that
    substructures of groups are groups, 
    substructures of rings are rings, 
    and substructures of integral domains are integral domains.
    Furthermore, 
    substructures of algebraically closed fields are 
    substructures of integral domains hence are integral domains.
    This becomes relevant when proving the equivalent condition on quantifier
    elimination.
\end{rmk}

\begin{nttn}
    If $A$ is a ring and 
    $I \subs A[x_1, \dots, x_n]$ is a set of polynomials then
    the vanishing of $I$ over $A$ is
    \[\V_{A}(I) := \set{c \in A^n \st \forall f \in I, f(c) = 0}\]
\end{nttn}

\begin{prop}
    \link{ACF_has_quantifier_elimination}
    $\ACF$ has quantifier elimination.
\end{prop}
\begin{proof}
    Let $\phi$ be a quantifier free $\Si_\RNG$-formula with a free variable $w$
    and index the rest by $S$.
    Let $K, L$ be algebraically closed field with $A$ embedding into each via
    $\io_K, \io_L$.
    Let $a \in A^S$.
    \linkto{improved_condition_for_q_e}{It suffices to show that}
    \[  
        K \model{\Si_\RNG} \exists w, \phi(\io_K(a),w) \implies
        L \model{\Si_\RNG} \exists w, \phi(\io_L(a),w)
    \]
    Crucially by the remark before this proposition $A$ 
    is an integral domain thus 
    we can consider $\io : A \to \Om$ the embedding into
    an algebraic closure of the field of fractions of $A$.
    As $K$ and $L$ are fields extending the fraction field of $A$,
    by the properties of field of fractions and algebraic closures
    there are field extensions $\ka : \Om \to K$ and $\la : \Om \to L$ 
    such that $\io_K = \ka \circ \io$ and $\io_L = \la \circ \io$.
    Let we write $a' := \io(a)$, 
    and have $\io_K(a) = \ka(a')$
    and $\io_L(a) = \la(a')$.
    \begin{cd}
        & & K\\
        A \ar[urr] \ar[drr] \ar[r] &F \ar[ur] \ar[dr] \ar[r] 
        & \Om \ar[u] \ar[d]\\
        & & L
    \end{cd}

    We can find the 
    \linkto{disjunctive_normal_form}{`disjunctive normal form' of} 
    $\phi$, i.e. 
    $\Si_\RNG$-terms $p_{ij}, q_{ij}$ such that any ring $\MM$
    satisfies 
    \[ 
        \MM \model{\Si} \bigforall{s \in S}{v_s},\forall w, \sqbrkt{\phi(v,w)
        \IFF \bigor{i \in I}{} 
        \brkt{\bigand{j \in J_{i0}}{} p_{ij}(v,w) = 0 \AND 
        \bigand{j \in J_{i1}}{} q_{ij}(v,w) \ne 0}}
    \]
    Hence assuming for some $b \in K$ that 
    \[
        K \model{\Si_\RNG} \bigor{i \in I}{} 
        \brkt{\bigand{j \in J_{i0}}{} p_{ij}(\ka(a'),b) = 0 \AND 
        \bigand{j \in J_{i1}}{} q_{ij}(\ka(a'),b) \ne 0}
    \] 
    it suffices to show that there exists a 
    $c \in L$ such that
    \[
        L \model{\Si_\RNG} \bigor{i \in I}{} 
        \brkt{\bigand{j \in J_{i0}}{} p_{ij}(\la(a'),c) = 0 \AND 
        \bigand{j \in J_{i1}}{} q_{ij}(\la(a'),c) \ne 0}
    \]
    Assume for a contradiction that $I$ is empty 
    then by convension we have the disjunctive normal form
    is $\bot$ and so $K \model{\Si_\RNG} \bot$ which implies $K$ is empty,
    which is a contradiction as the constant symbol $0$ 
    has an interpretation in $K$.

    Thus there exists $i \in I$ (we keep this $i$ for the rest of the proof) 
    such that 
    \[
        K \model{\Si_\RNG} \bigand{j \in J_{i0}}{} p_{ij}(\ka(a'),b) = 0 \AND 
        \bigand{j \in J_{i1}}{} q_{ij}(\ka(a'),b) \ne 0
    \]
    Now case on whether or not there exists a $j \in J_{i0}$ such that 
    $\modintp{K}{p_{ij}}$ is not the zero polynomial.
    If it exists then $\modintp{K}{p_{ij}}(\ka(a'),b) = 0$.
    As $p_{ij}$ is a $\Si_\RNG$-term and so 
    $\ka \modintp{\Om}{p_{ij}} = \modintp{K}{p_{ij}}$.
    (Intuitively it is a polynomial over $\<0,1\>$ as shown in
    \link{disjunctive_normal_form}{`disjunctive normal form'}.)
    Thus $b$ is algebraic over $\Om$, which is algebraically closed.
    Hence there is a $c \in \Om$ such that $\ka (c) = b$:
\begin{align*}
    &\quad 
        \bigand{j \in J_{i0}}{} \modintp{K}{p_{ij}}(\ka(a'),\ka(c)) = 0 \AND 
        \bigand{j \in J_{i1}}{} \modintp{K}{q_{ij}}(\ka(a'),\ka(c)) \ne 0 \\
    &\implies 
        \bigand{j \in J_{i0}}{} \ka \brkt{\modintp{\Om}{p_{ij}}(a',c)} = 0 \AND 
        \bigand{j \in J_{i1}}{} \ka \brkt{\modintp{\Om}{q_{ij}}(a',c)} \ne 0 \\
    &\implies
        \bigand{j \in J_{i0}}{} \brkt{\modintp{\Om}{p_{ij}}(a',c)} = 0 \AND 
        \bigand{j \in J_{i1}}{} \brkt{\modintp{\Om}{q_{ij}}(a',c)} \ne 0 
    \quad \quad \ka \text{ is injective} \\
    &\implies 
        \bigand{j \in J_{i0}}{} \la \brkt{\modintp{\Om}{p_{ij}}(a',c)} = 0 \AND 
        \bigand{j \in J_{i1}}{} \la \brkt{\modintp{\Om}{q_{ij}}(a',c)} \ne 0 \\
    &\implies
       \bigand{j \in J_{i0}}{} \brkt{\modintp{L}{p_{ij}}(\la(a'),\la(c))} = 0\AND 
       \bigand{j \in J_{i1}}{} \brkt{\modintp{L}{q_{ij}}(\la(a'),\la(c))} \ne 0\\
    &\implies
    L \model{\Si_\RNG} \bigor{i \in I}{} 
        \brkt{\bigand{j \in J_{i0}}{} p_{ij}(\la(a'),\la(c))) = 0 \AND 
        \bigand{j \in J_{i1}}{} q_{ij}(\la(a'),\la(c))) \ne 0}
\end{align*}
    Thus we are done with this case.
    
    In the other case we turn our attention to $J_{i1}$.
    If $j \in J_{i1}$, 
    we see $\modintp{\Om}{q_{ij}}(a',w)$ as a polynomial
    in $\Om[w]$.
    $\modintp{K}{q_{ij}}(\ka(a'),b) \ne 0$ and so 
    $\modintp{\Om}{q_{ij}}(a',w)$ is not the zero polynomial
    hence $\modintp{L}{q_{ij}}(\la(a'),w)$
    has finitely many zeros in $L$ by the division algorithm in 
    $\Om[w]$.
    Let 
    \[\V_L(q_{ij})_{j} = \set{c \in L \st \exists j \in J_{i1} 
    \st \modintp{L}{q_{ij}}(\la(a'),c) = 0}\]
    be the vanishing of the $q_{ij}$ for every $j \in J_{j1}$.
    Then $\V$ is finite and $L$ is 
    \linkto{card_of_alg_closed_fields}{infinite as it is algebraically closed}
    hence there exists a $c \in L$ 
    $L \model{\Si_\RNG} \bigand{j \in J_{i1}}{} q_{ij}(\la(a'),c) \ne 0$.
    Since each $f_{ij}$ are the zero polynomial we also have
    $L \model{\Si_\RNG} \bigand{j \in J_{i0}}{} p_{ij}(\la(a'),\la(c))) = 0$
    Hence we are done.
\end{proof}

\begin{prop}[Weak Nullstellensatz]
    If $K$ is an algebraically closed field and $\f{p}$ is a prime ideal of 
    $K[x_1, \dots, x_n]$, 
    then $\V_{K}(\f{p}) \ne \nothing$.
\end{prop}
\begin{proof}
    By \linkto{hilbert_basis}{Hilbert's basis theorem}
    $K[x_1, \dots, x_n]$ is Noetherian
    so we can find a finite set $\set{f_1, \dots, f_m}$ generating $\f{p}$.
    It suffices to find a common zero of $\set{f_1, \dots, f_m}$.
    We can write out each $f_i$ and use $a$ to represent 
    all the coefficients of all the $f_i$ as a tuple.
    We can construct some $\Si$-formula $\phi(v,x_1,\dots,x_n)$ 
    (a polynomial in variables $x_i$ with coefficients $v$
    corresponding to the tuple $a$)
    such that 
    for any $b \in K^n$,
    \[K \model{\Si_\RNG} \phi(a,b) \iff \bigand{j = 1}{m} f_i(b) = 0\]
    and for any extension $\io : K \to L$ and $b \in L^n$,
    \[  
        L \model{\Si_\RNG} \phi(\io(a),b) 
        \iff \bigand{j = 1}{m} \io(f_i)(b) = 0
    \]

    We can quotient $K[x_1, \dots, x_n] / \f{p}$ and take an algebraic closure
    $L$ (quotienting by prime ideals gives an integral domain).
    \begin{cd}
        K[x_1,\dots,x_n]  \ar[r] 
        & K[x_1,\dots,x_n] / \f{p} \ar[d, "\al"]\\
        K \ar[u] \ar[ur, "\ka"]\ar[ur, "\io"] & L
    \end{cd}
    We can then take $b := (x_1, \dots, x_n) \in (K[x_1, \dots, x_n])^n$
    and send it through to $\al(b) \in L^n$.
    \begin{align*}
        &\bigand{j = 1}{m} \ka(f_i)(b) = 0\\
        &\implies \bigand{j = 1}{m} \ka \circ \al (f_i)(\al(b)) = 0\\
        &\implies \bigand{j = 1}{m} \io(f_i)(\al(b)) = 0\\
        &\implies L \model{\Si_\RNG} \phi(\io(a),\al(b))\\
        &\implies L \model{\Si_\RNG} 
        \bigexists{i = 1}{n} x_i, \phi(\io(a),x)
    \end{align*}
    since $K$ and $L$ are both algebraically closed and 
    \linkto{ACF_has_quantifier_elimination}{$\ACF$ has quantifier elimination} 
    so it is \linkto{
        quantifier_elimination_implies_model_completeness}{model complete},
    which implies that the embedding is elementary.
    Hence we have
    \begin{align*}
        &\implies K \model{\Si_\RNG} 
        \bigexists{i = 1}{n} x_i, \phi(a,x)\\
        &\implies \exists c \in K^n, K \model{\Si_\RNG} \phi(a,c)\\
        &\implies \exists c \in K^n, \bigand{j = 1}{m} f_i(b) = 0
    \end{align*}
\end{proof}

\subsection{The Classical Zariski Topology, Chevalley, Vanishings}%

\begin{dfn}[Classical Zariski Topology]
    Let $K$ be an algebraically closed field and let 
    \[\set{\V_K(E) \st E \subs K[x_1, \dots, x_n]}\]
    be a closed basis for a topology on $K^n$.
    We call these Zariski closed sets.
\end{dfn}
The closed sets in this setting \linkto{zariski_correspondence}{correspond to }
closed sets in \linkto{prime_spec_zariski_top}{$\spec(K[x_1,\dots,x_n])$},
though the spaces are not homeomorphic.

\begin{prop}[Closed sets are finitely generated vanishings]
    \link{zariski_closed_sets_are_fin_gen}
    If $K$ is an algebraically closed field and 
    $V$ is a closed set of $K^n$ under the
    Zariski topology then $V = \V_K(S)$ for some finite subset 
    of the polynomial ring $S \subs K[x_1,\dots,x_n]$.
\end{prop}
\begin{proof}
    By definition of Zariski closed sets any element of the closed 
    basis is $\V_K(E)$ for some 
    $E \subs K[x_1,\dots,x_n]$
    We consider the ideal generated by $E$.
    Important point: This is finitely generated by the 
    \linkto{hilbert_basis}{Hilbert basis theorem}, 
    and so we can just require $E$ 
    to be finite without loss of generality.

    Moreover, 
    arbitrary intersection of such sets is also finitely generated:
    Let $E \in I$ be a collection of subsets of $K[x_1, \dots, x_n]$.
    Then $a \in \bigcap_{E \in I} \V_K(E)$ if and only if 
    for every $E \in I$ and every $f \in E, f(a) = 0$.
    This is true if and only if $a \in \V_K(\bigcup_{E \in I} E)$.
    Thus $\bigcap_{E \in I} \V_K(E) = \V_K(\bigcup_{E \in I} E)$
    which is finitely generated by the first point.

    Furthermore, any finite union of such sets is also finitely generated.
    We prove this by induction.
    For the empty case:
    \[\bigcup_{E \in \nothing} \V_K(E) = \nothing = 
    \V_K (K[x_1,\dots,x_n])\]
    which is finitely generated by the first point.
    It suffices to show that the union of two such sets is also finitely
    generated.
    \begin{align*}
        &a \in \V(F) \cup \V(G) 
        \iff \brkt{\bigand{f \in F}{} f(a) = 0} \OR 
        \brkt{\bigand{g \in G}{} g(a) = 0} \\
        & \iff \bigand{f \in F}{} \bigand{g \in G}{} 
        f(a) = 0 \OR g(a) = 0
        \iff \bigand{f \in F}{} \bigand{g \in G}{} 
        (fg)(a) = 0
    \end{align*}
    The last step is due to $K[x_1,\dots,x_n]$ being an integral domain.
    Hence we have a finite intersection
    $\V(F) \cup \V(G) = \V(\bigcap (fg)(a) = 0)$
    which is finitely generated by the first point.
\end{proof}

\begin{prop}[Constructable]
    \link{definable_is_constructable}
    Let $K$ be an algebraically closed field with the Zariski topology on $K^n$.
    Define the set $C$ inductively:
    \begin{itemize}
        \item[$\vert$] If $X \subs K^n$ is closed then it is in $C$.
        \item[$\vert$] If $X \subs K^n$ is in $C$ then $K^n \setminus X$
        is in $C$. 
        \item[$\vert$] If $X,Y \subs K^n$ are in $C$ then $X \cup Y$
        is in $C$. 
    \end{itemize}
    Then $C$ is the set of \linkto{constructable_dfn}{constructable} 
    and \linkto{definable_is_constructable_01}{equivalently} 
    the set of of definable sets in $K^n$
    (by \linkto{ACF_has_quantifier_elimination}{quantifier elimination}).

    $C$ consists of `finite boolean combinations' of closed sets, 
    and corresponds to the original definition `constructable'.
\end{prop}
\begin{proof}
    \begin{forward}
        First we show by induction on the set $C$ 
        that if $X$ is in $C$ then $X$ is constructble.
        If $X$ is closed then 
        \linkto{zariski_closed_sets_are_fin_gen}{$X = \V(S)$} for some 
        finite $S$. 
        Therefore $X$ is defined by 
        $\phi(v,b) := \bigand{f \in S}{} f(a) = 0$,
        where each $f$ is some $\Si_\RNG$ formula evaluated at $b \in K^m$.
        This is a finite and which is the negation of a finite or which is 
        (by induction) constructable.
        The rest of the induction follows immediately.
    \end{forward}
    
    \begin{backward}
        If $X$ is constructable then we show by induction that it is in $C$.
        If $X$ is defined by an atomic formula then it is either $\top$ 
        or $t = s$. If $\phi$ is $\top$ then $X = K^n$ which is closed
        hence in $C$.
        If $\phi$ is $t = s$ then 
        $\modintp{K}{t}(b)$ and $\modintp{K}{s}(b)$ are polynomials
        in $K[x_1,\dots,x_n]$.
        Writing $f = \modintp{K}{t}(b) - \modintp{K}{s}(b)$
        we have $X = \set{a \in K^n \st f(a) = 0}$,
        which is closed hence in $C$.
        The rest of the induction follows immediately.
    \end{backward}
\end{proof}

\begin{prop}[Chevalley]
    Over an algebraically closed field, 
    the image of a constructable set under a polynomial map is constructable.
\end{prop}
\begin{proof}
    Let $\rho : K^n \to K^m$ be a polynomial map defined by $(f_i)_{i=1}^m$. 
    Suppose $X \subs K^n$ is constructable.
    Then \linkto{definable_is_constructable}{
        as constructable is equivalent to definable over $K$} 
    there exists $\Si$-formula $\phi$ and $b \in K^l$ such that 
    \[X = \set{a \in K^n \st K \model{\Si_\RNG} \phi(a,b)}\]
    Then 
    \begin{align*}
        \rho(X) &= \set{c \in K^m \st \exists a \in K^m, K \model{\Si_\RNG}
        \AND \rho(a) = c}\\
        &= \set{c \in K^m \st \exists a \in K^m, K \model{\Si_\RNG}
            \AND \bigand{i = 1}{m} f_i(a) = c_i}\\
        &= \set{c \in K^m \st K \model{\Si_\RNG} \bigexists{j = 1}{n} x_j,
            \phi(x,b) \AND \bigand{i=1}{m} \phi_i(x,d) = c_i}
    \end{align*}
    The $d$ appearing in $\phi_i(x,d)$ is due to the fact that the polynomials 
    $f_i$ may have coefficients not from the language.
    Thus the image is constructable.
\end{proof}

\begin{nttn}[Radical]
    We write $r(\f{a})$ to be the radical of $\f{a}$.
\end{nttn}

\begin{dfn}[Ideal generated by subsets of $K^n$]
    If $K$ is a field, for a subset $X \subs K^n$, 
    we write $I(X)$ to mean the ideal of $X$ in $K[x_1,\dots,x_n]$ to mean
    \[\set{f \in K[x_1,\dots, x_n] \st \forall a \in X, f(a) = 0}\]
\end{dfn}

\begin{ex}[Taking ideals and vanishings are order reversing]
    \link{taking_ideals_order_reversing}
    Show that if $X \subs Y \subs K^n$ then $I(Y) \subs I(X)$.
    Show that if $E \subs F \subs K[x_1,\dots,x_n]$ then 
    $\V_K(F) \subs \V_K(E)$.
\end{ex}

\begin{prop}[Strong Nullstellensatz]
    \link{strong_nullstellensatz}
    Let $K$ be an algebraically closed field and suppose 
    $\f{a}$ is an ideal of $K[x_1,\dots,x_n]$.
    Then $r(\f{a}) = I(\V(\f{a}))$.
\end{prop}
\begin{proof}
    \linkto{strong_nullstellensatz_appendix}{See appendix.}
\end{proof}

\begin{prop}[The Zariski closed sets are Artinian]
    \link{zariski_closed_sets_artinian}
    Any descending chain of Zariski closed sets in $K^n$ for $K$ a field 
    stabilises.
\end{prop}
\begin{proof}
    Let $\dots \subs \V(\f{a}_1) \subs \V(\f{a}_0)$ 
    be a chain of Zariski closed sets.
    Then taking the ideals generated each of them we have an 
    \linkto{taking_ideals_order_reversing}{ascending chain}
    \[I(\V(\f{a}_0)) \subs I(\V(\f{a}_1)) \subs \dots\]
    This stabilises because 
    \linkto{hilbert_basis}{$K[x_1,\dots,x_n]$ is Noetherian}.
    Hence by \linkto{strong_nullstellensatz}{strong Nullstellensatz}
    we have that 
    \[r(\f{a}_0) \subs r(\f{a}_1) \subs \dots\]
    stabilises and
    taking the vanishing gives back
    the \linkto{taking_ideals_order_reversing}{descending chain} 
    $\dots \subs \V(\f{a}_1) \subs \V(\f{a}_0)$ which stabilises.
\end{proof}

\begin{dfn}[Irreducible, Variety]
    If $X$ is a topological space
    then the following are \linkto{irreducible_equiv_defs}{equivalent}:
    \begin{enumerate}
        \item Any non-empty open set is dense in $X$.
        \item Any pair of non-empty open subsets intersect non-trivially.
        \item Any two closed proper subsets do not form a cover of $X$.
    \end{enumerate}
    If any of the above hold then $X$ is said to be irreducible,
    and a subset of $X$ is irreducible if it is irreducible under the 
    subspace topology.

    A variety is a Zariski closed set that is irreducible.
\end{dfn}

\begin{cor}
    Zariski closed sets are finite unions of varieties.
\end{cor}
\begin{proof}
    For a contradiction suppose $\V_0$ 
    is a Zariski closed set that is not a finite union of varieties.
    Then $\V_0$ is not irreducible and so 
    there exists two closed proper subsets $\V_1, \V_1'$ that cover $\V_0$.
    If both $\V_1$ and $\V_1'$ are finite unions of varieties then we have a
    contradiction, 
    hence without loss of generality $\V_1$ is not a finite union of varieties.
    By induction we obtain 
    \[\dots \subset \V_1 \subset \V_0\]
    which \linkto{zariski_closed_sets_artinian}{stabilises}, a contradiction.
\end{proof}

\subsection{The Stone space and Spec}
\begin{prop}[Any polynomial is a formula]
    Let $K$ be an algebraically closed field
    and $v = (v_1,\dots,v_n)$ be variables.
    Then there exists a map 
    \[\eqzero : K[x_1,\dots,x_n] \to F(\Si_\RNG(K),v)\]
    such that for any $a \in K^n$ and any $f \in K[x_1,\dots,x_n]$,
    \[K \model{\Si_\RNG(K)} \eqzero_f(a) \iff f(a) = 0\]
    where $\eqzero_f$ is the image of an $f$.
\end{prop}
\begin{proof}
    This would be some nasty induction.
    I guess show that monomials can be made into formulas 
    and then sums of monomials can be made into formulas.
    Everything is finite so it should be okay.
\end{proof}

\begin{prop}[Bijection between the Stone space and spec]
    Let $K$ be an algebraically closed field.
    Then define the map
    \begin{align*}
        I : S_n(\eldiag{\Si}{K}) &\to \spec(K[x_1,\dots,x_n])\\
        p &\mapsto \set{f \in K[x_1,\dots,x_n] \st \eqzero_f \in p}
    \end{align*}
    We show that this is well-defined, continuous and a bijection.
    Hence $\spec(K[x_1,\dots,x_n])$ is compact.
    However, the spaces are \emph{not} homeomorphic.
\end{prop}
\begin{proof}
    Let $p \in S_n(\eldiag{\Si}{K})$.
    First we check that 
    $I_p := \set{f \in K[x_1,\dots,x_n] \st \eqzero_f \in p}$
    is a prime ideal.
    We will repeatedly use the following fact:
    since $p$ is consistent with $\eldiag{\Si}{K}$ we have that it is
    \linkto{finite_realisation_and_embeddings}{finitely realised in $K$.}

    Let $f,g \in I_p$, then $\eqzero_f,\eqzero_g \in p$.
    Suppose for a contradiction $f+g \notin I_p$, then by maximality of $p$,
    $\NOT \eqzero_{f+g} \in p$.
    Taking the finite subset of $p$ to be 
    $\set{\eqzero_f,\phi_g,\NOT \eqzero_{f+g}}$,
    by the above fact we obtain $a \in K^n$ such that 
    \[K \model{\Si_\RNG(K)} 
    \set{\eqzero_f,\eqzero_g,\NOT \eqzero_{f+g}}(a)\]
    By definition of $\eqzero$ this implies
    \[f(a) = 0,g(a) = 0,(f+g)(a) \ne 0\]
    which is a contradiction.
    Similarly we can let $f \in I_p$, $\la \in K$ 
    and suppose $\la f \notin I_p$ and so $\NOT \eqzero_{\la f} \in p$.
    We take the finite subset $\set{f, \la f} \subs p$ and get a contradiction.

    Let the product of two polynomials $fg$ be in $I_p$. 
    Suppose for a contradiction $f,g \notin I_p$.
    Then take the finite subset 
    $\set{\NOT \eqzero_f, \NOT \eqzero_g, \eqzero_{f+g}} \subs p$ and obtain
    an $a \in K^n$ such that $f(a) \ne 0, g(a) \ne 0$ but $f(a)g(a) = 0$,
    a contradiction.

    To show that the map is continuous let 
    $V(E) \subs \spec(K[x_1,\dots,x_n])$ be closed,
    where $E$ is a subset of $K[x_1,\dots,x_n]$ 
    ($V$ denotes the spec vanishing - 
    \linkto{prime_spec_zariski_top}{see appendix}).
    Then $p \in I^{-1}(U)$ if and only if $I_p \in U$
    if and only if $E \subs I_p$ if and only if 
    \[\forall f \in E, \eqzero_f \in p\]
    Hence the preimage is closed:
    \[  
        I^{-1}(U) = \set{p \in S_n(T) \st \forall f \in E, \eqzero_f \in p}
        = \bigcap_{f \in E} [\eqzero_f]
    \]
    as the basis elements $[\phi]$ are 
    \linkto{properties_of_stone_space}{clopen}.
    Thus $I$ is continuous.

    To show that it is injective suppose $I_p = I_q$ and let $\phi \in p$.
    By \linkto{ACF_has_quantifier_elimination}{quantifier-elimination 
        in $\ACF$} 
    we have $\phi$ is equivalent to quantifier free $\psi$ modulo $T$.
    Then $\psi \in p$ as $p$ is consistent with $T$
    and $\phi \in q$ if and only if $\psi \in q$ as $q$ is consistent with $T$.
    Take the \linkto{disjunctive_normal_form}{disjunctive normal form} of $\psi$
    \[ 
        \bigor{i \in I}{} 
        \brkt{\bigand{j \in J_{i0}}{} f_{ij}(v) = 0 \AND 
        \bigand{j \in J_{i1}}{} g_{ij}(v) \ne 0}
    \]
    Then (by maximality of $p$) 
    there exists some $i$ such that for all $j$,
    $f_{ij} = 0$ and $g_{ij} \ne 0$
    are in $p$.
    Thus $f_{ij} = 0$ and $g_{ij} \ne 0$ are in $q$ as they can be made into 
    polynomials in $I_p = I_q$.
    Hence by maximality of $q$ we have $\psi \in q$ and so $\phi \in q$.
    By symmetry $p = q$.

    For surjectivity, 
    let $\f{p}$ be a prime ideal of the polynomial ring.
    Consider the set
    \[  
        \eqzero(\f{p}) \cup 
        \set{\NOT \phi \st \phi \in \eqzero(K[x_1,\dots,x_n] \setminus \f{p})}
    \]
    Assuming for now that this set is 
    consistent with the elementary diagram, 
    \linkto{extend_to_maximal_type_zorn}{it can be extended 
    to $p$ a maximal $n$-type} of elementary diagram.
    Then 
    \begin{align*}
        &f \in \f{p} \implies \eqzero_f \in p \implies f \in I_p\\
        &f \notin \f{p} \implies (\NOT \eqzero_f) \in p \implies f \notin I_p
    \end{align*} 
    and so $I_p = \f{p}$ and we have found a preimage $p$.

    To show that the set is consistent 
    it suffices to show that it is 
    \linkto{finite_realisation_and_embeddings}{finitely realised in $K$}.
    Let $\De \subs \f{p}$ and $\Ga \subs K[x_1,\dots,x_n] \setminus \f{p}$
    be finite subsets.
    It suffices to show that 
    \[\exists a \in K^n, \forall f \in \De, \forall g \in \Ga, 
    f(a) = 0 \AND g(a) \ne 0\]
    which is equivalent to 
    \[\V_K(\De) \cap K^n \setminus \V_k(\Ga) \ne \nothing\]
    Suppose for a contradiction it is empty then $\V_k(\De) \subs \V_k(\Ga)$
    and taking ideals gives us 
    \[\Ga \subs r(\Ga) = I(\V_K(\Ga)) \subs I(\V_K(\De)) = r(\De) \subs \f{p}\]
    Here we used that 
    \linkto{taking_ideals_order_reversing}{taking ideals is order reversing},
    \linkto{strong_nullstellensatz}{strong Nullstellensatz}, 
    and the fact that the radical
    is a subset of any prime ideal containing its generators.
    Thus $\Ga \subs \f{p}$ which is a contradiction.
    Hence we have surjectivity.

    Suppose for a contradiction that the two spaces were homeomorphic.
    Then as the Stone space is Hausdorff, 
    $\spec(K[x_1,\dots,x_n])$ would also be Hausdorff,
    which is a \linkto{spec_not_hausdorff}{contradiction}.
\end{proof}
%\section{Morley Rank in Algebraically Closed Fields and Dimension}

We work towards the result that Krull dimension for algebraically closed 
fields is the same thing as Morley rank for varieties. 
The idea is that Morley rank of a variety corresponds to the Morley rank of 
types from elements of the variety, corresponds to model theoretic dimension,
corresponds to transcendence degree, 
which is the same thing as Krull dimension.
Strong minimality will play an important role in defining 
model theoretic algebraic closure,
which is then used to define dimension.

\subsection{Dimension}
In this section we will set up the important definitions needed to 
talk about dimension. 
A prerequisite for this section is knowledge of the basic results for 
\linkto{pregeometry_dfn}{pregeometries}, covered in the appendix.
\begin{dfn}[Algebraic, algebraic closure]
    Let $\MM$ be a $\Si$-structure and let $D$ be a subset of $\MM$.
    Let $A$ be a subset of $D$, 
    $a \in \MM$ is algebraic over $A$ if $a$ belongs to a finite 
    $\Si(A)$-definable set .
    Define the algebraic closure of $A$ over $D$ to be
    \[\acl_{\Si,D}(A) := \set{a \in D \st a \text{ is algebraic over } A}\]
    We drop the subscripts $\Si$ and $D$ when it is sufficiently obvious.
\end{dfn}

\begin{dfn}[Minimal, strongly minimal \cite{marker}]
    \link{strongly_minimal}
    Let $\MM$ be a $\Si$-structure.
    Let $D$ be an infinite $\Si(\MM)$-definable subset of $\MM^n$.
    $D$ is minimal in $\MM$ if any $\Si(\MM)$-definable subset of $D$
    if finite or cofinite.
    $D$ is strongly minimal if it is minimal in 
    $\NN$ for any elementary extension $\NN$ of $\MM$.
    A $\Si$-theory $T$ is strongly minimal if any $\Si$-model of $T$
    is strongly minimal 
    (note that any $\Si$-structure is definable by the formula $v = v$).
\end{dfn}

We have an equivalent definition of strong minimality, 
which if you like you can skip.
\begin{prop}[Strong minimality in terms of Morley rank and degree]
    Let $X$ be a definable subset of $\MM$, a $\Si$-structure.
    Then $X$ is strongly minimal if and only if $\MR{}{X} = \MD{X} = 1$.
\end{prop}
\begin{proof}
    \begin{forward}
        Suppose $X$ is strongly minimal.
        Then $X$ is infinite 
        \linkto{basic_facts_morley_rank_of_dfnbl_set}{hence} $1 \leq \MR{}{X}$.
        Let $\M$ be an 
        \linkto{om_sat_elem_ext_of_models}{$\om$-saturated extension of $\MM$}.
        If $2 \leq \MR{}{X}$ then there would be infinitely many 
        disjoint $\Si(\M)$ definable subsets of $X$ of Morley rank $1$
        (hence they are infinite).
        By strong minimality the only 
        $\Si(\M)$-definable subsets of $X$ are finite or cofinite.
        Thus these subsets must all be cofinite
        and so any two will intersect, a contradiction.
        Thus $1 = \MR{}{X}$.

        Since $\MR{}{X} = 1 \in \ord$ we have that $\MD{X} \in \N_{>0}$.
        Again if we have two disjoint 
        $\Si(\M)$-definable subsets of $X$ of Morley rank $1$ 
        we have a contradiction, hence $\MD{X} \leq 1$ and so $\MD{X} = 1$.
    \end{forward}

    \begin{backward}
        Suppose $X$ has Morley rank and degree $1$.
        Let $\NN$ be an elementary extension of $\MM$. 
        Let $A \subs X$ be a $\Si(\NN)$-definable subset of $X$;
        its complement is also $\Si(\NN)$-definable.
        Suppose both are infinite then they both have Morley rank greater 
        than or equal to $1$ and are disjoint,
        thus $2 \leq \MD{}{X} = 1$, which is a contradiction.
        Hence one is finite and the other cofinite.
    \end{backward}
\end{proof}

\begin{lem}[Some definable sets]
    \link{some_definable_sets}
    Let $\MM$ be a $\Si$-structure and let $B,C \subs \MM$ be $\Si$-definable
    set (i.e. $\Si(\nothing)$-definable).
    Let $\phi(x)$ be a $\Si$-formula with $n$ free variables.
    For $b \in B^n$, let $\psi(x,b)$ be a $\Si(B)$-formula 
    with $m$ free variables ($\psi(x,y)$
    is a $\Si$-formula with $n+m$ free variables).

    Then the following sets are definable by a $\Si$-formula:
    \begin{itemize}
        \item The intersection of $B$ and $C$, the union of $B$ and $C$ and
            the complement of $B$.
        \item The set of $b \in B^n$ that satisfy $\phi(x)$:
            \[\set{b \in B^n \st \MM \model{\Si} \phi(b)}\]
        \item The elements $b \in \MM^n$ such that $\psi(x,b)$ defines a set of 
            at most $k$ elements.
        \item The elements $b \in \MM^n$ such that $\psi(x,b)$ defines a set of 
            at least $k$ elements.
        \item The elements $b \in \MM^n$ such that $\psi(x,b)$ defines a set of 
            cardinality $k$.
            \[\set{b \in \MM^n \st \abs{\psi(\MM,b)} = k}\]
            and even $\set{b \in B^n \st \abs{\psi(\MM,b)} = k}$ by 
            taking the intersection of two definable sets.
    \end{itemize}
    We will become lazier when dealing with definable sets as we gain an idea 
    of what should and should not be definable.
\end{lem}
\begin{proof}
    \begin{itemize}
        \item This is clear.
        \item 
        Since $B$ is $\Si$-definable we can take $\chi(x)$ as the 
        $\Si$-formula defining $B$
        and consider the $\Si$-formula 
        \[\phi(x_1,\dots,x_n) \AND \bigand{i = 1}{n} \chi(x_i)\]
        Clearly this defines $\set{b \in B^n \st \MM \model{\Si} \phi(b)}$.
        \item 
        To make $\set{b \in \MM^n \st \abs{\psi(\MM,b)} \leq k}$
        we take the $\Si$-formula $\chi(x)$:
        \[
            \chi(x) = \bigforall{i = 1}{k + 1} x_i,
            \bigand{i = 1}{k + 1} \psi(x_i,y) \to \bigor{i \ne j}{x_i = x_j}
        \]
        where potentially $x_i$ represents $m$ variables, which we can 
        quantify over as it is finite.
        \item 
        To make $\set{b \in \MM^n \st k \leq \abs{\psi(\MM,b)}}$
        we take the $\Si$-formula $\chi(x)$:
        \[
            \chi(x) = \bigexists{i = 1}{k} x_i, {x_i \ne x_j}
        \]
    \end{itemize}
\end{proof}

\begin{prop}[Algebraic closure is a \linkto{pregeometry_dfn}{pregeometry}]
    \link{acl_d_is_pregeometry}
    Let $\MM$ be a $\Si$-structure. 
    Let $D$ be a minimal subset of $\MM$.
    Then $(D,\acl_{\Si,D})$ is a pregeometry.
\end{prop}
\begin{proof}\cite{fandom0}
    The signature we work in will always be $\Si$ and the 
    strongly minimal subset will always be $D$ so we drop the subscript here.
    Preserves order: 
    \emph{if $A \subs B \subs D$ then $\acl(A) \subs \acl(B)$.}
    Let $a \in \acl(A)$. 
    Then there exists a finite $\Si(A)$-definable set containing $a$.
    Any $\Si(A)$-formula is naturally a $\Si(B)$-formula thus $a \in \acl(B)$.

    Idempotence: \emph{for any $A \subs D$, $\acl(A) = \acl(\acl(A))$.}
    \begin{forward}
        We first show that for any subset $A \subs D$, $A \subs \acl(A)$.
        Let $a \in A$ then $a = x$ is a $\Si(A)$-formula that is defines a 
        finite set. 
        Thus $a \in \acl(A)$.
        Directly we have the corollary $\acl(A) \subs \acl(\acl(A))$.
    \end{forward}

    \begin{backward}
        We show that $\acl(\acl(A)) \subs \acl(A)$.
        Let $a \in \acl(\acl(A))$.
        Then there exists $\phi(x,v) = \phi(x,v_0,\dots,v_n)$ a $\Si$-formula 
        and $b_0,\dots, b_n \in \acl(A)$ such that $\phi(x,b)$
        defines a finite subset of $\MM$ containing $a$.
        Let $k$ be the finite cardinality of $\phi(\MM,b)$.
        \linkto{some_definable_sets}{There exists a $\Si$-formula $\psi(v)$} 
        that defines the set 
        $\set{b \in B \st \abs{\phi(\MM,b)} \leq n}$
        \[
            \phi'(x,v) := \phi(x,v) \AND \psi(v)
        \]
        We have that $a \in \phi(\MM,b) = \phi'(\MM,b)$
        and for any $c \in \MM^n$, $\phi'(\MM,c)$ is finite.
    
        For each $b_i$ appearing in $b$
        there exists a $\Si(A)$-formula $\psi_i(v_i)$ such that 
        $b_i \in \psi_i(\MM)$ and this definable set is finite.
        Define the $\Si(A)$-formula
        \[
            \phi''(x) := \bigexists{i = 1}{n} v_i,
            \phi'(x,v_0,\dots,v_n) \AND \bigand{i = 1}{n} \psi_i(v_i)
        \]
        Then $a \in \phi''(\MM)$ by taking the $v_i$ to be $b_i$ and 
        \begin{align*}
            d \in \phi''(\MM) 
            &\implies \exists c \in \MM^n, \MM \model{\Si}\phi'(d,c) 
            \text{ and for each $i$,} \MM \model{\Si}\psi_i(c_i)\\
            &\implies \text{there exist for each $i$ } c_i \in \psi_i(\MM),
            \MM \model{\Si}\phi'(d,c) \\
            &\implies d \in 
            \bigcup_{i = 0}^n \bigcup_{c_i \in \psi_i(\MM)} \phi'(\MM,c)
        \end{align*}
        The last expression is a finite union of finite sets which is finite.
        Hence $\phi''(\MM)$ is finite and $a \in \acl(A)$
    \end{backward}
    
    Finite character: \emph{if $A \subs D$ and $a \in \acl(A)$ then 
    there exists a finite subset $F \subs A$ such that $a \in \acl(F)$.}
    Take the $\Si(A)$-formula defining the finite set containing $a$.
    Pick out the (finitely many) constant symbols from $A$, 
    forming a finite subset $F \subs A$.
    Then $a \in \acl(F)$.

    Exchange: \emph{if $A \subs D$ and $a,b \in D$ such that 
    $a \in \acl(A,b)$ (shorthand for $A,\set{a}$)
    then $a \in \acl(A)$ or $b \in \acl(A,a)$.}
    Since $a \in \acl(A,b)$ there exists a $\Si(A)$-formula $\phi(v,w)$ such 
    that $a \in \phi(\MM,b)$ and $\phi(\MM,b)$ is finite - 
    say it has cardinality $n$ 
    (if $b$ does not appear in the formula then we immediately have 
    $a \in \acl(A)$).
    \linkto{some_definable_sets}{There exists a $\Si(A)$-formula} $\psi(w)$
    defining the set
    \[\psi(\MM) = \set{b' \in D \st n = \abs{\phi(\MM,b')}}\]
    As $\psi(\MM) \subs D$ and $D$ is minimal, 
    $\psi(\MM)$ is finite or cofinite.
    If it is finite then $b \in \psi(\MM)$ and so 
    $b \in \acl(A) \linkto{acl_d_is_pregeometry}{\subs} \acl(A,a)$.

    If it is $\psi(\MM)$ then consider the $\Si(A)$-formula 
    $\phi(v,w) \AND \psi(w)$.
    For each $a' \in D$ let $X(a')$ be the subset of $D$ defined by 
    $\phi(a',w) \AND \psi(w)$.
    Consider $b \in X(a)$, and case on whether it is finite or cofinite.
    If it is finite then $b \in \acl(A,a)$ as $\phi(a,w) \AND \psi(w)$
    is a $\Si(A)$-formula defining a finite set.

    If $X(a)$ is cofinite then let $m = \abs{D \setminus X(a)} \in \N$.
    \linkto{some_definable_sets}{There exists a $\Si(A)$-formula} $\chi(v)$
    defining the set
    \[\chi(\MM) = \set{a' \in D \st m = \abs{D \setminus X(a')}}\]
    If $\chi(\MM)$ is finite then $a \in \chi(\MM)$ and so $a \in \acl(A)$.
    If $\chi(\MM)$ is confinite then there exist $n + 1$ distinct elements 
    $a_i \in \chi(\MM)$ since $D$ is infinite by definition.
    Take the (finite) intersection of the cofinite $X(a_i)$,
    producing a non-empty (infinite) set.
    Take 
    \[b' \in \bigcap_{i = 1}^{n+1} X(a_i) = 
    \bigcap_{i = 1}^{n+1} \phi(a_i,\MM) \cap \psi(\MM)\] 
    Then for each $i$, $\MM \model{\Si} \phi(a_i,b')$, 
    hence $n + 1 \leq \abs{\phi(\MM,b')}$.
    However $\MM \model{\Si} \psi(b')$ implies $n = \abs{\phi(\MM,b')}$,
    a contradiction.
\end{proof}

The definition of \linkto{dimension_dfn}{dimension} 
    for pregeometries thus carries through for 
    subsets of $D$.
\begin{dfn}
    Let $\MM$ be a $\Si$-structure and let $X \subs D \subs \M$,
    where $D$ is minimal.
    We write $\dim_{\Si,D}(X)$ to mean the 
    \linkto{dimension_dfn}{dimension} 
    of $X$ in the pregeometry $(D,\acl_{\Si,D})$.
    We call this the $\Si$-dimension of $X$ in $D$.
\end{dfn}

\begin{lem}[$\acl$ preserves dimension]
    \link{acl_preserves_dimension}
    Let $\MM$ be a $\Si$-structure and let $X \subs D \subs \M$,
    where $D$ is minimal. 
    Then $\dim_{\Si,D}(X) = \dim_{\Si,D}(\acl_{\Si,D}(X))$.
\end{lem}
\begin{proof}
    Let $S \subs X$ be a basis of $X$. 
    Then it is an independent subset of $X \subs \acl(X)$ such that 
    \[\acl(X) \subs \acl(S) 
    \quad \text{ and by \linkto{acl_d_is_pregeometry}{idempotence} } \quad
    \acl(\acl(X)) \subs \acl(X) \subs \acl(S)\]
    Hence $S$ is a basis for $\acl(X)$.
\end{proof}

\subsection{The theory of algebrically closed fields is strongly minimal}
\begin{dfn}[Strongly minimal theory]
    A $\Si$-theory $T$ is (strongly) minimal if any $\Si$-model 
    of $T$ is (strongly) minimal.
\end{dfn}

\begin{lem}[Disjunctive normal form of definable sets]
    \link{dnf_for_definable_sets}
    Let $K$ be an algebraically closed field.
    Any definable set in $K^n$ can be written in the form 
    \[\bigcup_{i \in S}\brkt{V_i \cap U_i}\]
    where $V_i$ is a variety and $U_i$ is the complement of a variety in $K^n$.
\end{lem}
\begin{proof}
    Let $X$ be a definable set:
    \[X = \set{a \in K^m \st K \model{\Si_\RNG} \phi(a,b)}\]
    where $b \in K^n$ and 
    $\phi$ is some $\Si_\RNG$-formula with $n+m$ free variables.
    Then by \linkto{ACF_has_quantifier_elimination}{quantifier elimination in 
    $\ACF$}
    we have a quantifier free $\Si_\RNG$-formula $\psi$ such that 
    \[X = \set{a \in K^m \st K \model{\Si_\RNG} \psi(a,b)}\]
    We can find the `disjunctive normal form' of $\psi$ as 
    \linkto{disjunctive_normal_form}{it is quantifier free}.
    Hence for $a \in K^m$
    \begin{align*}
        &a \in X\\
        &\iff \bigor{i \in I}{} 
        \brkt{\bigand{j \in J_{i0}}{} p_{ij}(a,b) = 0 \AND 
        \bigand{j \in J_{i1}}{} q_{ij}(a,b) \ne 0}\\
        &\iff a \in \bigcup_{i \in I}
        \brkt{\bigcap_{j \in J_{i0}} \V_K(p_{ij}(x,b)) \cap 
        \bigcap_{j \in J_{i1}} K^m \setminus \V_K(q_{ij}(x,b))}\\
    \end{align*}
\end{proof}

\begin{lem}[Non-trivial vanishings are finite]
    \link{vanishing_finite_or_cofinite}
    If $K$ is a field and $S \subs K[x]$
    then $\V_K(S)$ is finite or $S = \set{0}$.
    In particular $\V_K(S)$ is either finite or cofinite in $K$.
\end{lem}
\begin{proof}
    If $\V(f)$ is finite we are done.
    If $\V(f)$ is infinite
    then each $f \in S$ has infinitely many distinct roots
    so $f = 0$ by the division algorithm.
    
    In particular, if $S = \set{0}$ then $\V(S)$ is $K$ and it is cofinite.
\end{proof}

\begin{prop}
    \link{ACF_strong_min}
    $\ACF$ is strongly minimal.
\end{prop}
\begin{proof}
    Let $K$ be an algebraically closed field.
    
    Let $D \subs K$ be definable.
    Any elementary extension of $K$ is also algebraically closed so 
    without loss of generality we only need to show minimality rather than 
    strong minimality.
    Then \linkto{dnf_for_definable_sets}{there exist 
    $p_{ij}(x,b),q_{ij}(x,b) \in K[x]$} 
    (note that the polynomials are in only one variable) such that 
        \[D = \bigcup_{i \in I}
        \brkt{\bigcap_{j \in J_{i0}} \V_K(p_{ij}(x,b)) \cap 
        \bigcap_{j \in J_{i1}} K \setminus \V_K(q_{ij}(x,b))}\]
    Which is a finite union and intersection of 
    \linkto{vanishing_finite_or_cofinite}{finite and cofinite sets},
    which is finite or cofinite.
    Hence $D$ is finite or cofinite and $\ACF$ is strongly minimal.
\end{proof}

\subsection{Dimension and transcendence degree}
We show that dimension and transcendence degree are the same thing.

\begin{lem}[Formulas defining finite sets and polynomials]
    \link{formulas_defining_finite_sets_give_poly}
    Suppose $K$ is an algebraically closed field, 
    $S$ is a subset of an extension field.
    $\phi$ is a $\Si_\RNG(K,S)$-formula with exactly $1$ free variable.
    If $\phi$ defines a finite set containing $a \in K(S)$ 
    then there exists a non-zero polynomial $p \in K(S)[x]$ such that 
    $p(a) = 0$.
\end{lem}
\begin{proof}
    First take the \linkto{disjunctive_normal_form}{
        disjunctive normal form of $\phi$}
    \[K(S) \model{\Si(K,S)} \forall v, \phi \IFF \bigor{i \in I}{} 
        \brkt{\bigand{j \in J_{i0}}{} p_{ij}(v) = 0 \AND 
        \bigand{j \in J_{i1}}{} q_{ij}(v) \ne 0}\]
    where $\Si(K,S)$-terms $p_{ij},q_{ij}$ are naturally 
    \linkto{terms_in_RNG_are_polynomials}{polynomials} in $K(S)[x]$.
    We see that there is some $i \in I$ such that 
    \[K(S) \model{\Si_\RNG(K,S)} 
        \bigand{j \in J_{i0}}{} p_{ij}(a) = 0 \AND 
        \bigand{j \in J_{i1}}{} q_{ij}(a) \ne 0
    \]
    The set defined by $\bigand{j \in J_{i0}}{} p_{ij}(v) = 0$ in $K(S)$
    is $\V_{K(S)}(\set{p_{ij} \st j \in J_{i0}})$,
    hence is \linkto{vanishing_finite_or_cofinite}{either finite
    (there exists a non-zero polynomial) 
    or all of $K(S)$ (all polynomials are zero)}.
    In the first case we obtain a non-zero polynomial $p \in K(S)[x]$ 
    such that $p(a) = 0$ and we are done.

    Assume for a contradiction the second case holds.
    For each $j \in J_{i1}$ 
    consider the set defined by $q_{ij}(v) \ne 0$ in $K(S)$,
    which is \linkto{vanishing_finite_or_cofinite}{cofinite or empty}.
    If it is empty then it is not satisfied by $a$ which is a contradiction.
    Hence 
    \[
        \bigand{j \in J_{i0}}{} p_{ij}(v) = 0 \AND 
        \bigand{j \in J_{i1}}{} q_{ij}(v) \ne 0
    \]
    defines a cofinite subset of $\phi(K(S))$, and so $\phi(K(S))$ is 
    \linkto{card_of_alg_closed_fields}{infinite}, a contradiction.
\end{proof}

A result that uses the previous lemma is that the usual algebraic closure
for fields is indeed a special case of our definition of algebraic closure.
This is included only because it is noteworthy and can be skipped.
\begin{prop}[Algebraic closure in $\ACF$ is an field theoretic 
    algebraic closure]
    Let $K \to M$ be a field extension with $M$ algebraically closed, 
    then as \linkto{ACF_strong_min}{$\ACF$ is strongly minimal} 
    we have that $M$ is strongly minimal.
    Consider a subset $A \subs M$.
    Then $\acl_{\Si(K),M}(A)$ is an algebraic closure of $K(A)$.
\end{prop}
\begin{proof}
    We write $\acl$ instead of $\acl_{\Si(K),M}$.
    We must show that $\acl(A)$ is a field that contains $K(A)$, 
    is algebraically closed and is an algebraic extension of 
    $K(A)$.

    To show that it is a field we only show closure for addition and leave
    the rest as an exercise. 
    Let $a, b \in \acl(A)$.
    Then there exist $\Si(K,A)$-formulas $\phi_a,\phi_b$ that define 
    finite subsets of $M$ containing $a$ and $b$ respectively.
    Then their sum is in the finite set defined by the formula with one free
    variable $z$:
    \[
        \exists x, \exists y, \phi_a(x) \AND \phi_b(y) \AND z = x + y
    \]

    It contains $K(A)$ if and only if it contains $A$ and the image of $K$.
    It contians $K$ since for any element $k \in K$ we can take the 
    $\Si(K,A)$-formula $x = k$. 
    Similarly for $A$.

    To show that it is algebraically closed let 
    $p \in \acl(A)[x]$ be a polynomial.
    Then we write $p$ out in terms of its coefficients $a_i \in \acl(A)$
    \[p = \sum_{i = 1}^m a_i x^i\]
    Each coefficient is in a finite subset of $M$
    defined by a $\Si(K,A)$-formula $\phi_i$.
    The formula with free variable $x$
    \[
        \bigexists{i = 1}{m} v_i, \bigand{i = 1}{m} \phi_i(v_i) \AND 
        \sum_{i = 1}^m v_i x^i
    \]
    defines the set of roots of $p$ in $M$.
    Since $M$ is algebraically closed this is all the roots and since
    $p$ only has finitely many roots it is finite.
    Hence all the roots of $p$ are in $\acl(A)$.

    To show that it is an algebraic extension of $K(A)$ we take $a \in \acl(A)$
    and obtain a formula defining a finite subset of $M$ containing $a$.
    Thus 
    \linkto{formulas_defining_finite_sets_give_poly}{there exists a polynomial}
    $p \in K(A)[x]$ with $a$ as a root, and so $a$ is algebraic over $K(A)$.
\end{proof}

\begin{prop}[Transcendence degree is dimension]
    \link{trans_deg_is_dim}
    Let $K \to L$ be a field extension.
    Suppose $S \subs L$ and 
    $M$ is an algebraically closed field extension of $L$.
    We consider the pregeometry $(M,\acl_{\Si_\RNG(K),M})$.
    \begin{enumerate}
        \item $S$ is algebraically independent over $K$ if and only if 
            $S$ is independent in the pregeometry.
        \item $K(S) \to L$ is an algebraic extension if and only if $S$ 
            spans $L$ in the pregeometry.
        \item $S$ is a transcendence basis for the extension $K \to L$ 
            if and only if 
            $S$ is a basis for $L$ in the pregeometry.
        \item Transcendence degree is the same thing as dimension:
            \[\tdeg(K \to K(S)) = \dim_{\Si(K),M}(L)\]
    \end{enumerate}
\end{prop}
\begin{proof}~
    We will write $\acl$ to mean $\acl_{\Si_\RNG(K),M}$.
    \begin{enumerate}
        \item \begin{forward}
            Let $a \in S$ and suppose for a contradiction that 
            $a \in \acl(S \setminus \set{a})$.
            Then there exists a $\Si_\RNG(K,S \setminus \set{a})$-formula $\phi$
            defining a finite subset of $\M$ containing $a$.
            Then \linkto{formulas_defining_finite_sets_give_poly}{
                there exists a non-zero polynomial} 
            $p \in K(S \setminus \set{a})[x]$ such that 
            $p(a) = 0$.
            This contradicts algebraic independence of $S$.
        \end{forward}

        \begin{backward}
            Suppose $p \in K[x_0,\dots,x_n]$ with distinct $s_0,\dots,s_n \in S$
            such that $p(s_0,\dots,s_n) = 0$.
            Then we have a polynomial in one variable
            \[p(x_0,s_1,\dots,s_n) \in K(S \setminus \set{s_0})[x_0]\]
            with $s_0$ as a root.
            Elements of $K(S \setminus \set{s_0})$ can be written as 
            polynomials over $K$ in variables from $S \setminus \set{s_0}$,
            which are $\Si_\RNG(K,S \setminus \set{s_0})$-terms;
            hence $p(x_0,s_1,\dots,s_n)$ is naturally a 
            $\Si_\RNG(K,S \setminus \set{s_0})$-term
            as well.
            Call $\phi(x_0)$ the $\Si_\RNG(K,S \setminus \set{s_0})$-formula 
            `$p(x_0,s_1,\dots,s_n) = 0$'.
            Since 
            \linkto{vanishing_finite_or_cofinite}{
                non-zero polynomials have finitely many roots} $\phi(M)$
            is finite or $p = 0$; since $s_0$ is a root, 
            \[M \model{\Si_\RNG (K,S \setminus \set{s_0})} \phi(s_0)\]
            and so if it is finite then $s_0 \in \acl(S \setminus \set{s_0})$,
            contradicting pregeometrical independence.
            Hence $p = 0$ and so $S$ is algebraically independent.
        \end{backward}
        \item \begin{forward}
            By \linkto{acl_d_is_pregeometry}{idempotence} of $\acl$, 
            for $S$ to be spanning in the pregeometry 
            it suffices to show that $L \subs \acl(S)$ 
            (take $\acl$ of both sides).
            Let $a \in L$.
            Since $K(S) \to L$ is an algebraic extension 
            there exists a non-zero polynomial 
            $p \in K(S)[x]$ such that $p(a) = 0$.
            We write $p$ as a polynomial over $K$ in variables from $S$,
            which is a $\Si_\RNG(K,S)$-term, 
            and so the $\Si_\RNG(K,S)$-formula `$p = 0$'
            defines a finite set containing $a$.
        \end{forward}
        
        \begin{backward}
            Let $a \in L$.
            We want to show that $a$ is algebraic over $K(S)$
            Since $S$ spans $L$
            \[a \in L \subs \acl(L) \subs \acl(S)\]
            and so there exists a $\Si(K,S)$-formula defining a finite set 
            containing $a$.
            \linkto{formulas_defining_finite_sets_give_poly}{Hence there exists
            a non-zero polynomial} $p \in K(S)[x]$ with $a$ as a root.
            Hence the extension is algebraic.
        \end{backward}

        \item
            $S$ is a transcendence basis of the extension $K \to L$ 
            \linkto{transcendence_bases_algebraic_extensions}{if and only if}
            $K(S) \to L$ is algebraic and $S$ is 
            algebraically independent over $K$.
            By the last two parts this is if and only if $S$ is independent 
            and spanning $L$ in the pregeometry, 
            which is if and only if $S$ is pregeometrical basis for $L$.
        
        \item This is clear.
    \end{enumerate}
\end{proof}

\subsection{Morley rank of types and dimension}
This section is purely model theoretic.
It allows us to connect our definition dimension together with Morley rank 
for types of tuples.

\begin{prop}[Projection and Morley rank]
    \link{proj_and_morley_rank}
    Suppose $\M$ is an $\om$-saturated $\Si$-structure.
    Let $\phi$ be a $\Si(\M)$-formula and let $v$ be a variable symbol.
    Then 
    \[\MR{}{\exists v, \phi} \leq \MR{}{\phi}\]
\end{prop}
\begin{proof}
    First we remove the case where $v$ is not a free variable of $\phi$ 
    by noting that if this is the case then $\exists v, \phi$ and $\phi$ 
    have the same number of free variables and so $a \in (\exists v, \phi)(\M)$
    if and only if $\M \modelsi \exists v, \phi(a)$ if and only if 
    $\M \modelsi \phi(a)$ if and only if $a \in \phi(\M)$.
    \linkto{implication_subset_inequality}{Hence} 
    they have the same Morley rank.

    Then we induct on $\al \in \ord$ to show that 
    $\al \leq \MR{}{\exists v, \phi}$ implies $\al \leq \MR{}{\phi}$.
    We will not bother with the cases $- \infty$ and $\infty$ and the former
    is trivial and the latter follows from the induction result.
    If $0 \leq \MR{}{\exists v, \phi}$ then there exists $a \in \M^n$
    such that $\M \modelsi \exists v, \phi(a,v)$. 
    Hence there is $b \in \M$ such that 
    $\M \modelsi \phi(a,b)$ and so $0 \leq \MR{}{\phi}$.

    The non-zero limit ordinal case is trivial. 
    Suppose $\al + 1 \leq \MR{}{\exists v, \phi}$.
    Then there exist $\Si(\M)$-formulas $\psi_i$ defining disjoint subsets of 
    $(\exists v, \phi) (\M)$ such that $\al \leq \MR{}{\psi_i}$.
    Then take $\Si(\M)$-formulas $\chi_i$ to be $\psi_i \AND \phi$.\footnote{
        It is important here that we notice $\psi_i \AND \phi$ and 
        $\phi$ have the same number of free variables, and so we can say that 
        $(\psi_i \AND \phi)(\M) \subs \phi(\M)$, which is not the case for 
        if we just took $\psi_i$, since it is one free variable short.
        (In general $\M^n \cap \M^{n+1} = \nothing$!)
        }
    If $a \in (\exists v, \phi)(\M)$ then 
    $a \in (\exists v, \psi \AND \phi)(\M)$ and so 
    \[
        \psi_i(\M) \subs (\exists v, \phi)(\M) 
        \subs (\exists v, \psi \AND \phi)(\M) \quad \implies \quad
        \al \leq \MR{}{\psi_i} \linkto{implication_subset_inequality}{\leq} 
        \MR{}{\exists v, \chi_i}
    \]
    by the induction hypothesis we have $\al \leq \chi_i$ and the
    $\chi_i$ define disjoint subsets of $\phi(\M)$, 
    hence $\al+1 \leq \MR{}{\phi}$.
\end{proof}

\begin{lem}[Morley rank of extended types \cite{marker}]
    \link{morley_rank_of_extended_types}
    Suppose $\M$ is a $\ka$-saturated, strongly minimal $\Si$-structure.
    Let $A \subs \M$ be such that $\abs{A} < \ka$.
    Let $a \in \M^n$ and $b \in \M$.
    Then 
    \[\MR{}{\subintp{A}{\M}{\tp}(a)} \leq \MR{}{\subintp{A}{\M}{\tp}(a,b)}\]
    Furthermore `if $b$ is dependent then removing it preserves Morley rank'
    \[b \in \acl_{\Si(A),\M}(\set{a_1,\dots,a_n}) \implies 
    \MR{}{\subintp{A}{\M}{\tp}(a)} = \MR{}{\subintp{A}{\M}{\tp}(a,b)}\]
    This says we can reduce finite sets to subsets that are independent 
    in the pregeometry whilst preserving Morley rank of the type.
\end{lem}
\begin{proof}
    We write $\MR{}{\star}$ to mean $\MR{}{\subintp{A}{\M}{\tp}(\star)}$.
    We use $x$ or $x_1,\dots,x_n$ to denote the variables corresponding to $a$
    and use $v$ to denote the variable corresponding to $b$.

    Let $\phi$ be a \linkto{morley_rank_for_types_dfn}{rank representative}
    for $\tp(a,b)$.
    Then $a \in (\exists v, \phi)(\M)$ and so $(\exists v, \phi) \in \tp(a)$.
    Hence
    \[
        \MR{}{a} \leq \MR{}{\exists v, \phi} \linkto{proj_and_morley_rank}{\leq} 
        \MR{}{\phi} \leq \MR{}{a,b}
    \]

    For the other inequality
    it suffices to show by induction on $\al \in \ord$ that 
    If $a \in \M^n$ and $b \in \M$.
    Then `if $b$ is dependent' then 
    \[\al \leq \MR{}{a,b} \implies \al \leq \MR{}{a}\]
    For the base case we note that $a \in \phi(\M)$ for any rank representative 
    $\phi$ of $\tp(a)$ and so $\phi(\M)$ is non-empty and 
    \[0 \leq \MR{}{\phi} = \MR{}{a}\]
    The non-zero limit ordinal case is trivial.

    Suppose for the successor case $\al + 1 \leq \MR{}{a,b}$. 
    Then $\al \leq \MR{}{a,b}$ and so by the induction 
    hypothesis $\al \leq \MR{}{a}$.
    \linkto{smallest_rank_rep}{There exists a `smallest' rank representative} 
    $\phi(x)$ of $\tp(a)$ such that 
    there is no $\Si(A)$-formula $\psi$ such that 
    \[\MR{}{\phi \AND \psi} = \MR{}{\phi \AND \NOT \psi} = \al\]
    Since $b \in \acl(a)$ we have a $\Si(A)$-formula $\psi(x,v)$ such that 
    $b \in \psi(a,\M)$ and $\psi(a,\M)$ is finite with cardinality $n$.
    Let $\abs{\psi(x,\M)} = n$ denote 
    \linkto{some_definable_sets}{the formula} defining the set of $a' \in \M^n$
    such that $\abs{\psi(a',\M)} = n$.
    Then define the $\Si(A)$-formula
    \[\Phi(x,v) := \phi(x) \AND \psi(x,v) \AND \abs{\psi(x,\M)} = n\]

    Note that $\Phi \in \tp(a,b)$ and so 
    $\al + 1 \leq \MR{}{a,b} \leq \MR{}{\Phi}$.
    Thus for $i \in \N$ there exist formulas $\theta_i$ with rank at least
    $\al$ defining disjoint subsets of $\Phi(\M)$.

    We show that for each $m \in \N_{>0}$ 
    \[\al \leq \MR{}{\bigand{i = 1}{m} \exists y, \theta_i}\]
    We first show for each $i$ that $\al \leq \MR{}{\exists v, \theta_i}$,
    covering the base case $m = 1$.
    \linkto{formulas_rep_by_types}{There exist representative types},
    element $c \in \M^n$ and $d \in \M$: $\theta_i \in \tp(c,d)$
    and $\MR{}{\theta_i} = \MR{}{c,d}$.
    If $d \in \acl(c)$ then by induction we have 
    \[\al \leq \MR{}{c,d} 
    \implies \al \leq \MR{}{c} \leq \MR{}{\exists v, \theta_i}\]
    Indeed $d \in \acl(c)$ since 
    $\theta_i(\M) \subs \Phi(\M)$ which says 
    \[d \in \psi(c, \M) \text{ and } \abs{\psi(c,\M)} = n\]

    Suppose it is true for $m$.
    Write $\chi(x) := \bigand{i = 1}{m} \exists v, \theta_i$.
    Then for each $i$, $(\exists v, \theta_i)(\M) \subs \phi(\M)$ implies 
    $\chi(\M) \subs \phi(\M)$ and 
    \[
        \al \leq \MR{}{\chi} \leq \MR{}{\phi} = \al 
        \implies \al = \MR{}{\chi} = \MR{}{\phi \AND \chi}
    \]
    We showed above that $\al \leq \exists y, \theta_m$.
    Partition $(\exists y, \theta_m)(\M)$ into its intersection with 
    $\chi(\M)$ and $\NOT \chi(\M)$.
    Supposing for a contradiction 
    \[\al \nleq \MR{}{\chi \AND \exists v, \theta_m}\]
    we see that $\al \leq \MR{}{\exists y, \theta_m}$ is the 
    \linkto{basic_facts_morley_rank_of_dfnbl_set}{maximum} of the two parts
    which must be $\MR{}{\NOT \chi \AND \exists v, \theta_m}$.
    Then 
    \[
        \al \leq \MR{}{\exists v, \theta_m \AND \NOT \chi} 
        \leq \MR{}{\phi \AND \NOT \chi} \leq \MR{}{\phi} = \al 
        \implies \al = \MR{}{\phi \AND \NOT \chi}
    \]
    This contradicts the property of $\phi$ being the 
    `smallest' rank representative.

    In particular we have shown, applying 
    \linkto{compactness_for_types}{compactness},
    that $\set{\exists v, \theta_i}$ is consistent and thus can be 
    \linkto{extend_to_maximal_type_zorn}{extended} to 
    an element of $S_n(\Theory_\M(A))$, which is realised by some $c \in \MM^n$ 
    as it is $\ka$-saturated.
    Thus we have for each $i$ that $\M \model{\Si(A)} \exists v, \theta_i(c,v)$,
    giving us $d_i \in \M$ such that $\M \model{\Si(A)} \theta_i(c,d_i)$.
    However the $\theta_i(c,\M)$ are disjoint subsets of 
    $\Phi(c,\M) \subs \psi(c,\M)$ and so the $d_i$ must be
    distinct elements of $\psi(c,\M)$ and so $\psi(c,\M)$ is infinite.
    However, $(c,d_1) \in \Phi(\M)$ so $\abs{\psi(c,\M)} = n$, a contradiction.
\end{proof}

\begin{lem}[Independent tuples have the same type \cite{marker}]
    \link{independent_tuples_same_type}
    Let $\MM$ be a $\Si$-structure with $A \subs D \subs \MM$,
    where $D$ is minimal and $\Si(A)$-definable.
    Let $a \in D^k$ and $b \in D^k$. 
    If $\set{a_1,\dots,a_k}$ and $\set{b_1,\dots,b_k}$ are both 
    independent in the pregeometry $(D,\acl_{\Si(A),D})$
    then 
    \[\subintp{A}{\MM}{\tp}(a) = \subintp{A}{\MM}{\tp}(b)\]
\end{lem}
\begin{proof}
    We induct on $k \in \N$. 
    We write $\tp$ for $\subintp{A}{\MM}{\tp}$.

    For the base case $k = 0$ we note that empty tuples define the same type
    \[{\tp}(a) = {\tp}(\nothing)= {\tp}(b)\]
    More concretely this is equal to $\Theory_{\MM}(A)$.

    Suppose it is true for $k$.
    Let $\set{a_1,\dots,a_{k+1}}$ and $\set{b_1,\dots,b_{k+1}}$ be independent
    in the pregeometry.
    Then by induction $\tp(a_1,\dots,a_k) = \tp(b_1,\dots,b_k)$.
    Let $\psi \in \tp(a_1,\dots,a_{k+1})$.
    It suffices to show that $\psi \in \tp(b_1,\dots,b_{k+1})$.
    Note that $a_k \in \psi(a_1,\dots,a_k, \MM) \cap D$, 
    which is a \linkto{some_definable_sets}{
        $\Si(A,a_1,\dots,a_k)$-definable subset} of $D$.
    Since by assumption $\set{a_1,\dots,a_{k+1}}$ is independent 
    this definable set must be infinite, 
    and by minimality of $D$ its complement 
    $\NOT \psi(a_1,\dots,a_k, \MM) \cap D$ is finite
    with cardinality $n$, say.
    
    \linkto{some_definable_sets}{There exists a $\Si(A)$-formula} $\chi$
    in free-variables $x_1,\dots,x_k$ defining the set of $c \in D^k$ such that
    \[\abs{\NOT \psi(c,\MM) \cap D} = n\]
    Since $\chi \in \tp(a_1,\dots,a_k) = \tp(b_1,\dots,b_k)$
    implies $\abs{\NOT \psi(b_1,\dots,b_k,\MM) \cap D} = n$,
    \[
        b_{k+1} \notin \psi(b_1,\dots,b_k,\MM) \implies 
        b_{k+1} \in \NOT \psi(b_1,\dots,b_k,\MM) \text{ finite} \implies 
        b_{k+1} \text{ is dependent, contradiction}
    \]
    We have $b_{k+1} \in \psi(b_1,\dots,b_k,\MM)$ and so 
    $\psi \in \tp(b_1,\dots,b_k)$.
\end{proof}

\begin{lem}[Monsterous structures have infinite dimension]
    \link{dim_of_monster_is_infinite}
    Let $\ka$ be a cardinal.
    Let $\M$ be a strongly minimal and $\ka$-saturated $\Si$-structure
    such that $\ka \leq \M$.
    Let $A \subs \M$ such that $\abs{A} < \ka$.
    Then $\dim_{\Si(A),\M}(\M)$ is infinite.
\end{lem}
\begin{proof}
    We work in the pregeometry $(\M,\acl_{\Si(A),\M})$ and write 
    $\dim$ for $\dim_{\Si(A),\M}$ and $\acl$ for $\acl_{\Si(A),\M}$.
    
    Suppose for a contradiction $B$ is a finite basis for $\M$ 
    in the pregeometry.
    We construct an injection to show that 
    \[\abs{\acl(B)} < \ka\]
    which is a contradiction as $\acl(B) = \M$.

    Let $b \in \acl(B)$. 
    Then there exists a $\Si(A,B)$ formula $\phi$ defining a finite subset of 
    $\M$ containing $b$.
    Then there exists $n \in \N$ and a bijection $\io_\phi : \phi(\M) \to n$.
    We define using choice
    \[
        f : \acl(B) \to \form{\Si(A,B)} \times \N, 
        \quad b \mapsto (\phi,\io_\phi(b))
    \]
    Then $f$ is injective and so 
    \[
        \abs{\acl(B)} \leq 
        \abs{\form{\Si(A,B)}} + \aleph_0 \leq 
        \aleph_0 + \abs{A} + \abs{B} = \aleph_0 + \abs{A}
        < \aleph_0 + \ka = \ka
    \]
    Hence we have our contradiction.
\end{proof}

\begin{prop}[Morley rank of types and dimension]
    \link{morley_rank_of_types_is_dim}
    Let $\ka$ be a cardinal.
    Let $\M$ be a strongly minimal and $\ka$-saturated $\Si$-structure
    such that $\ka \leq \M$.
    Let $A \subs \M$ such that $\abs{A} < \ka$.
    Let $a \in \M^k$. Then 
    \[\MR{}{\subintp{A}{\M}{\tp}(a)} = \dim_{\Si(A),\M}(\{a_1,\dots,a_k\})\]
\end{prop}
\begin{proof}
    We work in the pregeometry $(\M,\acl_{\Si(A),\M})$ and write 
    $\dim$ for $\dim_{\Si(A),\M}$ and $\acl$ for $\acl_{\Si(A),\M}$.
    We also write $\MR{}{a}$ for $\MR{}{\subintp{A}{\M}{\tp}(a)}$ and
    $\tp$ for $\subintp{A}{\M}{\tp}$.

    Let us first show that without loss of generality 
    $\{a_1,\dots,a_k\}$ is independent in the pregeometry.
    Concretely, we prove by induction on $k$ that there exists an independent
    subset $s \subs \set{a_1,\dots,a_k}$ such that 
    \[\MR{}{s} = \MR{}{a} \quad \text{ and } \quad \dim(s) = \dim(a)\]
    If $k = 0$ then it is okay as the empty set is trivially independent.
    If $0 < k$ then we case on if $\set{a_1,\dots,a_k}$ is independent
    or not.
    If it is then we are done. 
    Otherwise remove a dependent element $a_i$.
    By the theorem on \linkto{morley_rank_of_extended_types}{Morley 
        rank of extended types} we have that 
    \[\MR{}{a \setminus \set{a_i}} = \MR{}{a}\]
    Since \linkto{acl_preserves_dimension}{$\acl$ preserves dimension}
    and $a_i$ is dependent
    \[
        \dim(a \setminus \set{a_i}) = \dim(\acl(a \setminus \set{a_i}))
        = \dim(\acl(a)) = \dim(a)
    \]
    Hence by induction there is a subset $s \subs a \setminus \set{a_i} \subs a$
    such that 
    \[\MR{}{s} = \MR{}{a \setminus \set{a_i}} = \MR{}{a}
    \text{ and } \dim(s) = \dim(a \setminus \set{a_i}) = \dim(a)\]
    Hence $a$ can be replaced by an independent subset.

    Now we show that for independent $a_1,\dots,a_k$, $\MR{}{a} = k$.
    This will complete the proof since independent sets 
    are bases for themselves, which implies $\dim(a) = k = \MR{}{a}$.
    We prove this by induction on $k$.
    If $k = 0$ then $\subintp{A}{\M}{\tp}(a)$ 
    consists of $\Si(A)$-sentences satisfied by $\M$, 
    which \linkto{definable_set}{by convension} each define the set 
    $\set{\nothing}$. 
    Hence everything in $\tp(a)$ has 
    \linkto{basic_facts_morley_rank_of_dfnbl_set}{Morley rank $0$} 
    and so $\MR{}{a} = 0$.

    For the induction step (formally we should take out the $k=1$ case %? Lazy
    too but this is also straight forward) suppose 
    $a_1,\dots,a_{k+1}$ form an independent set in the 
    pregeometry.
    We first show that $k + 1 \leq \MR{}{a}$.
    Let $\phi(x_1,\dots,x_{k+1})$ be a rank representative of $a$.
    Since the \linkto{dim_of_monster_is_infinite}{dimension of $\M$ is infinite}
    we can take a countably infinite 
    subset of a basis $\set{b_i}_{i \in \N}$. 
    Let the formulas 
    \[
        \psi_i := \phi \AND (x_1 = b_i)
    \]
    define disjoint subsets of $\phi(\M)$.
    We need to show that these formulas have Morley rank at least $k$.
    Since \linkto{independent_tuples_same_type}{independent sets of the same 
        length have the same type}, for each $i \in \N$ 
    \[\tp(a_1,\dots,a_{k+1}) = \tp(b_i,\dots,b_{i+k})\]
    In particular $\phi \in \tp(b_i,\dots,b_{i+k})$ and so 
    $\psi_i \in \tp(b_i,\dots,b_{i+k})$.
    Hence, by induction and using the 
    \linkto{morley_rank_of_extended_types}{lemma on extended types}
    \[
        k = \MR{}{b_i,\dots,b_{i+k-1}} 
        {\leq} \MR{}{b_i,\dots,b_{i+k}}
        \leq \MR{}{\psi_i}
    \]
    Hence we have $k + 1 \leq \MR{}{\phi} = \MR{}{a}$.

    Suppose for a contradiction that $k + 2 \leq \MR{}{\phi}$.
    Then we have for each $i \in \N$, $\Si(\M)$-formulas $\chi_i$
    of Morley rank $k + 1$ defining disjoint subset of $\phi(\M)$.
    As they are disjoint there exists an $i$ such that $a \notin \chi_i(\M)$.
    We take $\tp(c) \in S_n(\Th_\M(A))$, 
    a \linkto{formulas_rep_by_types}{type representing this $\chi_i$}
    realised by some $c \in \M^{k+1}$ by $\ka$-saturation:
    \[\MR{}{c} = \MR{}{\chi_i} = k + 1 \text{ and } \chi_i \in \tp(c)\]
    Suppose for a contradiction that $c_1,\dots,c_{k+1}$ are independent in the 
    pregeometry.
    Then as \linkto{independent_tuples_same_type}{independent sets of the same 
        length have the same type} we have $\tp(a) = \tp(c)$,
    which implies $\chi_i \in \tp(a)$ so $a \in \chi_i(\M)$, a contradiction.
    There is a \linkto{independence_and_span_basic}{strictly smaller subset}
    $B \subs \set{c_1,\dots,c{k+1}}$ 
    that is a pregeometrical basis for $\set{c_1,\dots,c{k+1}}$.
    This will satisfy $\MR{}{B} = \MR{}{c}$ (by induction using the lemma on 
    \linkto{morley_rank_of_extended_types}{extended types}).
    By the induction hypothesis, noting that $B$ is independent, we have
    \[\MR{}{\chi_i} = \MR{}{c} = \MR{}{B} = \abs{B} < k + 1\]
    which contradicts $\MR{}{\chi_i} = k + 1$.
    Hence $k + 1 = \MR{}{\phi} = \MR{}{a}$.
\end{proof}

\subsection{Morley rank is Krull dimension}
We are now ready to show that Morley rank is the same thing as Krull dimension.
A prerequisite is the \linkto{transendence_deg_is_krull}{result} 
in algebraic geometry that transcendence degree 
is the same thing as Krull dimension.

\begin{rmk}[Defining formula for Zariski closed sets]
    \link{defining_formula_for_zariski_closed}
    Let $K$ be an alebraically closed field and $V \subs K^n$ 
    a Zariski closed set. 
    Then it is \linkto{zariski_closed_sets_are_fin_gen}{a 
        finitely generated vanishing} 
    so there exists a finite $S_V \subs K[x_1,\dots,x_n]$ such that 
    $V$ is defined by the $\Si(K)$-formula 
    \[\bigand{p \in S_V}{} p = 0\]
    where each polynomial is 
    \linkto{terms_in_RNG_are_polynomials}{naturally a $\Si(K)$-term}.
\end{rmk}

\begin{prop}[Morley rank is Krull dimension for algebraically closed fields
    \cite{marker}]
    Let $K$ be an algebraically closed field and $V \subs K^n$ a variety.
    Then the Morley rank of $V$ is equal to its 
    \linkto{dfn_krull_dimension}{Krull dimension}.
\end{prop}
\begin{proof}
    We write $\tp$ for $\subintp{K}{\M}{\tp}$.
    We show by induction that for each $n \in \N$
    if $\kdim(V) = n$
    then $\MR{}{V} = n$.
    Since \linkto{dfn_krull_dimension}{
        Krull dimension is always finite} this is sufficient.
    If $n = 0$ then \linkto{krull_dim_0}{$V$ is a singleton} 
    hence \linkto{basic_facts_morley_rank_of_dfnbl_set}{$\MR{}{V} = 0$}.

    Suppose $n > 0$ and $\kdim(V) = n$.
    Let $\ka$ be a cardinal such that $\abs{K} < \ka$.
    Then \linkto{upwards_lowenheim_skolem}{
        there exists an elementary extension of $K$
        with cardinality $\ka$};
    and then take $\M$ a 
    \linkto{existence_of_monster}{$\ka$-saturated elementary extension} of 
    that.
    We replace $K$ and $V$ with their images in $\M$ to make things simpler.
    The variety $V$ is Zariski closed by definition and so it is 
    \linkto{defining_formula_for_zariski_closed}{defined by a $\Si(K)$-formula}
    \[\phi := \bigand{p \in S_V}{} p = 0\]
    for some finite $S_V \subs K[x_1,\dots,x_n]$.
    
    Note that since 
    \linkto{formulas_rep_by_types}{formulas are represented by types} 
    and $\ka$-saturation we have
    \[
        \MR{}{V} = \MR{}{\phi} = 
        \max \set{\MR{}{q} \st \phi \in q \in S_n(\Theory_\M(K))}
        = \max \set{\MR{}{\tp(a)} \st a \in \phi(\M)}
    \]
    Thus there exists $a \in \phi(\M)$ such that $\MR{}{\tp(a)} = \MR{}{V}$.
    It suffices to show that $\MR{}{\tp(a)} = n$.

    Let $I(a)$ denote the ideal in $K[x_1,\dots,x_n]$ 
    of polyomials vanishing at $a$ 
    (extending our previous definition of $I$ beyond $K^n$).
    Then $V_a := \V_K(I(a))$ is Zariski closed and so it is 
    \linkto{defining_formula_for_zariski_closed}{defined by a $\Si(K)$-formula}
    \[\psi := \bigand{p \in S_{V_a}}{} p = 0\]
    for some finite $S_{V_a} \subs K[x_1,\dots,x_n]$.
    We first show that $V_a = V$.
    To show that $V_a \subs V$, we take $b \in V_a$ and note that 
    for each $p \in S$, $p(b) = 0$ by definition and so  
    $b \in \phi(K) = V$.
    Suppose for a contradiction $V_a \subset V$.
    Then the $\kdim(V_a) < \kdim(V) = n$.
    Hence by the induction hypothesis $\MR{}{V_a} < n$ 
    and since $a \in \psi(\M)$
    \begin{align*}
        \MR{}{V} 
        &= \MR{}{\tp(a)} \\
        &\leq \max\set{\MR{}{\tp(b) \st b \in \psi(\M)}}\\
        &= \max \set{\MR{}{q} \st \psi \in q \in S_n(\Theory_\M(K))}\\
        &\linkto{formulas_rep_by_types}{=} \MR{}{\psi} \\
        &= \MR{}{V_a} < n
    \end{align*}
    Hence by the induction hypothesis $V$ 
    has Krull dimension strictly less than $n$, a contradiction.
    Thus $V_a = V$.

    Now we have that $K(a)$ is isomorphic to the function field $K(V)$,
    and \linkto{iso_field_ext_same_trans_deg}{
        hence have the same transcendence degree}:
    \[
        K(a_1,\dots,a_n) \iso 
        K[x_1,\dots,x_n] / I(V_a) = 
        K[x_1,\dots,x_n] / I(V) = K(V)
    \]
    As \linkto{transendence_deg_is_krull}{Krull dimension is 
        transcendence degree, } 
    $n = \kdim(V) = \tdeg(K \to K(V)) = \tdeg(K \to K(a))$. 
    This in turn is the same as 
    $\dim_{\Si_\RNG(K),\M}(\set{a_1,\dots,a_n})$
    since \linkto{trans_deg_is_dim}{transcendence degree is dimension}.
    \linkto{ACF_strong_min}{$\ACF$ is strongly minimal}
    and $\ka$-saturated satisfying $\ka \leq \M$,
    thus \linkto{morley_rank_of_types_is_dim}{dimension 
        corresponds to Morley rank of the type}
    \[
        k = \dim_{\Si_\RNG(K),\M}(\set{a_1,\dots,a_n}) 
        = \MR{}{\tp(a)} = \MR{}{V}
    \]
\end{proof}
%\chapter{Appendix}
%\section{Algebraic closure}
\begin{dfn}
    Let $K$ be a field.
    $L$ is an algebraic closure of $K$ 
    if there exists an algebraic field extension 
    $\io : K \to L$ and $L$ is algebraically closed.
\end{dfn}

\begin{prop}[Existence of algebraic closure of Fields \cite{atiyah}]
    \link{algebraic_closure_of_fields}
    Let $K$ be a field. 
    Then there exists an algebraic closure of $K$.
\end{prop}
\begin{proof}
    We build $L$ by induction, 
    at each step making an algebraic extension to a greater field
    containing roots of irreducible polynomials in the previous field.
    Without loss of generality we just do this for $K$.

    Let $F$ be the set of irreducible polynomials $f \in K[x]$.
    Consider $K[x_f]_{f \in F}$, 
    where each $x_f$ is a variable and the ideal 
    $\f{a} = \sum_{f \in F} K f(x_f) $.
    We want to quotient by this ideal,
    but we also want this to result in a field.
    Thus we try to find a maximal ideal containing it.
    By an application of Zorn, all proper ideals are 
    contained in a maximal ideal, 
    hence it suffices to show it is proper.
    Clearly it is non-trivial 
    (it contains $x_f - 1$ for $f = x - 1$).
    Assume for a contradiction that it is the whole ring,
    then there is a finite set $S$ such that
    \[1 = \sum_{f \in S} c_f f(x_f)\]
    where $S$ is finite and each $c_f \in K$.
    By the existance of splitting fields over base field $K$, 
    we take the splitting field $K'$ of $\prod_{f \in S} f$.
    Then from $K'[x_f]_{f \in S}$ 
    we can evaluate at the roots $r_f$ for each $f$:
    \[1 = \sum_{f \in S} c_f f(r_f) = \sum_{f \in S} 0 = 0\]
    a contradiction.

    Let $\f{m}$ be the maximal ideal containing $\f{a}$ and let 
    $K(1) = K[x_f]_{f \in F} / \f{m}$.
    There is an obvious ring morphism $K \to K(1)$. 
    If $f \in K[x]$ is irreducible then $f(x_f) \in \f{m}$ so
    $0 = \bar{f(x_f)} = f(\bar{x_f})$ in $K(1)$.
    Hence $f$ has a root $x_f$ in $K(1)$.

    By the above construction, 
    we have a chain $K : \N \to \Mod{\Si_\RNG}$ and so we can take 
    \linkto{direct_limit_of_chains}{the direct limit}; 
    call this $L$.
    We obtain for free that $L$ is a field and there is a ring morphism 
    $K \to L$ that commutes with all the morphisms in the chain.
    If $f \in L[x]$ then there exists a $\be \in \N$ and 
    $b \in K(\be)[x]$ such that $b \mapsto f$ 
    in the map $K(\be)[x] \to L[x]$.
    Then $b$ has finite degree $n$, 
    thus completely splits in $K(\be + n)$.
    Hence $b$ splits in $L$, 
    $f$ splits in $L$ so $L$ is algebraically closed.

    To show that $L$ is an algebraic extension let $a \in L$.
    Then there exists a $\be \in \N$ and $b \in K(\be)$ such that 
    $b \mapsto a$ in the map $K(\be) \to L$. 
    Since $b$ is algebraic over $K$ 
    (each extension is finite hence algebraic)
    $a$ is algebraic over $K$.
\end{proof}

\begin{prop}[Basic facts about algebraic closures]
    \link{alg_closures_props}
    Suppose $K$ has an algebraic closure $\Om$. 
    Then 
    \begin{enumerate}
        \item If $M$ is an algebraically closed field extending $K$ then 
        there exists a field extension $\Om \to M$ 
        such that the diagram commutes
        \begin{cd}
            K \ar[r] \ar[dr]&\Om \ar[d, dashed]\\
            &M
        \end{cd}
        \item If $K \to L$ is an algebraic field extension then 
        there exists a field extension $L \to \Om$ 
        such that the diagram commutes
        \begin{cd}
            K \ar[r] \ar[d]&\Om\\
            L \ar[ru, dashed]
        \end{cd}
    \end{enumerate}
\end{prop}
\begin{proof}~
    \begin{enumerate}
        \item The map $K \to \Om$ is algebraic and so we can consider $\Om$ as 
        $K(S)$ for some minimal generating set $S \subs \Om$.
        $M$ is algebraically closed thus it splits all minimal polynomials of 
        $a \in S$ over $K$, thus by the embedding theorem (Galois theory), %
        we have an embedding 
        $\Om \to M$ that commutes with the diagram.
        \item If $K \to L$ is algebraic then we can find a generating set 
        $S \subs L$ such that $K(S) = L$.
        $\Om$ is algebraically closed thus it splits all minimal polynomials of 
        $a \in S$ over $K$, thus by the embedding theorem (Galois theory), %
        we have an embedding 
        $L \to \Om$ that commutes with the diagram.
    \end{enumerate}
\end{proof}

\begin{prop}[Algebraic closures of isomorphic fields are isomorphic]
    \link{alg_closures_of_iso_are_iso}
    Suppose $K_0 \iso K_1$. 
    Then any algebraic closures of these fields are (non-canonically) isomorphic.
\end{prop}
\begin{proof}
    Call the algebraic closures $\Om_0$ and $\Om_1$:
    \begin{cd}
        K_0 \ar[r, "\io_0"] \ar[d, "\si", "\sim"{swap}]     & \Om_0\\
        K_1 \ar[r, "\io_1"]                                 & \Om_1
    \end{cd}
    By the \linkto{alg_closures_props}{properties of algebraic closures}
    there exist
    $f : \Om_0 \to \Om_1$ and $g : \Om_1 \to \Om_0$ 
    that individually commute with the above diagram.
    We claim that the composition $g \circ f$ is a surjection.

    The composition fixes $K_0$ because 
    $\io_0 = g \circ \io_1 \circ \si = g \circ f \circ \io_0$.
    Fix $a \in \Om_0$.
    Let $\La$ be the set of roots of the minimal polynomial $\min(a,K_0)$.
    $g \circ f$ is a bijection on $\La$: let $\la \in \La$, then
    $g \circ f (\min(a,K_0))(\la) = \min(a, K_0)(g \circ f(\la))$ because 
    $g \circ f$ fixes $K_0$.
    Thus $g \circ f (\La) \subs \La$. 
    It is a therefore a bijection on $\La$ as 
    $\La$ is finite and $g \circ f$ is injective (it is a field embedding).
    Thus there exists an element that maps to $a$ under the composition.
    Hence
    \[\Om_1 = g \circ f (\Om_1) \subs f(\Om_0)\]
    Thus $f$ is surjective and injective (as it is a field embedding).
    Hence $f$ is an isomorphism.
\end{proof}
%\section{Transcendence degree and characteristic
determine \texorpdfstring{$\ACF_p$}{TEXT} up to isomorphism}
%?Ugly appearance of section title in page header
This part of the appendix mainly follows Hungerford's book \cite{hungerford}.
\begin{dfn}
    Suppose $\io: K \to L$ is a field embedding
    and $S \subs L$.
    Then $S$ is algebraically independent over $K$
    when for any $n \in \N$, any $f \in K[x_1,\dots, x_n]$
    and any distinct elements $s_i \in S$, 
    \[f(s) = 0 \implies f = 0\]
\end{dfn}

\begin{rmk}
    \begin{enumerate}
        \item If $S$ is algebraically independent then any subset is 
        algebraically independent over $K$.
        \item In particular $\nothing$ is algebraically over $K$.
        \item For the embedding $K \to K$, any non-empty subset if 
            algebraically dependent.
        \item The concept of algbebraic independence extends that of 
            linear independence.
    \end{enumerate}
\end{rmk}

\begin{dfn}
    Suppose $K \to L$ is a field embedding.
    Then $B \subs L$ is a transcendence basis of the extension
    if $B$ is algebraically independent over $K$ and is maximal
    with respect to $\subs$.
\end{dfn}
\begin{rmk}
    This is the analogue of a vector space basis.
\end{rmk}

\begin{prop}[Existence of transcendence basis]
    \link{existence_of_trans_basis}
    Suppose $K \to L$ is a field embedding.
    There there exists a transcendence basis.
\end{prop}
\begin{proof}
    Application of Zorn's lemma on the set of algebraically independent 
    subsets of $L$ (non-empty due to $\nothing$ being algebraically independent)
    with respect to inclusion.
\end{proof}

\begin{prop}[Algebraic elements over $K(S)$]
    \link{algebraic_elements_over_KS}
    Let $K \to L$ be a field embedding and let $S \subs L$ be algebraically
    independent over $K$.
    Let $u \in L \setminus S$.
    $S \cup \set{u}$ is algebraically dependent if and only if $u$
    is algebraic over $K(S)$.
\end{prop}
\begin{proof}
    \begin{forward}
        Suppose $S \cup \set{u}$ is algebraically dependent.
        Then there exists non-zero $f \in K[x_0, \dots, x_n]$ 
        and distinct $s_i \in S$
        such that $f(u, s_1, \dots, s_n) = 0$.
        Write \[f = \sum_{i = 0}^m h_i (x_0)^i\]
        for $h_i \in K[x_1, \dots, x_n]$.
        Then let 
        \[g(x_n):= \sum_{i = 0}^m h_i(s_1, \dots, s_n) (x_0)^i \in K(S)[x_0]\]
        which has the root $u$.
        Assuming for a contradiction that $u$ is not algebraic over $K(S)$.
        Since $g(u) = 0$ we must have $g(x_0) = 0$ and so for each $i$,
        $h_i(s_1, \dots, s_n) = 0$.
        By algebraic independence of $S$ over $K$ we have that each $h_i = 0$.
        Thus $f = 0$, a contradiction.
    \end{forward}

    \begin{backward}
        Suppose $u$ is algebraic over $K(S)$. 
        Then there exists non-zero 
        $f(x) \in K(S)[x]$ such that $f(u) = 0$.
        We can write 
        $f(x) = \sum_{i = 1}^n \frac{h_i}{g_i}(s_1, \dots , s_m) x^i$
        for $s_i \in K(S)$.
        Then we can factor all the $g_i$ out, 
        leaving
        \[f(x) = \frac{1}{\prod_{j = 1}^n g_j}
        \sum_{i = 1}^n f_i(s_1, \dots , s_m)
        \prod_{j \ne i} g_j (s_1, \dots , s_m) x^i\]
        Let \[h = \sum_{i = 1}^n f_i(x_1, \dots , x_m)
        \prod_{j \ne i} g_j (x_1, \dots , x_m) {x_{m+1}}^i \in 
        K[x_1, \dots, x_{m+1}]\]
        We see that $h(s_1, \dots, s_m, u) = 0$ as $g(s_1, \dots, s_m) \ne 0$
        ($S$ is algebraically independent).
        Suppose for a contradiction that $S \cup \set{u}$
        is algebraically independent, then $h = 0$ and so 
        for each $i$, 
        \[f_i(x_1, \dots , x_m) \prod_{j \ne i} g_j (x_1, \dots , x_m) = 0\]
        but $g_i$ were on the bottom of the fractions so
        they are non-zero,
        thus $f_i(x_1, \dots , x_m)=0$ as the polynomial ring is an 
        integral domain.
        This implies that $f = 0$, a contradiction.
    \end{backward}
\end{proof}

\begin{lem}[Composition of algebraic extensions]
    We use extensively the fact that the composition of 
    algebraic extensions is algebraic.
\end{lem}

\begin{prop}[(Key) Identifying transcendence bases]
    \link{transcendence_bases_algebraic_extensions}
    Let $K \to L$ be a field embedding.
    Suppose $S \subs L$ is algebraically independent over $K$.
    Then $L$ is algebraic over $K(S)$ if and only if 
    $S$ is a transcendence basis.
\end{prop}
\begin{proof}
    $\forall a \in L,$ $a$ is algebraic over $K(S)$
    \linkto{algebraic_elements_over_KS}{if and only if }
    $\forall a \in L,$ $S \cup \set{a}$ is algebraically independent
    if and only if $S$ is maximally algebraically independent
    if and only if $S$ is a transcendence basis.
\end{proof}

\begin{cor}[Subsets containing transcendence bases]
    Let $K \to L$ be a field embedding and let $X \subs L$.
    If $K(X) \to L$ is algebraic then $X$ contains a transcendence basis
    of the extension $K \to L$.
\end{cor}
\begin{proof}
    Using Zorn we have a maximally algebraically independent subset of $X$;
    call this $S$. 
    This is a transcendence basis of the extension $K \to K(X)$.
    Thus by \linkto{transcendence_bases_algebraic_extensions}{
        the previous proposition} 
    $K(S) \to K(X)$ is algebraic.
    The composition of algebraic extensions is algebraic,
    thus $K(S) \to L$ is algebraic.
    \linkto{transcendence_bases_algebraic_extensions}{Hence} 
    $S$ is a transcendence basis.
\end{proof}

\begin{prop}[Uniqueness of finite transcendence degree]
    \link{fin_trans_deg}
    Let $K \to L$ be a field embedding and let $S$ be a 
    finite transcendence basis of the extension.
    Then any other transcendence basis has the same cardinality as $S$.
\end{prop}
\begin{proof}
    If $S$ is empty then it is the unique transcendence basis.

    Otherwise, 
    let $S = \set{s_1 , \dots, s_n}$ and $T$ be another transcendence basis.
    We show that there exists a $t \in T$ such that 
    $\set{t, s_2, \dots, s_n}$ is a transcendence basis.

    To find such a $t$, 
    assume for a contradiction that for any $t \in T$,
    $\set{t, s_2, \dots, s_n}$ is algebraically dependent.
    \linkto{algebraic_elements_over_KS}{Then} each
    $t \in T$ is algebraic over $\set{s_2, \dots, s_n}$ and
    so $K(\set{s_2, \dots, s_n})(T)$ is algebraic over 
    $K(\set{s_2, \dots, s_n})$.
    Furthermore $L$ is algebraic over $K(\set{s_2, \dots, s_n})(T)$
    hence $L$ is algebraic over $K({s_2, \dots, s_n})$.
    Hence $s_1$ is algebraic over $K(s_2, \dots, s_n)$,
    a contradiction.

    Thus there exists such a $t \in T$.
    It suffices to show that $L$ is algebraic over 
    $K\brkt{t, s_2, \dots, s_n}$. 
    This is true if and only if $K(S)$ is algebraic over 
    $K\brkt{t, s_2, \dots, s_n}$, 
    if and only if $s_1$ is algebraic over
    $K\brkt{t, s_2, \dots, s_n}$,
    if and only if $\set{t, s_1, s_2, \dots, s_n}$
    is algebraically dependent over $K$,
    which it is since it is contains $S$ as a proper subset.
    Thus we have that $\set{t, s_1, s_2, \dots, s_n}$ is a transcendence basis
    of the same cardinality as $S$.

    By induction, replacing $s_i$ at each step, 
    we obtain a subset of $T$ that is a transcendence basis of the same
    cardinality as $S$.
    By maximality of transcendence bases this subset is $T$,
    thus $\abs{S} = \abs{T}$.
\end{proof}

\begin{nttn}[Minimal polynomial]
    \link{min_poly}
    If $K \to L$ is a field extension, 
    let the minimal polynomial of $a \in L$ over $K$ be denoted as
    $\min(a,K)$
\end{nttn}

\begin{lem}[Infinite transcendence bases inject into each other]
    \link{inf_trans_deg_inj}
    Let field embedding $K \to L$ have infinite transcendence bases $S$
    and $T$.
    Then $\abs{T} \leq \abs{S}$.
\end{lem}
\begin{proof}
    $S$ is infinite and hence non-empty;
    let $s \in S$ and consider $\min(s, K(T)) \in K(T)[x]$.
    Because polynomials are finite, 
    there exists $T_s$, a finite subset of $T$ such that 
    $\min(s, K(T)) \in K(T_s)[x]$.
    Hence $s$ is algebraic over $K(T_s)$.

    We claim that $\bigcup_{s \in S} T_s$ is a transcendence basis of $K \to L$.
    Indeed it is a subset of $T$ hence it is algebraically independent;
    by construction $K(S)$ is algebraic over 
    $K(\bigcup_{s \in S} T_s)$ and $L$ is algebraic over $K$ and so 
    $L$ is algebraic over $K(\bigcup_{s \in S} T_s)$.
    \linkto{transcendence_bases_algebraic_extensions}{Thus} 
    $\bigcup_{s \in S} T_s$ is a transcendence basis of $K \to L$.
    By maximality of transcendence bases $T = \bigcup_{s \in S} T_s$.

    We inject $T$ into $S$ 
    by writing it as a disjoint union of subsets of $T_s$.
    By the well-ordering principle (choice) we well-order $S$
    and define
    \[X_s := T_s \setminus \bigcup_{i < s} T_i\]
    Since $X_s \subs T_s$ we have 
    $\bigcup_{s \in S} X_s \subs \bigcup_{s \in S} T_s$.
    Conversely, if there exist $s \in S$ and $x \in T_s$ then
    the set of $i \in S$ such that $x \in T_i$ is non-empty and thus has a 
    minimum element $m$ (by well-ordering).
    Thus $x \in X_m$ and so 
    $\bigcup_{s \in S} X_s = \bigcup_{s \in S} T_s = T$.

    Define a map from $\bigcup_{s \in S} X_s \to S \times \N$ by the following:
    let $s \in S$. 
    Then since $X_s \subs T_s$ and $T_s$ is finite
    we can write $X_s = \set{x_0, \dots, x_{n_s}}$.
    we send $x_i$ to the element $(s,i) \in S \times \N$.
    This is well-defined because the $X_i$ are disjoint
    and is clearly injective.
    Since $S$ is infinite,
    \[\abs{T} = \abs{\bigcup_{s \in S} X_s} \leq \abs{S \times \N} = \abs{S}\]
\end{proof}

\begin{prop}[Uniqueness of infinite transcendence degree]
    \link{inf_trans_deg}
    Let field embedding $K \to L$ have a transcendence basis.
    Then any other transcendence basis has the same cardinality.
\end{prop}
\begin{proof}
    Let $S$ and $T$ be two transcendence bases.
    If one of them were finite then by 
    \linkto{fin_trans_deg}{uniqueness of finite transcendence degree}
    $S$ and $T$ have the same cardinality.
    Otherwise both are infinite and
    by the \linkto{inf_trans_deg_inj}{previous lemma} 
    $\abs{T} \leq \abs{S}$
    and $\abs{S} \leq \abs{T}$.
    By Schröder–Bernstein $\abs{S} = \abs{T}$.
\end{proof}

\begin{dfn}[Transcendence degree]
    \link{transcendence_degree_dfn}
    If $\io : K \to L$ is a field embedding then the transcendence degree
    is defined as the cardinality of a transcendence basis.
    It is well-defined as we showed that 
    \linkto{existence_of_trans_basis}{a basis exists}
    and any two bases have the 
    \linkto{inf_trans_deg}{same cardinality}.
    We use $\tdeg(\io)$ to denote the degree.
\end{dfn}

\begin{nttn}
    $K[x_1,\dots,x_n]$ is the polynomial ring. 
    $K(x_1,\dots,x_n)$ is the field of fractions of the polynomial ring.
\end{nttn}

\begin{lem}[Isomorphism with field of polynomial fractions]
    \link{iso_with_field_of_poly_frac}
    Suppose $\io: K \to L$ is a field embedding and $S \subs L$ is a finite
    set algebraically independent over $K$.
    Then there exists a (non-canonical) field isomorphism 
    \[K(S) \iso K(x_s)_{s \in S}\]
\end{lem}
\begin{proof}
    From Galois theory we have a (non-canonical) surjective ring morphism 
    $K[x_s]_{s\in S} \to K[S]$ given by $x_s \mapsto s$.
    It is injective due to $S$ being algebraically independence.
    By the \linkto{uni_prop_field_of_fractions}{
        universal property of field of fractions}
    there is a unique isomorphism $K(S) \iso K(x_s)_{s \in S}$
    that commutes with the other isomorphism.
    \begin{cd}
        K[S] \ar[r, "\subs"] \ar[d, "\iso"] & K(S) \ar[d, dashed]\\
        K[x_s]_{s \in S} \ar[r, "\subs"] &K(x_s)_{s \in S}
    \end{cd}
\end{proof}

\begin{prop}[Embedding algebraically independent sets]
    \link{embed_alg_ind_sets}
    Suppose we have the field embeddings
    \begin{cd}
        K_0 \ar[r] \ar[d, "\si"] & F_0\\
        K_1 \ar[r]  & F_1
    \end{cd} 
    and let $S \subs F_0$ be an algebraically independent over $K_0$.
    Suppose we have an injection $\phi: S \to F_1$ such that the image
    is algebraically independent over $K_1$.
    Then there exists a unique field embedding 
    $\bar{\si} : K_0(S) \to F_1$ such that $\res{\bar{\si}}{S} = \phi$
    and the following commutes  
    \begin{cd}
        K_0 \ar[r] \ar[d, "\si"] & K_0(S) \ar[d, "\bar{\si}"]\\
        K_1 \ar[r]  & F_1
    \end{cd} 
    Furthermore, if $\si$ is an isomorphism then $\bar{\si}$ is an isomorphism
    $K_0(S) \to K_1(\phi(S))$.
\end{prop}
\begin{proof}
    We define $\bar{\si}: K_0(S) \to F_1$ by 
    \[
        \frac{f(s_1, \dots, s_n)}{g(s_1, \dots, s_n)} \mapsto 
        \frac{\si(f) (\phi s_1, \dots, \phi s_n)}{
            \si(g) (\phi s_1, \dots, \phi s_n)}
    \]
    where $\si$ takes a polynomial over $K_0$ 
    as an argument (the induced map on the polynomial rings).
    To check that $\bar{\si}$ is well-defined we just need to check 
    uniqueness of the image.
    Suppose $\frac{f}{g}(s_1, \dots, s_n) \in K_0(S)$.
    Due to the 
    \linkto{uni_prop_field_of_fractions}{
        universal property of field of fractions},
    there is a unique field embedding 
    $\hat{\si} : K_0(x_i)_{i \leq n} \to K_1(x_i)_{i \leq n}$
    that commutes with the injective polynomial ring morphism
    $\si : K_0[x_i]_{i \leq n} \to K_1[x_i]_{i \leq n}$ 
    (which was induced by $\si$).
    By \linkto{iso_with_field_of_poly_frac}{
        the previous lemma}
    we have an isomorphisms $K_0(s_i)_{i \leq n} \iso K_0(x_i)_{i \leq n}$
    and $K_1(\phi (s_i))_{i \leq n} \iso K_0(x_i)_{i \leq n}$
    induced by the (not unique but suitably chosen) isomorphisms 
    $K_0[s_i]_{i \leq n} \iso K_0[x_i]_{i \leq n}$ and
    $K_1[\phi(s_i)]_{i \leq n} \iso K_1[x_i]_{i \leq n}$.
    Hence we have the diagram:
    \begin{cd}
        &K_0[s_i]_i \ar[r] \ar[d,"\iso"]       &K_0(s_i)_i \ar[d, "\iso"]  \\
    K_0 \ar[ru] \ar[r] \ar[d, "\si"]   
        &K_0[x_i]_i \ar[r] \ar[d, "\si"] &K_0(x_i)_i \ar[d, dashed, "\hat{\si}"] \\
    K_1 \ar[r] \ar[rd]    
        &K_1[x_i]_i  \ar[d,"\iso"] \ar[r]        &K_1(x_i)_i \ar[d,"\iso"]  \\
            &K_1[\phi(s_i)]_i \ar[r]     &K_1(\phi(s_i))_i   
    \end{cd}
    The composition of the three maps on the right hand side is $\bar{\si}$
    restricted to $K_0(s_i)_{i \leq n}$.
    Note that the composition is a well-defined injective ring morphism 
    that commutes with everything else. 
    Thus $\frac{f}{g}(s)$ is sent to a unique element of $F_1$.
    If $q \in K_0(S)$ maps to the same image under $\bar{\si}$
    then it lies in the image of the composition so it is $\frac{f}{g}(s)$.
    The composition commutes with everything and so for anything from $K_0$,
    going to $F_1$ via $\si$ is the same as going via $\hat{\si}$.
    
    Thus it is well-defined, injective and commutes.
    It is clearly a field embedding.
    It is unique because the map from $K_0[s_i] \to K_1[\phi(s_i)]$ was unique
    (though the intermediate isomorphisms were not unique).
    By definition, $\res{\bar{\si}}{S} = \phi$.

    The above construction shows that if $\si$ is an isomorphism
    then $\hat{\si}$ is an isomorphism.
    Hence $\bar{\si}$ restricted to the finite subset is an isomorphism.
    Since this is for any subset, $\bar{\si}$ is an isomorphism
    $K_0(S) \to K_1(\phi(S))$.
\end{proof}

\begin{prop}[Algebraically closed extensions of same transcendence degree
    are isomorphic]
    Suppose we have fields $K_0 \iso K_1$ and field extensions $K_0 \to L_0$
    and $K_1 \to L_1$ of equal transcendence degree such that 
    $L_0, L_1$ are algebraically closed,
    then $L_0$ and $L_1$ are (non-canonically) isomorphic.
\end{prop}
\begin{proof}
    Let $\si$ be the isomorphism $K_0 \to K_1$
    Let $S_0, S_1$ be transcendence bases of $K_0 \to L_0$ and $K_1 \to L_1$.
    They have the same cardinality thus we can biject $S_0, S_1$
    and \linkto{embed_alg_ind_sets}{
        produce an isomorphism $\bar{\si} : K_0(S_0) \to K_1(S_1)$.}
    The extensions $K_0(S_0) \to L_0$ and $K_1(S_1) \to L_1$ are 
    \linkto{transcendence_bases_algebraic_extensions}{algebraic}
    and $L_0,L_1$ are algebraically closed.
    Hence they are algebraic closures of isomorphic fields,
    \linkto{alg_closures_of_iso_are_iso}{
        which implies they they are (non-canonically) isomorphic}.
        \begin{cd}
        K_0 \ar[r] \ar[d, "\si"{swap}, "\sim"]  
        & K_0(S_0) \ar[r] \ar[d, "\bar{\si}"{swap}, "\sim"]
        & L_0 \ar[d, dashed]\\
        K_1 \ar[r] \ar[r]  
        & K_1(S_1) \ar[r]                    
        & L_1
    \end{cd}
\end{proof}

\begin{cor}[Transcendence degree and characteristic
    determine algebraically closed fields of characteristic $p$ 
    up to isomorphism]
    \link{appendix_trans_deg_and_char_determine_ACF_p}
    If $K_0,K_1$ are fields of the same characteristic and have the same
    transcendence degree over their minimal subfield ($\zmo{}$ or $\Q$).
    Then they are (non-canonically) isomorphic.
\end{cor}
\begin{proof}
    $K_0$ and $K_1$ have the same characteristic $p$ so they are
    extensions of isomorphic subfields (their minimal subfields).
    Thye are algebraically closed.
    They have the same transcendence degree thus by the previous proposition
    they are (non-canonically) isomorphic.
\end{proof}

%?Things that are needed for dimension:

\begin{prop}[Tower law of transcendence degree]
    Suppose $K \map{\io_L}{} L \map{\io_M}{} M$ are field embeddings. 
    Then 
    \[\tdeg(\io_L) + \tdeg(\io_M) = \tdeg(\io_M \circ \io_L)\]
\end{prop}
\begin{proof}
    Let $B_L$ and $B_M$ be transcendence bases for the extensions $\io_L$,
    $\io_M$ respectively.
    We show that $B_L \cup B_M$ is a transcendence basis for the composition.

    Since $B_L$ is a basis, 
    $K(B_L) \to L$ is algebraic and hence we can show that 
    $K(B_L \cup B_M) \to L(B_M)$ is algebraic.
    $L(B_M) \to M$ is algebraic as $B_M$ is a basis,
    thus the composition $K(B_L \cup B_M) \to M$ is algebraic.

    To show that it is algebraically independent,
    we first note that $B_L,B_M$ are disjoint,
    otherwise there exists $b$ in the intersection, 
    which is both in $B_M$ and in $L$ causing
    $B_M$ to be algebraically dependent over $L$.
    Let $f \in K[x_1, \dots, x_n]$ 
    and let $l_1,\dots,l_r,m_{r+1} \in B_L$
    and $m_{r+1}, \dots, m_n \in B_M$ be distinct elements such that
    \[f(l_1,\dots,l_r,m_{r+1}, \dots, m_n) = 0\]
    We can find some finite set $I$, $h_i \in K[x_1, \dots, x_r]$ and
    $k_i \in K[x_{r+1}, \dots, x_n]$ such that 
    \[f(x_1,\dots,x_n) = 
    \sum_{i \in I} h_i(x_1, \dots, x_r) k_i(x_{r+1}, \dots, x_{n})\]
    and each $k_i$ are linearly independent.
    \[g := \sum_{i \in I} h_i(l_1, \dots, l_r) k_i(x_{r+1}, \dots, x_{n}) \in 
    L[x_{r+1}, \dots, x_n] \quad \AND \quad g(m_{r+1}, \dots, m_n) = 0\]
    Since $B_M$ is algebraically independent $g = 0$.
    Thus (by linear independence) of $k_i$ each 
    $h_i(l_1, \dots, l_r) = 0$ and hence each 
    $h_i(x_1, \dots, x_r) = 0$ as $B_L$ is algebraically independent.
    Thus $f = 0$ and the union forms a transcendental basis and
    \[\tdeg(\io_M\circ\io_L) = \abs{B_L \cup B_M} 
    = \abs{B_L} + \abs{B_M} - \abs{B_L \cap B_M} = \tdeg(B_L) + \tdeg(B_M)\]
\end{proof} 

\begin{lem}[Isomorphic extensions have same transcendence degree]
    \link{iso_field_ext_same_trans_deg}
    Suppose $K \to L, K \to M$ are field extensions and 
    $L \to M$ is a an isomorphism that preserves $K$,
    then 
    \[\tdeg(K \to L) = \tdeg(K \to M)\]
\end{lem}
\begin{proof}
    Let $S$ be a transcendence basis for $K \to L$.
    We claim the image of $S$ under the isomorphism $\si: L \to M$ is 
    a transcendence basis for $K \to M$.

    Algebraic independence: Let $n \in \N$ and 
    let $p \in K[x_1,\dots,x_n]$. 
    Let $a \in \si(S)^n$ and suppose $p(a) = 0$.
    Then we apply $\si^{-1}$ to both sides, 
    noting that $p$ has coefficients 
    from $K$ and therefore commutes with $\si^{-1}$:
    \[0 = \si^{-1}(p(a)) = p(\si^{-1}(a))\]
    But $\si^{-1}(a)$ is an element of $S^n$
    and by algebraic independence of $S$ we have $p = 0$.

    Maximality: let $a \in M \setminus \si(S)$. 
    We show that $a \cup \si(S)$ is algebraically dependent.
    Consider $\si^-1(a) \in L$.
    Since $S$ is a basis 
    \linkto{transcendence_bases_algebraic_extensions}{$\si^{-1}(a)$
        is algebraic over $K(S)$}.
    This we have $p \in K(S)[x_0]$ such that 
    $p(\si^{-1}(a)) = 0$.
    We can identify $p$ with a polynomial $q$ in $K[x_0,\dots,x_n]$ 
    with the remaining $n$ coefficients from $S$,
    such that 
    \[p(x_0) = q(x_0,s_1,\dots,s_n)\]
    Since $\si$ preserves $K$ we have that 
    \[0 = \si(q(\si^{-1}(a),s_1,\dots,s_n)) = q(a, \si(s_1),\dots,\si(s_n))\]
    so $a \cup \si(S)$ is algebraically dependent.

    Since $\si$ is a field embedding it is injective and so 
    $S$ and $\si(S)$ have the same cardinality.
\end{proof}
%\section{Locally finite fields and polynomial maps}

\begin{dfn}[Locally finite]
    We say that a field is locally finite if for any finite subset,
    the minimal subfield `$K(S)$' containing the subset is finite.
\end{dfn}

\begin{dfn}[Polynomial map]
    Let $K$ be a field and $n$ a natural.
    Let $f : K^n \to K^n$ such that for each $a \in K^n$,
    \[f(a) = (f_1(a), \dots, f_n(a))\] for 
    $f_1, \dots, f_n \in K[x_1, \dots, x_n]$.
    Then we call $f$ a polynomial map over $K$.
\end{dfn}

\begin{lem}[Equivalences for locally finite over prime characteristic 
        \cite{stack0}]
    \link{locally_finite_over_prime}
    Let $K$ be a field of characteristic $p$ a prime.
    Then the following are equivalent:
    \begin{enumerate}
        \item $K$ is locally finite.
        \item $\F_p \to K$ is algebraic.
        \item $K$ embeds into an algebraic closure of $\F_p$.
    \end{enumerate}
    In particular, the algebraic closure of a finite field is locally finite.
\end{lem}
\begin{proof}~
    \begin{enumerate}
        \item[{}]$1.\implies 2.$ We show the contrapositive.
            Suppose there exists $a \in K$ such that $a$
            is not algebraic over $\F_p$. 
            Then $\F_p(a)$ is a isomorphic to the field of fractions
            of a polynomial ring over $K$ and so is infinite.
            Hence $K$ is not locally finite.
        \item[{}]$2. \implies 1.$ We show by induction that $K$ is locally 
            finite.  
            Let $S$ be a finite subset of $K$.
            If $S$ is empty then $\F_p(S) = \F_p$ and so it is finite.
            If $S = S' \cup {s}$ and $\F_p(S')$ is finite,
            then $s \in K$ is algebraic so by some Galois theory we can write%
            \[\F_p(S')[x] / (\min(s, \F_p(S'))) \iso \F_p(S)\]
            Where the left hand side is finite because it is a finite 
            dimensional vector space over a finite field.
            Hence $K$ is locally finite.
        \item[{}]$2. \implies 3.$ If $\F_p \to K$ is algebraic then 
            it can be written as $\F_p(S)$ 
            for some set $S$ algebraic over $\F_p$. 
            Any algebraic closure of $\F_p$ splits each $a \in S$ hence by the 
            embedding theorem (Galois Theory)%
            $K$ embeds into the algebraic closure.
        \item[{}]$3. \implies 2.$ If $a \in K$ then $a$ can be embedded into 
            the algebraic closure of $\F_p$ by assumption.
            Then it is algebraic over $\F_p$ thus $a$ is algebraic over $\F_p$.
    \end{enumerate}
    Any finite field is an algebraic extension over $\F_p$ where $p$ is its
    prime characteristic.
    Hence its algebraic closure is an algebraic extension over $\F_p$ and so 
    it is locally finite.
\end{proof}

\begin{cor}[Ax-Grothendieck for algebraic closure of finite fields]
    \link{appendix_algebraic_closure_ax_groth}
    If $\Om$ is an algebraic closure of a finite field
    then any injective polynomial map over $\Om$ is surjective.
\end{cor}
\begin{proof}
    Suppose for a contradiction 
    $f : \Om^n \to \Om^n$ is an injective polynomial map that is not surjective.
    Then there exists $b \in \Om^n$ such that $b \notin f(\Om^n)$.
    Writing $f = (f_1, \dots, f_n)$ for $f_i \in \Om[x_1, \dots, x_n]$
    we can find $A \subs \Om^n$, 
    the set of all the coefficients of all of the $f_i$.
    We want to find a contradiction by showing that $f$ is surjective on
    the subfield $K$ generated by $A \cup \set{b}$, 
    which contains $b$.
    $A \cup \set{b}$ is finite and \linkto{locally_finite_over_prime}{
        $\Om$ is locally finite},
    thus $K$ is finite.
    The restriction $\res{f}{K^n}$ is injective and has image inside $K^n$
    since each polynomial has coefficients in $K$ and is evaluated at 
    an element of $K^n$.
    Hence $\res{f}{K^n}$ is an injective endomorphism of a finite set thus 
    is surjective,
    a contradiction.
\end{proof}
%\section{Noetherian Modules}
All of the following material is from Atiyah and McDonald's book \cite{atiyah}.
\begin{dfn}[Ascending chain]
    If $(\Si,\leq)$ is a partially ordered set then an ascending chain 
    is a functor $F : (\N,\leq) \to (\Si,\leq)$.

    A chain is stationary if there exists a $m \in \N$ such that
    for any $n \in \N_{\ge m}$, $F(m) = F(n)$.
\end{dfn}

\begin{prop}[Noetherian chains]
    \link{noeth_chain}
    Let $\Si$ be a partially ordered set. 
    Then every non-empty subset has a maximal element
    if and only if every ascending chain is stationary.
\end{prop}
\begin{proof}
    \begin{forward}
        If $F$ is a chain then $I = \set{F(i) \st i \in \N}$
        has a maximal element $F(m)$. 
        Hence for any $n$ greater than or equal to $m$, 
        $F(m) = F(n)$.
    \end{forward}
    \begin{backward}
        Let $\nothing \ne I \subs \Si$ and
        suppose for a contradiction that for each $m \in I$
        there exists $a \in I$ such that $m < a$.
        Then create a chain by recursively taking such an $a$,
        starting with the non-empty element:
        $a_0 < a_1 < \dots$.
        This chain is not stationary.
    \end{backward}
\end{proof}

\begin{dfn}[Noetherian Module]
    \link{noetherian_definition}
    Order the submodules of an $A$-module $M$ by $\subs$.
    $M$ is Noetherian if it satisfies on one of the following
    equivalent definitions:
    \begin{itemize}
        \item Every non-empty set of submodules has a maximal element.
        \item Every ascending chain of submodules is stationary.
        \item Any submodule of $M$ is finitely generated.
    \end{itemize}
\end{dfn}
\begin{proof}
    $1. \iff 2.$ follows from \linkto{noeth_chain}{Noetherian chains}.
    $1. \implies 3.$: Let $N \leq M$ and take the set 
    \[I = \set{S \leq N \st S \text{ finitely generated}}\]
    Then $0 \in I$ hence it is non-empty hence has a maximal element,
    which we call $N_0$.
    Suppose for a contradiction that $N_0 < N$. 
    Then there exists $x \in N \setminus N_0$ such that $N_0 < N_0 + Ax \leq N$.
    This is finitely generated hence belongs to $I$, 
    contradicting with the maximality of $N_0$.
    $3. \implies 2.$: Let $N_0 \leq N_1 \leq \dots$ 
    be a chain of submodules.
    Their union $N$ is a submodule of $M$ hence is finitely generated.
    There exists $N_m$ such that all the generators are in $N_m$.
    At $m$ the chain becomes stationary.
\end{proof}

\begin{prop}[Exactness and Noetherian modules]
    \link{exact_noeth}
    If 
    \[0 \to N \to M \to P \to 0\]
    is exact then 
    $M$ is Noetherian if and only if both $N$ and $P$ are Noetherian.
\end{prop}
\begin{proof}
    Without loss of generality $N \subs M$ and $P = M / N$.
    \begin{forward}
        Any ascending chain of ideals in $N$ 
        is a chain in $M$ hence is stationary.
        By the third isomorphism theorem, ideals in $P$ 
        biject with ideals in $M$ containing $N$.
        Hence any ascending chain of ideals in $P$ is 
        stationary when sent back to $M$,
        thus is stationary.
    \end{forward}
    
    \begin{backward}
        Let $(M_i)_{i \in \N}$ be an ascending chain of submodules in $M$.
        Their preimages $(N_i)$ in $N$ - i.e. $(M_i \cap N)$ - and images
        $(P_i)$ in $P$ become stationary at $m_n, m_p$.
        We claim that $(M_i)$ becomes stationary at the maximum of the two.
        We first show that 
        \[0 \to N_i \to M_i \to P_i \to 0\]
        is exact.
        The map $M_i \to P_i$ is the restriction of the map $M \to P$ 
        on $M_i$, hence 
        \begin{align*}
            &\ker (M_i \to P_i) \\
            &= M_i \cap \ker (M \to P)\\
            &= M_i \cap N &\text{ by exactness}\\
            &= N_i
        \end{align*}
        Suppose $\max(m_n,m_p) \leq i$, then
        We apply the snake lemma to
        \begin{cd}
            N_i \ar[r]\ar[d] &M_i \ar[r] \ar[d]&P_i\ar[d]\\
            N_{i+1} \ar[r]  &M_{i+1} \ar[r]  &P_{i+1}
        \end{cd}
        Obtaining the exact sequences
        \[0 = \coker (N_i \to N_{i+1}) \to \coker (M_i \to M_{i+1}) 
        \to \coker (P_i \to P_{i+1}) = 0\]
        Noting that by stability the end points are $0$.
        This says that $\coker (M_i \to M_{i+1}) = 0$ and so 
        $M_i \to M_{i+1}$ is an isomorphism 
        (via $\subs$ which is already injective)
        so $M_i = M_{i+1}$.
    \end{backward}
\end{proof}

\begin{prop}[Direct sum of Noetherian is Noetherian]
    \link{direct_sum_of_noeth}
    If $M_i$ are Noetherian $A$-modules then the direct sum 
    $\bigoplus_{i = 1}^n M_i$ is Noetherian.
\end{prop}
\begin{proof}
    Induction. Consider the exact sequence
    \[0 \to M_{n+1} \to \bigoplus_{i = 1}^{n+1} M_i \to 
    \bigoplus_{i = 1}^{n} M_i \to 0\]
    Apply \linkto{exact_noeth}{the proposition on exact sequences}.
\end{proof}

\begin{dfn}[Noetherian ring]
    A ring is Noetherian if it is a Noetherian module over itself.
\end{dfn}

\begin{prop}[Finitely generated modules over Noetherian rings are Noetherian]
    \link{fin_gen_mod_over_noeth}
    If $A$ is a Noetherian ring then any finitely generated $A$-module is 
    Noetherian.
\end{prop}
\begin{proof}
    $M$ is the image of a projection of $A^n$ for some $n$.
    \linkto{exact_noeth}{It suffices that $A^n$ is Noetherian} by considering
    the exact sequence
    \[0 \to \ker \to A^n \to M \to 0\]
    \linkto{direct_sum_of_noeth}{$A^n$ is a direct sum of Noetherian
        modules hence is Noetherian.}
\end{proof}

\begin{prop}[Hilbert basis theorem]
    \link{hilbert_basis}
    If $A$ is Noetherian then the polynomial ring $A[x_1, \dots, x_n]$ 
    is Noetherian.
    In particular for any field $K$, $K[x_1, \dots, x_n]$ 
    is Noetherian.
\end{prop}
\begin{proof}
    By induction it suffices to show that $A[x]$ is Noetherian.
    Let $\f{a}$ be an ideal of $A[x]$;
    we show that it is finitely generated.
    The ideal $I$ of $A$ 
    formed by the leading coefficients of polynomials in $\f{a}$
    is finitely generated as $A$ is Noetherian.
    Hence there exists a finite subset of polynomials $S \subs \f{a}$
    such that the leading coefficients $a_f$ generate $I$.
    We consider the ideal $\sum_{f \in S}A[x]f$.
    Let $d$ denote $\max_{f \in S} \deg f$.

    Consider the $A$-module $M$ finitely generated by $\set{x_i}_{i = 0}^d$
    (\linkto{fin_gen_mod_over_noeth}{hence Noetherian}).
    Then $M \cap \f{a}$ (viewing $\f{a}$ as an $A$-module) is finitely generated
    as an $A$-module.
    We show that $\f{a} = \sum_{f \in S} A[x]f + M \cap \f{a}$.
    Hence any element of $\f{a}$ can be written as a sum 
    \[\sum_{f \in S} p_f f + \sum_{i = 0}^{d} a_i x^i\] 
    where $p_f$ are in $A[x]$ and $a_i$ are in $A \subs A[x]$.
    Hence $\f{a}$ is finitely generated.

    Indeed let $p \in \f{a}$.
    If $\deg p \leq d$ then $p$ is in $M$ hence in $M \cap \f{a}$.
    Otherwise $d < \deg p$ 
    and we write $a$ the leading coefficient of $p$ as a sum 
    $\sum_{f \in S} \la_f a_f$ (for $\la_f \in A$).
    So 
    \[
        q := \sum_{f \in S} \la_f f x^{\deg p - \deg f}
    \]
    is a polynomial in the finitely generated ideal 
    with the same degree and first coefficient as $p$
    such that $p - q \in \f{a}$ has degree strictly less than before.
    By induction on the degree of $p$ (base case covered by $\deg p \leq d$) 
    we have that $p$ is also in the finitely generated ideal,
    so we are done.

    For the `in particular' we note that fields only have two ideals,
    hence trivially satisfy the ascending chain condition.
\end{proof}
%%%Notes on Zariski Topology stuff
\section{Strong Nullstellensatz and Prime Spectra}
We use the `Rabinowitsch trick' to prove strong Nullstellensatz
from the weak version. 
This is also from Atiyah and McDonald's book \cite{atiyah}.
\begin{prop}[Strong Nullstellensatz]
    \link{strong_nullstellensatz_appendix}
    Let $K$ be an algebraically closed field and suppose 
    $\f{a}$ is an ideal of $K[x_1,\dots,x_n]$.
    Then $r(\f{a}) = I(\V(\f{a}))$.
\end{prop}
\begin{proof}
    \begin{forward}
        We show 
        $r(\f{a}) \subs I(\V(\f{a}))$.
        Clearly 
        $I(\V(\f{a})) \subs r(I(\V(\f{a})))$
        and if $f$ is in the radical then 
        $f^n (a) = 0$ and by induction 
        (using that $K[x_1,\dots,x_n]$ is an integral domain) $f(a) = 0$.
        Hence $I(\V(\f{a})) = r(I(\V(\f{a})))$.
        By opening up the definition of 
        $I(\V(\f{a}))$ we can show that 
        $\f{a} \subs I(\V(\f{a}))$.
        Thus \[r(\f{a}) \subs r(I(\V(\f{a}))) = I(\V(\f{a}))\]
    \end{forward}
    \begin{backward}
        Let $g \in I(\V(\f{a}))$.
        Consider the injective ring morphism 
        $\io : K[x_1,\dots,x_n] \to K[x_1,\dots,x_n][y]$ 
        and the polynomial $1 - y \io(g)$ of the codomain.
        Evaluation of this polynomial (in $y$) at any element of 
        $\V_K(\io(\f{a}))$
        gives us $1$. 
        If $\io(\f{a}) + \<1 - y \io(g)\>$ 
        were a proper ideal it would be contained in a maximal ideal 
        \cite{atiyah}
        which is prime, but that maximal ideal would have an empty vanishing
        as everything evaluates to $1$.
        By the weak Nullstellensatz this is a contradiction.
        Hence $\io(\f{a}) + \<1 - y \io(g)\> = \<1\>$
        and there exists a finite sum resulting in $1$:
        \[1 = \sum_{f \in S} \io(f)h_f(y) + (1 - y \io(g))h(y)\]
        If $\io(g) = 0$ then since $\io$ is injective $g = 0$ and 
        so is in $r(\f{a})$.
        Otherwise, we can make $A[\frac{1}{\io(g)}]$
        and evaluate the polynomial at $\frac{1}{\io(g)}$:
        \[1 = \sum_{f \in S} \io(f)h_f(\frac{1}{\io(g)})\]
        Hence there exist $H_f \in \io(\f{a})$ and $m \in \N$ such that
        \[1 = \sum_{f \in S} \io(f)\frac{H_f}{(\io(g))^m}\]
        Hence $\io(g) \in \io(\f{a})$ and so $g \in \f{a}$ 
        by injectivity of $\io$.
    \end{backward}
\end{proof}

\begin{cor}[Galois correspondence between ideals and vanishings]
    \link{galois_correspondence_ideals_vanishings}
    Let $K$ be an algebraically closed field.
    If $X \subs K^n$ is Zariski closed then $\V(I(X)) = X$. 
    Thus we have 
    \linkto{taking_ideals_order_reversing}{order reversing} bijections:
    \begin{cd}
        \set{\V \subs K^n \st \text{vanishing}} 
        \ar[r,rightharpoonup,"I(\star)",shift left = 1]
        \ar[r,leftharpoondown,"\V(\star)",swap,shift right = 1]
        &\set{\f{a} \leq K[x_1,\dots,x_n] \st r(\f{a}) = \f{a}}
    \end{cd}
\end{cor}
\begin{proof}
    There exists a prime ideal $\f{a} \leq K[x_1,\dots,x_n]$ such that 
    $X = \V(\f{a})$ and $\f{a}$ is the radical of itself.
    By \linkto{strong_nullstellensatz_appendix}{strong Nullstellensatz}
    we have that $I(\V(\f{a})) = \f{a}$ thus $\V(I(\V(\f{a}))) = \V(\f{a})$.
    Hence $\V(I(X)) = X$.
\end{proof}

\begin{prop}[Irreducible]
    \link{irreducible_equiv_defs}
    If $X$ is a non-empty topological space
    then the following are equivalent:
    \begin{enumerate}
        \item Any non-empty open set is dense in $X$.
        \item Any pair of non-empty open subsets intersect non-trivially.
        \item Any two closed proper subsets do not form a cover of $X$.
    \end{enumerate}
\end{prop}
\begin{proof}
    $(1. \implies 2.)$ Let $U,V \subs Y$ be open.
    $\bar{V} = X$ by assumption.
    Hence $\nothing \ne U \subs \bar{V}$ 
    and so their intersection is non-trivial.
    Hence $U \cap V$ is non-trivial.

    $(2. \implies 1.)$ Let $U$ be open and non-empty. 
    Then if $\bar{U} \ne X$ then its complement is non-empty and so 
    by assumption $U \cap X \setminus \bar{U}$ is non-empty,
    a contradiction.

    $(2. \iff 3.)$ is clear.
\end{proof}

\begin{dfn}[Regular map]
    Let $X \subs K^n$ and $Y \subs K^m$ be Zariski closed sets over field $K$.
    Then $\rho : X \to Y$ is regular when there exist polynomials 
    $\set{f_i}_{i = 1}^m \subs K[x_1,\dots,x_n]$ such that 
    $\rho$ is the restriction to $X$ of the map $K^n \to K^m$
    \[a \mapsto (f_i(a))_{i=1}^m\]
\end{dfn}

\begin{dfn}[Prime spectrum (Zariski topology)]
    \link{prime_spec_zariski_top}
    In commutative algebra, 
    for any ring $A$ there is a topology $\spec(A)$
    (the spectrum of $A$),
    the set of all prime ideals in $A$.
    This is generated by the closed sets, 
    namely for any $E \subs A$ the set 
    \[V(E) := \set{\f{p} \in \spec(A) \st E \subs \f{p}}\]
    the `vanishing' is closed. 
    (These will generalise the vanishings
    in the classical Zariski topolgy.)
    It can be shown that under finite union and arbitrary intersection of these
    sets are still closed thus it defines a topology on $\spec(A)$.
    Furthermore we have that for any $E \subs A$, 
    $V(E) = V(\<E\>)$ where the latter is the ideal generated by $E$.
\end{dfn}
We wish to relate all this to the classical setting.
We take our ring $A$ to be $K[x_1,\dots,x_n]$.
We will show that the set of vanishings in $\spec(K[x_1,\dots,x_n])$
bijects with
the set of Zariski closed sets in $K^n$.
Then we will take the varieties as the closed sets in $K[x_1,\dots,x_n]$.

\begin{nttn}[Ideal generated by varieties]
    For a subset $X \subs K^n$, 
    we write $I(X)$ to mean the ideal of $X$ in $K[x_1,\dots,x_n]$ to mean
    \[\set{f \in K[x_1,\dots, x_n] \st \forall a \in X, f(a) = 0}\]
\end{nttn}

\begin{prop}[Correspondence between Zariski topology and prime spectrum]
    \link{zariski_correspondence}
    Given $K$ a field and a vanishing in the spectrum of 
    $K[x_1,\dots, x_n]$, there exists a unique variety of some finite 
    subset of $K[x_1,\dots,x_n]$.
    and vice versa.
\end{prop}
\begin{proof}
    Let $V(E) \in B$ then 
    $\<E\>$ is finitely generated by the 
    \linkto{hilbert_basis}{Hilbert basis theorem}
    so there exists some finite subset $S \subs K[x_1, \dots, x_n]$ such that
    \[V(E) = V(\<E\>) = V(\<S\>)\]
    We send $V(E)$ to $\V_K(S)$.
    This is well defined: suppose $V(\<S\>) = V(\<T\>)$ then
    for any prime ideal $\f{p}$, 
    $\<S\> \subs \f{p} \iff \<T\> \subs \f{p}$.
    Hence 
    \[
        r(S) = \bigcap_{\<S\> \subs \f{p} \text{ prime }} \f{p}
        = \bigcap_{\<T\> \subs \f{p} \text{ prime }} \f{p}
        = r(T)
    \]
    Let $a \in \V_K(S)$, then for any $f \in <S>$, $f(a) = 0$.
    To show that $a \in \V_K(T)$,
    let $f \in T$,
    then there exists some $n$ such that $f^n \in <S>$ as $r(S) = r(T)$.
    Hence $f^n(a) = 0$ and since $K[x_1,\dots,x_n]$ is an integral domain,
    by induction $f(a) = 0$.
    Thus $a \in \V_K(T)$.

    This map is clearly surjective and by
    \linkto{strong_nullstellensatz}{strong Nullstellensatz}
    it is injective:
    \[\V_K(S) = \V_K(T) \implies I(\V_K(S)) = I(\V_K(T)) \implies 
    r(S) = I(\V_K(S)) = I(\V_K(T)) = r(T)\]
    hence 
    \[  
        \<S\> \subs \f{p} \iff r(S) \subs \f{p} 
        \iff r(T) \subs \f{p} \iff \<T\> \subs \f{p}
    \]
    Thus $V(S) = V(T)$.
\end{proof}

\begin{prop}
    If $K$ is an algebraically closed field then 
    $\spec(K[x_1,\dots,x_n])$ is not Hausdorff.
\end{prop}
\begin{proof}
    Let $\f{p}$ and $\f{q}$ be distinct prime ideals.
    We show that any opens containing 
    each of them have non-trivial intersection.

    Let $U_\f{p}, U_\f{q}$ be open sets containing the respective ideals.
    Closed sets are of the form $V(\f{a})$ for $\f{a}$ an ideal of 
    $K[x_1,\dots,x_n]$ which by the 
    \linkto{hilbert_basis}{Hilbert basis theorem} is finitely generated.
    Hence there exist finite sets $S_\f{p},S_\f{q} \subs K[x_1,\dots,x_n]$
    such that
    \[U_\f{p} = \set{y \in \spec \st r(S_\f{p}) \nsubseteq y}\]
    and similarly for $U_\f{q}$.
    If $S_\f{p}$ contains only $0$ then $U_\f{p}$ would be empty.
    Thus there exist 
    $f \in S_\f{p} \setminus \set{0}, g \in S_\f{p} \setminus \set{0}$.
    It suffices to show that there is a prime ideal that contains neither 
    $f$ nor $g$.

    As \linkto{card_of_alg_closed_fields}{algebraically closed fields} 
    are infinite and so we can inject $\io : \N \to K$.
    Define for each $m \in \N$ a prime ideal $\f{a}_m := \<x_1 + \io(m)\>$.
    If for all $m$, $f \in \f{a}_m$ then by the division algorithm $f$
    must be the $0$ polynomial, a contradiction. 
    (Similarly for $g$.) 
    Hence there is some ideal $\f{a}_m$ containing neither $f$ nor $g$.
\end{proof}
%\section{Krull Dimension}

\begin{lem}[Varieties and prime ideals]
    \link{ideal_of_variety_prime_ideal}
    Let $K$ be an algebraically closed field and $X \subs K^n$ 
    a Zariski closed set.
    Then $X$ is a variety (i.e. $X$ is irreducible) 
    if and only if $I(X)$ is prime.
\end{lem}
\begin{proof}
    \begin{forward}
        Suppose $X$ is irreducible.
        If $f,g \notin I(X)$ then there exist $a,b \in X$ such that 
        $f(a) \ne 0$ and $g(b) \ne 0$.
        Then $X \cap \V(f)$ and $X \cap \V(g)$ are proper subsets of $X$ and 
        are Zariski closed.
        Hence 
        \[X \cap (\V(fg)) = X \cap (\V(f) \cup \V(g)) \subset X\]
        This is a proper subset since $X$ is irreducible.
        Thus there exists $c \in X$ such that $fg(a) \ne 0$
        and so $fg \notin I(X)$.
    \end{forward}

    \begin{backward}
        Suppose $X$ is reducible, i.e. there are $U,V \subset X$ Zariski closed 
        proper subsets such that $U \cup V = X$.
        \linkto{galois_correspondence_ideals_vanishings}{By 
            the Galois correspondence between ideals and vanishings}
        we have $I(X) \subset I(U), I(V)$.
        Take $f \in I(U) \setminus I(X), g \in I(V) \setminus I(X)$ and note 
        that for any $a \in X$, 
        $a \in U$ or $a \in V$ so $f(a) = 0$ or $g(a) = 0$,
        hence $fg(a) = 0$ and $fg \in I(X)$.
        Hence $I(X)$ is not prime.
    \end{backward}
\end{proof}

\begin{dfn}[Krull dimension]
    \link{dfn_krull_dimension}
    Let $K$ be an algebraically closed field and $V \subs K^n$ be a variety.
    Let $S \subs \N$ be the set of naturals $m$ such that 
    there exists a chain of closed irreducible subspaces ending with $V$ with 
    length $m$:
    \[
        V_0 \subset V_1 \subset \dots \subset V_m = V
    \]
    By the \linkto{galois_correspondence_ideals_vanishings}{Galois
        correspondence between ideals and vanishings}
    this is the same as existance of a chain of prime ideals starting with the 
    \linkto{ideal_of_variety_prime_ideal}{prime ideal} $I(V)$
    \[  
        I(V) = \f{p}_m \subset 
        \dots \subset \f{p}_1 \subset \f{p}_0 \subset K[x_1,\dots,x_n]
    \]
    The set $S$ is bounded since $K[x_1,\dots,x_n]$ is 
    \linkto{hilbert_basis}{Noetherian}.
    Thus we can define the Krull dimension\footnote{
        We could define Krull dimension simply for any topological space
        by not requiring the last closed irreducible subspace to be equial to 
        the whole space. 
        However, this may not just be a natural number.
    } of $V$ 
    to be the maximum element of $S$.
    We denote this as $\kdim(V)$.
\end{dfn}
\begin{rmk}
    \link{krull_dim_0}
    $V$ has Krull dimension $0$ if and only if $I(V)$ is a maximal ideal 
    if and only if $V$ is a singleton:
    the first equivalence is clear.
    If $V$ is a singleton then trivially it has no irreducible subsets 
    (as irreducible requires non-empty) hence has dimension $0$.

    If $V$ has dimension $0$ firstly note that $V$ is non-empty by definition
    of irreducible.
    Then assume for a contradiction there are two distinct points $a, b$ in $V$.
    The singleton $\set{a}$ is closed as it is the vanishing of the polynomials
    $x_1 - a_1 \dots, x_n - a_n$.
    Then $\set{a}$ is a closed and irreducible subset of $V$,
    so the Krull dimension of $V$ is greater than or equal to $1$,
    a contradiction.
\end{rmk}

\begin{dfn}[Function field]
    Let $K$ be an algebraicaly closed field and $V \subs K^n$ a variety.
    Then we write $K(V)$ to mean the 
    \linkto{field_of_fractions}{field of fractions} of 
    \[K[x_1,\dots,x_n] / I(V)\]
    noting that this is well-defined since $I(V)$ is a prime ideal
    and so the quotient is an integral domain.
    We call $K(V)$ the function field of $V$.
\end{dfn}

\begin{prop}[Equivalent definition of Krull dimension]
    \link{transendence_deg_is_krull}
    Let $K$ be an algebraically closed field and $V \subs K^n$ be a variety.
    Then the Krull dimension of $V$ is equal to the 
    \linkto{transcendence_degree_dfn}{transcendence degree} of 
    the function field $K(V)$ over $K$.
\end{prop}
\begin{proof}
    See \cite{atiyah}. %????????
\end{proof}





\printbibliography

\end{document}